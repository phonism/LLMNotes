\section{稀疏注意力}
\label{sec:sparse-attention}

前两节介绍的FlashAttention和MLA分别从计算效率和内存占用角度优化了注意力机制,但它们都保留了完整的$O(N^2)$注意力计算——只是让这个计算更快、更省内存。本节探索另一条路径:如果大多数注意力权重本就接近零,我们能否\textbf{跳过这些无意义的计算}?

稀疏注意力的核心思想是:只计算"重要"的token对,将$O(N^2)$降至$O(Nk)$,其中$k \ll N$。与后面将介绍的线性注意力不同,稀疏注意力保留了精确的softmax计算,只是在更小的范围内进行。

\subsection{稀疏注意力概述}

\subsubsection{为什么需要稀疏注意力}

标准Softmax Attention的复杂度为$O(N^2)$,但实际上并非所有token对都同等重要:
\begin{itemize}
    \item 注意力分布通常是稀疏的(少数token获得大部分权重)
    \item 远距离token的注意力通常较弱
    \item 语义相关的token往往聚集在特定位置
\end{itemize}

稀疏注意力的核心思想:\textbf{只计算重要的注意力对},将复杂度从$O(N^2)$降至$O(Nk)$,其中$k \ll N$。

\subsubsection{稀疏注意力 vs 线性注意力}

\begin{table}[htbp]
\centering
\caption{稀疏注意力与线性注意力对比}
\label{tab:sparse_vs_linear}
\begin{tabular}{lcc}
\toprule
特性 & 稀疏注意力 & 线性注意力 \\
\midrule
复杂度 & $O(Nk)$ & $O(Nd^2)$ \\
注意力类型 & 精确Softmax & 近似/替代Softmax \\
长程精确检索 & 强 & 弱 \\
KV Cache & 需要完整 & 可压缩 \\
与原始Transformer兼容 & 高 & 中 \\
\bottomrule
\end{tabular}
\end{table}

\subsection{滑动窗口注意力}

滑动窗口注意力(Sliding Window Attention, SWA)是最直观的稀疏注意力形式:每个token只关注其周围固定窗口内的token。

\subsubsection{Mistral的滑动窗口~\citep{jiang2023mistral}}

Mistral 7B是首个将滑动窗口注意力规模化部署的开源模型,窗口大小为4096。

\paragraph{核心机制}
每个位置$t$的token只关注$[t-w, t]$范围内的token:
\begin{equation}
    \text{Attention}_t = \text{softmax}\left(\frac{q_t K_{[t-w:t]}^\top}{\sqrt{d}}\right) V_{[t-w:t]}
\end{equation}

\paragraph{层间信息传递}
滑动窗口的关键洞察是:通过Transformer的堆叠层,信息可以``跨窗口''传播。在第$k$层,位置$t$的token实际上可以访问到$[t - k \cdot w, t]$范围的信息。对于32层模型、窗口大小4096,理论感受野可达128K。

\paragraph{推理优化}
\begin{itemize}
    \item \textbf{滚动缓存}(Rolling Buffer):KV Cache只需保留最近$w$个token
    \item \textbf{内存节省}:8K序列长度下节省50\%缓存
    \item \textbf{速度提升}:配合FlashAttention和xFormers,16K序列上获得2倍加速
\end{itemize}

\subsubsection{Longformer与BigBird}

Longformer~\citep{beltagy2020longformer}和BigBird~\citep{zaheer2020bigbird}是早期将稀疏注意力系统化的工作。

\paragraph{Longformer}
结合三种注意力模式:
\begin{enumerate}
    \item \textbf{滑动窗口}:局部上下文建模
    \item \textbf{膨胀滑动窗口}(Dilated):扩大感受野
    \item \textbf{全局注意力}:特定token(如[CLS])关注全局
\end{enumerate}

\paragraph{BigBird}
在Longformer基础上增加\textbf{随机注意力}:
\begin{equation}
    A = A_{\text{sliding}} + A_{\text{global}} + A_{\text{random}}
\end{equation}
理论证明BigBird是完整注意力的通用近似器,支持最长8倍于BERT的序列。

\subsubsection{StreamingLLM~\citep{xiao2024streamingllm}}

StreamingLLM解决了一个重要问题:如何让LLM处理``无限长''的流式输入。

\paragraph{Attention Sink现象}
研究发现,无论输入多长,模型总是对\textbf{最开头的几个token}分配异常高的注意力权重——即使这些token在语义上并不重要。这被称为``注意力汇聚''(Attention Sink)。

\paragraph{SinkAttention}
StreamingLLM提出保留两部分KV Cache:
\begin{itemize}
    \item \textbf{Sink Tokens}:序列开头的4个token(固定)
    \item \textbf{滑动窗口}:最近的$w$个token
\end{itemize}
\begin{equation}
    \text{KV Cache} = \text{Sink}_{[1:4]} \cup \text{Window}_{[t-w:t]}
\end{equation}

\textbf{效果}:在4M+ token的流式场景下保持稳定性能,而普通滑动窗口在超过预训练长度后崩溃。

\subsection{KV Cache稀疏化}

KV Cache稀疏化是推理时的稀疏注意力:动态丢弃``不重要''的KV条目。

\subsubsection{H2O~\citep{zhang2024h2o}}

Heavy-Hitter Oracle(H2O)基于一个观察:少数``重击手''token累积了大部分注意力权重。

\paragraph{算法}
\begin{enumerate}
    \item 维护每个token的累积注意力分数
    \item 保留分数最高的Top-k token(Heavy Hitters)
    \item 结合最近的滑动窗口token
\end{enumerate}

\paragraph{层级策略}
不同层使用不同的保留策略,高层(更稀疏的注意力分布)可以更激进地压缩。

\subsubsection{SnapKV~\citep{li2024snapkv}}

SnapKV的核心洞察是:注意力模式在prefill阶段基本确定,可以``一次性''剪枝。

\paragraph{方法}
\begin{enumerate}
    \item 在prefill阶段末尾,分析最后一个窗口的注意力分布
    \item 识别整个上下文中的重要位置
    \item 永久保留这些位置的KV,丢弃其余
\end{enumerate}

\textbf{优势}:只需一次剪枝决策,无需每步动态更新。\\
\textbf{局限}:无法适应decoding阶段的动态变化。

\subsubsection{PyramidKV}

PyramidKV发现不同层的注意力稀疏度不同:
\begin{itemize}
    \item 低层:注意力较分散,需要更多KV
    \item 高层:注意力集中,可大幅压缩
\end{itemize}
因此采用\textbf{金字塔形}的KV分配:底层多、高层少。

\begin{table}[htbp]
\centering
\caption{KV Cache稀疏化方法对比}
\label{tab:kv_sparse}
\begin{tabular}{lccc}
\toprule
方法 & 剪枝时机 & 动态性 & 集成框架 \\
\midrule
H2O & 每步 & 动态 & vLLM \\
SnapKV & Prefill后 & 静态 & vLLM \\
StreamingLLM & 持续 & 静态 & -- \\
PyramidKV & 层级 & 静态 & -- \\
\bottomrule
\end{tabular}
\end{table}

\subsection{MoBA:块稀疏注意力}
\label{sec:moba}

MoBA(Mixture of Block Attention)~\citep{lu2025moba}是块级稀疏注意力的代表性工作,已部署于Kimi的长上下文服务。

\subsubsection{核心思想:将MoE应用于Attention}

MoBA的核心洞察是:\textbf{并非所有上下文对当前token都同等重要}。与其对整个序列计算注意力,不如让模型自主学习``关注哪些块''。

\paragraph{标准注意力}
\begin{equation}
    \text{Attn}(q, K, V) = \text{softmax}(qK^\top)V
\end{equation}

\paragraph{MoBA}
\begin{equation}
    \text{MoBA}(q, K, V) = \text{softmax}(qK_{[\mathcal{I}]}^\top)V_{[\mathcal{I}]}
    \label{eq:moba}
\end{equation}
其中$\mathcal{I} \subseteq [N]$是被选中的KV子集,由路由机制决定。

\subsubsection{块划分与路由机制}

\paragraph{块划分}
将长度为$N$的上下文均匀划分为$n$个块,每块大小$B = N/n$:
\begin{equation}
    \mathcal{I}_i = [(i-1) \cdot B + 1, \, i \cdot B], \quad i = 1, \ldots, n
\end{equation}

\paragraph{路由分数计算}
对每个query $q$,计算其与各块的亲和度分数:
\begin{equation}
    s_i = \langle q, \, \text{mean\_pool}(K_{[\mathcal{I}_i]}) \rangle
    \label{eq:moba_score}
\end{equation}
即query与块内所有key的平均向量的内积。这是一个\textbf{无参数}的路由机制。

\paragraph{Top-k选择}
选择分数最高的$k$个块进行注意力计算(典型设置$k = 3$):
\begin{equation}
    g_i = \begin{cases}
        1 & \text{if } s_i \in \text{Top-}k(\{s_j\}_{j=1}^n) \\
        0 & \text{otherwise}
    \end{cases}
\end{equation}

\subsubsection{因果性保证}

在自回归场景下:
\begin{enumerate}
    \item \textbf{未来块屏蔽}:对于位置$t$的query,所有$i > \lceil t/B \rceil$的块设$s_i = -\infty$
    \item \textbf{当前块强制选中}:query所在的块始终被路由
\end{enumerate}

\subsubsection{性能表现}

\begin{table}[htbp]
\centering
\caption{MoBA性能}
\label{tab:moba_performance}
\begin{tabular}{lcc}
\toprule
指标 & MoBA & Full Attention \\
\midrule
LM Loss差异 & \multicolumn{2}{c}{$< 10^{-3}$} \\
稀疏度 @32K & \multicolumn{2}{c}{95.31\%} \\
加速比 @1M & \multicolumn{2}{c}{6.5$\times$} \\
加速比 @10M & \multicolumn{2}{c}{16$\times$} \\
\bottomrule
\end{tabular}
\end{table}

\subsection{NSA:原生稀疏注意力}

Native Sparse Attention(NSA)~\citep{deepseek2025nsa}是DeepSeek提出的层级稀疏注意力机制,已部署于DeepSeek-V3.2。

\subsubsection{三条注意力路径}

NSA将注意力计算分解为三条并行路径:

\begin{figure}[htbp]
\centering
\begin{tikzpicture}[scale=0.7, font=\small]
    % Input
    \node[draw, rounded corners, fill=blue!20] (input) at (0, 0) {Query $q$};

    % Three paths
    \node[draw, rounded corners, fill=green!20] (comp) at (-4, -2) {Compression};
    \node[draw, rounded corners, fill=orange!20] (sel) at (0, -2) {Selection};
    \node[draw, rounded corners, fill=red!20] (sw) at (4, -2) {Sliding Window};

    % Arrows from input
    \draw[->, thick] (input) -- (comp);
    \draw[->, thick] (input) -- (sel);
    \draw[->, thick] (input) -- (sw);

    % Descriptions
    \node[below, text width=3cm, align=center] at (comp.south) {粗粒度\\块压缩};
    \node[below, text width=3cm, align=center] at (sel.south) {细粒度\\Top-k选择};
    \node[below, text width=3cm, align=center] at (sw.south) {局部\\滑动窗口};

    % Merge
    \node[draw, rounded corners, fill=purple!20] (merge) at (0, -5) {合并输出};
    \draw[->, thick] (comp) -- (merge);
    \draw[->, thick] (sel) -- (merge);
    \draw[->, thick] (sw) -- (merge);
\end{tikzpicture}
\caption{NSA的三条注意力路径}
\label{fig:nsa_paths}
\end{figure}

\paragraph{1. Compression Attention(压缩注意力)}
使用可学习的MLP将连续token压缩为块级表示:
\begin{equation}
    \tilde{K}_i = \text{MLP}(K_{[(i-1)l+1:il]}), \quad \tilde{V}_i = \text{MLP}(V_{[(i-1)l+1:il]})
\end{equation}
其中$l$是压缩块大小(NSA中$l=32$)。这捕获\textbf{全局粗粒度}信息。

\paragraph{2. Selection Attention(选择注意力)}
通过Lightning Indexer选择最相关的块保持原始精度:
\begin{enumerate}
    \item 计算query与所有块的相关性分数
    \item 选择Top-$n$个块(NSA中$n=16$个块,块大小$l'=64$)
    \item 对选中块进行精确Softmax注意力
\end{enumerate}
这保留\textbf{细粒度精确}信息。

\paragraph{3. Sliding Window Attention(滑动窗口注意力)}
对最近的$w$个token进行完整注意力(NSA中$w=512$):
\begin{equation}
    O_{sw} = \text{Attention}(q, K_{[t-w:t]}, V_{[t-w:t]})
\end{equation}
这保证\textbf{局部上下文}的精确建模。

\subsubsection{Lightning Indexer}

Lightning Indexer是NSA的核心创新,用于高效选择相关块:

\begin{itemize}
    \item 维护独立的\textbf{FP8量化}Key缓存(非MLA的KV Cache)
    \item 每个query计算与所有块的相关性分数
    \item 选择Top-k块(默认2048个token)
    \item 硬件优化:DeepGEMM实现的高效CUDA kernel
\end{itemize}

\textbf{关键设计}:索引计算与注意力计算分离,索引使用低精度快速完成。

\subsubsection{端到端可训练}

与ClusterKV、MagicPIG等依赖不可微操作的方法不同,NSA是\textbf{原生可训练}的——从预训练阶段就使用稀疏注意力。

\paragraph{可微的稀疏机制}
NSA的三个分支都支持梯度回传:
\begin{enumerate}
    \item \textbf{Token Compression}:可学习的MLP $\phi$将块级KV压缩为单个表示,带块内位置编码
    \item \textbf{Token Selection}:块重要性分数通过压缩token的注意力softmax计算,梯度可流过选择过程
    \item \textbf{门控机制}:通过MLP+Sigmoid学习三个分支的权重
\end{enumerate}

\paragraph{预训练配置}
DeepSeek使用27B参数模型(MoE架构,3B激活参数)验证NSA:
\begin{itemize}
    \item \textbf{模型规格}:30层,hidden\_dim=2560,GQA(4组,64头)
    \item \textbf{MoE结构}:72个路由专家 + 2个共享专家,top-k=6
    \item \textbf{训练数据}:270B tokens,8K序列长度
    \item \textbf{长上下文适配}:继续训练 + SFT在32K序列上,使用YaRN位置编码
\end{itemize}

\paragraph{训练加速}
NSA在训练阶段也实现了显著加速(64K序列长度,A100 GPU):
\begin{itemize}
    \item \textbf{前向传播}:9.0$\times$加速
    \item \textbf{反向传播}:6.0$\times$加速
    \item 加速比随序列长度增加:8K时4$\times$,16K时6.4$\times$,32K时9.1$\times$,64K时11.6$\times$
\end{itemize}

\paragraph{后训练:推理能力}
通过从DeepSeek-R1蒸馏,使用10B tokens的32K长度数学推理轨迹进行SFT。在AIME数学benchmark上,NSA-R比Full Attention-R提升+0.075,证明稀疏注意力不损害复杂推理能力。

\subsubsection{硬件对齐设计}

NSA针对现代GPU进行了深度优化,使用自定义Triton kernel:

\paragraph{Group-Centric Loading}
利用GQA的特性,将同一组内所有query head及其共享的稀疏KV块索引一次性加载到SRAM,避免重复内存访问。

\paragraph{Shared KV Fetching}
顺序加载被选中的连续KV块,最小化内存传输次数。由于不同query块的内循环长度几乎相同,可使用Triton的grid scheduler简化优化。

\paragraph{算术强度平衡}
通过组内共享消除冗余KV传输,在GPU流处理器间均衡计算负载,实现接近最优的算术强度。

\subsubsection{NSA参数配置}

\begin{table}[htbp]
\centering
\caption{NSA默认参数}
\label{tab:nsa_params}
\begin{tabular}{lcc}
\toprule
参数 & 符号 & 值 \\
\midrule
压缩块大小 & $l$ & 32 \\
选择块大小 & $l'$ & 64 \\
选择块数量 & $n$ & 16 \\
滑动窗口大小 & $w$ & 512 \\
滑动步长 & $d$ & 16 \\
\bottomrule
\end{tabular}
\end{table}

\subsection{DSA:DeepSeek稀疏注意力}

DSA(DeepSeek Sparse Attention,2025年9月)是DeepSeek在V3.2中部署的新一代稀疏注意力,与NSA有本质区别。DSA摒弃了NSA复杂的三分支设计,采用更简洁的\textbf{细粒度token级检索}。

\subsubsection{与NSA的核心区别}

\begin{table}[htbp]
\centering
\caption{NSA vs DSA设计对比}
\label{tab:nsa_vs_dsa}
\begin{tabular}{lcc}
\toprule
特性 & NSA & DSA \\
\midrule
选择粒度 & 块级(block) & Token级 \\
分支数量 & 3(压缩+选择+窗口) & 1(直接选择) \\
重要度计算 & 可学习MLP & 可学习$w$权重 \\
Attention变种 & GQA & MLA \\
验证模型 & 27B & 671B \\
\bottomrule
\end{tabular}
\end{table}

\subsubsection{算法设计}

DSA的核心思想:每个query只需关注固定数量$k$个最相关的token($k=2048$)。

\paragraph{重要度分数计算}
DSA引入可学习权重$w$计算token重要度:
\begin{equation}
    \text{score}_i = w \cdot f(q, k_i)
\end{equation}
这是一个折中方案——比NSA的MLP简单,但比MoBA的无参数mean-pooling更有表达力。

\paragraph{Top-k检索}
根据重要度分数选择Top-$k$个token进行精确注意力计算:
\begin{equation}
    \mathcal{I} = \text{Top-}k(\{\text{score}_i\}_{i=1}^N), \quad |\mathcal{I}| = 2048
\end{equation}

\paragraph{复杂度分析}
单query需访问固定$k$个token,因此整体复杂度为$O(Nk)$,是真正的\textbf{线性复杂度}。

\subsubsection{训练挑战与解决}

细粒度token级检索比块级更难训练。DSA采用以下技巧:
\begin{itemize}
    \item \textbf{Warm-up}:逐步增加稀疏度
    \item \textbf{知识蒸馏}:从Full Attention模型蒸馏
    \item \textbf{MLA适配}:专门针对Multi-Latent Attention优化
\end{itemize}

\subsubsection{工程实现}

\begin{itemize}
    \item \textbf{TileLang Kernel}:细粒度稀疏+MLA需要定制kernel,TileLang比Triton性能更优
    \item \textbf{vLLM/SGLang集成}:Day-0支持,使用DeepGEMM和FlashMLA
    \item \textbf{Blackwell优化}:与NVIDIA合作优化B200
\end{itemize}

\subsubsection{性能收益}

\begin{itemize}
    \item 长上下文API成本降低约\textbf{50\%}
    \item 64K序列上实现显著加速
    \item 671B模型上验证,质量几乎无损
\end{itemize}

\subsection{NSA vs MoBA vs DSA:三种方法深度对比}

2025年出现了三种重要的``学习式''稀疏注意力方法。本节从多个维度进行详细对比。

\subsubsection{算法设计对比}

\begin{table}[htbp]
\centering
\caption{三种方法算法设计对比}
\label{tab:three_algo_comparison}
\begin{tabular}{lccc}
\toprule
设计维度 & NSA & MoBA & DSA \\
\midrule
\multicolumn{4}{l}{\textit{基本信息}} \\
发布时间 & 2025.02 & 2025.02 & 2025.09 \\
提出者 & DeepSeek & Moonshot (Kimi) & DeepSeek \\
Attention基础 & GQA & GQA & MLA \\
\midrule
\multicolumn{4}{l}{\textit{稀疏策略}} \\
选择粒度 & 块级 & 块级 & Token级 \\
分支数量 & 3(压缩+选择+窗口) & 1 & 1 \\
路由机制 & 可学习MLP & 无参数mean-pool & 可学习$w$ \\
局部窗口 & 有($w$=512) & 当前块强制选中 & 无 \\
\bottomrule
\end{tabular}
\end{table}

\subsubsection{超参数与复杂度对比}

\begin{table}[htbp]
\centering
\caption{超参数与复杂度对比}
\label{tab:three_params_comparison}
\begin{tabular}{lccc}
\toprule
参数 & NSA & MoBA & DSA \\
\midrule
\multicolumn{4}{l}{\textit{关键超参}} \\
块大小 & $l$=32, $l'$=64 & $L$=4096 & -- \\
选择数量 & $n$=16块 & $k$=12块 & $k$=2048 tokens \\
滑动窗口 & $w$=512 & -- & -- \\
\midrule
\multicolumn{4}{l}{\textit{单query访问token数}} \\
@32K上下文 & $\sim$2560 & 49152 & 2048 \\
@128K上下文 & $\sim$5120 & 49152 & 2048 \\
是否随$N$增长 & 是($O(N/L)$) & 否(固定$kL$) & 否(固定$k$) \\
\midrule
\multicolumn{4}{l}{\textit{复杂度(vs原版$O(N^2)$)}} \\
理论复杂度 & $O(N^2/L)$ & $O(N \cdot kL)$ & $O(Nk)$ \\
实际系数 & $O(N^2/32)$ & $O(N \cdot 49152)$ & $O(N \cdot 2048)$ \\
@64K稀疏度 & $\sim$97\% & $\sim$23\% & $\sim$97\% \\
\bottomrule
\end{tabular}
\end{table}

\subsubsection{重要度计算机制}

三种方法采用不同的``检索''策略来确定哪些token值得关注:

\paragraph{NSA:可学习MLP压缩}
\begin{equation}
    \tilde{k}_i = \text{MLP}(K_{[(i-1)l+1:il]}), \quad \text{score}_i = q \cdot \tilde{k}_i
\end{equation}
使用一个额外的MLP将块内所有key压缩为单个表示,再计算与query的相似度。\textbf{优点}:表达力强。\textbf{缺点}:引入额外参数和计算。

\paragraph{MoBA:无参数Mean-Pooling}
\begin{equation}
    \text{score}_i = q \cdot \text{mean}(K_{[(i-1)L+1:iL]})
\end{equation}
直接对块内key做平均,再与query内积。\textbf{优点}:无额外参数,简洁优雅。\textbf{缺点}:表达力受限,块内信息被``抹平''。

\paragraph{DSA:可学习权重$w$}
\begin{equation}
    \text{score}_i = w \cdot f(q, k_i)
\end{equation}
介于两者之间——有可学习参数,但比MLP简单。\textbf{优点}:平衡表达力与复杂度。\textbf{缺点}:具体形式未完全公开。

\subsubsection{训练与工程对比}

\begin{table}[htbp]
\centering
\caption{训练与工程实现对比}
\label{tab:three_training_comparison}
\begin{tabular}{lccc}
\toprule
维度 & NSA & MoBA & DSA \\
\midrule
\multicolumn{4}{l}{\textit{训练}} \\
训练方式 & 原生预训练 & 继续训练 & 原生预训练 \\
验证模型规模 & 27B & 8B & 671B \\
训练数据量 & 270B tokens & -- & -- \\
训练序列长度 & 8K$\to$32K & -- & -- \\
需要蒸馏 & 可选(R1蒸馏) & 否 & 是(关键) \\
需要Warm-up & 否 & 否 & 是(关键) \\
\midrule
\multicolumn{4}{l}{\textit{工程实现}} \\
Kernel实现 & Triton & Triton & TileLang \\
与MLA兼容 & 否 & 否 & 是 \\
推理框架支持 & -- & -- & vLLM/SGLang \\
\midrule
\multicolumn{4}{l}{\textit{加速效果}} \\
前向加速@64K & 9$\times$ & 6.5$\times$@1M & -- \\
后向加速@64K & 6$\times$ & -- & -- \\
解码加速@64K & 11.6$\times$ & 16$\times$@10M & -- \\
\bottomrule
\end{tabular}
\end{table}

\subsubsection{设计哲学差异}

\paragraph{NSA:全面覆盖,层级融合}
NSA的设计理念是\textbf{不遗漏任何重要信息}:
\begin{itemize}
    \item \textbf{压缩分支}:捕获全局粗粒度信息
    \item \textbf{选择分支}:保留细粒度精确信息
    \item \textbf{窗口分支}:确保局部上下文不丢失
    \item 三分支通过可学习门控融合
\end{itemize}
\textbf{代价}:超参较多($l$, $l'$, $n$, $w$, $d$, 门控参数),调优和迁移复杂。

\paragraph{MoBA:简洁优雅,MoE思想}
MoBA的设计理念是\textbf{将MoE的路由思想应用于Attention}:
\begin{itemize}
    \item 把KV Cache视为``专家池'',每个块是一个``专家''
    \item 无参数路由,让注意力分数自然决定选择
    \item 当前块强制选中,保证局部信息
\end{itemize}
\textbf{优势}:设计最简洁,与现有架构兼容性好。很多人认为MoBA更有潜力成为下一代Transformer的基础组件。\\
\textbf{疑问}:块大小4096较大(约1页PDF),是否会退化为sliding window等trivial模式?

\paragraph{DSA:激进稀疏,端到端优化}
DSA的设计理念是\textbf{极限稀疏+精心训练}:
\begin{itemize}
    \item Token级选择,每个query只看2048个token(约3\%@64K)
    \item 与MLA深度集成,需要定制TileLang kernel
    \item 训练难度大,依赖warm-up和蒸馏
\end{itemize}
\textbf{优势}:稀疏度最高,推理效率最好,在671B超大模型上验证。\\
\textbf{挑战}:Token级检索是否会遗漏关键信息?小模型能否复现?

\subsubsection{稀疏模式可视化理解}

假设上下文长度$N$=32K,三种方法的稀疏模式差异:

\begin{table}[htbp]
\centering
\caption{@32K上下文的稀疏模式}
\label{tab:sparsity_pattern}
\begin{tabular}{lccc}
\toprule
特性 & NSA & MoBA & DSA \\
\midrule
块数量 & 1024块(每块32 token) & 8块(每块4096 token) & 32K个token \\
选中数量 & 16块+窗口 & 12块 & 2048 token \\
选中token数 & $\sim$2560 & $\sim$49152 & 2048 \\
稀疏度 & 92\% & 0\%(看全部) & 94\% \\
\bottomrule
\end{tabular}
\end{table}

\textbf{关键观察}:在32K这个长度上,MoBA实际上几乎没有稀疏(选中12块$\times$4096=49152 > 32K)!MoBA的稀疏优势在更长序列(如128K+)才能体现。

\subsubsection{适用场景建议}

\begin{table}[htbp]
\centering
\caption{适用场景建议}
\label{tab:when_to_use}
\begin{tabular}{lp{10cm}}
\toprule
方法 & 适用场景 \\
\midrule
NSA & 需要精确保留多尺度信息;可接受复杂超参调优;使用GQA架构 \\
MoBA & 追求简洁设计;希望无缝替换现有Attention;序列长度128K+ \\
DSA & 使用MLA架构;追求极致稀疏度;有足够资源做蒸馏训练;超大模型(100B+) \\
\bottomrule
\end{tabular}
\end{table}

\subsubsection{开放问题与未来方向}

\paragraph{固定$k$是否合理?}
DSA和MoBA都假设每个query只需看固定数量的token,不随$N$增长。这个假设值得深思:
\begin{itemize}
    \item \textbf{反对观点}:上下文很长时,固定$k$可能遗漏关键信息。Needle-in-haystack任务可能受影响。
    \item \textbf{支持观点}:每个token的``信息容量''有限。人类阅读长文档时也不会逐字关注。
    \item \textbf{折中思路}:可能需要自适应$k$,或结合全局压缩分支(如NSA)。
\end{itemize}

\paragraph{稀疏度应该自适应吗?}
目前三种方法的稀疏度都是人为设定的超参。未来方向可能包括:
\begin{itemize}
    \item 不同层使用不同稀疏度(类似PyramidKV的思想)
    \item 不同query根据任务复杂度动态调整$k$
    \item 通过强化学习或可微搜索自动确定稀疏策略
\end{itemize}

\paragraph{稀疏 vs 压缩:两条技术路线}
\begin{itemize}
    \item \textbf{稀疏路线}(NSA/MoBA/DSA):动态``检索''重要token,不改变KV Cache大小
    \item \textbf{压缩路线}(Qwen3-Next线性注意力):压缩KV Cache本身,信息更紧凑
    \item 两者可能结合:先压缩再稀疏,或用稀疏注意力指导压缩
\end{itemize}

\paragraph{Post-training方法是否被``降维打击''?}
原生稀疏注意力(NSA/DSA)的出现,对H2O、SnapKV、Quest等post-training稀疏方法形成挑战:
\begin{itemize}
    \item Post-training方法:在已训练好的模型上``事后''稀疏化
    \item 原生方法:从预训练开始就使用稀疏注意力
    \item 原生方法的优势:模型从一开始就学会``如何稀疏'',效果更好
    \item Post-training方法的优势:可应用于任意已有模型,无需重新训练
\end{itemize}

\subsection{Ring Attention与上下文并行}

当序列长度超过单GPU显存容量时,需要将注意力计算分布到多个设备上。

\subsubsection{Ring Attention~\citep{liu2023ringattention}}

Ring Attention将长序列分割到多个GPU,通过环形通信实现分布式注意力计算。

\paragraph{算法流程}
\begin{enumerate}
    \item 将Query、Key、Value按序列维度分割到$P$个GPU
    \item 每个GPU持有本地Query块,计算与本地KV的注意力
    \item KV块在GPU之间环形传递,累积计算全局注意力
    \item 使用online softmax避免数值溢出
\end{enumerate}

\paragraph{通信隐藏}
关键优化是\textbf{计算-通信重叠}:在计算当前KV块注意力的同时,异步传递下一个KV块。

\subsubsection{LLaMA 3的上下文并行}

LLaMA 3训练采用All-Gather方式的Context Parallelism~\citep{dubey2024llama3}:
\begin{itemize}
    \item 先All-Gather收集所有KV,再计算本地Query的注意力
    \item 为负载均衡,将序列分为$2 \times \text{CP}$个块并shuffle
    \item 支持128K上下文的高效训练
\end{itemize}

\paragraph{与Ring Attention对比}
\begin{itemize}
    \item Ring Attention:点对点通信,延迟可隐藏
    \item All-Gather:集合通信,实现更简单但延迟在关键路径
\end{itemize}

\subsection{稀疏注意力方法全景}

\begin{table}[htbp]
\centering
\caption{稀疏注意力方法全景对比}
\label{tab:sparse_comparison}
\begin{tabular}{lccccc}
\toprule
方法 & 稀疏策略 & 复杂度 & 全局信息 & 部署 \\
\midrule
Sliding Window & 固定窗口 & $O(Nw)$ & 无 & Mistral \\
Longformer & 窗口+全局 & $O(N(w+g))$ & 全局token & Longformer \\
BigBird & 窗口+全局+随机 & $O(N(w+g+r))$ & 全局+随机 & BigBird \\
StreamingLLM & Sink+窗口 & $O(N(s+w))$ & Sink tokens & -- \\
MoBA & 块路由 & $O(N \cdot kL)$ & Top-k块 & Kimi \\
NSA & 压缩+选择+窗口 & $O(N^2/L)$ & 压缩+选择 & -- \\
DSA & Token级检索 & $O(Nk)$ & Top-k token & DeepSeek-V3.2 \\
Ring Attention & 分布式 & $O(N^2/P)$ & 完整 & LLaMA 3 \\
\bottomrule
\end{tabular}
\end{table}

\begin{remark}[稀疏注意力的演进]
从Longformer/BigBird的``手工设计模式''到MoBA/NSA/DSA的``学习式稀疏'',稀疏注意力正经历范式转变。2025年,稀疏注意力首次在600B+规模模型上得到验证(DeepSeek-V3.2),标志着该技术从学术研究走向工业主流。
\end{remark}

\subsection{与其他方法的关系}

\subsubsection{与FlashAttention的关系}

稀疏注意力与FlashAttention是\textbf{正交}的优化:
\begin{itemize}
    \item FlashAttention优化\textbf{IO效率}(不改变计算量)
    \item 稀疏注意力减少\textbf{计算量}(只计算部分注意力)
    \item 两者可以结合:对选中的稀疏token使用FlashAttention计算
\end{itemize}

\subsubsection{与KV Cache压缩的关系}

\begin{itemize}
    \item \textbf{KV Cache压缩}(如SnapKV、H2O):推理时丢弃不重要的KV
    \item \textbf{稀疏注意力}:训练和推理时都只计算部分注意力
    \item 稀疏注意力是更根本的解决方案,从架构层面减少计算
\end{itemize}

\subsection{实践建议}

\paragraph{何时使用稀疏注意力}
\begin{itemize}
    \item 长上下文场景(32K+)
    \item 需要精确检索能力(passkey retrieval等)
    \item 希望保持Softmax注意力的特性
\end{itemize}

\paragraph{何时使用线性注意力}
\begin{itemize}
    \item 超长上下文(1M+)
    \item KV Cache显存受限
    \item 可以接受一定的精度损失
\end{itemize}

\begin{remark}[稀疏注意力的发展趋势]
稀疏注意力正从``手工设计模式''(如固定滑动窗口)向``学习式稀疏''演进:
\begin{itemize}
    \item MoBA:无参数路由,让模型学习关注哪些块
    \item NSA:可学习的压缩和选择机制
    \item 未来:更细粒度、更自适应的稀疏策略
\end{itemize}
\end{remark}

