% ============================================
% 15. 多模态大模型
% ============================================
\section{多模态大模型}
\label{sec:multimodal}

随着大语言模型(LLM)在文本理解和生成上取得突破性进展,研究者开始探索如何将视觉、音频等多模态信息与语言能力相结合。多模态大模型(Multimodal Large Language Models, MLLMs)已成为人工智能领域最活跃的研究方向之一。本章将系统介绍多模态大模型的架构设计,重点讨论"原生多模态"(Native Multimodal)这一新兴范式。

\subsection{多模态大模型概述}

\subsubsection{从单模态到多模态}

传统的大语言模型只能处理文本输入和输出。为了让模型具备"看"和"听"的能力,研究者提出了多种将视觉信息融入语言模型的方法。根据模态融合的深度和方式,多模态大模型可分为以下几类:

\begin{itemize}
    \item \textbf{级联式(Cascaded)}:多个独立模型串联,如先用视觉模型提取描述,再输入语言模型
    \item \textbf{适配器式(Adapter-based)}:在预训练LLM基础上添加视觉适配器,如LLaVA、BLIP-2
    \item \textbf{原生式(Native)}:从头开始在多模态数据上联合训练,如GPT-4o、Gemini
\end{itemize}

\subsubsection{核心挑战}

构建多模态大模型面临几个关键挑战:

\textbf{模态对齐(Modality Alignment)}:图像和文本存在于不同的表示空间,需要建立有效的跨模态映射。图像是连续的像素值,而文本是离散的token序列,如何让两者在同一语义空间中对齐是核心问题。

\textbf{信息压缩}:一张224×224的图像包含50176个像素,而典型的视觉编码器会产生196-576个视觉token。如何在保留关键信息的同时压缩视觉表示,避免对LLM造成过大的序列长度负担?

\textbf{理解与生成的统一}:视觉理解(如VQA)需要高层语义抽象,而图像生成需要细粒度的像素级信息。如何在单一模型中同时支持这两种看似矛盾的需求?

\subsection{视觉编码器}

视觉编码器是多模态大模型的"眼睛",负责将图像转换为语言模型可理解的表示。

\subsubsection{Vision Transformer (ViT)}

Vision Transformer \cite{dosovitskiy2020image}将Transformer架构应用于图像处理。其核心思想是将图像切分为固定大小的patch,然后像处理文本token一样处理这些patch:

\begin{equation}
    \mathbf{z}_0 = [\mathbf{x}_\text{class}; \mathbf{E}\mathbf{x}_1; \mathbf{E}\mathbf{x}_2; ...; \mathbf{E}\mathbf{x}_N] + \mathbf{E}_\text{pos}
\end{equation}

其中$\mathbf{x}_i \in \mathbb{R}^{P^2 \cdot C}$是第$i$个图像patch的展平向量,$\mathbf{E}$是patch embedding矩阵,$\mathbf{E}_\text{pos}$是位置编码。

\subsubsection{CLIP与对比学习}

CLIP(Contrastive Language-Image Pre-training)\cite{radford2021learning}通过对比学习在4亿图像-文本对上训练视觉编码器,使其输出的图像表示与对应文本描述在语义空间中对齐:

\begin{equation}
    \mathcal{L}_\text{CLIP} = -\frac{1}{N}\sum_{i=1}^{N}\left[\log\frac{\exp(\text{sim}(\mathbf{v}_i, \mathbf{t}_i)/\tau)}{\sum_{j=1}^{N}\exp(\text{sim}(\mathbf{v}_i, \mathbf{t}_j)/\tau)}\right]
\end{equation}

CLIP的视觉编码器(通常是ViT-L/14)因其强大的跨模态对齐能力,成为早期多模态大模型的标准选择。

\subsubsection{SigLIP与改进}

SigLIP \cite{zhai2023siglip}对CLIP的训练目标进行了改进,使用sigmoid损失替代softmax:

\begin{equation}
    \mathcal{L}_\text{SigLIP} = -\frac{1}{N}\sum_{i=1}^{N}\sum_{j=1}^{N}\log\sigma(y_{ij} \cdot \text{sim}(\mathbf{v}_i, \mathbf{t}_j) \cdot \tau)
\end{equation}

其中$y_{ij} = 1$当$i=j$,否则$y_{ij} = -1$。这种设计允许更大的batch size训练,且不需要全局负样本同步,使训练更加高效。SigLIP在InternVL、Qwen2-VL等新一代模型中广泛使用。

\subsection{模态融合机制}

将视觉特征注入语言模型的方式决定了多模态模型的架构设计。目前主流的融合机制包括:

\subsubsection{线性投影(Linear Projection)}

最简单的方式是使用线性层将视觉特征映射到语言模型的embedding空间:

\begin{equation}
    \mathbf{H}_v = \mathbf{W}_\text{proj} \cdot \mathbf{Z}_\text{vision} + \mathbf{b}
\end{equation}

\textbf{LLaVA} \cite{liu2023llava}最初采用这种方法,通过一个简单的线性投影矩阵连接CLIP ViT-L/14和Vicuna:

\begin{itemize}
    \item 保持视觉编码器和LLM的参数冻结
    \item 仅训练投影矩阵(约2M参数)
    \item 两阶段训练:预训练对齐 + 指令微调
\end{itemize}

LLaVA-1.5 \cite{liu2024llava15}将线性投影升级为两层MLP,显著提升了多模态能力:

\begin{equation}
    \mathbf{H}_v = \mathbf{W}_2 \cdot \text{GELU}(\mathbf{W}_1 \cdot \mathbf{Z}_\text{vision})
\end{equation}

\subsubsection{Q-Former(Querying Transformer)}

BLIP-2 \cite{li2023blip2}提出了Q-Former架构,使用可学习的query token通过交叉注意力从视觉特征中提取信息:

\begin{equation}
    \mathbf{Q}_\text{out} = \text{CrossAttn}(\mathbf{Q}_\text{learnable}, \mathbf{K}_\text{vision}, \mathbf{V}_\text{vision})
\end{equation}

Q-Former的核心设计:
\begin{itemize}
    \item 32个可学习的query embeddings(维度768)
    \item 基于BERT初始化的Transformer块
    \item 交叉注意力层与自注意力层交替堆叠
    \item 输出固定数量的视觉token(32个),无论输入图像分辨率
\end{itemize}

\textbf{两阶段预训练}:
\begin{enumerate}
    \item 视觉-语言表示学习:使用ITC、ITM、ITG三种损失训练Q-Former与冻结的视觉编码器对齐
    \item 视觉-语言生成学习:Q-Former输出接入冻结的LLM,训练生成能力
\end{enumerate}

BLIP-2在VQAv2零样本任务上超越Flamingo-80B达8.7\%,而可训练参数仅为后者的1/54。

\subsubsection{交叉注意力适配器(Cross-Attention Adapter)}

Flamingo \cite{alayrac2022flamingo}和LLaMA 3.2 Vision采用在LLM内部插入交叉注意力层的方式:

\begin{equation}
    \mathbf{h}_l' = \mathbf{h}_l + \text{CrossAttn}(\mathbf{h}_l, \mathbf{K}_\text{vision}, \mathbf{V}_\text{vision})
\end{equation}

\textbf{Flamingo}的架构特点:
\begin{itemize}
    \item Perceiver Resampler:将可变长度的视觉特征压缩为固定长度
    \item 在LLM的每几层插入gated cross-attention层
    \item 冻结LLM参数,仅训练新增的交叉注意力层
\end{itemize}

\textbf{LLaMA 3.2 Vision} \cite{dubey2024llama3}基于LLaMA 3.1构建:
\begin{itemize}
    \item 在冻结的LLaMA 3.1文本模型上添加视觉适配器
    \item 适配器包含多层交叉注意力,将图像编码器表示注入LLM
    \item 训练过程中更新视觉编码器和适配器,但冻结LLM参数
    \item 保持文本能力不变,实现LLaMA 3.1的"即插即用"替换
\end{itemize}

\subsubsection{融合机制对比}

\begin{table}[htbp]
\centering
\caption{模态融合机制对比}
\begin{tabular}{lcccl}
\toprule
\textbf{方法} & \textbf{代表模型} & \textbf{新增参数} & \textbf{视觉token数} & \textbf{特点} \\
\midrule
线性投影 & LLaVA & $\sim$2M & 576 & 简单高效 \\
MLP投影 & LLaVA-1.5 & $\sim$20M & 576 & 表达能力更强 \\
Q-Former & BLIP-2 & $\sim$107M & 32 & 压缩视觉信息 \\
Cross-Attention & LLaMA 3.2 & $\sim$1B & 可变 & 深度融合 \\
\bottomrule
\end{tabular}
\end{table}

\subsection{代表性多模态大模型}

\subsubsection{LLaVA系列}

LLaVA(Large Language and Vision Assistant)是最具影响力的开源多模态大模型之一。

\textbf{LLaVA-1.0}:
\begin{itemize}
    \item 视觉编码器:CLIP ViT-L/14(冻结)
    \item 语言模型:Vicuna-7B/13B(冻结)
    \item 连接方式:线性投影层
    \item 训练数据:595K图像-文本对(预训练)+ 158K视觉指令数据(微调)
\end{itemize}

\textbf{LLaVA-1.5}的改进:
\begin{itemize}
    \item MLP替代线性投影
    \item 输入分辨率从224提升到336
    \item 增加学术VQA数据
    \item 更大的语言模型(Vicuna-13B)
\end{itemize}

\textbf{LLaVA-NeXT}进一步支持动态分辨率,将图像切分为多个子图像分别编码。

\subsubsection{Qwen-VL系列}

\textbf{Qwen-VL} \cite{bai2023qwenvl}使用更大的视觉编码器和更高分辨率:
\begin{itemize}
    \item 视觉编码器:OpenCLIP ViT-bigG(448×448)
    \item 语言模型:Qwen-7B
    \item 连接方式:单层交叉注意力
\end{itemize}

\textbf{Qwen2-VL} \cite{wang2024qwen2vl}的创新:
\begin{itemize}
    \item \textbf{动态分辨率}:移除ViT的绝对位置编码,引入2D-RoPE,支持任意分辨率输入
    \item \textbf{M-RoPE}:Multimodal Rotary Position Embedding,将旋转位置编码分解为时间和空间(高度、宽度)三部分,同时捕获1D文本、2D图像、3D视频的位置信息
    \item \textbf{Token压缩}:MLP层将相邻2×2 token压缩为1个,224×224图像仅产生66个视觉token
\end{itemize}

\subsubsection{InternVL系列}

InternVL \cite{chen2024internvl}的独特设计在于视觉编码器的大规模化:
\begin{itemize}
    \item 视觉编码器扩展到60亿参数(InternViT-6B)
    \item 引入QLLaMA作为"胶水层"(8B参数)连接视觉和语言
    \item 三阶段训练:对比学习 → 生成学习 → 指令微调
\end{itemize}

InternVL 2.5是首个在MMMU基准上突破70\%的开源模型,达到GPT-4o水平。

\subsection{原生多模态模型}
\label{subsec:native-multimodal}

\subsubsection{什么是"原生多模态"?}

"原生多模态"(Native Multimodal)指的是模型从设计之初就具备多模态处理能力,而非在单模态模型基础上"嫁接"其他模态。

\textbf{非原生多模态}(如ChatGPT with GPT-4V):
\begin{itemize}
    \item 文本生成:GPT-4
    \item 图像理解:GPT-4V(独立的视觉模块)
    \item 语音识别:Whisper
    \item 图像生成:DALL-E 3
    \item 各模块独立,通过API或文本中转连接
\end{itemize}

\textbf{原生多模态}(如GPT-4o、Gemini):
\begin{itemize}
    \item 单一神经网络端到端处理所有模态
    \item 在多模态数据上从头联合训练
    \item 模态间共享表示空间,实现深度融合
    \item 无需模态间的文本中转,减少信息损失
\end{itemize}

\subsubsection{GPT-4o}

GPT-4o("o"代表"omni",全能)于2024年5月发布,是OpenAI首个原生多模态旗舰模型。

\textbf{核心特点}:
\begin{itemize}
    \item 单一模型端到端处理文本、音频、视觉输入
    \item 可直接生成文本、音频、图像输出
    \item 实时语音对话延迟降至232ms(接近人类反应速度)
    \item 音频输入保留语调、情感等非语义信息
\end{itemize}

\textbf{与GPT-4V的区别}:
\begin{itemize}
    \item GPT-4V:上传图像 → 视觉模型识别 → 转换为文本描述 → GPT-4处理 → 生成回复
    \item GPT-4o:上传图像 → 直接理解并生成回复(无中间转换)
\end{itemize}

由于GPT-4o架构未公开,具体技术细节不得而知,但其性能表明原生多模态训练的巨大优势。

\subsubsection{Google Gemini}

Gemini \cite{team2023gemini}是Google的原生多模态模型系列。

\textbf{技术报告声明}:
\begin{quote}
"Gemini models are natively multimodal, as they are trained jointly across text, image, audio, and video."
\end{quote}

\textbf{架构特点}:
\begin{itemize}
    \item 早期融合(Early Fusion)架构
    \item 从预训练阶段就在多模态数据上联合训练
    \item 支持32K(Gemini 1.0)到1M(Gemini 1.5/2.5)token上下文
\end{itemize}

\textbf{模型系列}:
\begin{itemize}
    \item Gemini Ultra:最大规模,MMLU首次超越人类专家水平
    \item Gemini Pro:均衡性能与效率
    \item Gemini Nano:端侧部署优化
    \item Gemini 2.5 Pro:2025年发布,加入"思考模型"能力
\end{itemize}

\subsubsection{Meta Chameleon}

Chameleon \cite{team2024chameleon}是Meta开源的原生多模态模型,采用彻底的早期融合架构。

\textbf{核心设计}:
\begin{itemize}
    \item 将所有模态(图像、文本、代码)表示为离散token
    \item 统一的词表包含文本、代码和图像token
    \item 使用标准Transformer架构处理混合模态序列
    \item 端到端从头训练,无需单独的图像编码器/解码器
\end{itemize}

\textbf{图像离散化}:
Chameleon使用改进的VQ-VAE将图像编码为离散token:
\begin{itemize}
    \item 图像编码为1024个离散token(32×32 latent grid)
    \item Codebook大小8192
    \item 与文本token共享统一的embedding空间
\end{itemize}

\textbf{训练规模}:
\begin{itemize}
    \item 7B和34B参数版本
    \item 约4.4万亿token训练数据(文本、图像-文本对、交错序列)
    \item 超过500万A100 GPU小时
\end{itemize}

\textbf{与Gemini的区别}:
Chameleon是完全端到端的稠密模型,没有路由组件或单独的图像解码器,而Gemini使用独立的图像解码器。

\subsection{统一理解与生成}

传统多模态模型要么专注于理解(如VQA),要么专注于生成(如文生图)。近期研究开始探索在单一模型中统一这两种能力。

\subsubsection{挑战与矛盾}

理解和生成对视觉表示有不同要求:
\begin{itemize}
    \item \textbf{理解}:需要高层语义抽象,关注"是什么"
    \item \textbf{生成}:需要细粒度细节,关注"怎么画"
\end{itemize}

使用同一个视觉编码器同时服务两种任务会产生冲突——语义编码器(如CLIP)擅长理解但生成的图像缺乏细节;像素编码器(如VQ-GAN)能重建细节但语义理解能力弱。

\subsubsection{Show-o}

Show-o \cite{xie2024showo}提出用单一Transformer统一理解和生成:

\textbf{核心设计}:
\begin{itemize}
    \item \textbf{Omni-Attention}:对文本token使用因果注意力,对图像token使用全注意力
    \item \textbf{混合建模}:文本使用自回归生成,图像使用离散扩散模型
    \item \textbf{统一词表}:文本token和图像token(VQ-GAN编码)共享词表
\end{itemize}

\textbf{任务能力}:
\begin{itemize}
    \item 图像描述(Image Captioning)
    \item 视觉问答(VQA)
    \item 文本生成图像(Text-to-Image)
    \item 图像编辑(Inpainting/Outpainting)
    \item 混合模态生成
\end{itemize}

Show-o在VQAv2上超越NExT-GPT和Chameleon等更大模型,同时在图像生成上达到FID 9.24(MSCOCO 30K)。

\subsubsection{Janus}

DeepSeek的Janus \cite{wu2024janus}采用"解耦编码、统一处理"的策略:

\textbf{核心洞察}:理解和生成需要不同的视觉编码,但可以共享语言模型处理。

\textbf{双编码器设计}:
\begin{itemize}
    \item \textbf{理解编码器}:SigLIP,提取高层语义特征
    \item \textbf{生成编码器}:VQ tokenizer,产生离散视觉表示
    \item \textbf{共享Transformer}:统一处理两种编码的token序列
\end{itemize}

\textbf{Janus-Pro}(2025年1月)进一步提升:
\begin{itemize}
    \item 基于DeepSeek-LLM-7B
    \item MMBench达到79.2(超越LLaVA-v1.5)
    \item 图像生成FID 8.53(MSCOCO 30K)
\end{itemize}

\subsubsection{JanusFlow}

JanusFlow将生成端从离散token改为连续流(Rectified Flow):
\begin{itemize}
    \item 理解端保持不变(SigLIP编码器)
    \item 生成端使用Rectified Flow替代VQ tokenizer
    \item 图像生成质量进一步提升
\end{itemize}

\subsection{视觉Tokenizer}

视觉tokenizer是原生多模态和统一模型的关键组件,负责将连续图像转换为离散token。

\subsubsection{VQ-VAE与VQ-GAN}

\textbf{VQ-VAE} \cite{van2017vqvae}首次提出将连续表示映射到可学习的离散codebook:

\begin{equation}
    z_q = \arg\min_{e_k \in \mathcal{C}} \|z_e - e_k\|_2
\end{equation}

其中$z_e$是编码器输出,$\mathcal{C}$是codebook。

\textbf{VQ-GAN} \cite{esser2021vqgan}在VQ-VAE基础上引入对抗损失:

\begin{equation}
    \mathcal{L}_\text{VQ-GAN} = \mathcal{L}_\text{rec} + \mathcal{L}_\text{commit} + \mathcal{L}_\text{GAN} + \mathcal{L}_\text{perceptual}
\end{equation}

VQ-GAN能将256×256图像编码为16×16=256个离散token,每个token来自大小为1024-16384的codebook。

\subsubsection{语义vs像素Tokenizer}

\begin{table}[htbp]
\centering
\caption{视觉Tokenizer类型对比}
\begin{tabular}{lccl}
\toprule
\textbf{类型} & \textbf{代表} & \textbf{Codebook} & \textbf{特点} \\
\midrule
像素级 & VQ-GAN & 8K-16K & 重建质量高,语义弱 \\
语义级 & CLIP-ViT & - & 语义强,无法重建 \\
混合 & SEED & 8K & 兼顾语义和重建 \\
统一 & TokenFlow & 16K & 双编码器+共享映射 \\
\bottomrule
\end{tabular}
\end{table}

\subsubsection{前沿进展}

\textbf{TokenFlow}首次证明离散视觉输入可以在理解任务上超越连续编码器(如LLaVA),通过双编码器和共享映射实现语义与细节的联合优化。

\textbf{UniTok}提出统一框架,将离散视觉和语言序列用统一的next-token prediction训练,复用code embedding到MLLM token空间。

\subsection{架构演进与对比}

\subsubsection{多模态架构发展脉络}

\begin{enumerate}
    \item \textbf{2022:适配器时代}
    \begin{itemize}
        \item Flamingo:Cross-Attention + Perceiver Resampler
        \item BLIP-2:Q-Former桥接冻结编码器和LLM
    \end{itemize}

    \item \textbf{2023:简化与规模化}
    \begin{itemize}
        \item LLaVA:简单MLP投影,证明"简单也有效"
        \item Qwen-VL:更大视觉编码器(ViT-bigG)
        \item Gemini:原生多模态训练
    \end{itemize}

    \item \textbf{2024:原生与统一}
    \begin{itemize}
        \item GPT-4o:原生多模态交互
        \item Chameleon:开源早期融合
        \item Show-o/Janus:统一理解与生成
        \item LLaMA 3.2 Vision:Cross-Attention适配器
    \end{itemize}

    \item \textbf{2025:深度融合}
    \begin{itemize}
        \item InternVL 2.5:6B视觉编码器
        \item Qwen2-VL:动态分辨率 + M-RoPE
        \item Janus-Pro:更强的统一模型
    \end{itemize}
\end{enumerate}

\subsubsection{架构对比总结}

\begin{table}[htbp]
\centering
\caption{主流多模态架构对比}
\resizebox{\textwidth}{!}{
\begin{tabular}{lccccc}
\toprule
\textbf{模型} & \textbf{视觉编码器} & \textbf{融合方式} & \textbf{LLM训练} & \textbf{原生} & \textbf{统一生成} \\
\midrule
LLaVA-1.5 & CLIP ViT-L & MLP投影 & 微调 & \ding{55} & \ding{55} \\
BLIP-2 & EVA-CLIP & Q-Former & 冻结 & \ding{55} & \ding{55} \\
LLaMA 3.2 & 自定义ViT & Cross-Attn & 冻结 & \ding{55} & \ding{55} \\
Qwen2-VL & DFN ViT & MLP & 全量微调 & \ding{55} & \ding{55} \\
InternVL 2.5 & InternViT-6B & MLP & 全量微调 & \ding{55} & \ding{55} \\
\midrule
Gemini & 内置 & 早期融合 & 联合预训练 & \ding{51} & \ding{51} \\
GPT-4o & 内置 & 端到端 & 联合预训练 & \ding{51} & \ding{51} \\
Chameleon & VQ-VAE & Token统一 & 从头训练 & \ding{51} & \ding{51} \\
\midrule
Show-o & VQ-GAN & Omni-Attn & 从LLM初始化 & 部分 & \ding{51} \\
Janus & SigLIP+VQ & 解耦编码 & 从LLM初始化 & 部分 & \ding{51} \\
\bottomrule
\end{tabular}
}
\end{table}

\subsection{设计选择与权衡}

\subsubsection{视觉编码器:冻结vs微调}

\textbf{冻结}(BLIP-2, LLaMA 3.2):
\begin{itemize}
    \item 优点:保持预训练视觉能力,训练效率高
    \item 缺点:无法针对特定任务优化视觉表示
\end{itemize}

\textbf{微调}(Qwen2-VL, InternVL):
\begin{itemize}
    \item 优点:视觉表示可针对性优化
    \item 缺点:可能遗忘预训练知识,需要更多训练资源
\end{itemize}

\subsubsection{视觉Token数量}

\begin{itemize}
    \item \textbf{固定少量}(32,Q-Former):计算高效,但可能丢失细节
    \item \textbf{固定中等}(576,LLaVA):平衡信息保留和效率
    \item \textbf{动态}(Qwen2-VL):根据图像复杂度调整,最灵活
\end{itemize}

\subsubsection{原生vs适配器}

\textbf{适配器方式}的优势:
\begin{itemize}
    \item 可复用强大的预训练LLM
    \item 训练成本相对较低
    \item 保持文本能力不退化
\end{itemize}

\textbf{原生多模态}的优势:
\begin{itemize}
    \item 模态间深度融合,减少信息损失
    \item 端到端优化,无接口瓶颈
    \item 支持实时多模态交互(如语音对话)
\end{itemize}

\subsection{多模态后训练}
\label{subsec:multimodal-posttraining}

与纯文本LLM类似,多模态大模型也需要通过后训练(Post-training)来提升指令遵循能力、减少幻觉、与人类偏好对齐。然而,多模态场景引入了新的挑战:视觉-语言对齐、跨模态幻觉、以及图像条件下的偏好学习。

\subsubsection{视觉指令微调}

视觉指令微调(Visual Instruction Tuning)是多模态后训练的基础阶段,通过高质量的多模态指令数据训练模型遵循视觉相关的指令。

\textbf{LLaVA的开创性工作} \cite{liu2023llava}:
\begin{itemize}
    \item 首次使用纯文本GPT-4生成多模态指令数据
    \item 基于COCO图像和边界框/描述,扩展为三类指令:对话式QA、详细描述、复杂推理
    \item 训练数据:558K预训练对齐数据 + 665K指令微调数据
\end{itemize}

\textbf{数据质量的重要性}:
LLaVA系列的演进(1.0 → 1.5 → NeXT)充分证明了数据质量对视觉指令微调的关键作用:
\begin{itemize}
    \item \textbf{数据多样性}:覆盖对话、描述、推理、学术VQA等多种任务
    \item \textbf{数据复杂度}:从简单问答到多轮对话、长文本描述
    \item \textbf{语言多样性}:ShareGPT4V等数据集支持多语言指令
\end{itemize}

\textbf{两阶段训练范式}:
\begin{enumerate}
    \item \textbf{预训练对齐}:在大规模图像-文本对上训练投影层,建立视觉-语言连接
    \item \textbf{指令微调}:在高质量指令数据上微调,获得指令遵循能力
\end{enumerate}

\subsubsection{多模态RLHF}

将RLHF(Reinforcement Learning from Human Feedback)从文本域扩展到多模态场景,面临独特挑战。

\textbf{LLaVA-RLHF} \cite{sun2023llava-rlhf}是首个开源的多模态RLHF工作:

\textbf{核心问题}:多模态幻觉——模型生成的文本与图像内容不符。

\textbf{方法}:Factually Augmented RLHF (Fact-RLHF)
\begin{itemize}
    \item 收集10K人类偏好数据,标注者比较两个回复并指出哪个更"幻觉"
    \item 使用图像描述或多选题答案作为额外信息校准奖励信号
    \item 通过PPO算法优化策略模型
\end{itemize}

\textbf{训练流程}:
\begin{enumerate}
    \item 在LLaVA-SFT+基础上添加LoRA模块
    \item 用10K偏好数据训练奖励模型
    \item 通过PPO进行强化学习
\end{enumerate}

\textbf{效果}:
\begin{itemize}
    \item LLaVA-Bench:达到GPT-4纯文本版94\%的水平(此前最佳仅87\%)
    \item MMHal-Bench:相比基线提升60\%
\end{itemize}

\textbf{RLHF-V} \cite{yu2024rlhfv}提出更细粒度的方法:
\begin{itemize}
    \item 收集片段级(segment-level)纠正标注,而非整体比较
    \item 使用Dense DPO直接在细粒度反馈上优化
    \item 仅需1.4K偏好数据,将幻觉率降低34.8\%
\end{itemize}

\subsubsection{多模态DPO}

DPO(Direct Preference Optimization)因其简单高效,在文本LLM对齐中广泛使用。然而,直接将DPO应用于多模态场景面临独特问题。

\textbf{无条件偏好问题}(Unconditional Preference Problem):
标准DPO在多模态场景中,模型可能忽略图像条件,仅基于文本模式学习偏好。这导致:
\begin{itemize}
    \item 模型过度依赖语言先验
    \item 视觉信息未被充分利用
    \item 对齐效果不稳定
\end{itemize}

\textbf{mDPO} \cite{wang2024mdpo}(Multimodal DPO)解决方案:

\textbf{核心改进}:
\begin{enumerate}
    \item \textbf{图像偏好优化}:在优化语言偏好的同时,显式优化图像偏好,防止语言偏好被过度优先
    \item \textbf{奖励锚点}(Reward Anchor):强制chosen响应的奖励为正,避免其likelihood下降——这是相对偏好优化的固有问题
\end{enumerate}

\textbf{mDPO损失函数}:
\begin{equation}
    \mathcal{L}_\text{mDPO} = \mathcal{L}_\text{DPO} + \lambda_\text{img} \mathcal{L}_\text{img-pref} + \lambda_\text{anchor} \mathcal{L}_\text{anchor}
\end{equation}

\textbf{实验效果}:在多个benchmark上显著降低幻觉率,特别是在需要精确视觉理解的任务上。

\textbf{其他多模态DPO变体}:
\begin{itemize}
    \item \textbf{V-DPO}:Vision-Guided DPO,通过视觉引导生成负样本
    \item \textbf{SymMPO}:对称多模态偏好优化,在正负样本间建立对称学习
    \item \textbf{负样本增强}:通过随机裁剪或扩散模型编辑图像生成负样本
\end{itemize}

\subsubsection{多模态幻觉与评估}

多模态幻觉(Multimodal Hallucination)是指模型生成的内容与输入图像不符,是多模态后训练需要重点解决的问题。

\textbf{幻觉类型}:
\begin{itemize}
    \item \textbf{对象幻觉}(Object Hallucination):描述图像中不存在的物体
    \item \textbf{属性幻觉}(Attribute Hallucination):错误描述物体的颜色、大小、位置等
    \item \textbf{关系幻觉}(Relation Hallucination):错误描述物体间的空间或语义关系
    \item \textbf{遗漏幻觉}(Omission):未提及图像中的重要内容
    \item \textbf{捏造幻觉}(Fabrication):添加图像中不存在的信息
\end{itemize}

\textbf{POPE评估基准}:
Polling-based Object Probing Evaluation将幻觉检测转化为二分类问题:
\begin{itemize}
    \item 提问:"图像中是否有\texttt{<object>}?"
    \item 三种难度:Random(随机物体)、Popular(常见物体)、Adversarial(相关但不存在的物体)
    \item 评估指标:Accuracy、Precision、Recall、F1
\end{itemize}

\textbf{MMHal-Bench}:
专门评估真实场景中的幻觉,包含96个图像-问题对,覆盖8类问题×12个物体主题。

\textbf{幻觉缓解方法}:
\begin{itemize}
    \item \textbf{对比解码}(Contrastive Decoding):比较原图和扰动图的输出分布
    \item \textbf{视觉对比解码}(VCD):通过高斯噪声等方式扰动图像
    \item \textbf{鲁棒指令微调}:在训练数据中加入幻觉对比样本
    \item \textbf{自省解码}(Self-Introspective Decoding):让模型自我检查输出的一致性
\end{itemize}

\subsubsection{LLaVA-Critic:多模态自我评估}

LLaVA-Critic \cite{xiong2024llavacritic}是首个开源的多模态通用评估模型,能够评估多模态模型在各种任务上的表现。

\textbf{核心能力}:
\begin{enumerate}
    \item \textbf{LMM-as-a-Judge}:提供可靠的评估分数,性能媲美或超越GPT模型
    \item \textbf{偏好学习}:为偏好学习生成奖励信号,提升模型对齐能力
\end{enumerate}

\textbf{训练数据}:
LLaVA-Critic-113k数据集:
\begin{itemize}
    \item 46K图像,113K评估指令样本
    \item 两种评估模式:
    \begin{itemize}
        \item Pointwise Scoring:为单个回复打分
        \item Pairwise Ranking:比较两个回复的相对质量
    \end{itemize}
    \item 30种成对评估prompt模板
\end{itemize}

\textbf{自我奖励(Self-Rewarding)效果}:
使用LLaVA-Critic的反馈训练LLaVA-OV-7B/72B:
\begin{itemize}
    \item 在6个开放式多模态benchmark上持续提升
    \item 效果优于人类偏好训练的奖励模型
    \item 虽然仅使用图像训练,也能提升视频理解能力
\end{itemize}

这表明多模态模型可以通过自我评估实现"超人类反馈"的自我改进路径。

\subsubsection{后训练流程总结}

\begin{table}[htbp]
\centering
\caption{多模态后训练方法对比}
\resizebox{\textwidth}{!}{
\begin{tabular}{lcccc}
\toprule
\textbf{方法} & \textbf{数据需求} & \textbf{训练复杂度} & \textbf{主要目标} & \textbf{代表工作} \\
\midrule
视觉指令微调 & 指令数据 & 低 & 指令遵循 & LLaVA, LLaVA-1.5 \\
RLHF & 偏好数据+奖励模型 & 高 & 减少幻觉 & LLaVA-RLHF \\
DPO/mDPO & 偏好数据 & 中 & 偏好对齐 & mDPO, V-DPO \\
细粒度反馈 & 片段级标注 & 中 & 精确纠错 & RLHF-V \\
自我奖励 & 无需人类标注 & 中 & 自我改进 & LLaVA-Critic \\
\bottomrule
\end{tabular}
}
\end{table}

\textbf{典型后训练流程}:
\begin{enumerate}
    \item \textbf{预训练}:在大规模图像-文本数据上训练模态对齐
    \item \textbf{指令微调}(SFT):在高质量指令数据上训练
    \item \textbf{偏好对齐}(可选):使用RLHF、DPO或mDPO进一步对齐
    \item \textbf{自我改进}(可选):使用Critic模型进行迭代优化
\end{enumerate}

\subsection{未来方向}

\subsubsection{统一所有模态}

当前多数模型主要处理图像和文本,未来将扩展到:
\begin{itemize}
    \item 音频/语音的原生支持
    \item 视频理解与生成
    \item 3D场景理解
    \item 具身智能(Embodied AI)
\end{itemize}

\subsubsection{更高效的视觉表示}

\begin{itemize}
    \item 自适应token数量:根据图像内容复杂度动态调整
    \item 更强的压缩:在更少token中保留更多信息
    \item 层次化表示:粗粒度语义 + 细粒度细节
\end{itemize}

\subsubsection{训练效率}

原生多模态训练需要海量计算资源(Chameleon: 500万A100小时)。未来方向:
\begin{itemize}
    \item 更高效的预训练方法
    \item 模块化训练:分阶段训练不同模态
    \item 合成数据增强
\end{itemize}

\subsubsection{评估与对齐}

\begin{itemize}
    \item 更全面的多模态benchmark
    \item 跨模态对齐的理论理解
    \item 多模态幻觉(hallucination)问题
\end{itemize}

