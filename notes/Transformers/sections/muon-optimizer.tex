\section{Muon优化器}
\label{sec:muon}

Adam优化器自2014年提出以来,一直是深度学习的标准选择。然而,Adam本质上是\textbf{逐元素}(element-wise)的优化器,没有利用神经网络参数的\textbf{矩阵结构}。Muon(MomentUm Orthogonalized by Newton-schulz)优化器~\citep{jordan2024muon}通过对动量进行矩阵正交化,实现了更高效的参数更新,已被Kimi等大规模模型采用~\citep{liu2025muonscalable}。

\subsection{从Adam到矩阵优化}

\subsubsection{Adam的局限}

Adam的更新规则为:
\begin{align}
    m_t &= \beta_1 m_{t-1} + (1 - \beta_1) g_t \\
    v_t &= \beta_2 v_{t-1} + (1 - \beta_2) g_t^2 \\
    \theta_t &= \theta_{t-1} - \eta \cdot \frac{\hat{m}_t}{\sqrt{\hat{v}_t} + \epsilon}
\end{align}
其中所有操作都是\textbf{逐元素}进行的。对于矩阵参数$W \in \R^{d_{out} \times d_{in}}$,Adam将其展平为向量处理,完全忽略了矩阵的行列结构。

\subsubsection{矩阵参数的特殊性}

神经网络的核心参数——线性层权重矩阵——具有天然的矩阵结构:
\begin{itemize}
    \item 行向量对应输出特征的组合权重
    \item 列向量对应输入特征的映射方向
    \item 奇异值分解揭示了参数的``主方向''
\end{itemize}

一个关键观察是:SGD和Adam产生的梯度更新通常具有\textbf{极高的条件数},即接近低秩矩阵。这意味着更新主要沿少数``主方向''进行,而``稀有方向''被严重抑制。

\subsection{Muon的核心思想:正交化动量}

\subsubsection{基本算法}

Muon的核心思想是:将动量矩阵$M$替换为其最近的半正交矩阵,即进行\textbf{极分解}(Polar Decomposition)。

设$M = U\Sigma V^\top$是$M$的奇异值分解,则:
\begin{equation}
    \text{msign}(M) = UV^\top
    \label{eq:msign}
\end{equation}
这称为矩阵的\textbf{符号函数}(matrix sign function),类似于标量的sign函数将所有奇异值映射为1。

\begin{algorithm}[htbp]
\caption{Muon优化器(朴素版)}
\label{alg:muon_naive}
\begin{algorithmic}[1]
\REQUIRE 学习率$\eta$,动量系数$\beta$,初始参数$W_0$
\STATE 初始化动量 $M_0 = 0$
\FOR{$t = 1, 2, \ldots$}
    \STATE 计算梯度 $G_t = \nabla_W \mathcal{L}(W_{t-1})$
    \STATE 更新动量 $M_t = \beta M_{t-1} + G_t$
    \STATE 计算正交化更新 $\Delta_t = \text{msign}(M_t)$
    \STATE 更新参数 $W_t = W_{t-1} - \eta \cdot \Delta_t$
\ENDFOR
\end{algorithmic}
\end{algorithm}

\subsubsection{为什么正交化有效?}

要理解正交化的作用,先看Adam的问题。设梯度矩阵 $G \in \R^{d_{out} \times d_{in}}$ 的奇异值分解为 $G = U\Sigma V^\top$,其中 $\Sigma = \text{diag}(\sigma_1, \ldots, \sigma_r)$,$\sigma_1 \gg \sigma_r$(高条件数)。

\paragraph{Adam的逐元素缩放}
Adam对每个元素 $g_{ij}$ 独立归一化,这相当于在"元素坐标系"下操作。问题在于:梯度的主方向($\sigma_1$ 对应)和稀有方向($\sigma_r$ 对应)在元素坐标系中是耦合的——无法通过逐元素缩放来均衡它们。

\paragraph{Muon的谱归一化}
$\text{msign}(G) = UV^\top$ 将所有奇异值映射为1:
\begin{equation}
    G = U \cdot \text{diag}(\sigma_1, \ldots, \sigma_r) \cdot V^\top \xrightarrow{\text{msign}} U \cdot \text{diag}(1, \ldots, 1) \cdot V^\top
\end{equation}

这意味着:
\begin{enumerate}
    \item \textbf{主方向被抑制}:$\sigma_1 \to 1$,更新幅度降低
    \item \textbf{稀有方向被放大}:$\sigma_r \to 1$,更新幅度提升
    \item \textbf{方向信息保留}:$U, V$ 不变,只改变"步长"
\end{enumerate}

\paragraph{几何直觉}
想象损失函数的等高线是一个狭长的椭圆(高条件数)。Adam沿梯度方向走,容易在窄谷中震荡。Muon将椭圆"压成圆"——沿各方向走相同步长,这正是牛顿法的效果,但无需计算Hessian。

\begin{remark}[与Shampoo的关系]
Muon可以理解为"无累积的Shampoo"。Shampoo通过累积梯度的二阶统计量 $L = \sum G G^\top$,$R = \sum G^\top G$,然后用 $L^{-1/4} G R^{-1/4}$ 预条件化更新。Muon直接用当前梯度的SVD达到类似效果,避免了:
\begin{itemize}
    \item 累积带来的延迟(需要多步才能准确估计统计量)
    \item 逆矩阵平方根的高昂计算成本
    \item 大规模分布式训练中统计量同步的通信开销
\end{itemize}
代价是Muon对单步梯度的噪声更敏感,因此使用动量 $M = \beta M + G$ 来平滑。
\end{remark}

\subsection{Newton-Schulz迭代}

直接计算SVD的复杂度为$O(\min(d_{out}, d_{in})^3)$,对于大矩阵不可接受。Muon使用\textbf{Newton-Schulz迭代}高效近似$\text{msign}(M)$。

\subsubsection{迭代公式}

Newton-Schulz迭代通过多项式逼近矩阵符号函数:
\begin{equation}
    X_{k+1} = aX_k + b(X_k X_k^\top)X_k + c(X_k X_k^\top)^2 X_k
    \label{eq:newton_schulz}
\end{equation}
其中$a = 3.4445$,$b = -4.7750$,$c = 2.0315$是优化过的系数。

\subsubsection{为什么Newton-Schulz迭代有效?}

Newton-Schulz迭代的本质是用多项式逼近函数 $f(x) = \text{sign}(x) = x / |x|$。对于矩阵,这变成逐奇异值操作:$f(\sigma_i) = 1$(将所有正奇异值映射到1)。

设 $X_0 = M / \|M\|_F$(归一化使奇异值落入 $(0, 1]$),迭代后:
\begin{equation}
    X_k = U \cdot \text{diag}(\varphi^{(k)}(\sigma_1/\|M\|_F), \ldots) \cdot V^\top
\end{equation}
其中 $\varphi(x) = ax + bx^3 + cx^5$ 是设计好的五次多项式。

\paragraph{多项式设计的直觉}
目标是让 $\varphi^{(k)}(x) \to 1$ 对所有 $x \in (0, 1]$ 快速收敛。系数 $a, b, c$ 的选择使得:
\begin{itemize}
    \item $\varphi(1) = 1$(不动点)
    \item $\varphi'(1) = 0$(在不动点处导数为0,加速收敛)
    \item 对于较小的 $x$,$\varphi(x) > x$(将小值推向1)
\end{itemize}

实践中\textbf{5次迭代}即可达到足够精度。更少迭代会欠近似,更多迭代无显著收益但增加计算成本。

\subsubsection{实现代码}

\begin{lstlisting}[language=Python, caption={Newton-Schulz迭代实现}]
def newton_schulz5(G, steps=5, eps=1e-7):
    """近似计算 msign(G)"""
    a, b, c = (3.4445, -4.7750, 2.0315)
    X = G.bfloat16()
    X = X / (X.norm() + eps)  # 归一化

    # 确保 d_out <= d_in(转置处理)
    transpose = G.size(0) > G.size(1)
    if transpose:
        X = X.T

    for _ in range(steps):
        A = X @ X.T
        B = b * A + c * A @ A
        X = a * X + B @ X

    if transpose:
        X = X.T
    return X
\end{lstlisting}

\textbf{计算复杂度}:每次迭代需要$O(d_{out} \cdot d_{in} \cdot \min(d_{out}, d_{in}))$的矩阵乘法,5次迭代的总开销约为$5m/B$($m$为模型维度,$B$为batch token数),通常$< 1\%$。

\subsection{Muon的四个版本}

苏剑林在~\citep{su2025muonguide}中总结了Muon的四个主要变体,它们的唯一区别是$\text{msign}$前的\textbf{缩放因子}:

\begin{table}[htbp]
\centering
\caption{Muon四个版本的缩放因子}
\label{tab:muon_versions}
\begin{tabular}{lll}
\toprule
版本 & 缩放因子 & 特点 \\
\midrule
朴素版 & $1$ & 最简单,但学习率不可迁移 \\
KellerJordan版 & $\sqrt{\max(1, d_{out}/d_{in})}$ & 默认版本,Keras实现 \\
MuP版 & $\sqrt{d_{out}/d_{in}}$ & 学习率可迁移 \\
Moonlight版 & $0.2 \times \sqrt{\max(d_{out}, d_{in})}$ & 可直接沿用Adam学习率 \\
\bottomrule
\end{tabular}
\end{table}

\paragraph{为什么需要缩放?}
不同形状的矩阵,$\text{msign}(M)$的Frobenius范数不同(等于$\min(d_{out}, d_{in})$)。缩放因子用于:
\begin{enumerate}
    \item 使不同层的更新幅度匹配
    \item 与AdamW的更新尺度对齐
    \item 实现学习率的跨模型迁移
\end{enumerate}

\paragraph{实践建议}
\begin{itemize}
    \item \textbf{Moonlight版}:可直接沿用Adam的学习率
    \item \textbf{其他版本}:需将Adam学习率乘以约$0.2 \times \sqrt{d_{hidden}}$
\end{itemize}

\subsection{$d_{in}$和$d_{out}$的识别}

不同深度学习框架对线性层的实现方式不同,需要正确识别$d_{in}$和$d_{out}$:

\paragraph{PyTorch}
使用$y = xW^\top + b$,即$W \in \R^{d_{out} \times d_{in}}$:
\begin{lstlisting}[language=Python]
d_out, d_in = W.shape[0], W.shape[1]
\end{lstlisting}

\paragraph{Keras/TensorFlow}
使用$y = xW + b$,即$W \in \R^{d_{in} \times d_{out}}$:
\begin{lstlisting}[language=Python]
d_in, d_out = W.shape[0], W.shape[1]
\end{lstlisting}

\textbf{注意}:搞错$d_{in}$和$d_{out}$会导致缩放因子计算错误,影响训练效果。

\subsection{哪些参数使用Muon?}

Muon专为\textbf{2D矩阵参数}设计,其他参数仍需使用AdamW:

\begin{table}[htbp]
\centering
\caption{参数类型与优化器选择}
\label{tab:muon_params}
\begin{tabular}{lll}
\toprule
参数类型 & 优化器 & 原因 \\
\midrule
隐藏层Linear权重 & Muon & 核心矩阵参数 \\
Attention的$W_Q, W_K, W_V, W_O$ & Muon & 矩阵参数 \\
MLP的$W_{gate}, W_{up}, W_{down}$ & Muon & 矩阵参数 \\
\midrule
Embedding层 & AdamW & 本质是查找表 \\
最终分类头 & AdamW & 与embedding对称 \\
LayerNorm参数 & AdamW & 1D向量 \\
Bias & AdamW & 1D向量 \\
\bottomrule
\end{tabular}
\end{table}

\subsection{大规模训练:Moonlight}

月之暗面(Moonshot AI)在Moonlight模型~\citep{liu2025muonscalable}中验证了Muon的大规模可扩展性。

\subsubsection{关键技术改进}

\begin{enumerate}
    \item \textbf{添加权重衰减}:原始Muon没有weight decay,大规模训练需要加入
    \item \textbf{参数级缩放}:精确调整每个参数的更新比例(即Moonlight版的缩放因子)
\end{enumerate}

\subsubsection{分布式训练挑战}

Muon需要完整的梯度矩阵来计算正交化更新,这与现有分布式策略(ZeRO、Megatron等按元素切分优化器状态)冲突。Moonlight论文提出了内存最优、通信高效的分布式Muon实现。

\subsubsection{性能对比}

\begin{table}[htbp]
\centering
\caption{Muon vs AdamW(Moonlight实验)}
\label{tab:muon_vs_adam}
\begin{tabular}{lcc}
\toprule
指标 & Muon & AdamW \\
\midrule
计算效率 & $\sim 2\times$ & 基准 \\
达到相同性能所需FLOPs & 52\% & 100\% \\
样本效率 & 1.92$\times$ & 基准 \\
\bottomrule
\end{tabular}
\end{table}

Moonlight模型规格:
\begin{itemize}
    \item 总参数:15.29B(MoE架构)
    \item 激活参数:2.24B
    \item 训练数据:5.7T tokens
\end{itemize}

\subsection{实践指南}

\paragraph{何时使用Muon?}
\begin{itemize}
    \item 大规模预训练:Muon的效率优势随规模增大
    \item 对训练效率敏感的场景
    \item 已有AdamW超参数,想进一步优化
\end{itemize}

\paragraph{超参数设置}
\begin{itemize}
    \item \textbf{动量系数}:$\beta = 0.95$(比Adam的0.9略大)
    \item \textbf{Newton-Schulz步数}:5步
    \item \textbf{学习率}:
    \begin{itemize}
        \item Moonlight版:直接用Adam学习率
        \item 其他版本:Adam学习率 $\times$ $0.2\sqrt{d_{hidden}}$
    \end{itemize}
    \item \textbf{权重衰减}:与AdamW相同
\end{itemize}

\paragraph{注意事项}
\begin{enumerate}
    \item Muon只用于2D矩阵参数,其他参数用AdamW
    \item 正确识别框架中的$d_{in}$和$d_{out}$
    \item 对于Attention,建议对$W_Q, W_K, W_V$分别做Newton-Schulz,而非合并
    \item bfloat16精度足够,无需float32
\end{enumerate}

\begin{remark}[从向量到矩阵的本质飞跃]
苏剑林评价~\citep{su2025muonguide}:``Muon相比Adam具有更优雅的设计,体现了从向量到矩阵的本质飞跃。''Adam将矩阵展平为向量处理,而Muon真正利用了矩阵的几何结构。这种``非Element-wise''的优化范式可能代表了未来优化器的发展方向。
\end{remark}
