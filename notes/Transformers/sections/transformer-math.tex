\section{Transformer计算分析}
\label{sec:transformer_math}

本节从计算角度系统分析Transformer架构,涵盖FLOPs计算、参数量统计、内存占用,以及推理优化。理解这些数学关系对于模型设计、训练规划和部署优化至关重要。

% ============================================
% 符号定义
% ============================================
\subsection{符号定义}

表~\ref{tab:notation}列出了本节使用的符号。图~\ref{fig:transformer_diagram}展示了Transformer单层的完整计算流程。

\begin{table}[htbp]
\centering
\caption{Transformer计算分析符号表}
\label{tab:notation}
\begin{tabular}{cl}
\toprule
符号 & 含义 \\
\midrule
$B$ & Batch size \\
$T$ & 序列长度(Sequence length) \\
$D$ & 模型维度(Hidden dimension) \\
$F$ & FFN中间维度(通常 $F = 4D$ 或 $\frac{8}{3}D$) \\
$L$ & Transformer层数 \\
$N$ & Query head数量 \\
$K$ & KV head数量(MHA: $K=N$,GQA: $K<N$,MQA: $K=1$) \\
$H$ & 每个head的维度(通常 $H = D/N$) \\
$V$ & 词表大小(Vocabulary size) \\
$P$ & 模型总参数量 \\
\bottomrule
\end{tabular}
\end{table}

\begin{figure}[H]
\centering
\includegraphics[width=0.75\textwidth]{figures/transformer-diagram.png}
\caption{Transformer层的计算流程与张量维度。上:Attention模块;下:MLP模块(SwiGLU)。}
\label{fig:transformer_diagram}
\end{figure}

% ============================================
% 基础运算
% ============================================
\subsection{基础运算}

矩阵运算是Transformer的计算基础。设矩阵维度为 $m, n, k$:

\begin{table}[htbp]
\centering
\caption{基本矩阵运算的FLOPs与数据传输}
\label{tab:basic_ops}
\begin{tabular}{lccc}
\toprule
运算 & 表达式 & FLOPs & 数据量 \\
\midrule
向量点积 & $\bm{x} \cdot \bm{y}$, $\bm{x}, \bm{y} \in \R^k$ & $2k$ & $2k$ \\
矩阵-向量乘 & $A\bm{x}$, $A \in \R^{m \times k}$ & $2mk$ & $mk + k$ \\
矩阵-矩阵乘 & $AB$, $A \in \R^{m \times k}, B \in \R^{k \times n}$ & $2mkn$ & $mk + kn$ \\
\bottomrule
\end{tabular}
\end{table}

\paragraph{前向与反向传播}
对于线性层 $Y = XW$($X \in \R^{m \times k}$,$W \in \R^{k \times n}$):
\begin{itemize}
    \item 前向:$Y = XW$,FLOPs $= 2mkn$
    \item 反向:$\frac{\partial L}{\partial X} = \frac{\partial L}{\partial Y} W^\top$($2mkn$)+ $\frac{\partial L}{\partial W} = X^\top \frac{\partial L}{\partial Y}$($2mkn$)
    \item 总计:$6mkn$ FLOPs
\end{itemize}

这导出了训练FLOPs的核心公式:
\begin{equation}
    \boxed{\text{Training FLOPs} \approx 6 \times P \times T_{\text{tokens}}}
    \label{eq:6nd}
\end{equation}
其中 $P$ 是参数量,$T_{\text{tokens}}$ 是训练token总数。

% ============================================
% 单层分析
% ============================================
\subsection{单层分析}

Transformer每层包含两个核心模块:Attention和MLP。

\subsubsection{MLP层}

MLP层(也称FFN)有两种常见形式:

\paragraph{Standard FFN}
\begin{equation}
    \text{FFN}(x) = W_2 \cdot \text{GELU}(W_1 x), \quad W_1 \in \R^{D \times F}, W_2 \in \R^{F \times D}
\end{equation}

\paragraph{SwiGLU FFN}
\begin{equation}
    \text{SwiGLU}(x) = W_2 \cdot (\text{SiLU}(W_1 x) \odot W_3 x), \quad W_1, W_3 \in \R^{D \times F}, W_2 \in \R^{F \times D}
\end{equation}

其中 $\odot$ 是逐元素乘法,起\textbf{门控}作用:$W_3 x$ 控制 $\text{SiLU}(W_1 x)$ 的信息流通。

\begin{table}[htbp]
\centering
\caption{MLP层参数与FLOPs(per layer, 输入shape $[B, T, D]$)}
\label{tab:mlp_flops}
\begin{tabular}{lccc}
\toprule
类型 & 参数量 & 前向FLOPs & 训练FLOPs \\
\midrule
Standard FFN & $2DF$ & $4BTDF$ & $12BTDF$ \\
SwiGLU & $3DF$ & $6BTDF$ & $18BTDF$ \\
\bottomrule
\end{tabular}
\end{table}

\begin{remark}[参数量一致性]
为保持总参数量一致,不同结构调整 $F$ 的取值:
\begin{itemize}
    \item Standard FFN:$F = 4D$ $\Rightarrow$ 参数量 $= 8D^2$
    \item SwiGLU:$F = \frac{8}{3}D$ $\Rightarrow$ 参数量 $= 8D^2$
\end{itemize}
现代模型(LLaMA、DeepSeek等)普遍采用SwiGLU。
\end{remark}

\subsubsection{Attention层}

Multi-Head Attention包含四个投影和注意力计算:
\begin{align}
    Q = XW_Q, \quad K = XW_K, \quad V = XW_V, \quad O = \text{Attn}(Q, K, V) W_O
\end{align}
其中 $W_Q, W_O \in \R^{D \times D}$,$W_K, W_V \in \R^{D \times KH}$。

\begin{equation}
    \text{Attn}(Q, K, V) = \text{softmax}\left(\frac{QK^\top}{\sqrt{H}}\right)V
\end{equation}

\begin{table}[htbp]
\centering
\caption{Attention层参数与FLOPs(per layer, GQA with $K$ KV heads)}
\label{tab:attn_flops}
\begin{tabular}{lcc}
\toprule
组件 & 参数量 & 训练FLOPs \\
\midrule
Q projection & $D^2$ & $6BTD^2$ \\
K projection & $DKH$ & $6BTDKH$ \\
V projection & $DKH$ & $6BTDKH$ \\
O projection & $D^2$ & $6BTD^2$ \\
$QK^\top$ & — & $6BT^2NH$ \\
$\text{softmax} \cdot V$ & — & $6BT^2NH$ \\
\midrule
\textbf{Total} & $2D^2 + 2DKH$ & $12BTD^2 + 12BTDKH + 12BT^2NH$ \\
\bottomrule
\end{tabular}
\end{table}

\paragraph{MHA / GQA / MQA对比}
\begin{itemize}
    \item \textbf{MHA}(Multi-Head Attention):$K = N$,每个head独立KV
    \item \textbf{GQA}(Grouped-Query Attention):$1 < K < N$,多个Q head共享KV
    \item \textbf{MQA}(Multi-Query Attention):$K = 1$,所有head共享一个KV
\end{itemize}

\subsubsection{Attention vs MLP 计算量对比}

简化假设($F = 4D$,$K \ll N$,$NH = D$)下:
\begin{equation}
    \frac{\text{Attention FLOPs}}{\text{MLP FLOPs}} \approx \frac{T}{8D}
\end{equation}

\begin{remark}
当 $T < 8D$ 时,\textbf{MLP计算量主导}。对于 $D = 8192$ 的模型,序列长度需超过 $65536$ 才能使注意力成为主要计算瓶颈。这解释了为什么长上下文场景才需要特别关注注意力效率。
\end{remark}

% ============================================
% 完整模型分析
% ============================================
\subsection{完整模型分析}

\subsubsection{总参数量}

完整Transformer模型的参数组成:
\begin{flalign}
    & P_{\text{embed}} = VD && \text{(词嵌入)} & \\
    & P_{\text{attn}} = L \cdot (2D^2 + 2DKH) && \text{(注意力层)} & \\
    & P_{\text{mlp}} = L \cdot 3DF && \text{(MLP层,SwiGLU)} & \\
    & P_{\text{norm}} = L \cdot 4D && \text{(LayerNorm)} & \\
    & P_{\text{head}} = DV && \text{(输出投影)} &
\end{flalign}

总参数量(不共享embedding时):
\begin{equation}
    \boxed{P_{\text{total}} = 2VD + L \cdot (2D^2 + 2DKH + 3DF + 4D)}
\end{equation}

\begin{example}[LLaMA-7B参数计算]
$D = 4096$, $F = 11008$, $L = 32$, $V = 32000$, $N = K = 32$(MHA), $H = 128$:
\begin{align}
    P_{\text{embed}} &= 32000 \times 4096 = 131\text{M} \\
    P_{\text{attn/layer}} &= 2 \times 4096^2 + 2 \times 4096 \times 32 \times 128 = 67\text{M} \\
    P_{\text{mlp/layer}} &= 3 \times 4096 \times 11008 = 135\text{M} \\
    P_{\text{total}} &\approx 2 \times 131\text{M} + 32 \times (67\text{M} + 135\text{M}) \approx \mathbf{6.7B}
\end{align}
\end{example}

\subsubsection{训练计算量}

每token的训练FLOPs(忽略注意力中的 $T^2$ 项):
\begin{equation}
    \text{FLOPs/token} \approx 6P + 12LT \cdot NH
\end{equation}

第一项 $6P$ 是参数相关计算(占主导),第二项是注意力的序列长度相关计算。

\begin{table}[htbp]
\centering
\caption{常见模型的计算量}
\label{tab:model_flops}
\begin{tabular}{lccc}
\toprule
模型 & 参数量 & 前向FLOPs/token & 训练FLOPs/token \\
\midrule
GPT-2 & 1.5B & 3B & 9B \\
LLaMA-7B & 6.7B & 13B & 40B \\
LLaMA-70B & 70B & 140B & 420B \\
GPT-4 (est.) & 1.8T & 3.6T & 11T \\
\bottomrule
\end{tabular}
\end{table}

\subsubsection{训练内存占用}

训练时的内存组成:

\begin{table}[htbp]
\centering
\caption{训练内存组成($P$ 为参数量)}
\label{tab:memory}
\begin{tabular}{lcc}
\toprule
组件 & 内存 & 说明 \\
\midrule
参数(bf16) & $2P$ & 模型权重 \\
梯度(bf16) & $2P$ & 反向传播梯度 \\
优化器状态(Adam, fp32) & $8P$ & momentum + variance \\
激活值 & $O(BTD \cdot L)$ & 中间结果,用于反向传播 \\
\midrule
\textbf{总计(无优化)} & $\approx 12P + \text{activations}$ & \\
\bottomrule
\end{tabular}
\end{table}

\paragraph{Activation Checkpointing}
通过只保存每层输入、重新计算中间激活来节省内存:
\begin{itemize}
    \item 内存:从 $O(L \cdot BTD)$ 降至 $O(BTD)$
    \item 代价:额外约33\%重计算($6P \to 8P$)
\end{itemize}

% ============================================
% 推理分析
% ============================================
\subsection{推理分析}

推理与训练有本质区别:无需梯度和优化器状态,但自回归生成引入新的挑战。

\subsubsection{KV Cache}

自回归生成时,每生成一个token需要attend到所有历史token。为避免重复计算,需缓存历史的K和V:
\begin{equation}
    \boxed{\text{KV Cache Size} = 2 \times B \times S \times L \times K \times H \times \text{bytes}}
\end{equation}
其中 $S$ 是当前序列长度,$K$ 是KV head数,$H$ 是head维度。

\begin{example}[KV Cache计算]
70B模型:$D = 8192$, $L = 80$, $K = 8$(GQA), $H = 128$, $S = 8192$, bf16精度:
\begin{equation}
    \text{KV Cache} = 2 \times 1 \times 8192 \times 80 \times 8 \times 128 \times 2 = \mathbf{2.1\text{ GB/request}}
\end{equation}
\end{example}

\paragraph{KV Cache的影响}
\begin{itemize}
    \item 内存瓶颈:限制最大batch size和上下文长度
    \item GQA/MQA:通过减少 $K$ 降低cache大小(如从32降到8,减少4倍)
    \item 量化:int8/int4进一步压缩(2-4倍)
\end{itemize}

\subsubsection{Prefill vs Decode}

推理分为两个阶段,具有不同的计算特性:

\begin{table}[htbp]
\centering
\caption{Prefill vs Decode对比}
\label{tab:prefill_decode_overview}
\begin{tabular}{lcc}
\toprule
 & Prefill & Decode \\
\midrule
输入 & 整个prompt($T$ tokens) & 单个token \\
计算模式 & 并行处理所有token & 逐token生成 \\
瓶颈 & Compute-bound & Memory-bound \\
主要开销 & 矩阵乘法计算 & 权重加载 + KV Cache读写 \\
优化方向 & 提高计算利用率 & 增大batch、减少访存 \\
\bottomrule
\end{tabular}
\end{table}

\subsubsection{推理优化策略}

\paragraph{Batching策略}
\begin{itemize}
    \item \textbf{Continuous Batching}:动态插入/移除请求,提高GPU利用率
    \item \textbf{Chunked Prefill}:将长prompt分块处理,与decode交错执行
\end{itemize}

\paragraph{投机解码(Speculative Decoding)}
用小模型快速生成候选token序列,大模型并行验证:
\begin{itemize}
    \item 小模型(draft):快速生成 $k$ 个候选token
    \item 大模型(verify):一次前向验证所有候选,接受正确的前缀
    \item 加速比:$1.5\times \sim 3\times$(取决于draft模型准确率)
\end{itemize}

\paragraph{量化}
\begin{itemize}
    \item \textbf{Weight-only}(W4A16):权重int4,激活fp16,减少权重加载
    \item \textbf{Full quantization}(W8A8):权重和激活都int8,利用INT8 Tensor Core
\end{itemize}
