\section{Transformer中的门控机制}
\label{sec:gating}

在前几节中,我们分析了Transformer的计算复杂度和位置编码设计。本节关注另一个核心问题:如何让模型学会\textbf{选择性地传递信息}。标准的线性变换对所有输入一视同仁,而门控机制赋予网络"开关"能力——根据上下文动态决定哪些信息应该通过、哪些应该被抑制。

\subsection{为什么需要门控?}

考虑标准的全连接层 $y = Wx + b$:无论输入 $x$ 的内容如何,权重 $W$ 始终相同。这种\textbf{静态映射}存在根本局限——网络无法根据输入内容调整自身行为。

门控机制引入了\textbf{数据依赖的动态性}:
\begin{equation}
    y = g(x) \odot f(x)
\end{equation}
其中 $g(x) \in [0, 1]^d$ 是门控信号,$f(x)$ 是候选输出。关键洞察是:$g$ 本身依赖于输入 $x$,使得变换从静态的 $f$ 变为动态的 $g \odot f$。

\paragraph{信息瓶颈视角}
从信息论角度,门控实现了\textbf{自适应压缩}。设 $I(X; Y)$ 为输入输出的互信息:
\begin{itemize}
    \item 无门控:$Y = f(X)$,互信息由 $f$ 的容量决定
    \item 有门控:$Y = g(X) \odot f(X)$,当 $g \to 0$ 时可主动丢弃信息
\end{itemize}
这种"主动遗忘"能力对于过滤噪声、聚焦关键信息至关重要。在LSTM中,遗忘门正是通过这一机制解决了长程依赖问题~\citep{hochreiter1997long}。

\paragraph{稀疏激活视角}
门控还天然诱导稀疏性。当 $g_i \approx 0$ 时,第 $i$ 个维度被"关闭"。这与ReLU的稀疏激活类似,但更灵活:ReLU的稀疏模式由权重固定,而门控的稀疏模式随输入动态变化。实验表明,门控网络的激活稀疏度可达60-80\%,显著高于非门控网络~\citep{shazeer2020glu}。

\subsection{MLP层的门控:SwiGLU}

如第~\ref{sec:transformer_math}节所述,现代Transformer普遍采用SwiGLU替代标准FFN:

\paragraph{标准FFN}
\begin{equation}
    \text{FFN}(x) = W_2 \cdot \text{GELU}(W_1 x)
\end{equation}

\paragraph{SwiGLU FFN}
\begin{equation}
    \text{SwiGLU}(x) = W_2 \cdot \underbrace{(\text{SiLU}(W_1 x) \odot W_3 x)}_{\text{gating}}
\end{equation}

这里 $W_3 x$ 作为门控信号,控制 $\text{SiLU}(W_1 x)$ 的信息流通。为保持参数量一致,SwiGLU将中间维度从 $F = 4D$ 调整为 $F = \frac{8}{3}D$。

\subsection{Attention层的门控:Gated Attention}

\citet{qiu2025gatedattention}提出在Scaled Dot-Product Attention(SDPA)输出后添加一个sigmoid门控:

\paragraph{标准SDPA}
\begin{equation}
    Y = \text{softmax}\left(\frac{QK^\top}{\sqrt{H}}\right)V
\end{equation}

\paragraph{Gated Attention}
\begin{equation}
    Y' = Y \odot \sigma(XW_g)
    \label{eq:gated_attn}
\end{equation}
其中 $\sigma$ 是sigmoid函数,$W_g \in \R^{D \times D}$ 是可学习的门控投影矩阵。

\begin{table}[htbp]
\centering
\caption{门控位置与激活函数的消融实验}
\label{tab:gating_ablation}
\begin{tabular}{lcc}
\toprule
Configuration & Effectiveness & Notes \\
\midrule
Gate on Values & \checkmark & 有效但非最优 \\
Gate on Keys & \checkmark & 有效但非最优 \\
Gate on Dense output & \checkmark & 有效但非最优 \\
\textbf{Gate on SDPA output} & \checkmark\checkmark & \textbf{最优位置} \\
\midrule
Sigmoid activation & \checkmark\checkmark & \textbf{最优激活} \\
SiLU activation & \checkmark & 效果略差 \\
Additive gating & $\times$ & 显著差于乘法门控 \\
\bottomrule
\end{tabular}
\end{table}

\paragraph{为什么在SDPA输出后门控最优?}

表~\ref{tab:gating_ablation}的消融实验揭示了一个有趣现象:门控位置对效果影响显著。理解这一点需要分析attention的信息流:

\begin{enumerate}
    \item \textbf{Gate on Values}:$Y = \text{softmax}(QK^\top/\sqrt{H}) \cdot (V \odot g)$

    问题:门控作用于V后,softmax仍强制将权重分配到所有位置。即使某个value被门控为零,对应的注意力权重仍会被计算和分配。

    \item \textbf{Gate on SDPA output}:$Y' = [\text{softmax}(QK^\top/\sqrt{H}) \cdot V] \odot g$

    优势:门控作用于最终输出,可以\textbf{整体抑制}整个attention head的贡献。这打破了softmax的强制分配约束——即使softmax必须分配权重,门控可以让整个输出趋近于零。
\end{enumerate}

本质上,SDPA输出门控实现了\textbf{head级别的动态剪枝}:网络可以学会在某些上下文中"关闭"特定的attention head,这是values门控无法实现的。

\subsection{Attention Sink:一个深层的结构性问题}

Attention Sink是理解注意力机制局限性的关键窗口。MIT HAN Lab在StreamingLLM~\citep{xiao2024streamingllm}中首次系统研究了这一现象,随后OpenAI、小米、月之暗面等团队提出了各自的解决方案。

\subsubsection{现象与发现}

\paragraph{现象观察}
在长序列任务中,研究者发现一个反直觉的现象:大量注意力权重集中在序列开头的少数token(通常是第一个token),即使这些token语义上并不重要(如标点符号或填充符)。更重要的是,\textbf{越深的层,这种现象越明显}。

StreamingLLM的关键发现是:当使用滑动窗口注意力时,一旦初始token被移出窗口,模型输出会\textbf{完全崩溃}——产生无意义的乱码。但只需保留最初的4个token,性能就能大幅恢复。

\begin{figure}[htbp]
\centering
\begin{tikzpicture}[scale=0.75]
    % 浅层注意力矩阵
    \begin{scope}[shift={(-3.2,0)}]
        \node[above] at (1.6, 3.5) {\small 浅层 Attention};
        % 绘制5x5网格
        \foreach \i in {0,...,4} {
            \foreach \j in {0,...,4} {
                \pgfmathsetmacro{\dist}{abs(\i-\j)}
                \pgfmathsetmacro{\intensity}{max(0, 85 - \dist*18 + (\j==0 ? 8 : 0))}
                \fill[blue!\intensity!white] (\j*0.65, 2.6-\i*0.6) rectangle (\j*0.65+0.6, 2.6-\i*0.6-0.55);
                \draw[gray!40] (\j*0.65, 2.6-\i*0.6) rectangle (\j*0.65+0.6, 2.6-\i*0.6-0.55);
            }
        }
    \end{scope}

    % 深层注意力矩阵(Attention Sink明显)
    \begin{scope}[shift={(2.2,0)}]
        \node[above] at (1.6, 3.5) {\small 深层 Attention};
        \foreach \i in {0,...,4} {
            \foreach \j in {0,...,4} {
                \pgfmathsetmacro{\sinkintensity}{(\j==0 ? 92 : 0)}
                \pgfmathsetmacro{\dist}{abs(\i-\j)}
                \pgfmathsetmacro{\localintensity}{max(0, 45 - \dist*12)}
                \pgfmathsetmacro{\intensity}{max(\sinkintensity, \localintensity)}
                \fill[red!\intensity!white] (\j*0.65, 2.6-\i*0.6) rectangle (\j*0.65+0.6, 2.6-\i*0.6-0.55);
                \draw[gray!40] (\j*0.65, 2.6-\i*0.6) rectangle (\j*0.65+0.6, 2.6-\i*0.6-0.55);
            }
        }
        % Sink标注
        \draw[thick, red!70!black, ->] (0.3, -0.7) -- (0.3, -0.35);
        \node[below, font=\scriptsize, red!70!black] at (0.3, -0.7) {Sink};
    \end{scope}
\end{tikzpicture}
\caption{Attention Sink现象示意图。浅层注意力主要沿对角线分布(关注局部上下文),深层则在第一列出现显著的注意力聚集(Sink)——所有query都倾向于attend到第一个token,即使该token语义上并不重要。}
\label{fig:attention_sink}
\end{figure}

\paragraph{表面原因:Softmax的概率约束}

问题的直接原因在于softmax的归一化特性:
\begin{equation}
    \sum_{j=1}^{T} \text{softmax}(q_i^\top k_j / \sqrt{H}) = 1
\end{equation}

这一约束意味着每个query \textbf{必须}将全部注意力分配出去,即使理想情况下应该"不关注任何token"。网络的应对策略是学习一个"垃圾桶"位置来吸收多余的注意力。

\subsubsection{深层原因:Context-Aware Identity Layer假设}

仅用softmax约束无法解释为什么sink token的value接近零、为什么越深层sink越严重。一个更深刻的假设是:

\begin{quote}
\textit{Attention Sink源于Transformer对"上下文感知的恒等层"的内在需求——模型需要能够根据上下文决定某个Attention Block不输出任何变化。}
\end{quote}

\paragraph{证据一:Sink Token的Value接近零}
实证研究表明,第一个token的value向量范数显著小于其他token。这意味着即使被attend到,对输出的贡献也很小。模型\textbf{主动学习}将sink token的value置零。

\paragraph{证据二:Early Decoding与层深相关性}
如果用语言模型头探测中间层的输出,会发现模型存在"early decoding"现象——从某一层开始,预测结果就与最终层相同。这说明浅层已经完成了主要计算,深层需要做的是\textbf{保持恒等变换}。

而实现恒等变换的方式就是让Attention输出零。由于softmax约束,模型必须通过attend到一个value为零的特殊token来实现这一点。

\paragraph{证据三:Sink Token的Key具有独立子空间}
实验表明,当sink发生时,sink token的key与其他所有key几乎正交或负相关,且模长显著更小。这说明模型为$k_{\text{sink}}$分配了一个专用子空间:当query想要输出零时,它会指向这个子空间,精确attend到sink token。

\subsubsection{解决方案谱系}

理解了Attention Sink的本质后,各种解决方案可以归为两类:\textbf{显式提供sink}和\textbf{消除对sink的需求}。

\paragraph{方案一:保留初始Token(StreamingLLM)}
最简单的方法是在滑动窗口中永久保留前几个token:
\begin{equation}
    \text{Attention Range} = \{1, 2, \ldots, k_{\text{sink}}\} \cup \{t - w + 1, \ldots, t\}
\end{equation}
StreamingLLM保留4个初始token,使模型能稳定处理\textbf{400万+token}的流式输入。

\paragraph{方案二:可学习的Softmax Bias}
OpenAI的GPT-OSS系列~\citep{openai2025gptoss}和小米的MiMo-V2-Flash都采用了在softmax分母中引入可学习偏置的方案:
\begin{equation}
    \text{Attention}_{ij} = \frac{\exp(q_i^\top k_j / \sqrt{d})}{\sum_k \exp(q_i^\top k_k / \sqrt{d}) + \exp(b_h)}
\end{equation}
其中$b_h$是每个attention head的可学习标量。当$b_h$很大时,分母增大,所有注意力权重被"稀释",等效于模型可以选择"不关注任何token"。这种方法仅增加每个head一个参数,但从根本上解决了softmax的强制分配问题。

MiMo-V2-Flash进一步将此技术与激进的128-token滑动窗口结合,实现了KV Cache减少约6倍,同时通过sink bias保持长上下文稳定性。

\paragraph{方案三:Output Gating(Kimi Linear)}
月之暗面的Kimi Linear~\citep{moonshot2025kimilinear}在线性注意力模块中采用output gating来缓解sink:
\begin{equation}
    Y' = Y \odot \text{gate}(X)
\end{equation}
这使得attention可以输出零向量,无需依赖sink token。其Kimi Delta Attention (KDA)模块将此机制与细粒度遗忘门结合,在保持线性复杂度的同时超越了全注意力性能。

\paragraph{方案四:SDPA Output Gating(Qwen)}
如前所述,Qwen的Gated Attention在SDPA输出后添加sigmoid门控,同样消除了对sink token的需求。

\begin{table}[htbp]
\centering
\caption{Attention Sink解决方案对比}
\label{tab:attention_sink_solutions}
\begin{tabular}{lccc}
\toprule
方案 & 额外参数 & 消除Sink & 代表模型 \\
\midrule
保留初始Token & 0 & 否(绕过) & StreamingLLM \\
Softmax Bias & $n_h$ & 是 & GPT-OSS, MiMo-V2-Flash \\
Output Gating & $D^2$ & 是 & Kimi Linear, Qwen \\
\bottomrule
\end{tabular}
\end{table}

\subsubsection{理论统一}

这些方案可以从统一的视角理解:它们都在解决\textbf{Attention如何输出零}的问题。

\begin{itemize}
    \item \textbf{StreamingLLM}:让模型继续使用学到的sink机制
    \item \textbf{Softmax Bias}:修改softmax使其可以输出接近零的分布
    \item \textbf{Output Gating}:在softmax之后乘以可学习门控,强制输出为零
\end{itemize}

值得注意的是,output gating不仅消除了sink,还\textbf{释放了被sink占用的维度}。模型不再需要为$k_{\text{sink}}$分配专用子空间,这些容量可以用于更有意义的表示学习。这可能是Gated Attention带来性能提升的深层原因之一。

\begin{remark}[非归一化注意力]
理论上,如果注意力分数不需要归一化(或可以为负),模型可以直接组合出零输出,不需要sink。实验表明非归一化注意力确实消除了sink现象,但这会带来训练稳定性问题。Softmax Bias和Output Gating是在保持训练稳定性的前提下解决sink的折中方案。
\end{remark}

\subsection{训练稳定性与长上下文}

Gated Attention还带来额外的训练优势:

\paragraph{训练稳定性}
\begin{itemize}
    \item 损失曲线更平滑,减少loss spike
    \item 可以使用更大的学习率($4.0 \times 10^{-3}$ 到 $4.5 \times 10^{-3}$)
\end{itemize}

\paragraph{长上下文外推}
结合YaRN位置编码插值:
\begin{itemize}
    \item 训练长度:32k tokens
    \item 外推测试:128k tokens
    \item 性能衰减显著小于基线模型
\end{itemize}

\subsection{工业应用}

Gated Attention已被集成到Qwen3-Next架构中,并部署于Qwen3-Next-80B-A3B-Instruct模型,验证了其在大规模工业应用中的有效性。该工作入选NeurIPS 2025 Oral(前1.5\%)。

\begin{remark}[MLP门控 vs Attention门控]
两种门控的作用互补:
\begin{itemize}
    \item \textbf{MLP门控(SwiGLU)}:在特征变换阶段选择性激活神经元
    \item \textbf{Attention门控}:在信息聚合阶段选择性传递attention输出
\end{itemize}
现代模型(如Qwen3-Next)同时采用两者。
\end{remark}
