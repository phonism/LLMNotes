\section{分词器(Tokenizer)}
\label{sec:tokenizer}

分词器是大语言模型的入口,负责将原始文本转换为模型可处理的token序列。分词策略的选择直接影响模型的词表大小、训练效率和多语言能力。

\subsection{为什么需要分词}

神经网络无法直接处理文本,需要将文本转换为数值表示:
\begin{equation}
    \text{``Hello world''} \xrightarrow{\text{Tokenizer}} [15496, 995] \xrightarrow{\text{Embedding}} \R^{2 \times d}
\end{equation}

分词粒度的选择面临\textbf{词表大小}与\textbf{序列长度}的权衡:

\begin{table}[htbp]
\centering
\caption{不同分词粒度的对比}
\label{tab:tokenization_granularity}
\begin{tabular}{lccc}
\toprule
粒度 & 词表大小 & 序列长度 & 问题 \\
\midrule
字符级 & $\sim$256 & 很长 & 序列过长,难以建模长距离依赖 \\
词级 & $\sim$100K+ & 短 & OOV问题,词表过大 \\
子词级 & $\sim$32K-128K & 适中 & 平衡,主流选择 \\
\bottomrule
\end{tabular}
\end{table}

\subsection{子词分词算法}

\subsubsection{Byte Pair Encoding (BPE)}

BPE~\citep{sennrich2016bpe}是最广泛使用的子词分词算法,源自数据压缩领域。

\paragraph{训练算法}
\begin{enumerate}
    \item 初始化词表为所有字符(或字节)
    \item 统计相邻token对的频率
    \item 将最高频的token对合并为新token,加入词表
    \item 重复步骤2-3,直到达到目标词表大小
\end{enumerate}

\begin{example}[BPE训练过程]
假设语料为:``low lower lowest''
\begin{enumerate}
    \item 初始:\texttt{l, o, w, e, r, s, t, \_}(\_表示词边界)
    \item 最高频对 \texttt{(l, o)} $\to$ 合并为 \texttt{lo}
    \item 最高频对 \texttt{(lo, w)} $\to$ 合并为 \texttt{low}
    \item 最高频对 \texttt{(low, e)} $\to$ 合并为 \texttt{lowe}
    \item ...
\end{enumerate}
最终词表包含常见子词如 \texttt{low}, \texttt{er}, \texttt{est} 等。
\end{example}

\paragraph{分词算法}
给定训练好的合并规则,按优先级依次应用:
\begin{lstlisting}
def bpe_tokenize(text, merges):
    tokens = list(text)  # 初始为字符
    for (a, b) in merges:  # 按训练顺序
        i = 0
        while i < len(tokens) - 1:
            if tokens[i] == a and tokens[i+1] == b:
                tokens = tokens[:i] + [a+b] + tokens[i+2:]
            else:
                i += 1
    return tokens
\end{lstlisting}

\subsubsection{Byte-level BPE}

GPT-2~\citep{radford2019language}引入的改进,直接在字节级别操作:
\begin{itemize}
    \item 基础词表为256个字节,无需预分词
    \item 可以表示任何UTF-8文本,无OOV问题
    \item 避免了不同语言的特殊处理
\end{itemize}

\begin{lstlisting}
# GPT-2的字节到Unicode映射
def bytes_to_unicode():
    # 将256个字节映射到可打印Unicode字符
    # 避免空白字符等控制字符的问题
    bs = list(range(33, 127)) + list(range(161, 173)) + list(range(174, 256))
    cs = bs[:]
    n = 0
    for b in range(256):
        if b not in bs:
            bs.append(b)
            cs.append(256 + n)
            n += 1
    return dict(zip(bs, [chr(c) for c in cs]))
\end{lstlisting}

\subsubsection{WordPiece}

WordPiece由Google提出,用于BERT~\citep{devlin2019bert}。与BPE的主要区别在于合并策略:

\begin{itemize}
    \item \textbf{BPE}:选择频率最高的token对
    \item \textbf{WordPiece}:选择使语言模型似然提升最大的token对
\end{itemize}

WordPiece的合并分数:
\begin{equation}
    \text{score}(a, b) = \frac{\text{freq}(ab)}{\text{freq}(a) \times \text{freq}(b)}
\end{equation}

WordPiece使用 \texttt{\#\#} 标记非词首子词:
\begin{verbatim}
"tokenization" -> ["token", "##ization"]
\end{verbatim}

\subsubsection{Unigram Language Model}

Unigram~\citep{kudo2018sentencepiece}采用相反的策略——从大词表开始,逐步删减:

\begin{enumerate}
    \item 初始化一个较大的候选词表
    \item 用EM算法估计每个token的概率
    \item 计算移除每个token对似然的影响
    \item 移除影响最小的token
    \item 重复直到达到目标词表大小
\end{enumerate}

Unigram的优势:
\begin{itemize}
    \item 同一文本可能有多种分词方式,支持采样
    \item 可用于数据增强(Subword Regularization)
\end{itemize}

\subsection{SentencePiece}

SentencePiece~\citep{kudo2018sentencepiece}是Google开源的分词工具,特点:

\begin{itemize}
    \item \textbf{语言无关}:将空格视为普通字符(用\_表示),无需预分词
    \item \textbf{支持多种算法}:BPE和Unigram
    \item \textbf{可逆}:分词结果可以无损还原原文
    \item \textbf{高效}:C++实现,训练和推理都很快
\end{itemize}

\begin{lstlisting}
import sentencepiece as spm

# 训练
spm.SentencePieceTrainer.train(
    input='corpus.txt',
    model_prefix='tokenizer',
    vocab_size=32000,
    model_type='bpe',  # 或 'unigram'
    character_coverage=0.9995,
)

# 使用
sp = spm.SentencePieceProcessor(model_file='tokenizer.model')
tokens = sp.encode('Hello world', out_type=str)
# ['▁Hello', '▁world']  (▁表示空格)
\end{lstlisting}

\subsection{Tiktoken}

Tiktoken是OpenAI开源的快速BPE实现,用于GPT-3.5/4:

\begin{itemize}
    \item \textbf{Rust实现}:比Python实现快3-6倍
    \item \textbf{正则预分词}:用正则表达式预先切分,提高效率
    \item \textbf{特殊token处理}:支持\texttt{<|endoftext|>}等特殊标记
\end{itemize}

\begin{lstlisting}
import tiktoken

# GPT-4使用cl100k_base编码
enc = tiktoken.get_encoding("cl100k_base")
tokens = enc.encode("Hello world")  # [9906, 1917]
text = enc.decode(tokens)  # "Hello world"

# 查看词表大小
print(enc.n_vocab)  # 100277
\end{lstlisting}

\paragraph{预分词正则}
Tiktoken使用复杂的正则表达式预切分文本:
\begin{lstlisting}
# cl100k_base的预分词正则(简化)
pat = r"""'s|'t|'re|'ve|'m|'ll|'d|
    [\p{L}]+|[\p{N}]{1,3}|
    [^\s\p{L}\p{N}]+|\s+(?!\S)|\s+"""
\end{lstlisting}

这确保了常见模式(如缩写、数字)被合理切分。

\subsection{词表大小的选择}

词表大小是重要的设计决策:

\begin{table}[htbp]
\centering
\caption{主流模型的词表配置}
\label{tab:vocab_sizes}
\begin{tabular}{lcc}
\toprule
模型 & 词表大小 & 分词器 \\
\midrule
GPT-2 & 50,257 & Byte-level BPE \\
GPT-3/3.5 & 50,257 & Byte-level BPE \\
GPT-4 & 100,277 & Byte-level BPE (cl100k) \\
BERT & 30,522 & WordPiece \\
LLaMA & 32,000 & SentencePiece BPE \\
LLaMA 2 & 32,000 & SentencePiece BPE \\
LLaMA 3 & 128,256 & Tiktoken BPE \\
Qwen & 151,936 & Byte-level BPE \\
DeepSeek & 102,400 & Byte-level BPE \\
\bottomrule
\end{tabular}
\end{table}

\paragraph{词表大小的影响}
\begin{itemize}
    \item \textbf{更大词表}:
    \begin{itemize}
        \item 更短的序列长度(同样文本用更少token)
        \item 更大的Embedding层参数($V \times d$)
        \item 更好的多语言支持
    \end{itemize}
    \item \textbf{更小词表}:
    \begin{itemize}
        \item 更长的序列(需要更多token表示)
        \item 更小的模型参数
        \item 稀有词可能被过度切分
    \end{itemize}
\end{itemize}

\paragraph{经验法则}
\begin{itemize}
    \item 英文为主:32K-50K通常足够
    \item 多语言:100K+可以更好覆盖非英语语言
    \item LLaMA 3将词表从32K扩展到128K,中文token效率提升约3倍
\end{itemize}

\subsection{特殊Token}

特殊token用于标记序列结构和控制生成:

\begin{table}[htbp]
\centering
\caption{常见特殊Token}
\label{tab:special_tokens}
\begin{tabular}{lll}
\toprule
Token & 含义 & 用途 \\
\midrule
\texttt{<|endoftext|>} & 文本结束 & 分隔文档,标记生成结束 \\
\texttt{<|im\_start|>} & 消息开始 & ChatML格式的消息边界 \\
\texttt{<|im\_end|>} & 消息结束 & ChatML格式的消息边界 \\
\texttt{[PAD]} & 填充 & 批处理时对齐序列长度 \\
\texttt{[CLS]} & 分类 & BERT的序列表示 \\
\texttt{[SEP]} & 分隔 & BERT的句子分隔 \\
\texttt{[MASK]} & 掩码 & BERT的MLM任务 \\
\bottomrule
\end{tabular}
\end{table}

\paragraph{ChatML格式}
现代Chat模型使用结构化格式区分角色:
\begin{verbatim}
<|im_start|>system
You are a helpful assistant.
<|im_end|>
<|im_start|>user
Hello!
<|im_end|>
<|im_start|>assistant
Hi there!
<|im_end|>
\end{verbatim}

\subsection{多语言分词:一个被低估的公平性问题}

分词器的设计深刻影响着不同语言用户的使用体验。这不仅是技术问题,更是公平性问题。

\subsubsection{Token效率差异的根源}

不同语言的token效率差异显著:

\begin{table}[htbp]
\centering
\caption{不同语言的Token效率(GPT-4,相同语义内容)}
\label{tab:token_efficiency}
\begin{tabular}{lcc}
\toprule
语言 & Token数 & 相对英语 \\
\midrule
英语 & 100 & 1.0$\times$ \\
西班牙语 & 120 & 1.2$\times$ \\
中文 & 150 & 1.5$\times$ \\
日语 & 180 & 1.8$\times$ \\
缅甸语 & 400 & 4.0$\times$ \\
\bottomrule
\end{tabular}
\end{table}

\paragraph{语言学解释}
效率差异源于\textbf{书写系统}与\textbf{训练数据}的双重因素:

\begin{enumerate}
    \item \textbf{字母文字 vs 表意文字}:英语26个字母组合成词,BPE容易学到常见子词(如 \texttt{-tion}, \texttt{-ing})。中文每个字是独立语素,约需3500个常用字覆盖99.9\%文本,但BPE难以学到跨字的有意义组合。

    \item \textbf{训练数据偏斜}:BPE按频率合并token对。当训练语料中英文占90\%时,英文子词被充分合并(\texttt{the}, \texttt{and}成为单token),而中文词汇因频率低而保持拆分状态。

    \item \textbf{UTF-8编码开销}:英文字符占1字节,中文字符占3字节。在Byte-level BPE中,一个中文字至少需要3个基础token才能表示,除非被合并。
\end{enumerate}

\paragraph{定量分析}
设语言$L$在训练语料中的占比为$p_L$,其token效率大致满足:
\begin{equation}
    \text{效率}_L \propto p_L^{\alpha}, \quad \alpha \approx 0.3\text{--}0.5
\end{equation}
这解释了为什么缅甸语(训练数据中几乎没有)效率是英语的1/4——缅甸文字几乎没有被合并,每个字符都被拆成多个字节token。

\subsubsection{实际影响}

对非英语用户而言:
\begin{itemize}
    \item \textbf{成本}:相同语义内容消耗1.5-4倍token,API费用相应增加
    \item \textbf{上下文}:有效上下文窗口缩短(128K tokens对中文用户相当于英文用户的85K)
    \item \textbf{延迟}:生成相同内容需要更多解码步骤
    \item \textbf{质量}:过度切分破坏词汇边界,可能影响语义理解
\end{itemize}

\subsubsection{改进策略}

\paragraph{扩大词表}
LLaMA 3将词表从32K扩展到128K,新增大量非英语token。效果:中文token效率提升约3倍。代价:Embedding层参数增加4倍(从32K$\times d$到128K$\times d$)。

\paragraph{平衡训练语料}
在分词器训练时对语言进行上采样,使低资源语言获得更多合并机会。但这会牺牲高资源语言的效率——存在帕累托权衡。

\paragraph{语言感知分词}
对特定语言使用专门的预分词规则。例如,中文按字切分后再BPE,避免跨字合并产生无意义token。日语可以利用形态分析工具(如MeCab)进行预切分。

\begin{remark}[公平性视角]
分词效率差异本质上是训练资源分配的结果。当商业模型按token计费时,这种差异直接转化为经济不平等——使用小语种的用户为相同服务付出更高代价。这是AI公平性讨论中容易被忽视的技术细节。
\end{remark}

\subsection{分词器的实现细节}

\subsubsection{Tokenizer的组成}

一个完整的分词器包含:
\begin{enumerate}
    \item \textbf{Normalizer}:文本标准化(Unicode归一化、大小写等)
    \item \textbf{Pre-tokenizer}:预分词(按空格、标点切分)
    \item \textbf{Model}:核心分词算法(BPE/WordPiece/Unigram)
    \item \textbf{Post-processor}:后处理(添加特殊token)
    \item \textbf{Decoder}:将token ID还原为文本
\end{enumerate}

\subsubsection{HuggingFace Tokenizers}

HuggingFace的\texttt{tokenizers}库提供高效实现:
\begin{lstlisting}
from tokenizers import Tokenizer, models, trainers, pre_tokenizers

# 创建BPE分词器
tokenizer = Tokenizer(models.BPE())
tokenizer.pre_tokenizer = pre_tokenizers.ByteLevel()

# 训练
trainer = trainers.BpeTrainer(vocab_size=32000, special_tokens=["<|endoftext|>"])
tokenizer.train(files=["corpus.txt"], trainer=trainer)

# 使用
output = tokenizer.encode("Hello world")
print(output.tokens)  # ['Hello', 'Ġworld']
\end{lstlisting}

\subsection{分词的计算开销}

分词通常不是瓶颈,但在某些场景需要注意:

\begin{itemize}
    \item \textbf{在线推理}:分词延迟约0.1-1ms,通常可忽略
    \item \textbf{大规模预处理}:TB级语料分词可能需要数小时
    \item \textbf{优化方法}:
    \begin{itemize}
        \item 使用Rust/C++实现(如Tiktoken)
        \item 并行处理
        \item 预分词缓存
    \end{itemize}
\end{itemize}

\begin{remark}[分词器选择建议]
\begin{itemize}
    \item \textbf{从头训练LLM}:使用SentencePiece或HuggingFace Tokenizers训练专用分词器
    \item \textbf{微调现有模型}:使用原模型的分词器,保持一致性
    \item \textbf{多语言场景}:选择词表较大(100K+)的分词器
    \item \textbf{特殊领域}(代码、数学):考虑在分词器训练时包含领域数据
\end{itemize}
\end{remark}
