\section{大模型数据工程}
\label{sec:data}

数据是大语言模型的基石。本节系统介绍预训练数据、后训练数据和评估数据的构建方法,以及数据配比的经验法则。

\subsection{预训练数据}

\subsubsection{数据来源}

现代LLM的预训练数据通常来自以下来源:

\begin{table}[htbp]
\centering
\caption{主要预训练数据来源}
\label{tab:pretrain_data_sources}
\begin{tabular}{llcc}
\toprule
数据源 & 描述 & 规模 & 质量 \\
\midrule
\textbf{网页数据} & & & \\
Common Crawl & 最大的网页爬虫数据 & PB级 & 低 \\
RefinedWeb & 过滤后的高质量网页 & 5T tokens & 中 \\
C4 & Colossal Clean Crawled Corpus & 800B tokens & 中 \\
\midrule
\textbf{高质量文本} & & & \\
Wikipedia & 百科全书 & 数十B tokens & 高 \\
Books & 书籍(Books3, Pile-Books) & 数十B tokens & 高 \\
arXiv & 学术论文 & 数十B tokens & 高 \\
\midrule
\textbf{代码} & & & \\
GitHub & 开源代码仓库 & 数百B tokens & 中-高 \\
The Stack & 去重的开源代码 & 3T tokens & 高 \\
StarCoder Data & 精选编程数据 & 1T tokens & 高 \\
\midrule
\textbf{对话/社交} & & & \\
Reddit & 社交媒体讨论 & 数百B tokens & 中 \\
StackExchange & 问答社区 & 数十B tokens & 高 \\
\bottomrule
\end{tabular}
\end{table}

\subsubsection{数据处理流程}

从原始数据到训练数据需要经过多个处理阶段:

\paragraph{1. 文本提取}
\begin{itemize}
    \item HTML解析:使用trafilatura、jusText等工具提取正文
    \item PDF解析:使用PyMuPDF、pdfplumber处理学术文档
    \item 代码处理:保留注释和文档字符串
\end{itemize}

\paragraph{2. 质量过滤}
\begin{itemize}
    \item \textbf{启发式规则}:
    \begin{itemize}
        \item 文档长度过滤(过短/过长)
        \item 特殊字符比例
        \item 重复行/段落比例
        \item 语言检测(fastText语言分类器)
    \end{itemize}
    \item \textbf{模型打分}:
    \begin{itemize}
        \item 困惑度过滤(用小模型计算PPL)
        \item 质量分类器(如GPT-3用的质量分类器)
        \item 教育价值评分(FineWeb-Edu使用的方法)
    \end{itemize}
\end{itemize}

\paragraph{3. 去重}
去重是预训练数据处理的关键步骤,重复数据会导致:
\begin{itemize}
    \item 训练效率下降
    \item 模型记忆而非泛化
    \item 隐私泄露风险
\end{itemize}

主要去重方法:
\begin{itemize}
    \item \textbf{精确去重}:基于哈希的完全匹配
    \item \textbf{模糊去重}:MinHash + LSH(Locality-Sensitive Hashing)
    \item \textbf{文档级 vs 段落级}:不同粒度的去重策略
\end{itemize}

\begin{lstlisting}
# MinHash去重示例
from datasketch import MinHash, MinHashLSH

def get_minhash(text, num_perm=128):
    m = MinHash(num_perm=num_perm)
    for word in text.split():
        m.update(word.encode('utf8'))
    return m

# 创建LSH索引
lsh = MinHashLSH(threshold=0.8, num_perm=128)
for doc_id, text in documents:
    minhash = get_minhash(text)
    lsh.insert(doc_id, minhash)
\end{lstlisting}

\paragraph{4. 敏感内容过滤}
\begin{itemize}
    \item PII(个人身份信息)移除:邮箱、电话、身份证号等
    \item 有害内容过滤:使用分类器检测
    \item 版权内容处理:根据许可证过滤
\end{itemize}

\subsubsection{数据配比}

数据配比(Data Mix)对模型能力有重要影响:

\begin{table}[htbp]
\centering
\caption{主流模型的数据配比(估计值)}
\label{tab:data_mix}
\begin{tabular}{lccccc}
\toprule
模型 & 网页 & 代码 & 书籍 & 学术 & 其他 \\
\midrule
GPT-3 & 60\% & 3\% & 16\% & - & 21\% \\
LLaMA & 67\% & 4.5\% & 4.5\% & 2.5\% & 21.5\% \\
LLaMA 2 & 89\% & - & - & - & 11\% \\
DeepSeek & 56\% & 18\% & - & 5\% & 21\% \\
Qwen & 类似LLaMA & 较高代码占比 & - & - & - \\
\bottomrule
\end{tabular}
\end{table}

\paragraph{配比原则}
\begin{itemize}
    \item \textbf{代码数据}:提升推理能力,通常占5-20\%
    \item \textbf{数学数据}:提升数学能力,但占比过高可能损害通用能力
    \item \textbf{多语言数据}:根据目标市场调整,中文模型通常30-50\%中文
    \item \textbf{高质量数据}:虽然量少,但应该多次采样(upsampling)
\end{itemize}

\subsubsection{数据规模与Scaling Law}

Chinchilla Scaling Law~\citep{hoffmann2022chinchilla}指出,计算最优的数据量与模型参数成正比:
\begin{equation}
    D_{opt} \approx 20 \times N
\end{equation}
其中$D$是token数,$N$是参数量。即7B模型需要约140B tokens。

然而,实践中常常使用更多数据(over-training):
\begin{itemize}
    \item LLaMA-7B:1T tokens(约143倍参数量)
    \item LLaMA 2-7B:2T tokens(约286倍)
    \item LLaMA 3-8B:15T tokens(约1875倍)
\end{itemize}

\paragraph{为什么Over-training?}
\begin{itemize}
    \item 推理成本固定,训练成本可摊销
    \item 更小的模型更容易部署
    \item 数据重复使用在一定程度内是有益的
\end{itemize}

\subsubsection{多轮训练与课程学习}

\paragraph{数据课程(Data Curriculum)}
按照特定顺序呈现数据可以提升模型性能:
\begin{enumerate}
    \item \textbf{阶段1}:通用网页数据,建立基础语言能力
    \item \textbf{阶段2}:增加高质量数据比例(书籍、维基百科)
    \item \textbf{阶段3}:增加代码和数学数据
    \item \textbf{阶段4}:Annealing阶段,使用最高质量数据,降低学习率
\end{enumerate}

\paragraph{退火阶段(Annealing)}
训练末期使用高质量数据的策略:
\begin{itemize}
    \item 学习率从正常值退火到接近0
    \item 数据切换为最高质量子集
    \item 通常占总训练的1-5\%
    \item LLaMA 3报告此阶段显著提升benchmark性能
\end{itemize}

\subsection{后训练数据}

后训练(Post-training)包括监督微调(SFT)和人类偏好对齐(RLHF/DPO),数据需求与预训练截然不同。

\subsubsection{SFT数据}

\paragraph{数据格式}
SFT数据通常是指令-响应对:
\begin{verbatim}
{
  "instruction": "将以下句子翻译成英文",
  "input": "今天天气很好",
  "output": "The weather is nice today."
}
\end{verbatim}

或多轮对话格式:
\begin{verbatim}
{
  "conversations": [
    {"role": "user", "content": "什么是机器学习?"},
    {"role": "assistant", "content": "机器学习是..."},
    {"role": "user", "content": "它有哪些应用?"},
    {"role": "assistant", "content": "机器学习的应用包括..."}
  ]
}
\end{verbatim}

\paragraph{数据来源}
\begin{itemize}
    \item \textbf{人工标注}:质量最高,成本最高
    \item \textbf{众包平台}:如Scale AI、Surge AI
    \item \textbf{开源数据集}:
    \begin{itemize}
        \item FLAN Collection:多任务指令数据
        \item OpenAssistant:社区标注对话
        \item ShareGPT:用户分享的ChatGPT对话
        \item Alpaca:GPT-4生成的指令数据
    \end{itemize}
    \item \textbf{合成数据}:用强模型生成(Self-Instruct、Evol-Instruct)
\end{itemize}

\paragraph{数据规模}
SFT数据量远小于预训练:
\begin{itemize}
    \item 早期:几千到几万条(Alpaca 52K)
    \item 现代:几十万到几百万条
    \item 质量比数量更重要(LIMA论文:1000条精选数据即可)
\end{itemize}

\paragraph{数据质量控制}
\begin{itemize}
    \item 响应长度:避免过短/过长的回复
    \item 格式一致性:统一的回复风格
    \item 多样性:覆盖不同任务类型
    \item 准确性:人工审核或模型验证
\end{itemize}

\subsubsection{偏好数据}

RLHF和DPO需要偏好对数据:

\paragraph{数据格式}
\begin{verbatim}
{
  "prompt": "解释量子计算",
  "chosen": "量子计算是一种利用量子力学原理...",
  "rejected": "量子计算就是很快的计算机..."
}
\end{verbatim}

\paragraph{收集方法}
\begin{enumerate}
    \item \textbf{人工标注}:
    \begin{itemize}
        \item 给定prompt,生成多个响应
        \item 人类标注者选择更好的响应
        \item 成本高,但质量可控
    \end{itemize}
    \item \textbf{AI反馈}:
    \begin{itemize}
        \item 用GPT-4等强模型做评判
        \item Constitutional AI:用规则约束AI反馈
        \item RLAIF(RL from AI Feedback)
    \end{itemize}
    \item \textbf{隐式反馈}:
    \begin{itemize}
        \item 用户点赞/点踩
        \item 用户选择哪个回复继续对话
        \item 用户编辑/重新生成
    \end{itemize}
\end{enumerate}

\paragraph{数据规模}
\begin{itemize}
    \item InstructGPT:约33K比较对
    \item LLaMA 2 Chat:约100万偏好数据
    \item 现代实践:几十万到几百万对
\end{itemize}

\subsubsection{合成数据}

合成数据在后训练中越来越重要:

\paragraph{Self-Instruct}
用模型自己生成指令和响应:
\begin{enumerate}
    \item 从种子任务开始
    \item 让模型生成新的指令
    \item 过滤低质量/重复的指令
    \item 让模型生成响应
\end{enumerate}

\paragraph{Evol-Instruct}
WizardLM提出的指令进化方法:
\begin{itemize}
    \item \textbf{深度进化}:增加复杂度、约束、推理步骤
    \item \textbf{广度进化}:变换话题、领域、风格
\end{itemize}

\begin{lstlisting}
# 深度进化prompt示例
"""
请将以下指令改写得更复杂,加入更多约束条件:
原始指令:写一首关于春天的诗
进化指令:写一首关于春天的七言绝句,要求包含至少三种春天的景物,
         使用对仗工整的句式,并在最后一句表达对时光流逝的感慨。
"""
\end{lstlisting}

\paragraph{蒸馏数据}
从强模型蒸馏知识到弱模型:
\begin{itemize}
    \item 用GPT-4/Claude生成高质量响应
    \item 收集推理过程(Chain-of-Thought)
    \item DeepSeek-R1蒸馏:用R1的推理轨迹训练小模型
\end{itemize}

\subsection{评估数据}

\subsubsection{通用能力评估}

\begin{table}[htbp]
\centering
\caption{主要评估基准}
\label{tab:eval_benchmarks}
\begin{tabular}{llll}
\toprule
基准 & 类型 & 规模 & 评估内容 \\
\midrule
\textbf{知识与推理} & & & \\
MMLU & 多选题 & 57科目 & 世界知识 \\
HellaSwag & 选择题 & 10K & 常识推理 \\
ARC & 选择题 & 7.8K & 科学推理 \\
WinoGrande & 选择题 & 44K & 常识推理 \\
\midrule
\textbf{数学} & & & \\
GSM8K & 应用题 & 8.5K & 小学数学 \\
MATH & 竞赛题 & 12.5K & 高中竞赛数学 \\
\midrule
\textbf{代码} & & & \\
HumanEval & 代码生成 & 164 & Python编程 \\
MBPP & 代码生成 & 974 & Python编程 \\
\midrule
\textbf{长文本} & & & \\
RULER & 多任务 & - & 长上下文理解 \\
LongBench & 多任务 & 4.75K & 长文本能力 \\
\bottomrule
\end{tabular}
\end{table}

\subsubsection{对话能力评估}

\begin{itemize}
    \item \textbf{MT-Bench}:多轮对话评估,GPT-4作为评判
    \item \textbf{AlpacaEval}:单轮指令跟随,与GPT-4比较
    \item \textbf{Arena Hard}:基于Chatbot Arena的困难子集
    \item \textbf{Chatbot Arena}:用户盲评,ELO排名
\end{itemize}

\subsubsection{安全性评估}

\begin{itemize}
    \item \textbf{TruthfulQA}:测试模型是否会生成虚假信息
    \item \textbf{ToxiGen}:测试有害内容生成
    \item \textbf{BBQ}:测试社会偏见
    \item \textbf{Red-teaming}:对抗性攻击测试
\end{itemize}

\subsection{数据工程实践}

\subsubsection{数据飞轮}

现代LLM公司通常建立数据飞轮:
\begin{enumerate}
    \item 部署模型服务用户
    \item 收集用户交互数据(带隐私保护)
    \item 标注高质量样本
    \item 训练更好的模型
    \item 回到步骤1
\end{enumerate}

\subsubsection{数据质量 vs 数量}

\begin{table}[htbp]
\centering
\caption{数据质量与数量的权衡}
\label{tab:quality_vs_quantity}
\begin{tabular}{lcc}
\toprule
阶段 & 优先级 & 理由 \\
\midrule
预训练早期 & 数量 & 需要大量数据建立基础能力 \\
预训练后期 & 质量 & Annealing阶段,精选数据提升性能 \\
SFT & 质量 $>$ 数量 & 几千条高质量数据可能胜过百万低质量 \\
RLHF & 质量 & 偏好标注错误会导致reward hacking \\
\bottomrule
\end{tabular}
\end{table}

\paragraph{LIMA的启示}
LIMA论文~\citep{zhou2023lima}表明:
\begin{itemize}
    \item 1000条精心策划的SFT数据可以产生强大的对话模型
    \item 数据多样性比数量更重要
    \item 响应风格的一致性很关键
\end{itemize}

\subsubsection{数据污染}

评估数据泄露到训练数据中是严重问题:

\paragraph{检测方法}
\begin{itemize}
    \item N-gram重叠检测
    \item 困惑度异常检测
    \item 模型记忆测试(completion)
\end{itemize}

\paragraph{预防措施}
\begin{itemize}
    \item 训练数据与评估数据去重
    \item 使用时间切分(训练数据早于评估数据)
    \item 定期更新评估基准
\end{itemize}

\subsection{开源数据集}

\begin{table}[htbp]
\centering
\caption{重要开源数据集}
\label{tab:open_datasets}
\begin{tabular}{llll}
\toprule
数据集 & 类型 & 规模 & 特点 \\
\midrule
\textbf{预训练} & & & \\
RedPajama & 通用 & 1.2T tokens & LLaMA数据复现 \\
The Pile & 通用 & 825GB & 多源混合 \\
FineWeb & 网页 & 15T tokens & 高质量过滤 \\
FineWeb-Edu & 网页 & 1.3T tokens & 教育内容 \\
StarCoder Data & 代码 & 1T tokens & 许可证友好 \\
\midrule
\textbf{SFT} & & & \\
FLAN Collection & 指令 & 数百万 & 多任务 \\
OpenAssistant & 对话 & 161K & 多语言 \\
UltraChat & 对话 & 1.5M & 合成对话 \\
\midrule
\textbf{偏好} & & & \\
HH-RLHF & 偏好对 & 170K & Anthropic发布 \\
UltraFeedback & 偏好对 & 64K & GPT-4标注 \\
\bottomrule
\end{tabular}
\end{table}

\begin{remark}[数据工程建议]
\begin{itemize}
    \item \textbf{预训练}:投资于数据处理流水线,去重和过滤至关重要
    \item \textbf{SFT}:质量优先,人工审核每一条数据
    \item \textbf{偏好对齐}:确保标注一致性,避免噪声标签
    \item \textbf{持续改进}:建立数据飞轮,不断收集和迭代
\end{itemize}
\end{remark}
