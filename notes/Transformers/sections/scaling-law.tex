\section{Scaling Law}
\label{sec:scaling_law}

Scaling Law(缩放定律)揭示了大语言模型性能与计算量、数据量、模型规模之间的幂律关系,是指导LLM训练资源分配的核心理论。

\subsection{基本概念}

\subsubsection{什么是Scaling Law}

Scaling Law描述了模型性能(通常用损失函数$L$衡量)如何随资源增加而改善:
\begin{equation}
    L = \left(\frac{N_c}{N}\right)^{\alpha_N} + \left(\frac{D_c}{D}\right)^{\alpha_D} + L_\infty
\end{equation}
其中:
\begin{itemize}
    \item $N$:模型参数量
    \item $D$:训练数据量(tokens)
    \item $L_\infty$:不可约损失(数据本身的熵)
    \item $N_c, D_c, \alpha_N, \alpha_D$:拟合常数
\end{itemize}

\subsubsection{为什么Scaling Law重要}

\begin{itemize}
    \item \textbf{资源分配}:给定计算预算,如何分配模型大小和数据量
    \item \textbf{性能预测}:在小规模实验预测大规模模型的性能
    \item \textbf{投资决策}:估算达到目标性能所需的资源
\end{itemize}

\subsection{Kaplan Scaling Law (2020)}

OpenAI的Kaplan等人~\citep{kaplan2020scaling}首次系统研究了LLM的Scaling Law。

\subsubsection{核心发现}

\paragraph{性能与规模的幂律关系}
\begin{align}
    L(N) &= \left(\frac{N_c}{N}\right)^{\alpha_N}, \quad \alpha_N \approx 0.076 \\
    L(D) &= \left(\frac{D_c}{D}\right)^{\alpha_D}, \quad \alpha_D \approx 0.095 \\
    L(C) &= \left(\frac{C_c}{C}\right)^{\alpha_C}, \quad \alpha_C \approx 0.050
\end{align}
其中$C$是计算量(FLOPs)。

\paragraph{关键结论}
\begin{enumerate}
    \item \textbf{模型规模主导}:在固定计算预算下,更大的模型(训练更少步数)比小模型(训练更多步数)更优
    \item \textbf{最优分配}:计算量增加10倍时,模型参数应增加约5.5倍,数据量增加约1.8倍
    \item \textbf{架构不敏感}:Scaling Law对Transformer的具体超参数(层数、宽度)不敏感
\end{enumerate}

\subsubsection{计算最优模型}

Kaplan建议的计算最优分配:
\begin{equation}
    N_{opt} \propto C^{0.73}, \quad D_{opt} \propto C^{0.27}
\end{equation}

这意味着随着计算预算增加,应该更多地投资于模型规模而非数据量。

\subsection{Chinchilla Scaling Law (2022)}

DeepMind的Hoffmann等人~\citep{hoffmann2022chinchilla}挑战了Kaplan的结论,提出了更均衡的分配策略。

\subsubsection{核心发现}

\paragraph{数据的重要性被低估}
Chinchilla研究表明,之前的模型普遍欠拟合(undertrained):
\begin{itemize}
    \item GPT-3 (175B):训练了300B tokens,但最优应该是约3.5T tokens
    \item Gopher (280B):训练了300B tokens,同样严重不足
\end{itemize}

\paragraph{新的最优分配}
\begin{equation}
    N_{opt} \propto C^{0.50}, \quad D_{opt} \propto C^{0.50}
\end{equation}

更简洁的表述:
\begin{equation}
    D_{opt} \approx 20 \times N
    \label{eq:chinchilla}
\end{equation}

即\textbf{最优训练数据量约为模型参数的20倍}。

\subsubsection{为什么是20倍?推导直觉}

这个看似任意的常数背后有清晰的数学逻辑。考虑计算预算约束 $C = 6ND$(FLOPs $\approx$ 6 × 参数 × tokens),目标是最小化损失:

\begin{equation}
    L(N, D) = \frac{A}{N^{\alpha}} + \frac{B}{D^{\beta}} + L_\infty
\end{equation}

在约束 $C = 6ND$ 下求极值,使用拉格朗日乘数法:
\begin{equation}
    \frac{\partial L}{\partial N} \bigg/ \frac{\partial L}{\partial D} = \frac{D}{N}
\end{equation}

代入幂律形式:
\begin{equation}
    \frac{\alpha A / N^{\alpha+1}}{\beta B / D^{\beta+1}} = \frac{D}{N} \implies \frac{D}{N} = \left(\frac{\alpha A}{\beta B}\right)^{\frac{1}{\alpha+\beta+2}}
\end{equation}

Chinchilla的实验测得 $\alpha \approx 0.34$,$\beta \approx 0.28$,代入后得到 $D/N \approx 20$。

\paragraph{物理直觉}
这个结果可以这样理解:模型参数需要足够的"训练信号"才能收敛到最优值。每个参数平均需要看到约20个tokens的信息才能学到有意义的表示。参数太多(相对于数据)会欠拟合——模型有容量但缺乏信号;数据太多(相对于参数)会浪费——模型容量已饱和,额外数据无法被利用。

\paragraph{Kaplan vs Chinchilla:争议的本质}
两个团队得出不同结论的根本原因在于\textbf{实验设计}:
\begin{itemize}
    \item Kaplan固定训练步数,变化模型大小——这倾向于发现"大模型更好"
    \item Chinchilla固定计算量,同时变化模型和数据——这才能找到真正的帕累托最优
\end{itemize}

Chinchilla的设计更接近实际资源分配问题:给定固定预算,如何分配?

\subsubsection{Chinchilla模型验证}

DeepMind训练了70B参数的Chinchilla模型,使用1.4T tokens:
\begin{itemize}
    \item 计算量与Gopher (280B)相同
    \item 性能全面超越Gopher
    \item 推理成本降低4倍(参数量是1/4)
\end{itemize}

\begin{table}[htbp]
\centering
\caption{Chinchilla vs Gopher性能对比}
\label{tab:chinchilla_vs_gopher}
\begin{tabular}{lccc}
\toprule
模型 & 参数量 & 训练Tokens & MMLU \\
\midrule
Gopher & 280B & 300B & 60.0\% \\
Chinchilla & 70B & 1.4T & 67.6\% \\
\bottomrule
\end{tabular}
\end{table}

\subsubsection{对工业界的影响}

Chinchilla Law深刻影响了后续模型的设计:

\begin{table}[htbp]
\centering
\caption{主流模型的参数-数据配比}
\label{tab:model_data_ratio}
\begin{tabular}{lccc}
\toprule
模型 & 参数量 & 训练Tokens & Tokens/参数 \\
\midrule
GPT-3 & 175B & 300B & 1.7$\times$ \\
Chinchilla & 70B & 1.4T & 20$\times$ \\
LLaMA & 65B & 1.4T & 21.5$\times$ \\
LLaMA 2 & 70B & 2T & 28.6$\times$ \\
LLaMA 3 & 70B & 15T & 214$\times$ \\
Qwen 2 & 72B & 7T+ & 97$\times$ \\
\bottomrule
\end{tabular}
\end{table}

\subsection{超越Chinchilla:推理最优的视角}

Chinchilla法则解决的是"训练最优"问题:给定计算预算,如何最小化训练后的损失?但在工业部署中,问题的形式不同:给定\textbf{总预算}(训练+推理),如何最大化用户价值?

\subsubsection{训练-推理权衡的数学框架}

设总成本为训练成本加上生命周期内的推理成本:
\begin{equation}
    \text{Total Cost} = C_{train} + n_{infer} \times C_{infer}(N)
\end{equation}

其中 $C_{train} = 6ND$,$C_{infer}(N) \propto N$(推理成本与参数量成正比)。

\paragraph{关键洞察}
Chinchilla最优假设 $n_{infer} = 0$——只考虑训练。但当 $n_{infer}$ 很大时,推理成本主导总成本。此时最优策略发生根本转变:
\begin{itemize}
    \item \textbf{小模型}:$C_{infer}$ 低,即使每个query成本微小,累积量也可观
    \item \textbf{Over-training}:用额外训练成本换取更小模型的性能,摊薄到海量推理中
\end{itemize}

\paragraph{定量分析}
假设目标性能固定,两种策略的成本对比:
\begin{enumerate}
    \item Chinchilla最优:70B模型 + 1.4T tokens,$C_{train} \propto 70 \times 1.4 = 98$
    \item Over-training:8B模型 + 15T tokens,$C_{train} \propto 8 \times 15 = 120$
\end{enumerate}

训练成本增加22\%,但推理成本降低88\%($8/70$)。当 $n_{infer} > 10^9$ 时(对于ChatGPT级产品很常见),Over-training策略的总成本显著更低。

\subsubsection{LLaMA的策略}

LLaMA系列采用激进的Over-training:
\begin{itemize}
    \item LLaMA-7B:1T tokens(143$\times$参数)
    \item LLaMA 2-7B:2T tokens(286$\times$参数)
    \item LLaMA 3-8B:15T tokens(1875$\times$参数)
\end{itemize}

\begin{remark}[Over-training的收益递减]
虽然Over-training有益,但收益递减:
\begin{itemize}
    \item 早期:每增加1倍数据,性能显著提升
    \item 后期:需要更多数据才能获得同样提升
    \item 存在实际上限:数据重复使用最终会导致过拟合
\end{itemize}
\end{remark}

\subsection{多维度Scaling Law}

\subsubsection{下游任务Scaling}

预训练Loss的Scaling Law不完全转化为下游任务性能:

\begin{itemize}
    \item \textbf{涌现能力}:某些能力在特定规模突然出现
    \item \textbf{任务依赖}:不同任务的Scaling曲线不同
    \item \textbf{评估敏感}:度量方式影响观察到的Scaling行为
\end{itemize}

\subsubsection{RLHF/DPO的Scaling}

后训练也遵循Scaling Law,但形式不同:

\begin{itemize}
    \item \textbf{基座模型规模}:更大的基座模型对齐效果更好
    \item \textbf{偏好数据量}:收益递减明显,质量比数量重要
    \item \textbf{Reward Model}:RM规模通常应与Policy规模匹配
\end{itemize}

\subsubsection{推理时Scaling(Test-time Compute)}

OpenAI o1等推理模型展示了新的Scaling维度:

\begin{equation}
    \text{Performance} = f(\text{Pretraining Compute}, \text{Inference Compute})
\end{equation}

\begin{itemize}
    \item 通过更多推理时计算(更长的思考链)提升性能
    \item 与预训练计算形成互补
    \item 开辟了``用推理换性能''的新范式
\end{itemize}

\subsection{Scaling Law的局限与争议}

\subsubsection{局限性}

\begin{itemize}
    \item \textbf{外推风险}:从小规模实验外推大规模可能失准
    \item \textbf{数据质量}:Scaling Law假设数据质量恒定
    \item \textbf{架构依赖}:不同架构可能有不同的Scaling行为
    \item \textbf{任务特定}:通用Scaling Law可能不适用于特定任务
\end{itemize}

\subsubsection{数据墙(Data Wall)}

\paragraph{高质量数据枯竭}
\begin{itemize}
    \item 互联网高质量文本有限(估计约10-15T tokens)
    \item 继续Scaling需要合成数据或其他来源
    \item 数据重复使用有上限
\end{itemize}

\paragraph{应对策略}
\begin{itemize}
    \item \textbf{合成数据}:用模型生成训练数据
    \item \textbf{数据增强}:改写、翻译、多样化
    \item \textbf{多模态数据}:图像、视频、音频
    \item \textbf{代码数据}:GitHub等代码仓库
\end{itemize}

\subsubsection{涌现能力的争议}

\paragraph{涌现是真实的吗?}
有研究认为``涌现能力''可能是评估度量的假象:
\begin{itemize}
    \item 使用连续度量(如token-level准确率)替代离散度量
    \item 重新分析后,很多``涌现''变成平滑的Scaling
\end{itemize}

\paragraph{实践意义}
无论涌现是否真实存在,大模型确实展现出小模型没有的能力,这对应用仍然重要。

\subsection{Scaling Law的实践应用}

\subsubsection{预测模型性能}

利用Scaling Law进行性能预测:
\begin{enumerate}
    \item 训练多个小规模模型(不同大小、不同数据量)
    \item 拟合Scaling Law参数
    \item 外推预测大规模模型性能
    \item 决定是否值得投资大规模训练
\end{enumerate}

\subsubsection{资源分配决策}

给定计算预算$C$,如何分配:

\paragraph{训练最优(Chinchilla)}
\begin{align}
    N &= \sqrt{C / 6D} \\
    D &= 20N
\end{align}

\paragraph{推理最优}
根据预期推理量$n_{infer}$调整:
\begin{itemize}
    \item $n_{infer}$小:接近Chinchilla最优
    \item $n_{infer}$大:Over-train更小的模型
\end{itemize}

\subsubsection{Scaling Law拟合示例}

\begin{lstlisting}
import numpy as np
from scipy.optimize import curve_fit

def scaling_law(x, a, b, c):
    """L = a / x^b + c"""
    return a / np.power(x, b) + c

# 实验数据:(参数量, 损失)
params = [125e6, 350e6, 760e6, 1.3e9, 2.7e9, 6.7e9]
losses = [3.50, 3.25, 3.10, 2.95, 2.80, 2.65]

# 拟合
popt, _ = curve_fit(scaling_law, params, losses)
a, b, c = popt

# 预测70B模型的损失
pred_loss = scaling_law(70e9, a, b, c)
print(f"Predicted loss for 70B: {pred_loss:.2f}")
\end{lstlisting}

\begin{remark}[Scaling Law使用建议]
\begin{itemize}
    \item \textbf{小规模验证}:在投入大规模训练前,用小模型验证假设
    \item \textbf{多点拟合}:至少3-5个不同规模的数据点
    \item \textbf{保守外推}:外推倍数不宜过大(通常$<$10倍)
    \item \textbf{任务相关}:关注目标任务的Scaling,而非只看Loss
    \item \textbf{持续监控}:大规模训练时监控是否符合预期
\end{itemize}
\end{remark}
