% ============================================
% 16. 推理大模型
% ============================================
\section{推理大模型}
\label{sec:reasoning}

前一章讨论的推理优化技术(KV Cache、投机解码等)都致力于同一个目标:\textbf{让模型更快地回答}。但2024年的一系列突破揭示了另一个维度:有时候,\textbf{让模型更慢地回答}反而能获得更好的结果。

这不是技术的倒退,而是对"推理"本身的重新理解。传统LLM的自回归生成是一种"快思考"——每个token的生成几乎不假思索。而人类解决复杂问题时,往往需要"慢思考":反复尝试、回溯检查、探索多条路径。推理大模型(o1、R1等)将这种慢思考机制引入LLM,用推理时的额外计算换取答案质量的提升。

\subsection{从快思考到慢思考}

\subsubsection{传统LLM的局限}

传统大语言模型采用自回归生成方式,给定输入后直接预测下一个token,这种"System 1"式的快速响应在许多任务上表现出色,但在需要复杂推理的任务上存在明显局限:

\begin{itemize}
    \item \textbf{推理深度受限}:每个token的生成只依赖前面的上下文,缺乏"回头检查"的能力
    \item \textbf{错误累积}:推理链中的早期错误会传播到后续步骤
    \item \textbf{缺乏规划}:无法预先规划解题路径,只能"边走边看"
\end{itemize}

\subsubsection{测试时计算(Test-Time Compute)}

推理大模型的核心思想是\textbf{测试时计算扩展}(Test-Time Compute Scaling):在推理阶段投入更多计算资源,换取更好的输出质量。

\textbf{Snell等人(2024)} \cite{snell2024scaling}的关键发现:
\begin{itemize}
    \item 测试时计算的扩展可以比扩展模型参数更有效
    \item 使用"计算最优"策略,测试时计算效率可提升4倍以上
    \item 在FLOPs匹配的评估中,小模型+测试时计算可超越14倍大的模型
\end{itemize}

测试时计算的主要方式:
\begin{enumerate}
    \item \textbf{搜索}:生成多个候选答案,使用验证器选择最佳
    \item \textbf{思考}:让模型"思考"更长时间,生成详细的推理过程
    \item \textbf{迭代}:多轮自我修正和优化
\end{enumerate}

\subsection{链式思考与自一致性}

\subsubsection{链式思考(Chain-of-Thought)}

链式思考(CoT)提示是推理大模型的基础技术,通过引导模型生成中间推理步骤来提升复杂任务的表现。

\textbf{基本形式}:
\begin{verbatim}
Q: Roger有5个网球,他又买了2罐网球,每罐3个。
   他现在有多少网球?
A: Roger一开始有5个球。2罐网球共有2*3=6个球。
   5+6=11。答案是11。
\end{verbatim}

\textbf{Zero-shot CoT}:仅需添加"Let's think step by step"即可激发模型的推理能力。

\subsubsection{自一致性(Self-Consistency)}

自一致性 \cite{wang2023selfconsistency}是对链式思考的重要改进,核心思想是:
\begin{itemize}
    \item 对同一问题生成多条推理路径
    \item 通过多数投票选择最一致的答案
    \item 利用"殊途同归"的直觉——正确答案应该可以通过多种方式得出
\end{itemize}

\textbf{效果提升}:
\begin{itemize}
    \item GSM8K:+17.9\%
    \item SVAMP:+11.0\%
    \item AQuA:+12.2\%
    \item StrategyQA:+6.4\%
\end{itemize}

\textbf{改进方法}:
\begin{itemize}
    \item \textbf{CISC}(Confidence-Informed SC):基于置信度加权投票,减少40\%以上的采样需求
    \item \textbf{RASC}(Reasoning-Aware SC):动态调整采样数量
    \item \textbf{LSC}(Latent SC):基于语义一致性选择,适用于长文本回答
\end{itemize}

\subsection{奖励模型与验证器}

\subsubsection{结果奖励模型(ORM)}

结果奖励模型(Outcome Reward Model)只对最终答案给出奖励信号:

\begin{equation}
    r_\text{ORM}(x, y) = \begin{cases} 1 & \text{if } y \text{ is correct} \\ 0 & \text{otherwise} \end{cases}
\end{equation}

\textbf{优点}:标注成本低,只需判断最终答案对错

\textbf{缺点}:
\begin{itemize}
    \item 信用分配困难:无法区分哪一步出错
    \item 反馈延迟:只有完成整个推理后才能获得奖励
\end{itemize}

\subsubsection{过程奖励模型(PRM)}

过程奖励模型(Process Reward Model)对推理的每一步给出奖励信号 \cite{lightman2023lets}:

\begin{equation}
    r_\text{PRM}(x, y_{1:t}) = \text{score}(y_t | x, y_{1:t-1})
\end{equation}

其中$y_t$是第$t$步推理,score通常为\{-1, 0, +1\}表示\{错误, 中性, 正确\}。

\textbf{OpenAI的实验结果}:
使用pre-RLHF GPT-4作为基础模型,PRM在MATH测试集上达到78.2\%准确率,显著优于ORM。

\textbf{PRM vs ORM对比}:

\begin{table}[htbp]
\centering
\caption{过程奖励模型与结果奖励模型对比}
\begin{tabular}{lcc}
\toprule
\textbf{特性} & \textbf{ORM} & \textbf{PRM} \\
\midrule
反馈粒度 & 整体结果 & 每步过程 \\
标注成本 & 低 & 高 \\
信用分配 & 困难 & 精确 \\
奖励黑客风险 & 低 & 较高 \\
搜索效率 & 较低 & 更高 \\
\bottomrule
\end{tabular}
\end{table}

\textbf{隐式PRM}:
最近研究发现,可以通过训练ORM然后将其作为PRM使用,获得"免费"的过程奖励,无需昂贵的步骤级标注。

\subsubsection{过程优势验证器(PAV)}

PAV(Process Advantage Verifier)\cite{setlur2024rewarding}结合了过程监督和优势估计:
\begin{itemize}
    \item 相比ORM,搜索准确率提升8\%以上
    \item 计算效率提升1.5-5倍
    \item 在线RL中样本效率提升5-6倍
\end{itemize}

\subsection{搜索与规划}

\subsubsection{Best-of-N采样}

最简单的搜索策略是生成N个候选答案,使用验证器选择最佳:

\begin{equation}
    y^* = \arg\max_{y \in \{y_1, ..., y_N\}} r(x, y)
\end{equation}

OpenAI o1在AIME 2024上的表现:
\begin{itemize}
    \item 单次采样(pass@1):74\%
    \item 64次采样+共识(consensus@64):83\%
\end{itemize}

\subsubsection{蒙特卡洛树搜索(MCTS)}

MCTS将推理过程建模为树搜索问题,每个节点是一个推理状态,边是推理步骤。

\textbf{基本流程}:
\begin{enumerate}
    \item \textbf{选择}(Selection):使用UCB公式选择有潜力的节点
    \item \textbf{扩展}(Expansion):生成新的推理步骤
    \item \textbf{模拟}(Simulation):完成推理并获得结果
    \item \textbf{回传}(Backpropagation):更新路径上所有节点的价值
\end{enumerate}

\textbf{UCB公式}:
\begin{equation}
    \text{UCB}(s, a) = Q(s, a) + c \sqrt{\frac{\ln N(s)}{N(s, a)}}
\end{equation}

\textbf{MCT Self-Refine (MCTSr)}:
结合LLM自我改进与MCTS,在奥林匹克级数学问题上取得优异表现。

\textbf{SC-MCTS*}:
使用对比解码设计可解释的奖励模型,结合推测解码加速,平均每节点速度提升51.9\%。在Blocksworld数据集上超越o1-mini 17.4\%。

\subsection{OpenAI o1}

\subsubsection{核心设计}

OpenAI o1(2024年9月发布)是首个大规模商用的推理大模型,其核心创新在于将链式思考内化为模型能力。

\textbf{关键特点}:
\begin{itemize}
    \item \textbf{推理token}(Reasoning Tokens):模型在回答前生成内部推理过程
    \item \textbf{隐藏思考}:推理token对用户不可见(但会计费)
    \item \textbf{强化学习训练}:通过大规模RL学习"如何思考"
\end{itemize}

\textbf{OpenAI官方描述}:
\begin{quote}
"Similar to how a human may think for a long time before responding to a difficult question, o1 uses a chain of thought when attempting to solve a problem. Through reinforcement learning, o1 learns to hone its chain of thought and refine the strategies it uses."
\end{quote}

\subsubsection{性能表现}

\begin{table}[htbp]
\centering
\caption{o1在主要benchmark上的表现}
\begin{tabular}{lccc}
\toprule
\textbf{Benchmark} & \textbf{GPT-4o} & \textbf{o1-preview} & \textbf{o1} \\
\midrule
AIME 2024 & 12\% & 44\% & 74\% \\
Codeforces Rating & 808 & 1673 & 1891 \\
GPQA Diamond & 50.6\% & 73.3\% & 78.0\% \\
MATH-500 & 60.3\% & 85.5\% & 94.8\% \\
\bottomrule
\end{tabular}
\end{table}

\subsubsection{扩展规律}

o1展示了两个维度的扩展规律:
\begin{enumerate}
    \item \textbf{训练时计算}:更多RL训练带来更强的推理能力
    \item \textbf{测试时计算}:更长的思考时间带来更好的答案质量
\end{enumerate}

这打开了一条新的扩展路径:不仅可以通过增加参数和训练数据来提升性能,还可以通过增加推理时的计算来提升。

\subsection{DeepSeek-R1}

\subsubsection{纯RL训练的突破}

DeepSeek-R1 \cite{deepseek2025r1}(2025年1月)是首个证明\textbf{纯强化学习可以激发推理能力}的开源模型。

\textbf{DeepSeek-R1-Zero}的关键发现:
\begin{itemize}
    \item 无需SFT,仅通过RL即可获得强大推理能力
    \item 涌现出自我反思、验证、动态策略调整等高级推理模式
    \item AIME 2024:从15.6\%提升到71.0\%(pass@1),多数投票达86.7\%
\end{itemize}

\subsubsection{GRPO算法}

DeepSeek使用Group Relative Policy Optimization(GRPO)进行强化学习训练:

\textbf{核心思想}:
\begin{itemize}
    \item 省去传统RLHF中与策略模型同等规模的Critic模型
    \item 使用组内相对分数作为基线估计
    \item 大幅降低训练成本
\end{itemize}

\textbf{GRPO优化目标}:
\begin{equation}
    \mathcal{L}_\text{GRPO} = -\mathbb{E}_{x, \{y_i\}}\left[\sum_i \frac{r(x, y_i) - \bar{r}}{\sigma_r} \log \pi_\theta(y_i|x)\right]
\end{equation}

其中$\bar{r}$是组内平均奖励,$\sigma_r$是组内奖励标准差。

\subsubsection{完整训练流程}

DeepSeek-R1的训练包含四个阶段:

\begin{enumerate}
    \item \textbf{冷启动数据}:少量高质量推理数据,解决R1-Zero的可读性问题
    \item \textbf{推理RL}:大规模RL训练,发现更好的推理模式
    \item \textbf{拒绝采样SFT}:收集RL模型的优质输出进行SFT
    \item \textbf{偏好RL}:与人类偏好对齐
\end{enumerate}

\subsubsection{涌现能力}

R1-Zero在训练过程中涌现出多种高级推理行为:
\begin{itemize}
    \item \textbf{自我反思}:"Wait, let me reconsider..."
    \item \textbf{验证}:检查中间步骤的正确性
    \item \textbf{回溯}:发现错误后退回重试
    \item \textbf{策略切换}:一种方法不行时尝试另一种
\end{itemize}

\subsection{开源推理模型}

\subsubsection{QwQ(Qwen with Questions)}

QwQ \cite{qwen2024qwq}是阿里巴巴Qwen团队发布的开源推理模型(2024年11月)。

\textbf{设计理念}:
\begin{quote}
"QwQ approaches every problem with genuine wonder and doubt. It knows that it knows nothing, and that's precisely what drives its curiosity."
\end{quote}

\textbf{技术特点}:
\begin{itemize}
    \item 32B参数,32K上下文长度
    \item 使用规则化强化学习嵌入推理能力
    \item 推理时生成长思考链
\end{itemize}

\textbf{性能表现}:
\begin{itemize}
    \item GPQA:65.2\%(研究生级科学推理)
    \item AIME 2024:50.0\%
    \item MATH-500:90.6\%
    \item LiveCodeBench:50.0\%
\end{itemize}

\textbf{已知局限}:
\begin{itemize}
    \item 可能混合语言或意外切换语言
    \item 可能陷入循环推理,产生过长输出
\end{itemize}

\subsubsection{Marco-o1}

阿里巴巴的另一个推理模型Marco-o1使用MCTS算法生成合成训练数据,结合CoT样本进行训练。

\subsection{知识蒸馏}

\subsubsection{推理能力的迁移}

DeepSeek开创性地证明了推理能力可以通过蒸馏迁移到小模型。

\textbf{蒸馏方法}:
\begin{itemize}
    \item 使用DeepSeek-R1生成800K推理样本
    \item 在小模型上进行SFT(无需额外RL)
    \item 小模型获得类似的推理能力
\end{itemize}

\subsubsection{蒸馏模型系列}

\begin{table}[htbp]
\centering
\caption{DeepSeek-R1蒸馏模型性能}
\begin{tabular}{lccc}
\toprule
\textbf{模型} & \textbf{基座} & \textbf{AIME 2024} & \textbf{MATH-500} \\
\midrule
R1-Distill-Qwen-1.5B & Qwen2.5-1.5B & 28.9\% & 83.9\% \\
R1-Distill-Qwen-7B & Qwen2.5-7B & 55.5\% & 92.8\% \\
R1-Distill-Qwen-14B & Qwen2.5-14B & 69.7\% & 93.9\% \\
R1-Distill-Qwen-32B & Qwen2.5-32B & 72.6\% & 94.3\% \\
R1-Distill-Llama-8B & Llama3.1-8B & 50.4\% & 89.1\% \\
R1-Distill-Llama-70B & Llama3.3-70B & 70.0\% & 94.5\% \\
\bottomrule
\end{tabular}
\end{table}

\textbf{关键发现}:
\begin{itemize}
    \item R1-Distill-Qwen-32B超越o1-mini
    \item 蒸馏效果优于同规模模型直接RL训练
    \item 蒸馏是获得推理能力的高效途径
\end{itemize}

\subsection{技术对比与分析}

\subsubsection{主要推理模型对比}

\begin{table}[htbp]
\centering
\caption{主流推理大模型对比}
\resizebox{\textwidth}{!}{
\begin{tabular}{lcccccc}
\toprule
\textbf{模型} & \textbf{参数} & \textbf{开源} & \textbf{训练方法} & \textbf{AIME} & \textbf{MATH} & \textbf{发布} \\
\midrule
GPT-4o & - & \ding{55} & SFT & 12\% & 60.3\% & 2024.05 \\
o1-preview & - & \ding{55} & RL & 44\% & 85.5\% & 2024.09 \\
o1 & - & \ding{55} & RL & 74\% & 94.8\% & 2024.12 \\
\midrule
QwQ-32B & 32B & \ding{51} & RL & 50\% & 90.6\% & 2024.11 \\
DeepSeek-R1 & 671B & \ding{51} & RL & 79.8\% & 97.3\% & 2025.01 \\
R1-Distill-32B & 32B & \ding{51} & 蒸馏 & 72.6\% & 94.3\% & 2025.01 \\
\bottomrule
\end{tabular}
}
\end{table}

\subsubsection{训练范式对比}

\begin{itemize}
    \item \textbf{o1}:大规模RL + 隐藏推理token + 闭源
    \item \textbf{DeepSeek-R1}:GRPO + 多阶段训练 + 完全开源
    \item \textbf{QwQ}:规则化RL + 开源权重
    \item \textbf{蒸馏模型}:SFT on推理数据 + 开源
\end{itemize}

\subsection{应用与局限}

\subsubsection{适用场景}

推理大模型特别适合:
\begin{itemize}
    \item \textbf{数学问题}:竞赛数学、定理证明
    \item \textbf{代码生成}:复杂算法、调试
    \item \textbf{科学推理}:物理、化学问题
    \item \textbf{逻辑推理}:规划、约束满足
\end{itemize}

\subsubsection{当前局限}

\begin{itemize}
    \item \textbf{延迟高}:思考时间长,不适合实时交互
    \item \textbf{成本高}:推理token消耗大量计算资源
    \item \textbf{过度思考}:简单问题也可能产生冗长推理
    \item \textbf{循环推理}:可能陷入无意义的思考循环
    \item \textbf{语言混杂}:思考过程中可能混合多种语言
\end{itemize}

\subsubsection{开放问题}

\begin{itemize}
    \item \textbf{最优思考长度}:如何确定何时停止思考?
    \item \textbf{思考可解释性}:隐藏的推理过程是否可信?
    \item \textbf{通用推理}:当前主要在数学/代码领域,如何扩展到更多领域?
    \item \textbf{效率优化}:如何在保持推理质量的同时降低计算成本?
\end{itemize}

\subsection{未来方向}

\subsubsection{计算最优策略}

根据任务难度动态调整测试时计算:
\begin{itemize}
    \item 简单问题:快速响应
    \item 困难问题:深度思考
    \item 预测难度并自动选择策略
\end{itemize}

\subsubsection{多模态推理}

将推理能力扩展到多模态:
\begin{itemize}
    \item 视觉推理:图像中的逻辑关系
    \item 视频理解:时序推理
    \item 具身智能:物理世界的规划
\end{itemize}

\subsubsection{推理与Agent}

推理大模型为AI Agent提供了更强的规划能力:
\begin{itemize}
    \item 任务分解与规划
    \item 工具使用决策
    \item 长期目标追踪
\end{itemize}

\subsubsection{效率提升}

\begin{itemize}
    \item \textbf{推测解码}:加速推理token生成
    \item \textbf{早停策略}:检测到答案收敛时提前停止
    \item \textbf{轻量化}:蒸馏更小的推理模型
    \item \textbf{稀疏激活}:仅激活与推理相关的参数
\end{itemize}

