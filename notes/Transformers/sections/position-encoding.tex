\section{位置编码}
\label{sec:position_encoding}

不同于RNN、CNN等模型,Transformer的自注意力机制本身是\textbf{置换不变的}(permutation invariant)——纯粹的Attention模块无法捕捉输入顺序,即无法区分不同位置的Token。为此,位置编码的引入是必不可少的。

位置编码大体有两种思路:
\begin{enumerate}
    \item \textbf{绝对位置编码}:将位置信息融入到输入中,即 $\bm{x}'_k = \bm{x}_k + \bm{p}_k$ 或 $\bm{x}'_k = \bm{x}_k \odot \bm{p}_k$
    \item \textbf{相对位置编码}:微调Attention结构,使其能分辨不同位置的Token
\end{enumerate}

\begin{table}[htbp]
\centering
\caption{位置编码方法分类与对比}
\label{tab:pos_encoding_comparison}
\begin{tabular}{llccl}
\toprule
类型 & 方法 & 作用位置 & 外推性 & 代表模型 \\
\midrule
\multirow{3}{*}{绝对} & Sinusoidal & Embedding & 差 & Transformer \\
 & Learned & Embedding & 差 & BERT, GPT \\
 & FLOATER & Embedding & 中 & — \\
\midrule
\multirow{4}{*}{相对} & 经典式 & Attention & 中 & NEZHA \\
 & T5 Bias & Attention score & 好 & T5 \\
 & ALiBi & Attention score & 好 & BLOOM, MPT \\
 & \textbf{RoPE} & Q/K向量 & 好 & LLaMA, Qwen \\
\bottomrule
\end{tabular}
\end{table}

\subsection{绝对位置编码}

绝对位置编码的一般形式是在输入的第 $k$ 个向量 $\bm{x}_k$ 中加入位置向量 $\bm{p}_k$,其中 $\bm{p}_k$ 只依赖于位置编号 $k$。

\subsubsection{训练式(Learned)}

最朴素的方案是将位置编码作为可训练参数。例如最大长度512、编码维度768,则初始化一个 $512 \times 768$ 的矩阵,随训练更新。BERT、GPT等模型采用此方案。

\paragraph{优点} 简单直接,让模型自己学习位置表示。

\paragraph{缺点} 缺乏外推性——如果预训练最大长度为512,则无法处理更长的序列。虽然可以将超长位置随机初始化后继续微调,但效果有限。

\subsubsection{三角函数式(Sinusoidal)}

原始Transformer~\citep{vaswani2017attention}采用正弦余弦函数生成位置编码:
\begin{align}
    p_{k,2i} &= \sin\left(k / 10000^{2i/d}\right) \\
    p_{k,2i+1} &= \cos\left(k / 10000^{2i/d}\right)
\end{align}
其中 $p_{k,2i}, p_{k,2i+1}$ 分别是位置 $k$ 的编码向量的第 $2i, 2i+1$ 个分量。

\paragraph{设计直觉}
不同维度对应不同频率的周期函数:低维变化快(捕捉局部位置),高维变化慢(捕捉全局位置)。由三角恒等式 $\sin(\alpha+\beta) = \sin\alpha\cos\beta + \cos\alpha\sin\beta$,位置 $\alpha+\beta$ 的向量可表示为位置 $\alpha$ 和 $\beta$ 的向量组合,这提供了表达相对位置信息的可能性。

\paragraph{局限性}
尽管有显式的生成规律,但实践中Sinusoidal编码的外推能力仍然有限,现代LLM已很少直接使用。

\subsubsection{递归式(FLOATER)}

ICML 2020的论文提出用微分方程建模位置编码:
\begin{equation}
    \frac{d\bm{p}_t}{dt} = h(\bm{p}_t, t)
\end{equation}
其中 $h$ 可以用神经网络建模。这种方法称为FLOATER,理论上具有较好的外推性(可以证明Sinusoidal是其特解),但牺牲了并行性。

\subsubsection{相乘式}

除了 $\bm{x}_k + \bm{p}_k$ 的加法融合,也可以考虑 $\bm{x}_k \odot \bm{p}_k$ 的逐位相乘。有实验表明相乘方式可能取得更好的效果,但目前主流模型仍以相加为主。

\subsection{相对位置编码}

相对位置编码不完整建模每个输入的位置信息,而是在计算Attention时考虑当前位置与被attend位置的\textbf{相对距离}。由于自然语言更依赖相对位置,这类方法通常表现优秀。

\subsubsection{经典式}

Google的论文《Self-Attention with Relative Position Representations》首次提出相对位置编码。考虑带绝对位置编码的Attention:
\begin{equation}
    q_i k_j^\top = (x_i + p_i)W_Q W_K^\top (x_j + p_j)^\top
\end{equation}

展开后将 $p_j W_K$ 替换为二元位置向量 $R^K_{i-j}$(只依赖相对距离),并进行截断:
\begin{equation}
    R^K_{i-j} = p^K[\text{clip}(i-j, p_{\min}, p_{\max})]
\end{equation}

这样只需有限个位置编码就可以表达任意长度的相对位置。华为的NEZHA模型采用了这种编码。

\subsubsection{Transformer-XL / XLNet式}

Transformer-XL对 $q_i k_j^\top$ 完全展开:
\begin{equation}
    q_i k_j^\top = x_i W_Q W_K^\top x_j^\top + x_i W_Q W_K^\top p_j^\top + p_i W_Q W_K^\top x_j^\top + p_i W_Q W_K^\top p_j^\top
\end{equation}

将 $p_j$ 替换为相对位置向量 $R_{i-j}$(使用Sinusoidal生成),$p_i$ 替换为可训练向量 $u, v$:
\begin{equation}
    x_i W_Q W_K^\top x_j^\top + x_i W_Q W_{K,R}^\top R_{i-j}^\top + u W_K^\top x_j^\top + v W_{K,R}^\top R_{i-j}^\top
\end{equation}

\subsubsection{T5式}

T5进一步简化,认为"输入-位置"交互项应该解耦,直接删除中间两项:
\begin{equation}
    x_i W_Q W_K^\top x_j^\top + \beta_{i,j}
\end{equation}

其中 $\beta_{i,j}$ 是只依赖于 $(i,j)$ 的可训练偏置。T5的特色是对相对位置进行\textbf{分桶}处理:邻近位置(0$\sim$7)精细区分,远距离位置共用编码,距离越远共用范围越大。

\subsubsection{DeBERTa式}

DeBERTa保留"输入-位置"和"位置-输入"项,去掉"位置-位置"项:
\begin{equation}
    q_i k_j^\top = x_i W_Q W_K^\top x_j^\top + x_i W_Q W_K^\top R_{i,j}^\top + R_{j,i} W_Q W_K^\top x_j^\top
\end{equation}

DeBERTa还提出了混合使用相对位置(前11层)和绝对位置(后2层)的策略,在SuperGLUE上取得了优异成绩。

\subsection{旋转位置编码(RoPE)}

\citet{su2021roformer}提出的RoPE是目前最主流的位置编码方法,被LLaMA、Mistral、Qwen等模型广泛采用。RoPE的核心思想是\textbf{融合绝对位置与相对位置}:通过在q、k上施加绝对位置的旋转操作,使得内积自然地只依赖于相对位置。

\begin{remark}[RoPE的理论起源]
RoPE的设计灵感来自复数的性质:两个复数的内积 $\langle q e^{im\theta}, k e^{in\theta} \rangle = \text{Re}[q \bar{k} e^{i(m-n)\theta}]$ 只依赖相对位置 $m-n$。这一洞察使得我们可以用绝对位置的操作(乘以 $e^{im\theta}$)达到相对位置的效果。
\end{remark}

\subsubsection{问题设定}

我们希望通过下述运算给 $\bm{q}, \bm{k}$ 添加绝对位置信息:
\begin{equation}
    \tilde{\bm{q}}_m = f(\bm{q}, m), \quad \tilde{\bm{k}}_n = f(\bm{k}, n)
\end{equation}

由于Attention的核心是内积,我们希望内积结果带有\textbf{相对位置信息},即存在恒等关系:
\begin{equation}
    \langle f(\bm{q}, m), f(\bm{k}, n) \rangle = g(\bm{q}, \bm{k}, m-n)
    \label{eq:rope_goal}
\end{equation}

初始条件设为 $f(\bm{q}, 0) = \bm{q}$,$f(\bm{k}, 0) = \bm{k}$。

\subsubsection{二维情况的求解}

借助复数来求解。在复数中有 $\langle \bm{q}, \bm{k} \rangle = \text{Re}[\bm{q} \bar{\bm{k}}]$,所以:
\begin{equation}
    \text{Re}[f(\bm{q}, m) \overline{f(\bm{k}, n)}] = g(\bm{q}, \bm{k}, m-n)
\end{equation}

假设存在复数 $g(\bm{q}, \bm{k}, m-n)$ 使得 $f(\bm{q}, m) \overline{f(\bm{k}, n)} = g(\bm{q}, \bm{k}, m-n)$。用复数的指数形式:
\begin{align}
    f(\bm{q}, m) &= R_f(\bm{q}, m) e^{i\Theta_f(\bm{q}, m)} \\
    f(\bm{k}, n) &= R_f(\bm{k}, n) e^{i\Theta_f(\bm{k}, n)}
\end{align}

代入后得到方程组:
\begin{align}
    R_f(\bm{q}, m) R_f(\bm{k}, n) &= R_g(\bm{q}, \bm{k}, m-n) \\
    \Theta_f(\bm{q}, m) - \Theta_f(\bm{k}, n) &= \Theta_g(\bm{q}, \bm{k}, m-n)
\end{align}

\paragraph{求解模长} 代入 $m = n$:
\begin{equation}
    R_f(\bm{q}, m) R_f(\bm{k}, m) = R_g(\bm{q}, \bm{k}, 0) = R_f(\bm{q}, 0) R_f(\bm{k}, 0) = \|\bm{q}\| \|\bm{k}\|
\end{equation}
最简单的解是 $R_f(\bm{q}, m) = \|\bm{q}\|$,$R_f(\bm{k}, m) = \|\bm{k}\|$,即\textbf{模长不依赖于位置}。

\paragraph{求解幅角} 同样代入 $m = n$:
\begin{equation}
    \Theta_f(\bm{q}, m) - \Theta_f(\bm{k}, m) = \Theta_g(\bm{q}, \bm{k}, 0) = \Theta(\bm{q}) - \Theta(\bm{k})
\end{equation}
这说明 $\Theta_f(\bm{q}, m) - \Theta(\bm{q})$ 只与 $m$ 相关,记为 $\varphi(m)$。再代入 $n = m - 1$,可得 $\{\varphi(m)\}$ 是等差数列,即 $\varphi(m) = m\theta$。

\paragraph{最终解} 综上,二维情况下的RoPE为:
\begin{equation}
    \boxed{f(\bm{q}, m) = \|\bm{q}\| e^{i(\Theta(\bm{q}) + m\theta)} = \bm{q} e^{im\theta}}
\end{equation}

这正是向量乘以旋转因子 $e^{im\theta}$,对应\textbf{旋转角度 $m\theta$}。

\subsubsection{矩阵形式与高维扩展}

二维旋转的矩阵形式:
\begin{equation}
    f(\bm{q}, m) = \begin{pmatrix} \cos m\theta & -\sin m\theta \\ \sin m\theta & \cos m\theta \end{pmatrix} \begin{pmatrix} q_0 \\ q_1 \end{pmatrix}
\end{equation}

由于内积满足线性叠加性,$d$ 维向量可分成 $d/2$ 对,每对独立旋转:
\begin{equation}
    f(\bm{q}, m) = \begin{pmatrix} R_{\theta_0}(m) & & \\ & R_{\theta_1}(m) & \\ & & \ddots \end{pmatrix} \bm{q}
\end{equation}

关键性质:给位置 $m$ 的 $\bm{q}$ 乘 $R_m$,位置 $n$ 的 $\bm{k}$ 乘 $R_n$,则:
\begin{equation}
    (R_m \bm{q})^\top (R_n \bm{k}) = \bm{q}^\top R_m^\top R_n \bm{k} = \bm{q}^\top R_{n-m} \bm{k}
\end{equation}
内积自动包含相对位置信息。此外,$R_m$ 是正交矩阵,\textbf{不改变向量模长},保持模型稳定性。

\subsubsection{旋转矩阵形式}

完整的RoPE公式为:
\begin{equation}
    \text{RoPE}(\bm{x}, m) = \underbrace{\begin{pmatrix}
    \cos m\theta_1 & -\sin m\theta_1 & & & \\
    \sin m\theta_1 & \cos m\theta_1 & & & \\
    & & \cos m\theta_2 & -\sin m\theta_2 & \\
    & & \sin m\theta_2 & \cos m\theta_2 & \\
    & & & & \ddots
    \end{pmatrix}}_{R_\Theta(m)} \bm{x}
    \label{eq:rope_matrix}
\end{equation}

\paragraph{频率参数}
频率 $\theta_i$ 采用指数递减的形式:
\begin{equation}
    \theta_i = \text{base}^{-2(i-1)/d}, \quad i = 1, 2, \ldots, d/2
    \label{eq:rope_theta}
\end{equation}
其中 $\text{base} = 10000$ 是原始设置。不同 $\theta_i$ 对应不同的"波长",小 $i$ 对应高频(短波长,捕捉局部位置),大 $i$ 对应低频(长波长,捕捉全局位置)。

\subsubsection{Base选择原则}

\citet{men2024base}从理论上分析了RoPE的base值与上下文长度的关系,揭示了\textbf{base存在下界}这一重要性质(参见苏剑林的解读~\citep{su2024base})。

\paragraph{语义聚合性质}
设 $\bm{q}$ 是query向量,$\bm{k}^* = \bm{q} + \bm{\epsilon}$ 是语义相似的key($\bm{\epsilon}$ 为小扰动),$\bm{k}$ 是随机key。定义\textbf{语义区分度}为模型区分相似token和随机token的能力:
\begin{equation}
    B_{m,\theta} = \frac{1}{2\sigma^2}\left(\E[\bm{q}^\top R_{m,\theta} \bm{k}^*] - \E[\bm{q}^\top R_{m,\theta} \bm{k}]\right) = \sum_{i=1}^{d/2} \cos(m\theta_i)
    \label{eq:semantic_discrimination}
\end{equation}

\paragraph{Long-term Decay}
随着相对距离 $m$ 增大,$B_{m,\theta}$ 逐渐减小(注意力衰减),且当base过小时会变为负值——此时模型反而给随机token更高的注意力,丧失语义理解能力。

\paragraph{Base下界}
为保证所有上下文位置的语义区分度非负,base必须满足:
\begin{equation}
    b^* = \inf\left\{b \mid B_{m,\theta} \geq 0, \forall m \in \{0, 1, \ldots, L-1\}\right\}
\end{equation}

数值求解得到不同上下文长度对应的base下界:

\begin{table}[htbp]
\centering
\caption{RoPE Base下界与上下文长度的关系}
\label{tab:base_lower_bound}
\begin{tabular}{lcccccc}
\toprule
Context Length $L$ & 4K & 8K & 32K & 128K & 1M \\
\midrule
Base下界 $b^*$ & $4.5 \times 10^4$ & $8.4 \times 10^4$ & $6.4 \times 10^5$ & $3.4 \times 10^6$ & $6.5 \times 10^7$ \\
\bottomrule
\end{tabular}
\end{table}

渐近分析表明 $b^* \approx O(L)$,即base应随上下文长度\textbf{线性增长}。

\paragraph{实际模型的Base选择}
\begin{itemize}
    \item \textbf{LLaMA 3}:训练长度8192,但base选择了500000,远超下界($8.4 \times 10^4$),可能是为更长上下文预留
    \item \textbf{Mixtral}:base = 1000000,支持128K上下文
\end{itemize}

\paragraph{与Position Interpolation的联系}
语义聚合视角也解释了PI的原理:同一个base下,$B_{m,\theta}$ 的非负区间是固定的。PI通过缩小位置间隔($0, 1/s, 2/s, \ldots$)使更多位置落入非负区间,而非增大base。两种方法殊途同归。

\begin{remark}[OOD理论的不足]
早期基于Out-of-Distribution(OOD)的分析认为base越小越好(使位置遍历整个圆)。但语义聚合视角揭示了base存在下界——过小的base会导致模型虽然保持低困惑度,却丧失长距离信息检索能力,形成"表面长上下文能力"。
\end{remark}

\subsubsection{远程衰减性质}

RoPE具有一个重要的理论性质:\textbf{随着相对距离增大,内积会逐渐衰减}。这与自然语言的局部性假设相吻合——距离越近的token通常关联越强。

设 $\bm{q}, \bm{k}$ 是随机向量,RoPE编码后的内积期望为:
\begin{equation}
    \E[\langle f(\bm{q}, m), f(\bm{k}, n) \rangle] \propto \sum_{i=1}^{d/2} \cos((m-n)\theta_i)
\end{equation}

当 $m = n$ 时(同一位置),内积最大;随着 $|m-n|$ 增大,各维度的余弦项相位错开,求和结果减小。这种衰减是\textbf{非单调}的(因为余弦函数),但长程平均效果是衰减的。

\paragraph{与Sinusoidal编码的对比}
Sinusoidal编码虽然在设计时也考虑了相对位置表达,但它只是将位置信息\textbf{加入}到向量中,相对位置需要模型自己学习。而RoPE通过旋转操作,\textbf{在数学上保证}内积只依赖于相对位置,无需学习。

\subsubsection{几何直觉}

\begin{figure}[htbp]
\centering
\begin{tikzpicture}[scale=1.5]
    % 坐标轴
    \draw[->] (-1.5,0) -- (1.8,0) node[right] {$x_1$};
    \draw[->] (0,-1.5) -- (0,1.8) node[above] {$x_2$};

    % 原始向量
    \draw[->, thick, blue] (0,0) -- (1.2,0.4) node[right] {$\bm{q}$};

    % 旋转后的向量
    \draw[->, thick, red] (0,0) -- (0.8,1.0) node[above right] {$f(\bm{q}, m)$};

    % 旋转角度
    \draw[dashed] (0.6,0.2) arc (18:51:0.6);
    \node at (0.75,0.55) {$m\theta$};

    % 单位圆(部分)
    \draw[gray, dashed] (0,0) circle (1.26);
\end{tikzpicture}
\caption{RoPE的几何直觉:位置编码 = 向量旋转。位置 $m$ 处的向量被旋转 $m\theta$ 角度。}
\label{fig:rope_intuition}
\end{figure}

RoPE的核心直觉可以总结为:

\begin{quote}
\textit{"根据位置旋转query和key向量,让点积自然地揭示它们的相对距离。"}
\end{quote}

\begin{itemize}
    \item \textbf{相同位置偏移保持角度不变}:如果 $\bm{q}$ 和 $\bm{k}$ 同时偏移相同的位置(绝对位置改变,相对位置不变),两者会被旋转相同的角度,它们之间的夹角不变,点积也不变
    \item \textbf{距离衰减}:相对位置越远,累积旋转角度越大,高频维度的点积贡献会因相位错开而减小,形成自然的距离衰减
\end{itemize}

\subsubsection{高效实现}

直接使用旋转矩阵需要 $O(d^2)$ 计算。但注意到旋转矩阵是\textbf{分块对角}的,每个2×2块独立作用,可以展开为逐元素运算。

对于二维情况:
\begin{equation}
    \begin{pmatrix} \cos m\theta & -\sin m\theta \\ \sin m\theta & \cos m\theta \end{pmatrix} \begin{pmatrix} q_0 \\ q_1 \end{pmatrix} = \begin{pmatrix} q_0 \cos m\theta - q_1 \sin m\theta \\ q_0 \sin m\theta + q_1 \cos m\theta \end{pmatrix}
\end{equation}

推广到 $d$ 维,可以写成向量形式:
\begin{equation}
    \text{RoPE}(\bm{x}, m) = \bm{x} \odot \cos(m\bm{\theta}) + \text{rotate\_half}(\bm{x}) \odot \sin(m\bm{\theta})
    \label{eq:rope_efficient}
\end{equation}
其中 $\bm{\theta} = (\theta_1, \theta_1, \theta_2, \theta_2, \ldots)$ 是频率向量(每个频率重复两次),$\odot$ 表示逐元素乘法。这种实现的复杂度为 $O(d)$。

\begin{remark}[实现风格]
存在两种实现风格,区别在于如何定义 $\text{rotate\_half}$ 操作:
\begin{itemize}
    \item \textbf{GPT-NeoX风格}:将向量分成前后两半
    \begin{equation*}
        \text{rotate\_half}([x_0, \ldots, x_{d/2-1}, x_{d/2}, \ldots, x_{d-1}]) = [-x_{d/2}, \ldots, -x_{d-1}, x_0, \ldots, x_{d/2-1}]
    \end{equation*}
    \item \textbf{GPT-J风格}:相邻两个维度配对(interleaved)
    \begin{equation*}
        \text{rotate\_half}([x_0, x_1, x_2, x_3, \ldots]) = [-x_1, x_0, -x_3, x_2, \ldots]
    \end{equation*}
\end{itemize}
LLaMA官方实现采用GPT-J风格,但Hugging Face Transformers库默认采用GPT-NeoX风格。两者数学上等价(只是维度排列不同),但加载预训练权重时需要注意兼容性。
\end{remark}

\subsection{RoPE的应用}

\subsubsection{Dense模型}

RoPE已成为主流Dense模型的标准配置:

\begin{table}[htbp]
\centering
\caption{RoPE在Dense模型中的应用}
\label{tab:rope_dense}
\begin{tabular}{lccc}
\toprule
Model & Base & Max Position & Context Length \\
\midrule
LLaMA & 10000 & 2048 & 2K \\
LLaMA 2 & 10000 & 4096 & 4K \\
Mistral-7B & 10000 & 32768 & 32K \\
Qwen-7B & 10000 & 8192 & 8K \\
\bottomrule
\end{tabular}
\end{table}

\subsubsection{MoE模型}

MoE模型同样采用RoPE,但通常配置更大的base值以支持更长上下文:

\begin{table}[htbp]
\centering
\caption{RoPE在MoE模型中的应用}
\label{tab:rope_moe}
\begin{tabular}{lcccc}
\toprule
Model & Base & Max Position & Experts & Notes \\
\midrule
Mixtral-8x7B & 1000000 & 131072 & 8 & Sliding window attention \\
Mixtral-8x22B & 1000000 & 65536 & 8 & — \\
DeepSeek-MoE & 10000 & 4096 & 64 & Fine-grained experts \\
\bottomrule
\end{tabular}
\end{table}

\paragraph{MoE的特殊考虑}
MoE模型中不同expert可能专门处理不同位置模式或模态。LLaMA 4引入了\textbf{iRoPE}(interleaved RoPE)架构:部分层使用RoPE,部分层不使用位置编码,依靠因果attention mask和学习到的模式推断位置。这种混合方法使LLaMA 4 Scout能够支持高达1000万token的上下文窗口(仅在256K token上训练)。

\subsection{长度外推方法}

RoPE的一个关键问题是\textbf{长度外推}(length extrapolation):如何让模型处理超出训练长度的序列?

\subsubsection{Position Interpolation (PI)}

最简单的方法是缩放位置索引~\citep{chen2023extending}:
\begin{equation}
    m' = \frac{m}{s}, \quad s = \frac{L'}{L}
\end{equation}
其中 $L$ 是训练长度,$L'$ 是目标长度。

\paragraph{直觉}
将长序列"压缩"到原始位置范围内。例如,要将4K模型扩展到16K,令 $s=4$,则位置16000被映射到位置4000。

\paragraph{问题}
均匀缩放会破坏高频信息,因为高频维度对位置变化敏感。

\subsubsection{NTK-aware Interpolation}

基于Neural Tangent Kernel理论,调整base而非均匀缩放~\citep{bloc97}:
\begin{equation}
    \text{base}' = \text{base} \cdot s^{d/(d-2)}
\end{equation}

\paragraph{直觉}
将插值压力分散到不同维度:高频维度(编码局部位置)少插值,低频维度(编码全局位置)多插值。

\subsubsection{NTK-by-parts}

进一步改进,对不同维度采用不同的插值策略:
\begin{equation}
    \theta'_i = \begin{cases}
    \theta_i & \text{if } \lambda_i > \beta \cdot L \text{ (高频,不插值)} \\
    \theta_i / s & \text{if } \lambda_i < \alpha \cdot L \text{ (低频,完全插值)} \\
    \text{interpolate} & \text{otherwise (中间,渐进插值)}
    \end{cases}
\end{equation}
其中 $\lambda_i = 2\pi / \theta_i$ 是波长。对LLaMA模型,$\alpha = 1$,$\beta = 32$ 效果最优。

\subsubsection{YaRN}

YaRN(Yet another RoPE extensioN)~\citep{peng2023yarn}结合NTK-by-parts和\textbf{注意力温度缩放}:

\begin{equation}
    \text{Attention}'_{ij} = \frac{\bm{q}_i^\top \bm{k}_j}{\sqrt{d} \cdot t}
\end{equation}

温度系数 $t$ 根据扩展比例 $s$ 设定:
\begin{equation}
    \sqrt{1/t} = 0.1 \cdot \ln(s) + 1
\end{equation}

\paragraph{效果}
YaRN只需约400步微调即可将LLaMA 2从4K扩展到64K,相比PI减少10倍数据和2.5倍训练步数。

\begin{table}[htbp]
\centering
\caption{长度外推方法对比}
\label{tab:extrapolation_comparison}
\begin{tabular}{lccc}
\toprule
Method & Extrapolation & Finetuning & Notes \\
\midrule
PI & 2$\times$ & Required & 均匀缩放 \\
NTK-aware & 32$\times$ & Optional & 无需微调时效果好 \\
NTK-by-parts & 16$\times$ & Required & 分维度处理 \\
YaRN & 16$\times$ & Minimal & 结合温度缩放 \\
Dynamic & 64$\times$ & None & 推理时动态调整 \\
\bottomrule
\end{tabular}
\end{table}

\subsubsection{Dynamic Scaling}

动态缩放在推理时根据当前序列长度自适应调整插值因子:
\begin{equation}
    s = \max\left(1, \frac{\text{current\_length}}{L}\right)
\end{equation}

\paragraph{优势}
\begin{itemize}
    \item 无需微调
    \item 短序列保持原始性能
    \item 长序列自动启用插值
\end{itemize}

\subsection{RoPE vs ALiBi}

ALiBi~\citep{press2021alibi}是另一种流行的相对位置编码方法,直接在attention score上添加距离惩罚:
\begin{equation}
    \text{Attention}_{ij} = \frac{\bm{q}_i^\top \bm{k}_j}{\sqrt{d}} - m \cdot |i - j|
\end{equation}
其中 $m$ 是每个head的斜率参数。

\begin{table}[htbp]
\centering
\caption{RoPE与ALiBi的详细对比}
\label{tab:rope_vs_alibi}
\begin{tabular}{lcc}
\toprule
Feature & RoPE & ALiBi \\
\midrule
编码位置 & Q/K向量 & Attention score \\
参数量 & 0 & 0 \\
外推能力 & 中等(需扩展方法) & 好 \\
训练速度 & 基准 & 更快 \\
KV Cache友好 & 是(旋转一次写入) & 是 \\
采用模型 & LLaMA, Mistral, Qwen & BLOOM, MPT \\
\bottomrule
\end{tabular}
\end{table}

\paragraph{选择建议}
\begin{itemize}
    \item 如果需要极长上下文且不想微调,ALiBi可能更合适
    \item 如果使用主流开源模型生态(LLaMA系),RoPE是默认选择
    \item RoPE配合YaRN等扩展方法可以达到与ALiBi相当的外推能力
\end{itemize}

\subsection{最新进展}

\paragraph{Resonance RoPE (2024)}
针对OOD(Out-of-Distribution)位置的特征插值问题进行优化,进一步提升长度外推能力。

\paragraph{iRoPE (LLaMA 4)}
混合使用RoPE层和无位置编码层,配合推理时注意力温度缩放,实现从256K训练长度到10M上下文窗口的极端外推。

\paragraph{2D/3D RoPE}
将RoPE扩展到二维(图像)和三维(视频)位置编码,应用于Vision Transformer和多模态模型。

\begin{remark}[为什么RoPE成为主流]
RoPE成功的关键因素:
\begin{enumerate}
    \item \textbf{零参数}:不增加模型参数
    \item \textbf{流式友好}:KV Cache只需旋转一次
    \item \textbf{可扩展}:配合PI/YaRN等方法可处理超长上下文
    \item \textbf{生态支持}:LLaMA系模型的标准配置
\end{enumerate}
\end{remark}
