\section{LLM推理优化}
\label{sec:inference}

LLM推理面临独特的挑战:模型参数量巨大、自回归生成逐token进行、KV Cache随序列长度增长。本章介绍主流的推理优化技术和推理引擎。

\subsection{推理基础}

\subsubsection{两阶段推理}

LLM推理分为两个阶段:

\paragraph{Prefill阶段}
处理输入prompt的所有token,生成KV Cache和第一个输出token:
\begin{itemize}
    \item \textbf{计算特性}:并行处理所有输入token,计算密集型(Compute-bound)
    \item \textbf{瓶颈}:矩阵乘法的计算量
    \item \textbf{指标}:Time To First Token(TTFT)
\end{itemize}

\paragraph{Decode阶段}
逐个生成后续token,每步只处理1个新token:
\begin{itemize}
    \item \textbf{计算特性}:自回归生成,内存带宽受限(Memory-bound)
    \item \textbf{瓶颈}:加载模型参数和KV Cache的内存带宽
    \item \textbf{指标}:Inter-Token Latency(ITL)或Tokens Per Second(TPS)
\end{itemize}

\begin{table}[htbp]
\centering
\caption{Prefill vs Decode对比}
\label{tab:prefill_decode}
\begin{tabular}{lcc}
\toprule
特性 & Prefill & Decode \\
\midrule
Token数 & $N$(输入长度) & 1 \\
计算模式 & 并行 & 串行 \\
瓶颈 & 计算 & 内存带宽 \\
GPU利用率 & 高 & 低 \\
\bottomrule
\end{tabular}
\end{table}

\subsubsection{KV Cache}

KV Cache是推理优化的核心。每个Transformer层在处理token时会生成Key和Value向量,这些向量在后续生成中被重复使用。

\paragraph{KV Cache大小}
\begin{equation}
    \text{KV Cache} = 2 \times L \times N \times h_{kv} \times d \times \text{bytes}
\end{equation}
其中$L$是层数,$N$是序列长度,$h_{kv}$是KV头数,$d$是头维度。

\textbf{示例}:LLaMA-13B,序列长度8K,BF16精度:
\begin{equation}
    2 \times 40 \times 8192 \times 40 \times 128 \times 2 = 6.7\text{GB}
\end{equation}
单个请求的KV Cache就需要6.7GB,成为主要的显存瓶颈。

\paragraph{KV Cache优化方向}
\begin{enumerate}
    \item \textbf{减少KV头数}:GQA/MQA(见第\ref{sec:mla}节)
    \item \textbf{量化}:INT8/FP8压缩KV Cache
    \item \textbf{稀疏化}:只保留重要的KV对
    \item \textbf{共享}:跨层共享KV Cache
\end{enumerate}

\subsection{批处理策略}

\subsubsection{静态批处理}

传统方法:等待凑齐一批请求后统一处理。

\textbf{问题}:
\begin{itemize}
    \item 请求长度不一,短请求需等待长请求完成
    \item 填充(Padding)浪费计算资源
    \item 吞吐量低,延迟高
\end{itemize}

\subsubsection{Continuous Batching}

Continuous Batching~\citep{yu2022orca}动态管理请求:
\begin{itemize}
    \item 请求完成后立即释放资源,新请求立即加入
    \item 无需等待整批完成
    \item 迭代级别调度,而非请求级别
\end{itemize}

\begin{figure}[htbp]
\centering
\begin{tikzpicture}[scale=0.6, font=\small]
    % Static batching
    \node[left] at (-1, 2) {\textbf{静态}};
    \fill[blue!30] (0, 1.5) rectangle (4, 2.5);
    \fill[blue!50] (0, 0.5) rectangle (6, 1.5);
    \fill[blue!70] (0, -0.5) rectangle (3, 0.5);
    \fill[gray!30] (3, -0.5) rectangle (6, 0.5);
    \fill[gray!30] (4, 1.5) rectangle (6, 2.5);
    \node at (5, 1) {Padding};

    % Continuous batching
    \node[left] at (-1, -2) {\textbf{连续}};
    \fill[blue!30] (0, -2.5) rectangle (4, -1.5);
    \fill[green!50] (4, -2.5) rectangle (7, -1.5);
    \fill[blue!50] (0, -3.5) rectangle (6, -2.5);
    \fill[blue!70] (0, -4.5) rectangle (3, -3.5);
    \fill[orange!50] (3, -4.5) rectangle (5, -3.5);
    \fill[red!50] (5, -4.5) rectangle (7, -3.5);

    % Timeline
    \draw[->, thick] (0, -5.5) -- (8, -5.5);
    \node[below] at (4, -5.5) {Time};
\end{tikzpicture}
\caption{静态批处理 vs Continuous Batching}
\label{fig:continuous_batching}
\end{figure}

\subsubsection{Chunked Prefill}

长prompt的Prefill会阻塞Decode请求。Chunked Prefill~\citep{agrawal2024sarathi}将Prefill分块:
\begin{itemize}
    \item 长Prefill拆分为多个小块
    \item 每个块与Decode请求混合执行
    \item 减少Decode请求的排队延迟
\end{itemize}

\subsubsection{Prefill-Decode分离}

DistServe~\citep{zhong2024distserve}提出将Prefill和Decode部署到不同的GPU:
\begin{itemize}
    \item \textbf{Prefill服务器}:计算密集,可用更高的张量并行度
    \item \textbf{Decode服务器}:内存密集,优化批处理大小
    \item \textbf{KV Cache传输}:通过NVLink/PCIe在服务器间传递
\end{itemize}

\textbf{优势}:
\begin{itemize}
    \item Prefill不会阻塞Decode,降低ITL
    \item 可独立扩展两种资源
    \item 针对不同阶段优化并行策略
\end{itemize}

\subsection{PagedAttention与vLLM}

vLLM~\citep{kwon2023efficient}引入PagedAttention,借鉴操作系统虚拟内存的思想管理KV Cache。

\subsubsection{传统KV Cache的问题}

传统实现预分配连续内存存储KV Cache:
\begin{itemize}
    \item 按最大序列长度预分配,造成\textbf{内存浪费}
    \item 不同请求长度不一,产生\textbf{碎片化}
    \item 无法动态扩展,限制\textbf{并发请求数}
\end{itemize}

\subsubsection{PagedAttention原理}

PagedAttention将KV Cache分成固定大小的\textbf{Page}(块):
\begin{itemize}
    \item 每个Page存储固定数量token的KV
    \item Page可以非连续存储(类似虚拟内存)
    \item 按需分配,用完即释放
\end{itemize}

\begin{figure}[htbp]
\centering
\begin{tikzpicture}[scale=0.7, font=\small]
    % Logical view
    \node at (1.5, 3) {\textbf{逻辑视图}};
    \draw[fill=blue!20] (0, 0) rectangle (3, 2.5);
    \foreach \i in {0,1,2,3,4} {
        \draw (0, \i*0.5) -- (3, \i*0.5);
        \node[left] at (0, \i*0.5+0.25) {\tiny Block \i};
    }

    % Arrow
    \draw[->, thick] (3.5, 1.25) -- (5, 1.25);
    \node[above] at (4.25, 1.25) {Page Table};

    % Physical memory
    \node at (7.5, 3) {\textbf{物理显存}};
    \draw (5.5, 0) rectangle (9.5, 2.5);
    \fill[blue!40] (5.5, 2) rectangle (6.5, 2.5);
    \fill[blue!20] (7, 1.5) rectangle (8, 2);
    \fill[blue!60] (8.5, 0.5) rectangle (9.5, 1);
    \fill[blue!30] (5.5, 0) rectangle (6.5, 0.5);
    \fill[blue!50] (7.5, 1) rectangle (8.5, 1.5);
    \node at (6, 2.25) {\tiny 0};
    \node at (7.5, 1.75) {\tiny 1};
    \node at (9, 0.75) {\tiny 2};
    \node at (6, 0.25) {\tiny 3};
    \node at (8, 1.25) {\tiny 4};
\end{tikzpicture}
\caption{PagedAttention:非连续KV Cache存储}
\label{fig:paged_attention}
\end{figure}

\subsubsection{vLLM性能}

vLLM结合PagedAttention和Continuous Batching:
\begin{itemize}
    \item 相比HuggingFace Transformers,吞吐量提升\textbf{24倍}
    \item 显存利用率接近100\%(无碎片)
    \item 支持更大的并发批处理
\end{itemize}

\subsection{Prefix Caching与SGLang}

\subsubsection{Prefix Caching动机}

许多应用场景存在\textbf{共享前缀}:
\begin{itemize}
    \item 多轮对话:System Prompt相同
    \item Few-shot学习:示例相同
    \item 批量处理:相同的指令模板
\end{itemize}

重复计算相同前缀的KV Cache造成浪费。

\subsubsection{RadixAttention}

SGLang~\citep{zheng2024sglang}提出RadixAttention,使用\textbf{基数树}(Radix Tree)管理KV Cache:
\begin{itemize}
    \item 树的每条边对应一段token序列
    \item 共享前缀的请求共享KV Cache
    \item LRU策略管理缓存淘汰
\end{itemize}

\textbf{优势}:
\begin{itemize}
    \item 自动检测和复用共享前缀
    \item 相比vLLM,吞吐量提升可达\textbf{5-6倍}
    \item 支持复杂的LLM程序(多次调用、分支)
\end{itemize}

\subsubsection{SGLang特性}

SGLang是目前最快的开源推理引擎之一:
\begin{itemize}
    \item \textbf{RadixAttention}:自动Prefix Caching
    \item \textbf{结构化输出}:压缩有限状态机加速JSON生成
    \item \textbf{投机解码}:支持EAGLE等方法
    \item \textbf{多模态}:支持Vision-Language模型
\end{itemize}

\subsection{投机解码}

投机解码(Speculative Decoding)~\citep{leviathan2023fast,chen2023accelerating}是加速自回归生成的重要技术,其核心思想是``先用小模型快速猜测,再用大模型批量验收''。

\subsubsection{核心思想与工作流程}

自回归生成的瓶颈在于每步只生成1个token,且由于大模型是Memory-bound的,输入1个token和输入$K$个token的前向时间几乎相同——计算不是瓶颈,读取权重才是。投机解码正是利用了这一特性。

\paragraph{工作流程}
\begin{enumerate}
    \item \textbf{Draft阶段}:用小模型(Draft Model)自回归生成$K$个候选token
    \item \textbf{Verify阶段}:将这$K$个token\textbf{并行}输入大模型(Target Model),一次前向得到每个位置的next token预测
    \item \textbf{Accept/Reject}:从头开始逐个比对,连续一致的token直接接受;一旦不一致,在分歧点由大模型给出正确token,后续draft token全部丢弃
    \item 重复上述过程直到生成结束
\end{enumerate}

\textbf{关键保证}:即使draft全部猜错,也能从大模型获得至少1个正确token,因此投机解码的下限就是普通自回归解码。

\subsubsection{采样场景下的验证机制}

上述流程在贪婪解码(temperature=0)下很直观:直接比较token是否相等。但在采样场景下,大模型本身的输出也是随机的,如何定义``猜对''?

设Draft模型在位置$t$生成token $x$的分布为$q(x)$,Target模型的分布为$p(x)$。投机解码需要保证\textbf{最终输出的分布严格等于$p(x)$}(无偏性),而不仅仅是加速。

\paragraph{接受概率}
对于draft生成的token $x$,以如下概率接受:
\begin{equation}
    a(x) = \min\left(1, \frac{p(x)}{q(x)}\right)
\end{equation}

直觉理解:
\begin{itemize}
    \item 若$q(x) \leq p(x)$:大模型比小模型更认可这个token,必然接受
    \item 若$q(x) > p(x)$:小模型``过于自信'',以概率$p(x)/q(x)$接受
\end{itemize}

\paragraph{拒绝后的重采样:为什么需要残差分布?}

一个自然的问题:拒绝后为什么不直接从$p(x)$采样?因为接受分支已经``预支''了部分概率质量。

\textbf{概率账本分析}:被接受分支输出token $x$的概率为:
\begin{equation}
    \Pr[\text{accept and output } x] = q(x) \cdot a(x) = \min(p(x), q(x))
\end{equation}

要使最终分布等于$p(x)$,拒绝分支必须补齐剩余的概率质量:
\begin{equation}
    p(x) - \min(p(x), q(x)) = \max(0, p(x) - q(x))
\end{equation}

因此,拒绝时从\textbf{残差分布}重采样:
\begin{equation}
    p'(x) = \frac{\max(0, p(x) - q(x))}{\sum_{x'} \max(0, p(x') - q(x'))}
\end{equation}

\paragraph{无偏性的严格证明}

记总接受概率为$\beta = \sum_x \min(p(x), q(x))$,则拒绝概率为$1-\beta$。对任意token $x$,其最终输出概率为:
\begin{align}
    \Pr[\text{output } x] &= \underbrace{\min(p(x), q(x))}_{\text{接受分支}} + \underbrace{(1-\beta) \cdot \frac{\max(0, p(x)-q(x))}{1-\beta}}_{\text{拒绝分支}} \\
    &= \min(p(x), q(x)) + \max(0, p(x)-q(x)) \\
    &= p(x) \quad \checkmark
\end{align}
最后一步利用了恒等式$\min(a,b) + \max(0, a-b) = a$。这证明了投机解码在采样场景下是\textbf{无损}的——输出分布严格等于只用Target模型逐token采样。

\paragraph{替代方案:全局上界Rejection Sampling}

经典rejection sampling的另一种做法是:先计算全局上界$M = \max_x \frac{p(x)}{q(x)}$,然后以概率$\frac{p(x)}{M \cdot q(x)}$接受。在这种方案下:
\begin{itemize}
    \item 接受分支得到的是$p/M$(缩放版的目标分布)
    \item 拒绝后可以直接从$p$重采样(因为没有``预支''问题)
    \item 但接受率为$1/M$,通常远低于残差方案的$\sum_x \min(p(x), q(x))$
\end{itemize}
残差采样之所以成为标准做法,正是因为它在保证无偏性的同时,最大化了接受率。

\begin{remark}[两token例子]
设词表$\{A, B\}$,$p(A)=0.9, p(B)=0.1$,$q(A)=0.5, q(B)=0.5$。
\begin{itemize}
    \item A的接受概率:$\min(1, 0.9/0.5)=1$,接受分支贡献$0.5$
    \item B的接受概率:$\min(1, 0.1/0.5)=0.2$,接受分支贡献$0.5 \times 0.2=0.1$
    \item 拒绝概率:$1 - 0.5 - 0.1 = 0.4$
    \item 残差分布:A剩余$0.9-0.5=0.4$,B剩余$0.1-0.1=0$,归一化后$p'(A)=1$
    \item 最终:$A = 0.5 + 0.4 \times 1 = 0.9$,$B = 0.1 + 0.4 \times 0 = 0.1$ \checkmark
\end{itemize}
若拒绝后直接从$p$采样,B会变成$0.1 + 0.4 \times 0.1 = 0.14$,分布已偏离!
\end{remark}

\paragraph{贪婪解码作为特例}

当temperature=0时,$p$和$q$退化为在argmax上概率为1的delta分布。此时验证规则简化为:
\begin{itemize}
    \item 若draft token等于Target的argmax $\Rightarrow$ 接受($p(x)=q(x)=1$)
    \item 否则 $\Rightarrow$ 拒绝并由Target给出argmax
\end{itemize}
这解释了为什么在贪婪解码下,``猜对''直观上就是``token相等''——它是采样验证机制的退化情况。

\subsubsection{加速比分析}

设Draft模型的平均接受率为$\alpha$,每次投机$K$个token:
\begin{equation}
    \text{平均接受长度} = \frac{1 - \alpha^{K+1}}{1 - \alpha}
\end{equation}

\textbf{加速比}取决于:
\begin{itemize}
    \item 接受率$\alpha$:Draft与Target越接近越好
    \item 投机长度$K$:过大会降低接受率
    \item Draft模型速度:需要比Target快很多(通常10-50倍小)
    \item 场景特点:batch较小、交互式低延迟场景收益更明显
\end{itemize}

\subsubsection{EAGLE系列:从Chain到Tree}

EAGLE~\citep{li2024eagle}是目前SOTA的投机解码方法,其核心创新是利用Target模型的\textbf{隐状态}来指导Draft生成。

\paragraph{EAGLE-1:Chain模式}

与普通Draft模型的区别在于输入:
\begin{equation}
    \text{Draft Input} = \text{Embedding}(x_t) + \text{Project}(h_t^{\text{target}})
\end{equation}
其中$h_t^{\text{target}}$是Target模型的隐状态。这使得Draft模型能够``看到''Target模型的中间表示,显著提高猜测准确率。

\textbf{Stable KV}:验证后,清除Draft模型中由自身隐状态生成的KV Cache,只保留由Target隐状态生成的部分,确保下一轮的输入分布与训练时一致。

\paragraph{EAGLE-2:Tree模式}

Chain模式每次只生成一条候选序列。Tree模式的改进是:每步生成top-$k$个候选token,形成树状结构,增加命中概率。

\begin{itemize}
    \item \textbf{Expand阶段}:每层选top-$k$,动态维护累计得分最高的路径
    \item \textbf{Select阶段}:对所有叶节点按累计得分排序,选top-$K$送入Target验证
    \item \textbf{Tree Attention}:需要特殊的attention mask,确保每个token只关注其祖先节点
\end{itemize}

\paragraph{EAGLE-3:统一训练与推理}

EAGLE-1/2的一个问题是:训练时Draft模型看到的全是Target隐状态(stable KV),但推理时大部分隐状态来自Draft自身(unstable KV)。这种train-test mismatch导致:
\begin{itemize}
    \item 第一个draft token命中率随数据量提升
    \item 第二个及之后的token命中率急剧下降(累积误差)
\end{itemize}

EAGLE-3的解决方案:
\begin{enumerate}
    \item 训练时加入unstable KV情况,统一train与test
    \item 去除隐状态监督损失,只保留next token prediction
    \item 使用多层Target隐状态(low/mid/high features)
\end{enumerate}

这些改进使EAGLE-3获得了scaling性质——更多训练数据带来持续的性能提升。

\begin{table}[htbp]
\centering
\caption{投机解码方法对比}
\label{tab:speculative_methods}
\begin{tabular}{lcccc}
\toprule
方法 & 额外模型 & 训练需求 & 加速比 & 特点 \\
\midrule
独立Draft & 是 & 无 & 2-3$\times$ & 简单通用 \\
Self-Speculative & 否 & 无 & 1.5-2$\times$ & 显存友好 \\
Medusa & 否 & 训练Head & 2-3$\times$ & 并行预测 \\
EAGLE-1 & 否 & 训练Head & 2.5-3$\times$ & Chain + 隐状态 \\
EAGLE-2 & 否 & 训练Head & 3-4$\times$ & Tree扩展 \\
EAGLE-3 & 否 & 训练Head & 4-5$\times$ & 统一train/test \\
\bottomrule
\end{tabular}
\end{table}

\subsubsection{工业应用现状}

投机解码已成为主流推理引擎的标配优化:
\begin{itemize}
    \item \textbf{Google}:官方博客明确提到speculative decoding用于加速生成,``used across Google products''
    \item \textbf{NVIDIA TensorRT-LLM}:完整支持speculative sampling,特定配置下可达3.6$\times$吞吐提升
    \item \textbf{vLLM/SGLang}:支持draft model和n-gram等多种proposal方式
    \item \textbf{OpenAI}:API提供``Predicted Outputs''功能,用户提供预测文本,模型统计accepted/rejected tokens——思路与投机解码一致,只是proposal来自用户而非draft model
\end{itemize}

\begin{remark}[适用场景]
投机解码在以下场景收益最明显:
\begin{itemize}
    \item 交互式低延迟(QPS中等、batch小):大模型逐token生成时GPU利用率低
    \item Draft与Target分布接近:验收率高才有稳定收益
    \item 生成内容可预测性强:代码补全、格式化输出等
\end{itemize}
对于大batch高吞吐场景,GPU利用率本已较高,投机解码的收益会减弱。
\end{remark}

\subsection{KV Cache压缩}

\subsubsection{KV Cache量化}

将KV Cache从FP16/BF16量化到低精度:

\paragraph{INT8/FP8量化}
vLLM、TensorRT-LLM支持的标准方法:
\begin{itemize}
    \item 显存减少50\%
    \item 精度损失很小
    \item 硬件原生支持
\end{itemize}

\paragraph{更激进的量化}
KVQuant~\citep{hooper2024kvquant}实现3-4bit量化:
\begin{itemize}
    \item Per-channel量化 + 离群值处理
    \item 支持百万级上下文(单卡A100)
    \item 1\%离群值保留可保持精度
\end{itemize}

KIVI~\citep{liu2024kivi}实现2bit量化:
\begin{itemize}
    \item Key用2bit,Value用2bit
    \item 无需训练,即插即用
    \item 显存减少约4倍
\end{itemize}

\subsubsection{KV Cache稀疏化}

并非所有KV对都同等重要,可以选择性保留。

\paragraph{H2O (Heavy-Hitter Oracle)}
动态识别重要token~\citep{zhang2024h2o}:
\begin{itemize}
    \item 基于注意力分数累积判断重要性
    \item 保留"Heavy Hitter"token
    \item 吞吐量提升40\%+
\end{itemize}

\paragraph{SnapKV}
基于观察窗口选择重要KV~\citep{li2024snapkv}:
\begin{itemize}
    \item 每个Head只关注部分位置
    \item 聚类保留关键KV
    \item 16K输入:3.6倍加速,8.2倍显存节省
\end{itemize}

\subsection{推理引擎对比}

\begin{table}[htbp]
\centering
\caption{主流LLM推理引擎对比}
\label{tab:inference_engines}
\begin{tabular}{lcccc}
\toprule
引擎 & 核心技术 & Prefix Cache & 投机解码 & 特点 \\
\midrule
vLLM & PagedAttention & 支持 & 支持 & 最广泛使用 \\
SGLang & RadixAttention & 原生 & EAGLE等 & 结构化输出快 \\
TensorRT-LLM & 深度优化 & 支持 & 多种 & NVIDIA官方 \\
llama.cpp & CPU优化 & 有限 & 支持 & 本地部署 \\
\bottomrule
\end{tabular}
\end{table}

\subsubsection{vLLM}

vLLM是目前最流行的开源推理引擎:
\begin{itemize}
    \item \textbf{PagedAttention}:高效KV Cache管理
    \item \textbf{Continuous Batching}:动态批处理
    \item \textbf{张量并行}:多GPU推理
    \item \textbf{量化}:FP8、AWQ、GPTQ
    \item \textbf{生态}:支持HuggingFace模型、OpenAI兼容API
\end{itemize}

\subsubsection{SGLang}

SGLang专注于复杂LLM应用:
\begin{itemize}
    \item \textbf{RadixAttention}:自动Prefix Caching
    \item \textbf{Frontend语言}:简化多次调用的编程
    \item \textbf{结构化输出}:JSON Schema约束生成
    \item \textbf{性能}:在某些场景比vLLM快5-6倍
\end{itemize}

\subsubsection{TensorRT-LLM}

NVIDIA官方推理库:
\begin{itemize}
    \item 深度优化的CUDA Kernel
    \item 支持Hopper特性(FP8、TMA)
    \item In-flight Batching
    \item 与Triton Inference Server集成
\end{itemize}

\subsection{推理优化最佳实践}

\paragraph{延迟优先场景}
\begin{itemize}
    \item 使用Prefix Caching(SGLang)
    \item 投机解码(EAGLE)
    \item 小批处理 + 高并行度
\end{itemize}

\paragraph{吞吐优先场景}
\begin{itemize}
    \item Continuous Batching(vLLM)
    \item 大批处理
    \item KV Cache量化
\end{itemize}

\paragraph{长上下文场景}
\begin{itemize}
    \item KV Cache量化(KVQuant、KIVI)
    \item KV Cache稀疏化(SnapKV、H2O)
    \item Prefill-Decode分离
\end{itemize}

\begin{remark}[推理优化的权衡]
没有万能方案,需要根据场景选择:
\begin{itemize}
    \item \textbf{交互式应用}:优先TTFT和ITL
    \item \textbf{批处理任务}:优先吞吐量
    \item \textbf{边缘部署}:优先显存效率
\end{itemize}
\end{remark}

