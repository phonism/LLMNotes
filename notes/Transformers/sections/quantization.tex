\section{模型量化}
\label{sec:quantization}

模型量化是将神经网络中的浮点数表示转换为低精度表示的技术,是大语言模型高效部署的核心技术之一。本节从基础理论到LLM专用方法,系统介绍量化技术。

\subsection{量化基础}

\subsubsection{为什么需要量化}

现代LLM的参数规模带来严峻的部署挑战:
\begin{itemize}
    \item \textbf{内存需求}:70B参数模型以FP16存储需要140GB显存
    \item \textbf{带宽瓶颈}:推理时主要受限于内存带宽而非计算
    \item \textbf{能耗成本}:数据移动消耗的能量远超计算本身
\end{itemize}

量化通过降低数值精度来解决这些问题:
\begin{table}[htbp]
\centering
\caption{不同精度的存储与计算特性}
\label{tab:precision_comparison}
\begin{tabular}{lccc}
\toprule
精度 & 位宽 & 相对内存 & 典型用途 \\
\midrule
FP32 & 32 & 1$\times$ & 训练梯度累积 \\
FP16/BF16 & 16 & 0.5$\times$ & 标准训练与推理 \\
FP8 & 8 & 0.25$\times$ & 高效训练(Hopper+) \\
INT8 & 8 & 0.25$\times$ & 量化推理 \\
INT4 & 4 & 0.125$\times$ & 激进量化推理 \\
\bottomrule
\end{tabular}
\end{table}

\subsubsection{量化的数学定义}

\textbf{均匀量化}(Uniform Quantization)是最常用的量化方式。给定浮点数$x$,量化过程为:
\begin{equation}
    Q(x) = \text{clamp}\left( \left\lfloor \frac{x}{s} \right\rceil + z, 0, 2^b - 1 \right)
\end{equation}
其中$s$是缩放因子(scale),$z$是零点(zero-point),$b$是目标位宽,$\lfloor \cdot \rceil$表示四舍五入。

\textbf{反量化}(Dequantization)恢复近似值:
\begin{equation}
    \hat{x} = s \cdot (Q(x) - z)
\end{equation}

\paragraph{对称量化 vs 非对称量化}

\textbf{对称量化}:$z = 0$,量化范围关于零点对称:
\begin{equation}
    s = \frac{\max(|x|)}{2^{b-1} - 1}
\end{equation}

\textbf{非对称量化}:允许$z \neq 0$,更好地适应非对称分布:
\begin{align}
    s &= \frac{\max(x) - \min(x)}{2^b - 1} \\
    z &= \left\lfloor -\frac{\min(x)}{s} \right\rceil
\end{align}

对称量化实现更简单(无需存储$z$),但非对称量化对偏斜分布更有效。

\subsubsection{量化粒度}

量化参数$s, z$的计算粒度影响精度与开销的权衡:

\begin{itemize}
    \item \textbf{Per-Tensor}:整个张量共享一组$(s, z)$,开销最小但精度损失大
    \item \textbf{Per-Channel}:每个输出通道独立量化,常用于权重
    \item \textbf{Per-Token}:每个token独立量化,常用于激活值
    \item \textbf{Per-Group}:将通道分组,每组独立量化,精度与开销的折中
\end{itemize}

\begin{example}[Group Quantization]
对于权重矩阵$W \in \R^{m \times n}$,设group size为$g$,则需要存储$\lceil n/g \rceil$组量化参数。常见设置$g = 128$,在INT4量化时有效位宽约为$4 + 32/128 = 4.25$位。
\end{example}

\subsection{训练后量化(PTQ)}

训练后量化(Post-Training Quantization)在模型训练完成后进行,无需重新训练,是LLM量化的主流方法。

\subsubsection{基本PTQ流程}

\begin{enumerate}
    \item \textbf{校准}(Calibration):使用少量代表性数据(通常几百到几千样本)统计激活值分布
    \item \textbf{确定量化参数}:根据统计信息计算$s, z$
    \item \textbf{量化权重}:将浮点权重转换为低精度表示
    \item \textbf{(可选)校正}:通过额外优化减少量化误差
\end{enumerate}

\subsubsection{校准策略}

\paragraph{MinMax校准}
最简单的方法,使用观测到的最大最小值:
\begin{equation}
    s = \frac{\max(x) - \min(x)}{2^b - 1}
\end{equation}
问题:对离群值敏感。

\paragraph{百分位校准}
使用第$p$和$100-p$百分位数代替最大最小值,$p$通常取0.1-1\%。

\paragraph{MSE校准}
最小化量化误差:
\begin{equation}
    s^* = \arg\min_s \mathbb{E}\left[ (x - \hat{x})^2 \right]
\end{equation}

\paragraph{KL散度校准}
最小化原始分布与量化分布的KL散度,由TensorRT采用。

\subsubsection{权重量化 vs 激活量化}

\textbf{权重量化}(W-only):
\begin{itemize}
    \item 权重在推理前确定,可以离线计算量化参数
    \item 分布通常接近高斯,易于量化
    \item 常见配置:W8A16、W4A16
\end{itemize}

\textbf{激活量化}:
\begin{itemize}
    \item 激活值依赖输入,需要动态量化或校准
    \item LLM中存在大量\textbf{离群值}(outliers),极大增加量化难度
    \item 常见配置:W8A8、W4A8
\end{itemize}

\begin{figure}[htbp]
\centering
\begin{tikzpicture}[scale=0.9]
    \draw[->] (0, 0) -- (8, 0) node[right] {值};
    \draw[->] (0, 0) -- (0, 3) node[above] {频率};
    
    \draw[blue, thick, domain=1:6, smooth] plot (\x, {2.5*exp(-0.5*(\x-3.5)^2/0.8)});
    \node[blue] at (3.5, 3) {权重分布};
    
    \draw[red, thick] (0.3, 0.2) -- (0.5, 0.3) -- (1, 0.5) -- (2, 1.5) -- (3, 2) -- (4, 1.8) -- (5, 0.8) -- (6, 0.3) -- (7, 0.1) -- (7.5, 0.08);
    \draw[red, thick, dashed] (7.5, 0.08) -- (7.8, 0.15);
    \node[red] at (6.5, 1.5) {激活分布};
    \node[red] at (7.8, 0.5) {离群值};
\end{tikzpicture}
\caption{权重与激活值的典型分布对比}
\label{fig:weight_activation_dist}
\end{figure}

\subsection{量化感知训练(QAT)}

量化感知训练(Quantization-Aware Training)在训练过程中模拟量化效果,让模型学会适应低精度表示。

\subsubsection{直通估计器(STE)}

量化操作$Q(x)$是阶梯函数,梯度几乎处处为零。\textbf{直通估计器}(Straight-Through Estimator)绕过这一问题:
\begin{equation}
    \frac{\partial \mathcal{L}}{\partial x} = \frac{\partial \mathcal{L}}{\partial Q(x)}
\end{equation}
即前向传播使用量化值,反向传播直接传递梯度。

\subsubsection{伪量化}

QAT的核心是在前向传播中插入\textbf{伪量化}(Fake Quantization)操作:
\begin{equation}
    \hat{x} = s \cdot \left( \text{clamp}\left( \left\lfloor \frac{x}{s} \right\rceil, q_{min}, q_{max} \right) \right)
\end{equation}
伪量化保持浮点计算,但模拟量化的离散化效果。

\subsubsection{QAT vs PTQ}

\begin{table}[htbp]
\centering
\caption{QAT与PTQ对比}
\label{tab:qat_vs_ptq}
\begin{tabular}{lcc}
\toprule
特性 & PTQ & QAT \\
\midrule
计算成本 & 低(分钟级) & 高(需重新训练) \\
数据需求 & 少量校准数据 & 完整训练数据 \\
精度损失 & 较大(尤其低位宽) & 较小 \\
适用场景 & 快速部署、8位量化 & 极低位宽、精度敏感 \\
\bottomrule
\end{tabular}
\end{table}

对于LLM,QAT的计算成本通常不可接受,因此PTQ是主流选择。

\subsection{LLM量化的挑战与方法}

\subsubsection{激活值离群值问题}

LLM的一个关键特性是激活值中存在\textbf{离群值}(outliers):极少数通道包含数值远大于其他通道的激活值。这些离群值:
\begin{itemize}
    \item 出现在特定通道,跨token一致
    \item 数值可达正常值的100倍以上
    \item 移除这些通道会导致模型性能崩溃
\end{itemize}

标准量化方法被迫扩大量化范围以覆盖离群值,导致正常值的量化精度严重下降。

\subsubsection{SmoothQuant}

SmoothQuant~\citep{xiao2023smoothquant}是解决激活离群值问题的突破性方法,实现了LLM的W8A8量化。

\paragraph{核心思想}
将量化难度从激活``迁移''到权重。观察到离群值出现在固定通道,可以通过per-channel缩放来平滑激活:
\begin{equation}
    Y = (X \cdot \text{diag}(s)^{-1}) \cdot (\text{diag}(s) \cdot W) = \hat{X} \hat{W}
\end{equation}
其中$s$是迁移因子。$\hat{X}$的分布更均匀,更易量化;$\hat{W}$吸收了部分难度,但权重本身分布良好,影响有限。

\paragraph{迁移因子选择}
\begin{equation}
    s_j = \frac{\max(|X_j|)^\alpha}{\max(|W_j|)^{1-\alpha}}
\end{equation}
其中$\alpha \in [0, 1]$控制迁移程度。$\alpha = 0.5$对大多数模型效果良好。

\paragraph{性能}
SmoothQuant在OPT-175B上实现了INT8量化,精度损失小于0.5\%,推理速度提升1.5倍。

\subsubsection{GPTQ}

GPTQ~\citep{frantar2022gptq}是基于二阶信息的权重量化方法,可将LLM压缩至4位精度。

\paragraph{问题形式化}
逐层优化,最小化量化后的输出误差:
\begin{equation}
    \arg\min_{\hat{W}} \| WX - \hat{W}X \|_2^2
\end{equation}

\paragraph{基于Hessian的优化}
利用Optimal Brain Quantization (OBQ)的思想,但做了关键优化使其适用于LLM规模:
\begin{enumerate}
    \item 以固定顺序(而非贪心顺序)量化权重列
    \item 批量处理多列,使用高效的矩阵运算
    \item 利用Cholesky分解高效更新Hessian逆
\end{enumerate}

\paragraph{性能}
\begin{itemize}
    \item 175B参数模型可在单GPU上4小时内完成量化
    \item 3-4位量化精度损失极小(困惑度增加$<0.5$)
    \item 广泛应用于开源LLM的量化分发
\end{itemize}

\subsubsection{AWQ}

AWQ(Activation-aware Weight Quantization)~\citep{lin2024awq}基于一个关键观察:不同权重通道的重要性差异巨大。

\paragraph{核心观察}
对于矩阵乘法$Y = XW$,如果激活值$X$的某些通道数值较大,则对应权重通道的量化误差影响更大。

\paragraph{方法}
不均匀地保护重要权重通道:
\begin{equation}
    s_j = \sqrt{\frac{\max(|X_j|)}{1 + \epsilon}}
\end{equation}
通过per-channel缩放$W_j' = W_j \cdot s_j$放大重要通道,量化后再除以$s_j$恢复。

\paragraph{性能}
在4位量化下,AWQ的困惑度通常优于GPTQ,且支持更高效的实时推理。

\subsubsection{GGUF/GGML}

GGUF(GPT-Generated Unified Format)是llama.cpp项目定义的模型存储格式,广泛用于本地LLM部署。

\paragraph{特点}
\begin{itemize}
    \item 支持多种量化格式:Q4\_0, Q4\_K\_M, Q5\_K\_M, Q8\_0等
    \item ``K''表示使用k-quant方法,对重要层使用更高精度
    \item 针对CPU推理优化,支持Apple Metal、CUDA等后端
\end{itemize}

\paragraph{常用配置}
\begin{table}[htbp]
\centering
\caption{GGUF量化格式对比}
\label{tab:gguf_formats}
\begin{tabular}{lcc}
\toprule
格式 & 有效位宽 & 精度损失 \\
\midrule
Q8\_0 & 8.5 bits & 极小 \\
Q5\_K\_M & 5.5 bits & 小 \\
Q4\_K\_M & 4.8 bits & 中等 \\
Q4\_0 & 4.5 bits & 较大 \\
Q2\_K & 3.4 bits & 大 \\
\bottomrule
\end{tabular}
\end{table}

\subsection{FP8量化}

FP8是一种8位浮点格式,相比INT8保留了动态范围,成为现代GPU训练和推理的重要选择。

\subsubsection{FP8格式}

两种主要格式:
\begin{itemize}
    \item \textbf{E4M3}:4位指数、3位尾数,动态范围更大,适合前向传播
    \item \textbf{E5M2}:5位指数、2位尾数,精度更高,适合梯度
\end{itemize}

\begin{table}[htbp]
\centering
\caption{FP8格式对比}
\label{tab:fp8_formats}
\begin{tabular}{lccc}
\toprule
格式 & 动态范围 & 最大值 & 最小正数 \\
\midrule
E4M3 & $2^{15}$ & 448 & $2^{-9}$ \\
E5M2 & $2^{31}$ & 57344 & $2^{-16}$ \\
FP16 & $2^{31}$ & 65504 & $2^{-24}$ \\
\bottomrule
\end{tabular}
\end{table}

\subsubsection{FP8训练}

NVIDIA Hopper和Blackwell架构原生支持FP8 Tensor Core。DeepSeek-V3~\citep{deepseek2024v3}展示了FP8训练的工业应用:

\paragraph{混合精度策略}
\begin{itemize}
    \item 权重和激活以FP8存储和计算
    \item 累加器使用FP32保持精度
    \item 主权重副本保持FP32/BF16用于优化器更新
\end{itemize}

\paragraph{Block-wise量化}
将张量分块独立量化,每块$128 \times 128$元素共享一个缩放因子,平衡精度与开销。

\paragraph{性能收益}
\begin{itemize}
    \item 内存带宽需求减半
    \item H100上FP8吞吐量是FP16的2倍
    \item DeepSeek-V3训练仅需2.788M H800 GPU小时
\end{itemize}

\subsubsection{FP8推理}

FlashAttention-3支持FP8注意力计算:
\begin{itemize}
    \item 使用Block Quantization减少量化误差
    \item Incoherent Processing(基于Hadamard变换)进一步提升精度
    \item 接近1.2 PFLOPS的吞吐量
\end{itemize}

\subsection{极低位宽量化}

\subsubsection{INT4/W4A16}

4位权重量化是目前实用的最低位宽:
\begin{itemize}
    \item 模型大小减少4倍
    \item 推理速度提升2-4倍(受限于内存带宽)
    \item 精度损失可控(困惑度增加通常$<1$)
\end{itemize}

\paragraph{关键技术}
\begin{itemize}
    \item Group quantization(组大小128)减少误差
    \item 对敏感层(如第一层、最后一层)使用更高精度
    \item GPTQ/AWQ的二阶优化
\end{itemize}

\subsubsection{BitNet与1.58-bit量化}

BitNet 1.58~\citep{ma2024era}提出了极端量化:权重仅取$\{-1, 0, +1\}$三个值。

\paragraph{量化函数}
\begin{equation}
    W_q = \text{RoundClip}\left( \frac{W}{\gamma}, -1, 1 \right)
\end{equation}
其中$\gamma = \frac{1}{nm}\sum_{ij}|W_{ij}|$是平均绝对值。

\paragraph{计算优势}
矩阵乘法退化为加减运算,无需乘法器:
\begin{equation}
    Y = X \cdot W_q = \sum_j X_j \cdot W_{qj} = \sum_{j: W_{qj}=1} X_j - \sum_{j: W_{qj}=-1} X_j
\end{equation}

\paragraph{当前状态}
\begin{itemize}
    \item 需要从头训练,不适用于已有模型
    \item 在同等模型容量下,精度接近FP16模型
    \item 能效提升71倍(BitNet b1.58 2B4T模型)
    \item 尚未在超大规模模型上验证
\end{itemize}

\subsection{KV Cache量化}

长上下文推理的内存瓶颈主要来自KV Cache而非模型权重。KV Cache量化成为长上下文场景的关键技术。

\subsubsection{KV Cache的内存需求}

对于序列长度$L$、batch大小$B$、层数$N_L$、注意力头数$N_H$、头维度$d$:
\begin{equation}
    \text{KV Cache大小} = 2 \times B \times L \times N_L \times N_H \times d \times \text{sizeof(dtype)}
\end{equation}

以LLaMA-70B(80层、64头、128维)为例:
\begin{itemize}
    \item 100K上下文、batch=1、FP16:约40GB
    \item 同等设置使用INT4:约10GB
\end{itemize}

\subsubsection{KIVI方法}

KIVI~\citep{liu2024kivi}是KV Cache量化的代表性工作:

\paragraph{非对称量化}
Key和Value有不同的分布特性,分别使用per-channel和per-token量化:
\begin{itemize}
    \item Key:per-channel量化(通道间差异大)
    \item Value:per-token量化(token间差异大)
\end{itemize}

\paragraph{2-bit量化}
KIVI可将KV Cache压缩至2-bit,内存减少8倍,困惑度增加极小($<0.1$)。

\subsubsection{实践建议}

\begin{itemize}
    \item 中等上下文($<$32K):4-bit KV Cache足够
    \item 长上下文(32K-128K):考虑2-bit量化
    \item 超长上下文($>$128K):结合KV Cache压缩(如H2O、StreamingLLM)
\end{itemize}

\subsection{量化的系统支持}

\subsubsection{硬件支持}

不同硬件对量化精度的支持差异显著:

\begin{table}[htbp]
\centering
\caption{GPU架构的量化支持}
\label{tab:hw_quant_support}
\begin{tabular}{lccc}
\toprule
架构 & INT8 & FP8 & INT4 \\
\midrule
Ampere (A100) & Tensor Core & 无 & 软件模拟 \\
Hopper (H100) & Tensor Core & E4M3/E5M2 & 软件模拟 \\
Blackwell (B200) & Tensor Core & 增强 & FP4原生 \\
\bottomrule
\end{tabular}
\end{table}

\subsubsection{推理框架}

主流推理框架的量化支持:
\begin{itemize}
    \item \textbf{vLLM}:支持GPTQ、AWQ、FP8、SqueezeLLM
    \item \textbf{TensorRT-LLM}:INT8/INT4 weight-only、FP8、SmoothQuant
    \item \textbf{llama.cpp}:GGUF格式,支持Q2-Q8多种精度
    \item \textbf{ExLlama}:专注于4-bit量化的高效实现
\end{itemize}

\subsection{量化最佳实践}

\paragraph{精度选择}
\begin{itemize}
    \item 内存充足:FP16/BF16,无精度损失
    \item 一般部署:INT8或FP8,精度损失极小
    \item 边缘设备:INT4(GPTQ/AWQ),精度损失可接受
    \item 极限压缩:2-3 bit,需仔细评估任务影响
\end{itemize}

\paragraph{量化方法选择}
\begin{itemize}
    \item 快速部署:直接使用预量化模型(HuggingFace上大量可用)
    \item 自定义量化:AWQ通常是最佳起点
    \item 极致压缩:GPTQ可达到更低位宽
    \item 激活量化需求:SmoothQuant处理离群值
\end{itemize}

\paragraph{评估指标}
\begin{itemize}
    \item \textbf{困惑度}:最基础的质量指标
    \item \textbf{下游任务}:MMLU、GSM8K等benchmark
    \item \textbf{推理速度}:tokens/s
    \item \textbf{内存占用}:峰值显存需求
\end{itemize}

\begin{remark}[量化的trade-off]
量化不是免费的午餐。在选择量化策略时需权衡:
\begin{itemize}
    \item 精度 vs 效率:位宽越低,压缩率越高,但精度损失也越大
    \item 量化时间 vs 推理性能:复杂量化方法(如GPTQ)需要更长校准时间
    \item 通用性 vs 专用性:针对特定任务微调的模型可能需要重新量化
\end{itemize}
当前4-bit量化(GPTQ/AWQ)是精度与效率的良好平衡点,8-bit量化(INT8/FP8)则几乎无精度损失。
\end{remark}

