\section{评测与Benchmark}
\label{sec:evaluation}

大语言模型的评测是一个复杂且快速演进的领域。本节系统介绍主流评测基准,重点关注2024年以来顶级模型(DeepSeek-V3、LLaMA 3、Qwen 2.5、GPT-4o、Claude 3.5等)普遍采用的benchmark。

\subsection{评测体系概述}

\subsubsection{为什么需要多维度评测}

单一benchmark无法全面反映模型能力:
\begin{itemize}
    \item \textbf{能力多样性}:知识、推理、代码、指令遵循等维度独立
    \item \textbf{数据污染}:训练数据可能包含测试集,导致虚高
    \item \textbf{评测饱和}:旧benchmark被刷榜,区分度下降
\end{itemize}

\subsubsection{现代评测框架}

主流模型发布时通常报告以下类别的benchmark:

\begin{table}[htbp]
\centering
\caption{评测维度与代表性Benchmark}
\label{tab:eval_dimensions}
\begin{tabular}{lll}
\toprule
维度 & 核心Benchmark & 备注 \\
\midrule
知识与理解 & MMLU, MMLU-Pro, C-Eval & 多学科知识 \\
推理能力 & GPQA, ARC-C, BBH & 复杂推理 \\
数学能力 & GSM8K, MATH-500, AIME & 从小学到竞赛 \\
代码能力 & HumanEval, LiveCodeBench & 代码生成与执行 \\
指令遵循 & IFEval, MT-Bench & 指令理解与执行 \\
长上下文 & RULER, LongBench & 长文本处理 \\
多语言 & MGSM, C-Eval & 非英语能力 \\
安全对齐 & TruthfulQA, BBQ & 真实性与偏见 \\
\bottomrule
\end{tabular}
\end{table}

\subsection{知识与理解}

\subsubsection{MMLU (Massive Multitask Language Understanding)}

MMLU是最广泛使用的知识评测基准,覆盖57个学科:
\begin{itemize}
    \item \textbf{规模}:约14,000道四选一题目
    \item \textbf{学科}:STEM、人文、社科、其他
    \item \textbf{难度}:从高中到研究生水平
\end{itemize}

\paragraph{评测方式}
\begin{itemize}
    \item Zero-shot或Few-shot(通常5-shot)
    \item 计算模型对A/B/C/D选项的概率
    \item 报告整体准确率和各学科准确率
\end{itemize}

\paragraph{当前水平}
\begin{table}[htbp]
\centering
\caption{MMLU性能对比(5-shot)}
\label{tab:mmlu_scores}
\begin{tabular}{lcc}
\toprule
模型 & MMLU & 发布时间 \\
\midrule
GPT-4o & 88.7\% & 2024.05 \\
Claude 3.5 Sonnet & 88.7\% & 2024.06 \\
DeepSeek-V3 & 88.5\% & 2024.12 \\
Qwen2.5-72B & 86.1\% & 2024.09 \\
LLaMA 3.1-405B & 88.6\% & 2024.07 \\
LLaMA 3.1-70B & 86.0\% & 2024.07 \\
\bottomrule
\end{tabular}
\end{table}

\subsubsection{MMLU-Pro}

MMLU的升级版,解决原版的问题:
\begin{itemize}
    \item \textbf{更多选项}:从4选1变为10选1,降低猜测收益
    \item \textbf{更难题目}:过滤简单题,保留需要推理的题目
    \item \textbf{减少噪声}:修正原MMLU中的错误标注
\end{itemize}

\paragraph{区分度更强}
\begin{itemize}
    \item MMLU上GPT-4与Claude差距约1\%
    \item MMLU-Pro上差距扩大到5-10\%
    \item 更能反映真实能力差异
\end{itemize}

\subsubsection{GPQA (Graduate-Level Google-Proof QA)}

针对研究生水平的专业问题:
\begin{itemize}
    \item \textbf{来源}:物理、化学、生物领域的博士生出题
    \item \textbf{特点}:问题设计为``Google-proof'',搜索引擎难以直接找到答案
    \item \textbf{难度}:领域专家准确率约65\%,非专家约30\%
\end{itemize}

GPQA-Diamond是其中最难的子集,是区分顶级模型的关键benchmark。

\begin{table}[htbp]
\centering
\caption{GPQA-Diamond性能对比}
\label{tab:gpqa_scores}
\begin{tabular}{lc}
\toprule
模型 & GPQA-Diamond \\
\midrule
DeepSeek-R1 & 71.5\% \\
o1-preview & 73.3\% \\
DeepSeek-V3 & 59.1\% \\
Claude 3.5 Sonnet & 59.4\% \\
GPT-4o & 53.6\% \\
LLaMA 3.1-405B & 51.1\% \\
Qwen2.5-72B & 49.0\% \\
\bottomrule
\end{tabular}
\end{table}

\subsection{推理能力}

\subsubsection{BBH (BIG-Bench Hard)}

BIG-Bench中最具挑战性的23个任务:
\begin{itemize}
    \item \textbf{任务类型}:逻辑推理、因果判断、算法执行等
    \item \textbf{特点}:之前模型表现接近随机
    \item \textbf{评测}:通常使用Chain-of-Thought prompting
\end{itemize}

\subsubsection{ARC (AI2 Reasoning Challenge)}

科学推理问题:
\begin{itemize}
    \item \textbf{ARC-Easy}:简单科学问题
    \item \textbf{ARC-Challenge}:需要多步推理的难题
    \item \textbf{来源}:美国3-9年级科学考试
\end{itemize}

\subsubsection{HellaSwag}

常识推理与句子补全:
\begin{itemize}
    \item 给定场景描述,选择最合理的后续
    \item 测试模型的常识理解能力
    \item 当前顶级模型准确率$>$95\%,区分度下降
\end{itemize}

\subsection{数学能力}

\subsubsection{GSM8K}

小学数学应用题:
\begin{itemize}
    \item \textbf{规模}:8,500道题目
    \item \textbf{难度}:2-8步推理
    \item \textbf{特点}:需要理解题意并进行多步计算
\end{itemize}

当前顶级模型准确率$>$95\%,已接近饱和。

\subsubsection{MATH}

竞赛级数学问题:
\begin{itemize}
    \item \textbf{来源}:AMC、AIME等数学竞赛
    \item \textbf{难度分级}:Level 1-5,Level 5最难
    \item \textbf{领域}:代数、几何、数论、概率等
\end{itemize}

\paragraph{MATH-500}
从MATH数据集中精选的500道高难度题目,是当前主流评测标准。

\begin{table}[htbp]
\centering
\caption{数学Benchmark性能对比}
\label{tab:math_scores}
\begin{tabular}{lccc}
\toprule
模型 & GSM8K & MATH-500 & AIME 2024 \\
\midrule
o1 & 96.4\% & 96.4\% & 74\% (44.6/60) \\
DeepSeek-R1 & 97.3\% & 97.3\% & 79.8\% (47.9/60) \\
DeepSeek-V3 & 91.1\% & 90.2\% & 39.2\% \\
Claude 3.5 Sonnet & 96.4\% & 78.3\% & - \\
GPT-4o & 95.8\% & 76.6\% & - \\
Qwen2.5-72B & 95.8\% & 83.1\% & - \\
\bottomrule
\end{tabular}
\end{table}

\subsubsection{AIME (American Invitational Mathematics Examination)}

美国数学邀请赛:
\begin{itemize}
    \item 15道填空题,每题答案为0-999的整数
    \item 代表高中竞赛最高水平
    \item 是区分推理模型(o1、R1)与普通模型的关键benchmark
\end{itemize}

\subsection{代码能力}

\subsubsection{HumanEval}

Python函数生成:
\begin{itemize}
    \item \textbf{规模}:164道题目
    \item \textbf{形式}:给定函数签名和docstring,生成实现
    \item \textbf{评测}:Pass@k(k次采样至少一次通过)
\end{itemize}

\paragraph{HumanEval+}
增加更多测试用例,减少假阳性。

\subsubsection{MBPP (Mostly Basic Python Problems)}

更大规模的Python编程问题:
\begin{itemize}
    \item \textbf{规模}:974道题目
    \item \textbf{难度}:相对简单,入门级编程
    \item \textbf{用途}:与HumanEval互补
\end{itemize}

\subsubsection{LiveCodeBench}

\textbf{2024年最重要的代码评测创新},解决数据污染问题:
\begin{itemize}
    \item \textbf{持续更新}:从LeetCode、AtCoder、CodeForces持续收集新题
    \item \textbf{时间标记}:每道题有发布日期,可验证是否在训练数据截止日期之后
    \item \textbf{多维度}:代码生成、自我修复、测试输出预测
\end{itemize}

\paragraph{为什么LiveCodeBench重要}
\begin{itemize}
    \item HumanEval已被``刷榜'',很多模型在训练数据中见过
    \item LiveCodeBench的新题确保公平评测
    \item 是当前评估代码能力的金标准
\end{itemize}

\subsubsection{SWE-bench}

软件工程真实任务:
\begin{itemize}
    \item \textbf{任务}:修复GitHub上真实的issue
    \item \textbf{形式}:给定代码仓库和issue描述,生成patch
    \item \textbf{难度}:需要理解大型代码库,非常具有挑战性
\end{itemize}

\begin{table}[htbp]
\centering
\caption{代码Benchmark性能对比}
\label{tab:code_scores}
\begin{tabular}{lccc}
\toprule
模型 & HumanEval & LiveCodeBench & SWE-bench Verified \\
\midrule
Claude 3.5 Sonnet & 92.0\% & 41.4\% & 50.8\% \\
DeepSeek-V3 & 82.6\% & 40.5\% & 42.0\% \\
GPT-4o & 90.2\% & 34.2\% & 38.4\% \\
o1-preview & 92.4\% & - & 41.3\% \\
Qwen2.5-72B & 86.6\% & 32.8\% & - \\
\bottomrule
\end{tabular}
\end{table}

\subsection{指令遵循}

\subsubsection{IFEval (Instruction Following Evaluation)}

测试模型严格遵循指令的能力:
\begin{itemize}
    \item \textbf{规模}:500+条带约束的指令
    \item \textbf{约束类型}:
    \begin{itemize}
        \item 长度约束:``写超过400字''
        \item 格式约束:``用JSON格式输出''
        \item 内容约束:``至少提到3次AI''
        \item 结构约束:``分成5个段落''
    \end{itemize}
    \item \textbf{评测}:约束是否被满足(可程序验证)
\end{itemize}

\paragraph{两种指标}
\begin{itemize}
    \item \textbf{Prompt-level}:整个prompt的所有约束都满足
    \item \textbf{Instruction-level}:单个约束的满足率
\end{itemize}

IFEval是Open LLM Leaderboard的核心benchmark之一。

\subsubsection{MT-Bench}

多轮对话评测:
\begin{itemize}
    \item \textbf{形式}:80个两轮对话
    \item \textbf{评分}:GPT-4作为评判,1-10分
    \item \textbf{类别}:写作、角色扮演、推理、数学等8类
\end{itemize}

\subsubsection{AlpacaEval}

单轮指令跟随:
\begin{itemize}
    \item 与GPT-4的输出进行比较
    \item 报告Win Rate(胜率)
    \item AlpacaEval 2.0使用更严格的评判标准
\end{itemize}

\subsubsection{Arena-Hard}

基于Chatbot Arena的困难子集:
\begin{itemize}
    \item 从真实用户对话中筛选的500道困难问题
    \item GPT-4-Turbo作为评判
    \item 与Chatbot Arena排名高度相关
\end{itemize}

\subsection{长上下文评测}

\subsubsection{RULER}

长上下文能力的系统评测:
\begin{itemize}
    \item \textbf{任务类型}:
    \begin{itemize}
        \item Needle-in-a-Haystack:在长文本中找到特定信息
        \item Multi-hop QA:需要整合多处信息
        \item Aggregation:统计或汇总信息
    \end{itemize}
    \item \textbf{长度范围}:4K到128K+
    \item \textbf{评测}:不同长度下的准确率衰减曲线
\end{itemize}

\subsubsection{LongBench}

多任务长文本评测:
\begin{itemize}
    \item 6大类21个任务
    \item 平均长度约15K tokens
    \item 涵盖单文档/多文档QA、摘要、代码补全等
\end{itemize}

\subsubsection{Needle-in-a-Haystack}

最简单但直观的长上下文测试:
\begin{itemize}
    \item 在长文本的随机位置插入一个``针''(关键信息)
    \item 测试模型能否准确检索
    \item 生成位置-长度的热力图
\end{itemize}

\subsection{多语言评测}

\subsubsection{C-Eval / CMMLU}

中文知识评测:
\begin{itemize}
    \item \textbf{C-Eval}:52个学科,覆盖中国教育体系
    \item \textbf{CMMLU}:中文版MMLU
    \item 是评估中文能力的核心benchmark
\end{itemize}

\subsubsection{MGSM (Multilingual GSM)}

多语言数学推理:
\begin{itemize}
    \item GSM8K翻译成10种语言
    \item 测试非英语数学推理能力
    \item 揭示模型的语言偏差
\end{itemize}

\subsection{安全与对齐}

\subsubsection{TruthfulQA}

测试模型是否会生成虚假但常见的错误信息:
\begin{itemize}
    \item 817个问题,涵盖常见误解
    \item 人类因为偏见经常答错的问题
    \item 测试模型是否学习了人类的错误信念
\end{itemize}

\subsubsection{SimpleQA}

事实准确性评测(OpenAI 2024发布):
\begin{itemize}
    \item 简单的事实性问题
    \item 测试模型是否会``幻觉''错误信息
    \item 评估拒绝回答(``我不知道'')的能力
\end{itemize}

\subsection{综合评测平台}

\subsubsection{Open LLM Leaderboard}

Hugging Face维护的开放评测平台:
\begin{itemize}
    \item \textbf{当前版本(v2)包含}:
    \begin{itemize}
        \item IFEval(指令遵循)
        \item BBH(复杂推理)
        \item MATH Level 5(高难度数学)
        \item GPQA(研究生水平QA)
        \item MuSR(多步推理)
        \item MMLU-Pro(知识理解)
    \end{itemize}
    \item 任何人可以提交模型评测
    \item 透明、可复现
\end{itemize}

\subsubsection{Chatbot Arena}

基于真实用户投票的评测:
\begin{itemize}
    \item 用户盲评两个模型的回复
    \item 使用ELO排名系统
    \item 被认为是最能反映真实用户偏好的评测
    \item 但难以控制变量,不够``科学''
\end{itemize}

\subsubsection{LiveBench}

抗污染的动态评测:
\begin{itemize}
    \item 每月更新题目
    \item 严格的时间控制防止数据污染
    \item 涵盖数学、代码、推理、语言等多个维度
\end{itemize}

\subsection{评测最佳实践}

\subsubsection{避免数据污染}

\begin{itemize}
    \item \textbf{使用新benchmark}:LiveCodeBench、LiveBench等持续更新
    \item \textbf{时间切分}:确保评测数据晚于训练数据截止日期
    \item \textbf{多源验证}:同一能力用多个benchmark交叉验证
\end{itemize}

\subsubsection{评测配置标准化}

\begin{itemize}
    \item 明确报告Few-shot数量
    \item 统一prompt模板
    \item 使用相同的解码参数(temperature、top\_p等)
\end{itemize}

\subsubsection{选择合适的Benchmark}

\begin{table}[htbp]
\centering
\caption{评测场景与推荐Benchmark}
\label{tab:eval_recommendations}
\begin{tabular}{ll}
\toprule
评测目标 & 推荐Benchmark \\
\midrule
通用能力快速评估 & MMLU-Pro, GPQA-Diamond \\
数学推理 & MATH-500, AIME \\
代码生成 & LiveCodeBench, SWE-bench \\
指令遵循 & IFEval \\
长上下文 & RULER, Needle-in-Haystack \\
中文能力 & C-Eval, CMMLU \\
真实用户偏好 & Chatbot Arena, Arena-Hard \\
安全性 & TruthfulQA, SimpleQA \\
\bottomrule
\end{tabular}
\end{table}

\begin{remark}[评测的局限性]
\begin{itemize}
    \item \textbf{Benchmark ≠ 真实能力}:高分不代表实际应用效果好
    \item \textbf{优化目标错位}:过度优化benchmark可能损害通用能力
    \item \textbf{评测演进}:benchmark会饱和,需要持续更新
    \item \textbf{人类评估}:某些能力(创意、共情)难以自动评测
\end{itemize}
\end{remark}
