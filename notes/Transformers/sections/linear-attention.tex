\section{线性注意力}
\label{sec:linear-attention}

稀疏注意力通过"只计算重要的token对"将$O(N^2)$降至$O(Nk)$,但它仍然保留了softmax的计算形式。本节介绍一条更激进的路径:\textbf{彻底改变注意力的数学形式},使复杂度降至真正的$O(N)$。

线性注意力的核心洞察是:softmax注意力之所以需要$O(N^2)$,是因为必须先算出完整的$N \times N$注意力矩阵再做归一化。如果我们用其他函数替代softmax,使得计算可以"重新排列",就能避免显式构造这个矩阵。这种重排不是免费的——我们用数学等价性换取计算效率,代价是改变了注意力的语义。

\subsection{核心思想}

\subsubsection{从Softmax Attention到线性化}

标准自注意力(单头)的计算为:
\begin{equation}
    \text{Attention}(Q, K, V) = \underbrace{\text{softmax}\left(\frac{QK^\top}{\sqrt{d}}\right)}_{n \times n} V
\end{equation}
其中需要显式构造$n \times n$的注意力矩阵,时间和空间复杂度均为$O(n^2)$。

线性注意力的核心思想是:将softmax或$QK^\top$改写为可分解的形式,利用乘法结合律改变计算顺序:
\begin{equation}
    \text{Attention}(Q, K, V) \approx \phi(Q) \cdot \underbrace{(\phi(K)^\top V)}_{d \times d}
    \label{eq:linear_attn_kernel}
\end{equation}
其中$\phi(\cdot)$是某种特征映射函数。关键在于:先计算$\phi(K)^\top V$(与长度$n$线性相关的$d \times d$矩阵),再左乘$\phi(Q)$,总复杂度降为$O(nd^2)$,当$d \ll n$时近似$O(n)$。

\subsubsection{递推形式:Transformer即RNN}

在自回归(causal)场景下,线性注意力可以写成类似RNN的递推式~\citep{katharopoulos2020linear}。设$q_t, k_t, v_t$分别为第$t$步的query、key、value向量,定义状态矩阵$S_t \in \R^{d \times d}$:
\begin{align}
    S_t &= S_{t-1} + v_t k_t^\top \\
    o_t &= S_t \cdot q_t
    \label{eq:linear_attn_rnn}
\end{align}
这揭示了一个深刻联系:\textbf{线性Attention本质上是一种RNN},其隐状态$S_t$累积了历史信息。

\begin{remark}[线性注意力的双模式]
线性注意力支持两种等价的计算模式:
\begin{itemize}
    \item \textbf{并行模式}:训练时使用矩阵乘法,充分利用GPU并行性
    \item \textbf{递推模式}:推理时使用RNN形式,实现$O(1)$的增量更新
\end{itemize}
这种双模式特性使线性注意力在训练和推理阶段都能获得最优效率。
\end{remark}

\subsection{经典线性注意力方法}

\subsubsection{Linear Transformer~\citep{katharopoulos2020linear}}

最早系统化提出``Transformer即RNN''的工作。核心思想是将softmax attention重写为核函数形式:
\begin{equation}
    \text{Attention}(Q, K, V) = \frac{\phi(Q) (\phi(K)^\top V)}{\phi(Q) (\phi(K)^\top \bm{1})}
\end{equation}
其中$\phi(x) = \text{elu}(x) + 1$保证非负性。实验表明在自回归任务上可获得高达4000倍的加速。

\textbf{局限}:简单的特征映射难以精确近似softmax的行为,在复杂语言任务上存在性能差距。

\subsubsection{Performer~\citep{choromanski2020performer}}

提出FAVOR+(Fast Attention Via positive Orthogonal Random features)方法,用随机特征近似softmax核:
\begin{equation}
    \text{softmax}(q^\top k) \approx \phi(q)^\top \phi(k)
\end{equation}
其中$\phi$通过正交随机特征构造,具有无偏或近似无偏的理论保证。

\textbf{优势}:与原始Transformer完全兼容,可作为drop-in replacement。\\
\textbf{局限}:随机近似在实际任务中仍有精度损失。

\subsubsection{cosFormer~\citep{qin2022cosformer}}

不再硬近似softmax,而是基于softmax的两个关键性质设计线性替代:
\begin{enumerate}
    \item \textbf{非负性}:注意力权重应为非负
    \item \textbf{分布集中性}:注意力应集中在相关位置
\end{enumerate}

cosFormer使用ReLU保证非负性,并引入基于余弦的位置再加权机制:
\begin{equation}
    \text{Attention}_{ij} = \text{ReLU}(q_i)^\top \text{ReLU}(k_j) \cdot \cos\left(\frac{\pi}{2} \cdot \frac{i - j}{n}\right)
\end{equation}

在Long-Range Arena等长序列benchmark上取得当时最优性能,是``好用的线性Attention''的代表。

\subsection{带遗忘门的线性注意力}

原始线性注意力的一个根本问题是:状态矩阵$S_t$只能累加,无法遗忘。随着序列增长,历史信息``挤在一起'',导致检索能力下降。

\subsubsection{RetNet~\citep{sun2023retnet}}

微软提出的Retentive Network引入指数衰减因子$\gamma \in (0, 1)$:
\begin{align}
    S_t &= \gamma S_{t-1} + v_t k_t^\top \\
    o_t &= S_t \cdot q_t
    \label{eq:retnet}
\end{align}
这相当于对历史信息施加指数衰减,强调近期token的重要性。

\textbf{Multi-Scale Retention}:不同attention head使用不同的$\gamma$值,实现多尺度的记忆保持:
\begin{itemize}
    \item 小$\gamma$:关注近期信息(短程依赖)
    \item 大$\gamma$:保留更长历史(长程依赖)
\end{itemize}

\textbf{三种计算模式}:
\begin{enumerate}
    \item \textbf{并行模式}:训练时的矩阵计算
    \item \textbf{递推模式}:$O(1)$推理
    \item \textbf{分块递推模式}:长序列的高效处理
\end{enumerate}

性能:7B模型在8k序列长度下,推理速度比Transformer快8.4倍,内存减少70\%。

\subsubsection{Lightning Attention~\citep{minimax2025}}

MiniMax提出的Lightning Attention是目前\textbf{首个规模化到商业级的线性注意力架构}。核心创新:

\textbf{分块计算策略}:将注意力计算分为intra-block和inter-block两部分:
\begin{itemize}
    \item \textbf{Intra-block}:块内使用``左乘''形式,可并行计算
    \item \textbf{Inter-block}:块间使用``右乘''形式,累积状态
\end{itemize}
这种分解避免了传统线性注意力中缓慢的cumsum操作。

\textbf{混合架构}:每8层中,7层使用Lightning Attention,1层使用标准Softmax Attention,平衡效率与精度。

\begin{table}[htbp]
\centering
\caption{MiniMax-01模型规格}
\label{tab:minimax_spec}
\begin{tabular}{lc}
\toprule
参数 & 值 \\
\midrule
总参数量 & 456B \\
激活参数量(MoE) & 45.9B \\
专家数量 & 32 \\
训练上下文长度 & 1M tokens \\
推理外推长度 & 4M tokens \\
\bottomrule
\end{tabular}
\end{table}

\subsection{DeltaNet:基于Delta Rule的线性注意力}

\subsubsection{动机:解决记忆过载问题}

原始线性注意力的核心缺陷是\textbf{记忆过载}(memory overload):只能添加新的key-value关联,无法擦除已有信息。这导致随着序列增长,检索错误累积。

\subsubsection{Delta Rule更新}

DeltaNet~\citep{yang2024deltanet}引入``除旧迎新''的Delta Rule:
\begin{equation}
    S_t = S_{t-1} - \underbrace{(S_{t-1} \cdot k_t - v_t)}_{\text{delta}} \cdot k_t^\top
    \label{eq:deltanet}
\end{equation}

直观理解:
\begin{enumerate}
    \item $S_{t-1} \cdot k_t$:用当前key检索记忆中的value
    \item $S_{t-1} \cdot k_t - v_t$:计算检索值与真实值的差异(delta)
    \item 根据delta修正记忆,实现``精准更新''
\end{enumerate}

\subsubsection{Gated DeltaNet~\citep{yang2025gateddeltan}}

ICLR 2025工作进一步引入门控机制:
\begin{equation}
    S_t = \alpha_t \odot S_{t-1} + \beta_t \odot (v_t - S_{t-1} \cdot k_t) \cdot k_t^\top
\end{equation}
其中$\alpha_t$控制遗忘,$\beta_t$控制更新强度。

\textbf{互补性}:门控实现快速记忆擦除,Delta Rule实现精准记忆更新,两者结合在多个benchmark上超越Mamba2和原始DeltaNet。

\textbf{工业采用}:Gated DeltaNet已被Qwen3-Next采用作为线性注意力层。

\subsection{与状态空间模型的联系}

\subsubsection{Mamba与结构化状态空间对偶}

Mamba~\citep{gu2023mamba}是另一条重要的高效序列建模路线,基于选择性状态空间模型(Selective SSM)。Mamba-2论文~\citep{dao2024mamba2}揭示了\textbf{结构化状态空间对偶}(Structured State Space Duality, SSD):

\begin{quote}
    ``与标准自注意力相比,SSD只有两个区别:去掉softmax归一化,并应用一个独立的逐元素掩码矩阵。''
\end{quote}

这表明线性注意力和SSM可以视为同一框架的不同实例:
\begin{itemize}
    \item 线性注意力:通过特征映射分解attention矩阵
    \item SSM:通过状态空间方程建模序列
    \item 两者都有线性复杂度和递推形式
\end{itemize}

\subsubsection{混合架构}

实践中,纯线性模型在某些任务上仍有差距,因此出现了混合架构:
\begin{itemize}
    \item \textbf{Jamba}(AI21):Mamba + Attention
    \item \textbf{MiniMax-01}:Lightning Attention + 稀疏Softmax Attention
    \item \textbf{Qwen3-Next}:Gated DeltaNet + SwiGLU
\end{itemize}

\subsection{Test-Time Training视角}

苏剑林在``线性注意力简史''~\citep{su2025linearhistory}中指出,现代线性注意力的核心思想可以统一到\textbf{Test-Time Training}(TTT)框架:

\begin{quote}
    ``将序列建模视为在线学习问题,用优化器构建RNN。不同的损失函数对应不同的RNN模型。''
\end{quote}

\begin{table}[htbp]
\centering
\caption{线性注意力方法的TTT视角}
\label{tab:ttt_view}
\begin{tabular}{lll}
\toprule
方法 & 更新规则 & 对应优化器 \\
\midrule
Linear Attention & $S_t = S_{t-1} + v_t k_t^\top$ & 累积梯度 \\
RetNet & $S_t = \gamma S_{t-1} + v_t k_t^\top$ & 带衰减的累积 \\
DeltaNet & $S_t = S_{t-1} - (S_{t-1}k_t - v_t)k_t^\top$ & Delta Rule \\
Gated DeltaNet & 门控Delta Rule & 自适应学习率 \\
\bottomrule
\end{tabular}
\end{table}

这个视角为设计新型线性注意力提供了原则性指导:选择合适的``优化器''来更新记忆状态。

\subsection{工业部署现状}

\begin{table}[htbp]
\centering
\caption{线性注意力的工业应用}
\label{tab:linear_attn_industry}
\begin{tabular}{llll}
\toprule
公司/模型 & 架构类型 & 上下文长度 & 特点 \\
\midrule
MiniMax-01 & Lightning Attention + MoE & 1M-4M & 首个商业级线性Attention LLM \\
MiniMax-M1 & Lightning Attention & 1M+80k生成 & 开源reasoning模型 \\
Qwen3-Next & Gated DeltaNet & -- & 线性层 + 门控Attention \\
\bottomrule
\end{tabular}
\end{table}

\textbf{关键观察}:MiniMax是目前唯一将线性注意力规模化到商业级的厂商。其他厂商(如Kimi、DeepSeek)更倾向于稀疏注意力路线(见第\ref{sec:sparse-attention}节)。

\subsection{线性注意力的局限与展望}

\paragraph{当前局限}
\begin{enumerate}
    \item \textbf{精度差距}:在复杂推理任务上,纯线性注意力仍落后于Softmax Attention
    \item \textbf{in-context learning能力}:线性模型的few-shot能力通常弱于Transformer
    \item \textbf{长程精确检索}:passkey retrieval等任务上表现不稳定
\end{enumerate}

\paragraph{发展趋势}
\begin{enumerate}
    \item \textbf{混合架构}:结合线性层和稀疏Softmax层
    \item \textbf{门控机制}:更精细的记忆管理(如Gated DeltaNet)
    \item \textbf{知识蒸馏}:从Softmax模型蒸馏到线性模型(如LAWCAT)
    \item \textbf{TTT原则}:基于优化器视角设计新架构
\end{enumerate}

\begin{remark}[历史评价]
苏剑林的评价~\citep{su2025linearhistory}:``线性注意力已从单纯模仿Softmax Attention发展到`反哺'它——通过核技巧将DeltaNet改进应用于Softmax Attention。这表明该领域方兴未艾,仍有广阔探索空间。''
\end{remark}
