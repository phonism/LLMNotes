% ==========================================
% 强化学习笔记 - 导言区配置
% ==========================================

% ---------- 中文字体配置----------
% 文档类使用 ctexbook,支持章节结构
% 使用冬青黑体(现代清晰风格)
\setCJKmainfont{Hiragino Sans GB}    % 冬青黑体(正文)
\setCJKsansfont{Heiti SC}            % 黑体(无衬线)
\setCJKmonofont{Heiti SC}            % 等宽字体

% ---------- 中文引号配置 ----------
% 使用 csquotes 包自动处理引号
\usepackage[autostyle=true]{csquotes}
% 设置中文引号样式:外层 "",内层 ''
\DeclareQuoteStyle{chinese}
  {\textquotedblleft}{\textquotedblright}   % 外层:" "
  {\textquoteleft}{\textquoteright}          % 内层:' '
\setquotestyle{chinese}
% 让 "..." 自动转换为正确引号
\MakeOuterQuote{"}

% ---------- 页面设置 ----------
% 单页模式:左右对称边距
\usepackage[a4paper,
    top=2.5cm,
    bottom=2.5cm,
    left=2.5cm,
    right=2.5cm
]{geometry}

% 页眉页脚设置(单页模式:所有页统一格式)
\usepackage{fancyhdr}
\pagestyle{fancy}
\fancyhf{}
% 章节名在左,页码在右
\fancyhead[L]{\small\leftmark}
\fancyhead[R]{\small\thepage}
\renewcommand{\headrulewidth}{0.4pt}
\setlength{\headheight}{15pt}

% 章节首页使用plain样式(只有页码,居中底部)
\fancypagestyle{plain}{%
    \fancyhf{}
    \fancyfoot[C]{\small\thepage}
    \renewcommand{\headrulewidth}{0pt}
}

% 行间距和段落间距优化
\linespread{1.05}                    % 行间距 (1.0 = 单倍行距)
\setlength{\parskip}{3pt plus 1pt minus 1pt}  % 段落间距
\setlength{\parindent}{0em}          % 无首行缩进

% ---------- 数学宏包 ----------
\usepackage{amsmath}                 % AMS数学
\usepackage{amssymb}                 % AMS符号
\usepackage{amsthm}                  % AMS定理环境
\usepackage{mathtools}               % 数学工具
\usepackage{bm}                      % 粗体数学符号
\usepackage{cancel}                  % 数学删除线

% ---------- 图表相关 ----------
\usepackage{graphicx}                % 插图
\usepackage{subfigure}               % 子图
\usepackage{float}                   % 浮动体控制
\usepackage{caption}                 % 标题设置
\captionsetup{
    font=small,                      % 图表标题字体大小
    labelfont=bf,                    % 标签加粗(如"图 1")
    textfont={small},                % 说明文字(中文不使用斜体)
    skip=6pt,                        % 标题与图表间距
    justification=justified,         % 两端对齐
    singlelinecheck=false            % 即使单行也按设置对齐
}
\usepackage{booktabs}                % 三线表
\usepackage{multirow}                % 表格多行合并
\usepackage{tikz}                    % 绘图
\usepackage{pgfplots}                % 图表绘制
\pgfplotsset{compat=1.18}
\usetikzlibrary{shapes,arrows,positioning,calc,decorations.pathreplacing}

% ---------- 代码高亮 ----------
\usepackage{minted}                  % 代码高亮(需要Pygments)
\usemintedstyle{colorful}            % 代码配色方案 (更专业)
\setminted{
    breaklines,                      % 自动换行
    breakanywhere,                   % 任意位置换行
    fontsize=\footnotesize,          % 字体大小(更小更紧凑)
    linenos,                         % 行号
    numbersep=3pt,                   % 行号间距(减小)
    frame=lines,                     % 边框
    framesep=1.5mm,                  % 边框间距(减小)
    tabsize=4,                       % Tab宽度
    bgcolor=gray!3,                  % 背景色(更浅)
    baselinestretch=1.0,             % 代码行间距
    xleftmargin=0pt,                 % 左边距
    xrightmargin=0pt                 % 右边距
}

% ---------- 算法伪代码 ----------
\usepackage[ruled,vlined,linesnumbered]{algorithm2e}
\SetKwComment{Comment}{/* }{ */}
\SetKwInput{KwInput}{输入}
\SetKwInput{KwOutput}{输出}
\SetKwProg{Fn}{函数}{:}{}
\SetAlgorithmName{算法}{算法}{算法列表}
% 优化算法环境间距
\SetAlgoSkip{medskip}                % 算法前后间距
\SetAlCapSkip{1ex}                   % 标题间距
\setlength{\algomargin}{1em}         % 算法左边距

% ---------- 超链接 ----------
% 注意:cleveref 必须在 titlesec 配置之后加载,见文件末尾
\usepackage[colorlinks=true, linkcolor=blue, citecolor=red, urlcolor=purple]{hyperref}

% ---------- 参考文献 ----------
\usepackage[backend=biber, style=numeric, sorting=none]{biblatex}
\addbibresource{references.bib}

% ---------- 其他工具包 ----------
\usepackage{enumerate}               % 枚举环境
\usepackage{enumitem}                % 列表定制
\setlist{
    itemsep=2pt,                     % 列表项间距
    parsep=2pt,                      % 段落间距
    topsep=4pt,                      % 列表前后间距
    partopsep=0pt                    % 段落列表额外间距
}
\usepackage{color,xcolor}            % 颜色
\usepackage{tcolorbox}               % 彩色文本框
\tcbuselibrary{most}

% ---------- 标题格式优化 ----------
\usepackage{titlesec}                % 标题定制

% Chapter标题格式(居中,大字号)
\titleformat{\chapter}[display]
    {\normalfont\huge\bfseries\centering}
    {\chaptertitlename\ \thechapter}
    {20pt}
    {\Huge}
\titlespacing*{\chapter}
    {0pt}{0pt}{40pt}

% Section标题格式(左对齐,加粗)
\titleformat{\section}
    {\normalfont\Large\bfseries}
    {\thesection}
    {1em}
    {}
\titlespacing*{\section}
    {0pt}{12pt plus 4pt minus 2pt}{6pt plus 2pt minus 1pt}

% Subsection标题格式(左对齐,加粗)
\titleformat{\subsection}
    {\normalfont\large\bfseries}
    {\thesubsection}
    {1em}
    {}
\titlespacing*{\subsection}
    {0pt}{10pt plus 3pt minus 2pt}{4pt plus 1pt minus 1pt}

% Subsubsection标题格式(左对齐,加粗)
\titleformat{\subsubsection}
    {\normalfont\normalsize\bfseries}
    {\thesubsubsection}
    {1em}
    {}
\titlespacing*{\subsubsection}
    {0pt}{8pt plus 2pt minus 1pt}{3pt plus 1pt minus 1pt}

% 设置目录深度(显示到subsection)
\setcounter{tocdepth}{2}
% 设置章节编号深度(编号到subsection)
\setcounter{secnumdepth}{3}

% ---------- 智能引用 ----------
% 注意:titlesec 与 cleveref 存在兼容性问题,section 引用会出错
% 解决方案:使用自定义的 \secref 命令代替 \cref 引用章节
\usepackage{cleveref}
% cleveref 中文配置(用于非 section 类型的引用)
\crefname{figure}{图}{图}
\Crefname{figure}{图}{图}
\crefname{table}{表}{表}
\Crefname{table}{表}{表}
\crefname{equation}{式}{式}
\Crefname{equation}{式}{式}
\crefname{theorem}{定理}{定理}
\Crefname{theorem}{定理}{定理}
\crefname{lemma}{引理}{引理}
\Crefname{lemma}{引理}{引理}
\crefname{definition}{定义}{定义}
\Crefname{definition}{定义}{定义}
\crefname{algorithm}{算法}{算法}
\Crefname{algorithm}{算法}{算法}

% 自定义章节引用命令(绕过 titlesec + cleveref 兼容性问题)
\newcommand{\chapref}[1]{第\ref{#1}章}
\newcommand{\secref}[1]{第\ref{#1}节}

% ==========================================
% 自定义命令 - RL专用
% ==========================================

% ---------- 数学符号 ----------
\newcommand{\E}{\mathbb{E}}                    % 期望
\newcommand{\R}{\mathbb{R}}                    % 实数
\newcommand{\N}{\mathbb{N}}                    % 自然数
\newcommand{\prob}{\mathbb{P}}                 % 概率
\newcommand{\ind}{\mathbb{I}}                  % 指示函数

% ---------- RL核心符号 ----------
\newcommand{\state}{s}                         % 状态
\newcommand{\action}{a}                        % 动作
\newcommand{\reward}{r}                        % 奖励
\newcommand{\policy}{\pi}                      % 策略
\newcommand{\Val}{V}                           % 状态价值函数
\newcommand{\Qval}{Q}                          % 动作价值函数
\newcommand{\advantage}{A}                     % 优势函数
\newcommand{\discount}{\gamma}                 % 折扣因子
\newcommand{\trajectory}{\tau}                 % 轨迹
\newcommand{\transition}{\mathcal{T}}          % 状态转移
\newcommand{\statespace}{\mathcal{S}}          % 状态空间
\newcommand{\actionspace}{\mathcal{A}}         % 动作空间

% ---------- 数学算子 ----------
\DeclareMathOperator*{\argmax}{argmax}
\DeclareMathOperator*{\argmin}{argmin}
\newcommand{\defeq}{\coloneqq}                 % 定义为(:=)
\newcommand{\norm}[1]{\left\| #1 \right\|}    % 范数
\newcommand{\abs}[1]{\left| #1 \right|}       % 绝对值

% ==========================================
% 自定义环境
% ==========================================

% ---------- 定理环境 ----------
% Book类中定理环境按章编号(如:定理1.1表示第1章第1个定理)
\newtheorem{theorem}{定理}[chapter]
\newtheorem{lemma}[theorem]{引理}
\newtheorem{proposition}[theorem]{命题}
\newtheorem{corollary}[theorem]{推论}
\newtheorem{definition}{定义}[chapter]
\newtheorem{example}{例}[chapter]
\newtheorem{remark}{注}[chapter]

% ---------- 彩色文本框 ----------
\newtcolorbox{keypoint}{
    colback=blue!5!white,
    colframe=blue!75!black,
    title=关键点,
    fonttitle=\bfseries\small,
    top=4pt, bottom=4pt,             % 减小上下边距
    left=6pt, right=6pt,             % 减小左右边距
    boxsep=2pt,                      % 减小盒子间距
    before skip=8pt, after skip=8pt  % 减小前后间距
}

\newtcolorbox{note}{
    colback=green!5!white,
    colframe=green!75!black,
    title=注意,
    fonttitle=\bfseries\small,
    top=4pt, bottom=4pt,
    left=6pt, right=6pt,
    boxsep=2pt,
    before skip=8pt, after skip=8pt
}

\newtcolorbox{important}{
    colback=red!5!white,
    colframe=red!75!black,
    title=重要,
    fonttitle=\bfseries\small,
    top=4pt, bottom=4pt,
    left=6pt, right=6pt,
    boxsep=2pt,
    before skip=8pt, after skip=8pt
}

% ---------- 代码环境快捷方式 ----------
\newcommand{\pycode}[1]{\mintinline{python}{#1}}
\newcommand{\bashcode}[1]{\mintinline{bash}{#1}}

% ==========================================
% 文档信息设置
% ==========================================
\newcommand{\settitle}[1]{\title{#1}}
\newcommand{\setauthor}[1]{\author{#1}}
\newcommand{\setdate}[1]{\date{#1}}
