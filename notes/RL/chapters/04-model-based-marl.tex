% ==========================================
% 第四章:基于模型的方法与多智能体学习
% ==========================================

\chapter{基于模型的方法与多智能体学习}
\label{chap:model-based-marl}

% ------------------------------------------
\section{引言:样本效率的追求}
\label{sec:mbrl-intro}

\subsection{核心问题}

前三章介绍的 Model-Free 方法(Q-Learning、Policy Gradient、PPO)虽然强大,但有一个共同的缺陷:

\begin{quote}
\textbf{样本效率极低}——训练一个 Atari 游戏 agent 需要数亿帧画面,相当于人类玩数百小时。而人类通常只需几分钟就能学会基本操作。为什么会有如此巨大的差距?
\end{quote}

关键区别在于:\textbf{人类在脑中有一个世界模型}。当我们想象"如果我这样做会发生什么"时,我们在用这个模型进行\textbf{心智模拟}(mental simulation),而不需要真正尝试。

Model-Based RL 的核心思想正是:\textbf{学习或利用环境模型,通过规划(Planning)来提高样本效率}。

\subsection{Model-Free vs Model-Based}

根据是否使用环境模型,RL 方法分为两大类:

\begin{definition}[Model-Free 与 Model-Based]
\leavevmode
\begin{itemize}
    \item \textbf{Model-Free}:不学习或使用环境模型,直接从真实经验中学习价值函数或策略
    \item \textbf{Model-Based}:学习或利用环境模型 $\hat{P}(s'|s,a)$, $\hat{R}(s,a)$,在模型中进行规划
\end{itemize}
\end{definition}

\begin{figure}[H]
    \centering
    \begin{tikzpicture}[
        box/.style={draw, rounded corners, minimum width=3cm, minimum height=1cm, align=center},
        arrow/.style={->, thick, >=stealth}
    ]
        % Model-Free
        \begin{scope}[shift={(-4,0)}]
            \node[box, fill=blue!20] (env1) at (0, 1.5) {真实环境};
            \node[box, fill=orange!20] (policy1) at (0, -1) {策略/价值函数};
            \draw[arrow, red, very thick] (env1) -- node[right, font=\small] {大量真实经验} (policy1);
            \node[font=\bfseries] at (0, 3) {Model-Free};
            \node[font=\scriptsize, text=red] at (0, -2.3) {样本效率低};
        \end{scope}

        % Model-Based
        \begin{scope}[shift={(4,0)}]
            \node[box, fill=blue!20] (env2) at (0, 1.5) {真实环境};
            \node[box, fill=green!20] (model) at (0, 0) {环境模型};
            \node[box, fill=orange!20] (policy2) at (0, -1.5) {策略/价值函数};
            \draw[arrow] (env2) -- node[right, font=\small] {少量经验} (model);
            \draw[arrow, green!60!black, very thick] (model) -- node[right, font=\small] {大量模拟} (policy2);
            \draw[arrow, dashed] (env2.west) to[out=180, in=180] node[left, font=\small] {校正} (policy2.west);
            \node[font=\bfseries] at (0, 3) {Model-Based};
            \node[font=\scriptsize, text=green!60!black] at (0, -2.8) {样本效率高};
        \end{scope}
    \end{tikzpicture}
    \caption{Model-Free vs Model-Based:后者用模型生成模拟经验,减少真实交互需求}
    \label{fig:mf-vs-mb}
\end{figure}

\begin{table}[H]
    \centering
    \begin{tabular}{@{}lcc@{}}
        \toprule
        & \textbf{Model-Free} & \textbf{Model-Based} \\
        \midrule
        环境模型 & 不需要 & 需要(已知或学习) \\
        样本效率 & 低 & 高 \\
        计算开销 & 低 & 高(规划) \\
        模型误差 & 无 & 可能累积 \\
        适用场景 & 模型难以获取 & 模型已知或易学 \\
        典型算法 & Q-Learning, PPO & Dyna, MCTS, MuZero \\
        \bottomrule
    \end{tabular}
    \caption{Model-Free 与 Model-Based 的对比}
\end{table}

\subsection{本章路线图}

本章将介绍两个重要方向:

\begin{enumerate}
    \item \textbf{Model-Based RL}(\secref{sec:mbrl-overview}-\secref{sec:alphago}):如何利用环境模型进行规划
    \begin{itemize}
        \item World Model 的定义与学习
        \item Dyna 架构:结合直接学习与规划
        \item MCTS:Decision-time Planning 的代表
        \item AlphaGo/Zero:MCTS + 深度学习的里程碑
    \end{itemize}

    \item \textbf{Multi-Agent RL}(\secref{sec:marl}-\secref{sec:self-play}):多个 agent 的交互与博弈
    \begin{itemize}
        \item 博弈论基础:Normal-form Game
        \item Nash 均衡:稳定的策略组合
        \item Self-Play:训练博弈 AI 的强大方法
    \end{itemize}
\end{enumerate}

AlphaGo/AlphaZero 是两者的完美结合:用 MCTS 进行规划,用 Self-Play 进行训练。

% ------------------------------------------
\section{Model-Based RL 概述}
\label{sec:mbrl-overview}

\subsection{World Model 的定义}

\begin{definition}[World Model]
World Model 是对环境动力学的估计,包括:
\begin{itemize}
    \item \textbf{状态转移模型}:$\hat{P}(s'|s,a) \approx P(s'|s,a)$
    \item \textbf{奖励模型}:$\hat{R}(s,a) \approx R(s,a)$
\end{itemize}
有了 World Model,agent 可以在"脑中"模拟动作的后果,而不需要真正执行。
\end{definition}

World Model 的来源有两种:

\begin{enumerate}
    \item \textbf{已知规则}:如棋类游戏的规则、物理引擎的方程
    \begin{itemize}
        \item 优点:模型精确,无误差
        \item 缺点:仅适用于规则完全已知的领域
    \end{itemize}

    \item \textbf{从数据学习}:用神经网络从交互经验中学习
    \begin{itemize}
        \item 优点:适用于复杂环境
        \item 缺点:模型存在误差
    \end{itemize}
\end{enumerate}

\subsection{学习 World Model}

学习 World Model 本质上是一个监督学习问题。给定经验数据 $\{(s_t, a_t, r_t, s_{t+1})\}$:

\begin{enumerate}
    \item \textbf{确定性模型}:直接预测下一状态
    \[
        \hat{s}_{t+1} = f_\theta(s_t, a_t), \quad L = \|s_{t+1} - \hat{s}_{t+1}\|^2
    \]

    \item \textbf{概率模型}:预测状态分布
    \[
        \hat{P}_\theta(s'|s, a), \quad L = -\log \hat{P}_\theta(s_{t+1}|s_t, a_t)
    \]

    \item \textbf{隐空间模型}:在低维隐空间中预测
    \[
        z_{t+1} = f_\theta(z_t, a_t), \quad z_t = \text{Encoder}(s_t)
    \]
\end{enumerate}

\begin{note}
现代 World Model 方法(如 Dreamer、MuZero)通常在隐空间中进行预测,避免直接预测高维原始观测(如图像),大大降低了学习难度。
\end{note}

\subsection{Model Bias 问题}

\begin{definition}[Model Bias / Model Error]
当学习的模型 $\hat{P}, \hat{R}$ 与真实环境 $P, R$ 存在差异时,在模型中规划得到的策略在真实环境中可能表现不佳,这称为 \textbf{Model Bias}。
\end{definition}

Model Bias 的关键问题是\textbf{误差累积}(Error Compounding):

\begin{figure}[H]
    \centering
    \begin{tikzpicture}[
        state/.style={circle, draw, fill=blue!20, minimum size=0.8cm},
        pred/.style={circle, draw, dashed, fill=red!20, minimum size=0.8cm},
        arrow/.style={->, thick, >=stealth}
    ]
        % 真实轨迹
        \node[state] (s0) at (0, 0) {$s_0$};
        \node[state] (s1) at (2, 0) {$s_1$};
        \node[state] (s2) at (4, 0) {$s_2$};
        \node[state] (s3) at (6, 0) {$s_3$};
        \node[state] (s4) at (8, 0) {$s_4$};

        \draw[arrow] (s0) -- node[above, font=\scriptsize] {$a_0$} (s1);
        \draw[arrow] (s1) -- node[above, font=\scriptsize] {$a_1$} (s2);
        \draw[arrow] (s2) -- node[above, font=\scriptsize] {$a_2$} (s3);
        \draw[arrow] (s3) -- node[above, font=\scriptsize] {$a_3$} (s4);

        % 预测轨迹
        \node[pred] (h1) at (2, -1) {$\hat{s}_1$};
        \node[pred] (h2) at (4, -1.5) {$\hat{s}_2$};
        \node[pred] (h3) at (6, -2.2) {$\hat{s}_3$};
        \node[pred] (h4) at (8, -3) {$\hat{s}_4$};

        \draw[arrow, dashed, red] (s0) -- (h1);
        \draw[arrow, dashed, red] (h1) -- (h2);
        \draw[arrow, dashed, red] (h2) -- (h3);
        \draw[arrow, dashed, red] (h3) -- (h4);

        % 误差标注
        \draw[<->, gray] (s1) -- node[right, font=\scriptsize] {$\epsilon_1$} (h1);
        \draw[<->, gray] (s2) -- node[right, font=\scriptsize] {$\epsilon_2$} (h2);
        \draw[<->, gray] (s3) -- node[right, font=\scriptsize] {$\epsilon_3$} (h3);
        \draw[<->, gray] (s4) -- node[right, font=\scriptsize] {$\epsilon_4$} (h4);

        % 图例
        \node[font=\small] at (4, 1) {真实轨迹(实线)};
        \node[font=\small, red] at (4, -4) {预测轨迹(虚线)——误差逐步累积};
    \end{tikzpicture}
    \caption{Model Error Compounding:每一步的预测误差会累积,导致长期预测严重偏离}
    \label{fig:model-error}
\end{figure}

\begin{theorem}[误差累积上界]
设单步模型误差为 $\epsilon = \max_{s,a} \|\hat{P}(\cdot|s,a) - P(\cdot|s,a)\|_1$,则 $H$ 步规划的总变差距离上界为:
\[
    \text{TV}(\hat{P}^H, P^H) \leq H \cdot \epsilon
\]
即误差随规划步数\textbf{线性累积}。
\end{theorem}

缓解 Model Bias 的策略:
\begin{enumerate}
    \item \textbf{短期规划}:只用模型做短期预测(如 Dyna 中的 1-step)
    \item \textbf{集成模型}:训练多个模型,用不确定性指导探索
    \item \textbf{持续校正}:用真实数据不断更新模型
    \item \textbf{隐空间规划}:在抽象空间中规划(如 MuZero)
\end{enumerate}

% ------------------------------------------
\section{Planning 方法}
\label{sec:planning}

有了环境模型,下一步是利用模型进行\textbf{规划}(Planning)。根据规划时机,分为两类:

\begin{definition}[Background Planning vs Decision-time Planning]
\leavevmode
\begin{itemize}
    \item \textbf{Background Planning}:在与真实环境交互之外,利用模型生成模拟经验来训练策略
    \item \textbf{Decision-time Planning}:在需要做决策时,利用模型进行前向搜索,选择最优动作
\end{itemize}
\end{definition}

\begin{figure}[H]
    \centering
    \begin{tikzpicture}[
        box/.style={draw, rounded corners, fill=blue!10, minimum width=3.5cm, minimum height=1.2cm, align=center},
        arrow/.style={->, thick, >=stealth}
    ]
        % Background Planning
        \begin{scope}[shift={(-4.5, 0)}]
            \node[box, fill=green!20] (bg) at (0, 0) {Background\\Planning};
            \node[font=\small, align=center] at (0, -2) {离线生成经验\\训练策略网络\\代表:Dyna};
            \node[font=\bfseries] at (0, 1.5) {训练时规划};
        \end{scope}

        % Decision-time Planning
        \begin{scope}[shift={(4.5, 0)}]
            \node[box, fill=orange!20] (dt) at (0, 0) {Decision-time\\Planning};
            \node[font=\small, align=center] at (0, -2) {在线搜索决策\\不训练网络\\代表:MCTS};
            \node[font=\bfseries] at (0, 1.5) {决策时规划};
        \end{scope}
    \end{tikzpicture}
    \caption{两种规划方式的对比}
    \label{fig:planning-types}
\end{figure}

\subsection{Dyna 架构}

Dyna 是 Background Planning 的经典框架,由 Sutton 于 1991 年提出。其核心思想是:
\textbf{每次真实交互后,用模型生成多次模拟经验来加速学习}。

\begin{figure}[H]
    \centering
    \begin{tikzpicture}[scale=0.95,
        box/.style={draw, rounded corners, minimum width=2.8cm, minimum height=1cm, align=center},
        arrow/.style={->, thick, >=stealth}
    ]
        % 组件
        \node[box, fill=blue!20] (env) at (0, 2) {真实环境};
        \node[box, fill=green!20] (model) at (5, 2) {环境模型\\$\hat{P}, \hat{R}$};
        \node[box, fill=orange!20] (policy) at (2.5, -1) {策略/价值函数\\$\Qval(s,a)$};
        \node[box, fill=purple!15] (exp) at (-2.5, -1) {经验缓存\\$(s,a,r,s')$};

        % 连接
        \draw[arrow] (env) -- node[above, font=\small] {学习模型} (model);
        \draw[arrow] (env) -- node[left, font=\small, pos=0.3] {真实经验} (exp);
        \draw[arrow] (exp) -- node[below, font=\small] {直接学习} (policy);
        \draw[arrow, green!60!black, very thick] (model) -- node[right, font=\small, pos=0.3] {模拟经验\\($n$ 次)} (policy);
        \draw[arrow, dashed] (policy.north) to[out=120, in=240] node[left, font=\small] {动作} (env.south);

        % 标注
        \node[font=\scriptsize, red] at (5, 0.3) {每步真实交互};
        \node[font=\scriptsize, red] at (5, -0.1) {可生成 $n$ 步模拟};
    \end{tikzpicture}
    \caption{Dyna 架构:结合直接学习(从真实经验)与规划(从模拟经验)}
    \label{fig:dyna}
\end{figure}

\begin{algorithm}[H]
\caption{Dyna-Q}
\label{alg:dyna-q}
\KwInput{规划步数 $n$,学习率 $\alpha$,探索率 $\epsilon$}
初始化 $\Qval(s,a) \leftarrow 0$,表格模型 $\text{Model}(s,a) \leftarrow \emptyset$\;
\ForEach{episode}{
    初始化状态 $s$\;
    \While{$s$ 不是终止状态}{
        $a \leftarrow \epsilon\text{-greedy}(\Qval(s, \cdot))$\;
        执行 $a$,观察 $r, s'$\;

        \tcp{直接 RL 学习}
        $\Qval(s,a) \leftarrow \Qval(s,a) + \alpha[r + \discount \max_{a'} \Qval(s', a') - \Qval(s,a)]$\;

        \tcp{更新模型}
        $\text{Model}(s,a) \leftarrow (r, s')$ \Comment{确定性模型}

        \tcp{规划:从模型中学习}
        \For{$i = 1$ \KwTo $n$}{
            随机选择之前访问过的状态-动作对 $(\tilde{s}, \tilde{a})$\;
            $(\tilde{r}, \tilde{s}') \leftarrow \text{Model}(\tilde{s}, \tilde{a})$\;
            $\Qval(\tilde{s},\tilde{a}) \leftarrow \Qval(\tilde{s},\tilde{a}) + \alpha[\tilde{r} + \discount \max_{a'} \Qval(\tilde{s}', a') - \Qval(\tilde{s},\tilde{a})]$\;
        }
        $s \leftarrow s'$\;
    }
}
\end{algorithm}

\begin{keypoint}
Dyna 的核心优势:
\begin{enumerate}
    \item \textbf{样本效率提升}:每次真实交互可产生 $n$ 次模拟学习
    \item \textbf{灵活的计算-样本权衡}:增大 $n$ 可用更多计算换取更少真实交互
    \item \textbf{渐进收敛}:当模型准确时,理论上与直接学习收敛到相同策略
\end{enumerate}
\end{keypoint}

\subsection{Decision-time Planning}

与 Background Planning 不同,Decision-time Planning 在每次决策时进行规划:

\begin{enumerate}
    \item 从当前状态出发,用模型模拟多条可能的轨迹
    \item 评估每条轨迹的回报
    \item 选择最优的第一步动作
    \item 执行后重新规划(不保存中间结果)
\end{enumerate}

Decision-time Planning 的特点:
\begin{itemize}
    \item \textbf{计算集中}:所有计算都为当前决策服务
    \item \textbf{动态精度}:可根据需要调整搜索深度和广度
    \item \textbf{无需训练}:可直接用于测试时
\end{itemize}

最著名的 Decision-time Planning 方法是 \textbf{Monte Carlo Tree Search (MCTS)}。

% ------------------------------------------
\section{Monte Carlo Tree Search (MCTS)}
\label{sec:mcts}

MCTS 是一种基于树搜索的 Decision-time Planning 方法,广泛应用于棋类游戏和组合优化问题。

\subsection{MCTS 的核心思想}

MCTS 的目标是在有限的计算预算内,估计当前状态下各动作的价值。其核心思想是:
\textbf{有选择性地扩展搜索树,把计算资源集中在最有希望的分支上}。

\begin{quote}
\textit{如何决定哪个分支"最有希望"?这需要平衡\textbf{利用}(选择已知好的分支)和\textbf{探索}(尝试不确定的分支)。}
\end{quote}

\subsection{MCTS 四步流程}

MCTS 的每次迭代包含四个步骤:

\begin{figure}[H]
    \centering
    \begin{tikzpicture}[scale=0.8, every node/.style={scale=0.8},
        treenode/.style={circle, draw, minimum size=0.7cm},
        selected/.style={treenode, fill=blue!30, very thick},
        expanded/.style={treenode, fill=green!30, dashed},
        evaluated/.style={treenode, fill=orange!30},
        backed/.style={treenode, fill=red!20}
    ]
        % Step 1: Selection
        \begin{scope}[shift={(0, 0)}]
            \node[selected] (r1) at (0, 0) {};
            \node[selected] (a1) at (-0.8, -1) {};
            \node[treenode] (b1) at (0.8, -1) {};
            \node[selected] (c1) at (-1.2, -2) {};
            \node[treenode] (d1) at (-0.4, -2) {};

            \draw[very thick, blue, ->] (r1) -- (a1);
            \draw (r1) -- (b1);
            \draw[very thick, blue, ->] (a1) -- (c1);
            \draw (a1) -- (d1);

            \node[font=\small\bfseries] at (0, 0.8) {1. Selection};
            \node[font=\scriptsize, align=center] at (0, -3) {沿树用 UCB\\选择子节点};
        \end{scope}

        % Step 2: Expansion
        \begin{scope}[shift={(4, 0)}]
            \node[treenode] (r2) at (0, 0) {};
            \node[treenode] (a2) at (-0.8, -1) {};
            \node[treenode] (b2) at (0.8, -1) {};
            \node[treenode] (c2) at (-1.2, -2) {};
            \node[expanded] (new) at (-0.4, -2) {};

            \draw (r2) -- (a2);
            \draw (r2) -- (b2);
            \draw (a2) -- (c2);
            \draw[thick, green!60!black, dashed] (a2) -- (new);

            \node[font=\small\bfseries] at (0, 0.8) {2. Expansion};
            \node[font=\scriptsize, align=center] at (0, -3) {扩展一个\\新子节点};
        \end{scope}

        % Step 3: Evaluation
        \begin{scope}[shift={(8, 0)}]
            \node[treenode] (r3) at (0, 0) {};
            \node[treenode] (a3) at (-0.8, -1) {};
            \node[treenode] (b3) at (0.8, -1) {};
            \node[evaluated] (c3) at (-1.2, -2) {};

            \draw (r3) -- (a3);
            \draw (r3) -- (b3);
            \draw (a3) -- (c3);

            % Rollout
            \draw[thick, orange, ->] (c3) -- ++(0.3, -0.8) -- ++(0.2, -0.6) -- ++(-0.1, -0.5);
            \node[font=\scriptsize] at (-0.3, -3.5) {$v = ?$};

            \node[font=\small\bfseries] at (0, 0.8) {3. Evaluation};
            \node[font=\scriptsize, align=center] at (0, -4.5) {Rollout 或\\价值网络};
        \end{scope}

        % Step 4: Backup
        \begin{scope}[shift={(12, 0)}]
            \node[backed] (r4) at (0, 0) {$\uparrow$};
            \node[backed] (a4) at (-0.8, -1) {$\uparrow$};
            \node[treenode] (b4) at (0.8, -1) {};
            \node[backed] (c4) at (-1.2, -2) {$v$};

            \draw[thick, red, <-] (r4) -- (a4);
            \draw (r4) -- (b4);
            \draw[thick, red, <-] (a4) -- (c4);

            \node[font=\small\bfseries] at (0, 0.8) {4. Backup};
            \node[font=\scriptsize, align=center] at (0, -3) {沿路径\\更新统计};
        \end{scope}
    \end{tikzpicture}
    \caption{MCTS 的四个步骤}
    \label{fig:mcts-steps}
\end{figure}

\begin{enumerate}
    \item \textbf{Selection(选择)}:
    从根节点开始,使用\textbf{树策略}(如 UCB)递归选择子节点,直到到达叶节点(未完全扩展的节点)。

    \item \textbf{Expansion(扩展)}:
    如果叶节点不是终止状态,根据可行动作扩展一个或多个新的子节点。

    \item \textbf{Evaluation(评估)}:
    评估新扩展节点的价值。传统方法使用 \textbf{rollout}(随机模拟到终局);现代方法使用\textbf{价值网络}直接估计。

    \item \textbf{Backup(回溯)}:
    将评估值沿选择路径回传,更新路径上所有节点的访问次数 $N$ 和价值估计 $\Qval$。
\end{enumerate}

\subsection{UCB 公式}

Selection 阶段的核心是 \textbf{UCB(Upper Confidence Bound)}公式,它优雅地平衡了利用与探索:

\begin{theorem}[UCB for Trees (UCT)]
在 Selection 阶段,选择最大化以下值的动作:
\begin{equation}
    \text{UCB}(s, a) = \underbrace{\Qval(s, a)}_{\text{利用}} + \underbrace{c \sqrt{\frac{\ln N(s)}{N(s, a)}}}_{\text{探索}}
    \label{eq:ucb}
\end{equation}
其中:
\begin{itemize}
    \item $\Qval(s, a)$:动作 $a$ 的平均价值估计(从历史模拟中统计)
    \item $N(s)$:状态 $s$ 的总访问次数
    \item $N(s, a)$:在状态 $s$ 执行动作 $a$ 的次数
    \item $c$:探索系数,控制探索-利用权衡
\end{itemize}
\end{theorem}

\begin{figure}[H]
    \centering
    \begin{tikzpicture}
        \begin{axis}[
            width=10cm, height=6cm,
            xlabel={$N(s,a)$(访问次数)},
            ylabel={探索奖励},
            domain=1:100,
            samples=100,
            legend pos=north east,
            grid=major
        ]
            \addplot[thick, blue] {sqrt(ln(100)/x)};
            \addlegendentry{$\sqrt{\frac{\ln N(s)}{N(s,a)}}$}
        \end{axis}
    \end{tikzpicture}
    \caption{UCB 探索项随访问次数的变化}
    \label{fig:ucb-exploration}
\end{figure}

\begin{note}
UCB 的直觉理解:
\begin{itemize}
    \item \textbf{利用项} $\Qval(s,a)$:选择历史表现好的动作
    \item \textbf{探索项}:选择访问次数少的动作(不确定性高)
    \item 随着访问次数增加,探索奖励逐渐减小,最终由利用主导
    \item $c$ 越大,越倾向探索;$c$ 越小,越倾向利用
\end{itemize}
\end{note}

\subsection{MCTS 算法}

\begin{algorithm}[H]
\caption{Monte Carlo Tree Search}
\label{alg:mcts}
\KwInput{当前状态 $s_0$,搜索预算 $B$(迭代次数),探索系数 $c$}
\KwOutput{最优动作 $a^*$}
初始化根节点 $\text{root} = s_0$,$N(\text{root}) = 0$\;
\For{$i = 1$ \KwTo $B$}{
    \tcp{Selection}
    $\text{node} \leftarrow \text{root}$\;
    \While{node 已完全扩展 且 不是终止状态}{
        $a \leftarrow \argmax_a \text{UCB}(\text{node}, a)$\;
        $\text{node} \leftarrow \text{child}(\text{node}, a)$\;
    }

    \tcp{Expansion}
    \If{node 不是终止状态}{
        选择一个未扩展的动作 $a$\;
        $\text{node} \leftarrow$ 扩展子节点 $\text{child}(\text{node}, a)$\;
    }

    \tcp{Evaluation}
    $v \leftarrow \text{Evaluate}(\text{node})$ \Comment{Rollout 或价值网络}

    \tcp{Backup}
    \While{node $\neq$ null}{
        $N(\text{node}) \leftarrow N(\text{node}) + 1$\;
        $\Qval(\text{node}) \leftarrow \Qval(\text{node}) + \frac{v - \Qval(\text{node})}{N(\text{node})}$\;
        $\text{node} \leftarrow \text{parent}(\text{node})$\;
    }
}
\Return{$\argmax_a N(\text{root}, a)$} \Comment{选择访问次数最多的动作}
\end{algorithm}

\begin{important}
MCTS 最终选择动作的标准:
\begin{itemize}
    \item 训练/搜索时:用 UCB(平衡探索-利用)
    \item 最终决策时:选择\textbf{访问次数最多}的动作(更稳健)
\end{itemize}
访问次数而非平均价值,因为高访问次数意味着高置信度。
\end{important}

% ------------------------------------------
\section{AlphaGo 与 AlphaZero}
\label{sec:alphago}

AlphaGo 和 AlphaZero 是 MCTS + 深度学习 + Self-Play 的里程碑式成果。

\subsection{围棋的挑战}

围棋被认为是 AI 最难攻克的棋类游戏:

\begin{itemize}
    \item \textbf{搜索空间巨大}:平均每步有 $\sim 200$ 种合法走法,一局棋约 $200$ 步,总状态数 $\sim 10^{170}$
    \item \textbf{局面评估困难}:不像国际象棋有明确的子力价值,围棋的局面优劣难以量化
    \item \textbf{长期规划}:需要考虑数十步后的战略影响
\end{itemize}

传统围棋 AI 使用穷举搜索 + 手工评估函数,水平仅达业余段位。

\subsection{AlphaGo 架构(2016)}

AlphaGo 在 2016 年以 4:1 击败世界冠军李世石,其架构包括:

\begin{figure}[H]
    \centering
    \begin{tikzpicture}[
        box/.style={draw, rounded corners, minimum width=3cm, minimum height=1.2cm, align=center},
        arrow/.style={->, thick, >=stealth}
    ]
        % 输入
        \node[box, fill=blue!20] (input) at (0, 0) {棋盘状态\\$19 \times 19$};

        % Policy Network
        \node[box, fill=green!20] (pn) at (-3.5, -2.5) {Policy Network\\$p_\theta(a|s)$};

        % Value Network
        \node[box, fill=orange!20] (vn) at (3.5, -2.5) {Value Network\\$v_\phi(s)$};

        % MCTS
        \node[box, fill=purple!20, minimum width=4cm] (mcts) at (0, -5) {MCTS 搜索};

        % 输出
        \node[box, fill=red!15] (output) at (0, -7.5) {最终动作};

        % 连接
        \draw[arrow] (input) -- (pn);
        \draw[arrow] (input) -- (vn);
        \draw[arrow] (pn) -- node[left, font=\small] {指导选择} (mcts);
        \draw[arrow] (vn) -- node[right, font=\small] {评估叶节点} (mcts);
        \draw[arrow] (mcts) -- (output);

        % 训练方式标注
        \node[font=\scriptsize, align=left] at (-6.5, -2.5) {监督学习\\(人类棋谱)\\+ RL 微调};
        \node[font=\scriptsize, align=right] at (6.5, -2.5) {监督学习\\(自我对弈\\结果预测)};
    \end{tikzpicture}
    \caption{AlphaGo 架构}
    \label{fig:alphago}
\end{figure}

\begin{enumerate}
    \item \textbf{Policy Network} $p_\theta(a|s)$:
    \begin{itemize}
        \item 输入:棋盘状态(多通道特征)
        \item 输出:每个位置的落子概率
        \item 训练:先用人类棋谱监督学习,再用 Policy Gradient 自我对弈强化
    \end{itemize}

    \item \textbf{Value Network} $v_\phi(s)$:
    \begin{itemize}
        \item 输入:棋盘状态
        \item 输出:当前局面的胜率估计 $v_\phi(s) \approx \E[z|s]$,其中 $z \in \{-1, +1\}$
        \item 训练:用自我对弈生成的 $(s, z)$ 数据监督学习
    \end{itemize}

    \item \textbf{改进的 MCTS}:
    \begin{itemize}
        \item Selection:使用 Policy Network 指导(PUCT 公式)
        \[
            \text{UCB}(s,a) = \Qval(s,a) + c \cdot p_\theta(a|s) \cdot \frac{\sqrt{N(s)}}{1 + N(s,a)}
        \]
        \item Evaluation:混合 Value Network 和 rollout
        \[
            v = (1-\lambda) v_\phi(s) + \lambda z_{\text{rollout}}
        \]
    \end{itemize}
\end{enumerate}

\subsection{AlphaZero 的简化与超越(2017)}

AlphaZero 在 2017 年大幅简化了 AlphaGo 的设计,却取得了更强的性能:

\begin{table}[H]
    \centering
    \begin{tabular}{@{}lcc@{}}
        \toprule
        & \textbf{AlphaGo} & \textbf{AlphaZero} \\
        \midrule
        人类棋谱 & 需要(监督预训练) & \textbf{不需要} \\
        网络结构 & Policy + Value 分离 & \textbf{统一网络} \\
        Rollout & 需要 & \textbf{不需要} \\
        特征工程 & 手工设计特征 & \textbf{原始棋盘输入} \\
        训练时间 & 数月 & \textbf{数小时} \\
        适用游戏 & 仅围棋 & \textbf{围棋、国际象棋、将棋} \\
        \bottomrule
    \end{tabular}
    \caption{AlphaGo 与 AlphaZero 的对比}
\end{table}

\begin{figure}[H]
    \centering
    \begin{tikzpicture}[
        box/.style={draw, rounded corners, minimum width=3.5cm, minimum height=1cm, align=center},
        arrow/.style={->, thick, >=stealth}
    ]
        % 统一网络
        \node[box, fill=blue!20] (input) at (0, 0) {棋盘状态 $s$};
        \node[box, fill=purple!25, minimum height=2cm] (net) at (0, -2.5) {ResNet\\(统一网络)};

        % 双头输出
        \node[box, fill=green!20] (policy) at (-2.5, -5) {$p_\theta(a|s)$\\策略头};
        \node[box, fill=orange!20] (value) at (2.5, -5) {$v_\theta(s)$\\价值头};

        \draw[arrow] (input) -- (net);
        \draw[arrow] (net) -- (policy);
        \draw[arrow] (net) -- (value);

        \node[font=\small, align=center] at (0, -6.5) {单个网络同时输出策略分布和价值估计\\共享底层表示,参数更少,训练更高效};
    \end{tikzpicture}
    \caption{AlphaZero 的统一网络架构}
    \label{fig:alphazero-net}
\end{figure}

\subsection{AlphaZero 训练循环}

AlphaZero 的训练是一个\textbf{自我增强}的循环:

\begin{figure}[H]
    \centering
    \begin{tikzpicture}[
        box/.style={draw, rounded corners, minimum width=3cm, minimum height=1.2cm, align=center},
        arrow/.style={->, very thick, >=stealth}
    ]
        % 三个组件
        \node[box, fill=green!20] (selfplay) at (0, 0) {Self-Play\\生成对弈数据};
        \node[box, fill=orange!20] (train) at (5, -3) {网络训练\\学习搜索结果};
        \node[box, fill=blue!20] (network) at (-5, -3) {神经网络\\$(p_\theta, v_\theta)$};

        % 循环箭头
        \draw[arrow, green!60!black] (selfplay) -- node[right, font=\small, pos=0.5] {$(s, \policy_{\text{MCTS}}, z)$} (train);
        \draw[arrow, orange] (train) -- node[below, font=\small, yshift=-3pt] {更新 $\theta$} (network);
        \draw[arrow, blue] (network) -- node[left, font=\small, pos=0.5] {指导搜索} (selfplay);

        % 中心说明 - 上移避免与底部箭头重叠
        \node[font=\small, align=center, text=gray] at (0, -1.8) {正向循环\\不断变强};
    \end{tikzpicture}
    \caption{AlphaZero 的自我增强训练循环}
    \label{fig:alphazero-loop}
\end{figure}

\begin{algorithm}[H]
\caption{AlphaZero 训练}
\label{alg:alphazero}
初始化网络参数 $\theta$(随机初始化)\;
\Repeat{收敛}{
    \tcp{Self-Play 生成数据}
    \For{每局对弈}{
        \For{每步 $t$}{
            用当前网络 + MCTS 搜索,得到 $\policy_{\text{MCTS}}(a|s_t)$\;
            按 $\policy_{\text{MCTS}}$ 采样动作 $a_t$\;
            记录 $(s_t, \policy_{\text{MCTS}})$\;
        }
        游戏结束,得到胜负 $z \in \{-1, +1\}$\;
        将 $(s_t, \policy_{\text{MCTS}}, z)$ 加入训练数据\;
    }

    \tcp{网络训练}
    从训练数据中采样 batch\;
    最小化损失:
    \[
        L(\theta) = \underbrace{(z - v_\theta(s))^2}_{\text{价值损失}} - \underbrace{\policy_{\text{MCTS}}^\top \log p_\theta(s)}_{\text{策略损失}} + \underbrace{c \|\theta\|^2}_{\text{正则化}}
    \]
}
\end{algorithm}

\begin{keypoint}
AlphaZero 的核心洞察:
\begin{enumerate}
    \item \textbf{MCTS 作为策略改进}:搜索产生的 $\policy_{\text{MCTS}}$ 比原始网络 $p_\theta$ 更好
    \item \textbf{网络学习搜索}:网络被训练去模仿 MCTS 的输出
    \item \textbf{正向循环}:更好的网络 $\to$ 更好的搜索 $\to$ 更好的训练数据 $\to$ 更好的网络
\end{enumerate}
这个循环不需要任何人类知识,完全从零开始(tabula rasa)学习。
\end{keypoint}

% ------------------------------------------
\section{Multi-Agent RL 基础}
\label{sec:marl}

当环境中存在多个 agent 时,问题变得更加复杂。本节介绍 Multi-Agent RL 的基础概念。

\subsection{从单 Agent 到多 Agent}

\begin{quote}
\textbf{核心问题}:当其他 agent 也在学习和改变策略时,环境对单个 agent 来说是\textbf{非稳态}的。这打破了 MDP 的基本假设。
\end{quote}

\begin{definition}[Multi-Agent 环境的非稳态性]
在 Multi-Agent 环境中,状态转移和奖励不仅依赖于自己的动作,还依赖于其他 agent 的动作:
\[
    P(s'|s, a_1, a_2, \ldots, a_n), \quad R_i(s, a_1, a_2, \ldots, a_n)
\]
当其他 agent 的策略 $\policy_{-i}$ 在学习过程中变化时,从 agent $i$ 的视角看,环境是非稳态的。
\end{definition}

非稳态性的影响:
\begin{itemize}
    \item 单 agent RL 的收敛性保证不再适用
    \item 最优响应策略随对手策略而变化
    \item 可能出现策略震荡,无法收敛
\end{itemize}

\subsection{博弈论基础}

Multi-Agent 问题可以用博弈论的语言来描述。

\begin{definition}[Normal-form Game]
一个 $n$ 人博弈由以下元素组成:
\begin{itemize}
    \item \textbf{玩家集合}:$\mathcal{N} = \{1, 2, \ldots, n\}$
    \item \textbf{策略空间}:每个玩家 $i$ 的策略集合 $\mathcal{A}_i$
    \item \textbf{效用函数}:$u_i: \mathcal{A}_1 \times \cdots \times \mathcal{A}_n \to \R$,表示每种策略组合下玩家 $i$ 的收益
\end{itemize}
\end{definition}

\begin{example}[囚徒困境]
两个嫌犯被分开审讯,各自选择"合作"(沉默)或"背叛"(坦白):

\begin{center}
\begin{tabular}{cc|c|c|}
    & \multicolumn{1}{c}{} & \multicolumn{2}{c}{玩家 2} \\
    & \multicolumn{1}{c}{} & \multicolumn{1}{c}{合作} & \multicolumn{1}{c}{背叛} \\
    \cline{3-4}
    \multirow{2}{*}{玩家 1} & 合作 & $(-1, -1)$ & $(-3, 0)$ \\
    \cline{3-4}
    & 背叛 & $(0, -3)$ & $(-2, -2)$ \\
    \cline{3-4}
\end{tabular}
\end{center}

分析:
\begin{itemize}
    \item 无论对方如何选择,"背叛"对自己都更有利(支配策略)
    \item 结果:双方都背叛,各获 $-2$
    \item 但如果双方都合作,各获 $-1$(帕累托更优)
\end{itemize}
\end{example}

\subsection{合作与竞争设定}

Multi-Agent 场景主要分为两类:

\begin{enumerate}
    \item \textbf{合作(Cooperative)}:所有 agent 共享奖励,最大化团队总收益
    \begin{itemize}
        \item 例:多机器人协作搬运、多 agent 协调导航
        \item 挑战:信用分配(Credit Assignment)——哪个 agent 贡献了多少?
        \item 方法:集中训练、分布式执行(CTDE)
    \end{itemize}

    \item \textbf{竞争/零和(Competitive/Zero-sum)}:一方获益等于另一方损失
    \begin{itemize}
        \item 例:棋类游戏、对抗博弈
        \item 特点:$u_1 + u_2 = 0$
        \item 目标:找到 Nash 均衡
    \end{itemize}
\end{enumerate}

\subsection{Nash 均衡}

\begin{definition}[Nash 均衡]
策略组合 $(\policy_1^*, \policy_2^*, \ldots, \policy_n^*)$ 是 \textbf{Nash 均衡},当且仅当没有任何玩家有动机单方面改变自己的策略:
\begin{equation}
    \forall i, \forall \policy_i: \quad u_i(\policy_i^*, \policy_{-i}^*) \geq u_i(\policy_i, \policy_{-i}^*)
\end{equation}
其中 $\policy_{-i}^*$ 表示除玩家 $i$ 外所有玩家的策略。
\end{definition}

\begin{note}
Nash 均衡的含义:
\begin{itemize}
    \item 每个玩家都在对其他玩家的策略做\textbf{最优响应}
    \item 是一种\textbf{稳定状态}:没有人有动机单方面偏离
    \item 不一定是全局最优(如囚徒困境中双方背叛是 Nash 均衡,但不是帕累托最优)
\end{itemize}
\end{note}

\begin{example}[石头剪刀布的 Nash 均衡]
石头剪刀布是一个零和博弈:

\begin{center}
\begin{tabular}{cc|c|c|c|}
    & \multicolumn{1}{c}{} & \multicolumn{3}{c}{玩家 2} \\
    & \multicolumn{1}{c}{} & \multicolumn{1}{c}{石头} & \multicolumn{1}{c}{剪刀} & \multicolumn{1}{c}{布} \\
    \cline{3-5}
    \multirow{3}{*}{玩家 1} & 石头 & $(0, 0)$ & $(1, -1)$ & $(-1, 1)$ \\
    \cline{3-5}
    & 剪刀 & $(-1, 1)$ & $(0, 0)$ & $(1, -1)$ \\
    \cline{3-5}
    & 布 & $(1, -1)$ & $(-1, 1)$ & $(0, 0)$ \\
    \cline{3-5}
\end{tabular}
\end{center}

Nash 均衡:双方都以 $(\frac{1}{3}, \frac{1}{3}, \frac{1}{3})$ 的概率随机选择。

任何确定性策略都可以被对手利用,只有\textbf{混合策略}(随机化)才能达到均衡。
\end{example}

\begin{theorem}[Nash 均衡存在性]
每个有限博弈(有限玩家、有限策略)至少存在一个 Nash 均衡(可能是混合策略均衡)。
\end{theorem}

% ------------------------------------------
\section{Self-Play 方法}
\label{sec:self-play}

Self-Play 是训练博弈 AI 的强大方法,也是 AlphaGo/AlphaZero 成功的关键之一。

\subsection{Self-Play 的定义}

\begin{definition}[Self-Play]
Agent 与自己(或自己的历史版本)进行对弈,从对弈经验中学习改进策略。
\end{definition}

\begin{figure}[H]
    \centering
    \begin{tikzpicture}[
        box/.style={draw, rounded corners, minimum width=2.5cm, minimum height=1cm, align=center},
        arrow/.style={->, thick, >=stealth}
    ]
        % 当前策略
        \node[box, fill=blue!20] (current) at (0, 0) {当前策略\\$\policy_\theta$};

        % 对手(自己的副本)
        \node[box, fill=blue!10] (opponent) at (5, 0) {对手\\$\policy_\theta$ 或 $\policy_{\theta'}$};

        % 对弈
        \node[box, fill=green!20] (game) at (2.5, -2.5) {对弈};

        % 经验
        \node[box, fill=orange!20] (exp) at (2.5, -5) {对弈经验\\$(s, a, r, s')$};

        % 更新
        \draw[arrow] (current) -- (game);
        \draw[arrow] (opponent) -- (game);
        \draw[arrow] (game) -- (exp);
        \draw[arrow] (exp) to[out=180, in=270] node[left, font=\small] {更新} (current);

        % 可能的历史对手池
        \node[box, fill=gray!20, dashed] (pool) at (8, -2.5) {历史对手池\\$\{\policy_{\theta_1}, \ldots\}$};
        \draw[arrow, dashed] (pool) -- (opponent);
    \end{tikzpicture}
    \caption{Self-Play 训练示意图}
    \label{fig:self-play}
\end{figure}

\subsection{Self-Play 的优势}

\begin{enumerate}
    \item \textbf{无限数据}:可以生成任意多的对弈数据,不受人类对局数量限制

    \item \textbf{自适应难度}:对手随自己一起变强,始终提供适当的挑战
    \begin{itemize}
        \item 初期:对手弱,容易学习基本策略
        \item 后期:对手强,推动学习高级策略
    \end{itemize}

    \item \textbf{发现新策略}:不受人类先验知识限制,可能发现人类未知的创新策略
    \begin{itemize}
        \item AlphaGo 的"肩冲"等新招法震惊了职业棋手
    \end{itemize}

    \item \textbf{逼近 Nash 均衡}:在零和博弈中,Self-Play 在理论上收敛到 Nash 均衡
\end{enumerate}

\subsection{Self-Play 的挑战}

\begin{enumerate}
    \item \textbf{策略遗忘}:
    \begin{itemize}
        \item 当策略更新后,可能"忘记"如何对付旧策略
        \item 解决:维护历史对手池(Opponent Pool),随机抽取对手
    \end{itemize}

    \item \textbf{局部最优}:
    \begin{itemize}
        \item 可能陷入"只擅长对付自己"的局部最优
        \item 例:两个版本互相克制,形成循环
        \item 解决:添加多样性奖励,或从对手池采样
    \end{itemize}

    \item \textbf{评估困难}:
    \begin{itemize}
        \item 没有固定基准来衡量进步
        \item 解决:用 Elo 评分系统或与固定对手对战
    \end{itemize}
\end{enumerate}

\begin{keypoint}
Self-Play 与 Nash 均衡的关系:
\begin{itemize}
    \item 在两人零和博弈中,如果 Self-Play 收敛,则收敛到 Nash 均衡
    \item 直觉:Nash 均衡是"最优响应的不动点",Self-Play 就是迭代求最优响应
    \item 但收敛不保证——可能出现策略循环
\end{itemize}
\end{keypoint}

% ------------------------------------------
\section{本章总结}
\label{sec:chap4-summary}

\begin{keypoint}
本章核心内容:
\begin{enumerate}
    \item \textbf{Model-Based RL} 利用环境模型提高样本效率
    \begin{itemize}
        \item World Model = 状态转移 + 奖励模型
        \item Model Bias:模型误差会累积,需要短期规划或持续校正
    \end{itemize}

    \item \textbf{Dyna 架构}结合直接学习与规划
    \begin{itemize}
        \item 每次真实交互后,用模型生成 $n$ 次模拟经验
        \item 提供计算-样本的灵活权衡
    \end{itemize}

    \item \textbf{MCTS} 是 Decision-time Planning 的代表
    \begin{itemize}
        \item 四步流程:Selection, Expansion, Evaluation, Backup
        \item UCB 公式平衡探索与利用
    \end{itemize}

    \item \textbf{AlphaGo/AlphaZero} 展示了 MCTS + 深度学习 + Self-Play 的强大组合
    \begin{itemize}
        \item AlphaZero 从零开始,无需人类知识
        \item 核心循环:MCTS 改进策略 $\to$ 网络学习搜索 $\to$ 正向增强
    \end{itemize}

    \item \textbf{Multi-Agent RL} 面临非稳态性挑战
    \begin{itemize}
        \item Nash 均衡:稳定的策略组合
        \item Self-Play:训练博弈 AI 的有效方法
    \end{itemize}
\end{enumerate}
\end{keypoint}

\begin{figure}[H]
    \centering
    \begin{tikzpicture}[
        box/.style={draw, rounded corners, fill=blue!10, minimum width=2.5cm, minimum height=0.8cm, align=center, font=\small},
        arrow/.style={->, thick, >=stealth}
    ]
        % 层次结构
        \node[box, fill=blue!25] (rl) at (0, 0) {RL 方法};

        \node[box, fill=green!20] (mf) at (-4, -1.5) {Model-Free};
        \node[box, fill=orange!20] (mb) at (4, -1.5) {Model-Based};

        \node[box, minimum width=2.2cm] (vb) at (-6.5, -3) {Value-Based};
        \node[box, minimum width=2.2cm] (pb) at (-4, -3) {Policy-Based};
        \node[box, minimum width=2.2cm] (ac) at (-1.5, -3) {Actor-Critic};

        \node[box] (bg) at (2.5, -3) {Background\\Planning};
        \node[box] (dt) at (5.5, -3) {Decision-time\\Planning};

        \node[font=\scriptsize, gray] at (-6.5, -3.8) {DQN};
        \node[font=\scriptsize, gray] at (-4, -3.8) {REINFORCE};
        \node[font=\scriptsize, gray] at (-1.5, -3.8) {PPO, SAC};
        \node[font=\scriptsize, gray] at (2.5, -3.8) {Dyna};
        \node[font=\scriptsize, gray] at (5.5, -3.8) {MCTS};

        % AlphaZero 横跨
        \node[box, fill=purple!20, minimum width=3cm] (az) at (4, -5) {AlphaZero};

        \draw[arrow] (rl) -- (mf);
        \draw[arrow] (rl) -- (mb);
        \draw[arrow] (mf) -- (vb);
        \draw[arrow] (mf) -- (pb);
        \draw[arrow] (mf) -- (ac);
        \draw[arrow] (mb) -- (bg);
        \draw[arrow] (mb) -- (dt);
        \draw[arrow, dashed] (dt) -- (az);
        \draw[arrow, dashed] (ac.south) to[out=-45, in=180] (az.west);

        \node[font=\scriptsize, align=center] at (0, -5.5) {AlphaZero = MCTS + Policy Network + Value Network + Self-Play};
    \end{tikzpicture}
    \caption{本章内容在 RL 算法体系中的位置}
    \label{fig:chap4-overview}
\end{figure}
