% ==========================================
% 第三章:基于策略的强化学习
% ==========================================

\chapter{基于策略的强化学习}
\label{chap:policy-based}

% ------------------------------------------
\section{引言:直接优化策略}
\label{sec:policy-intro}

\subsection{核心问题}

在第二章中,我们介绍了 Value-Based 方法:先学习 $\Qval^*$,再通过 $\argmax$ 导出策略。这种方法在离散动作空间中效果很好,但遇到以下问题时会遇到困难:

\begin{quote}
\textbf{如果动作空间是连续的(如机器人关节角度),如何计算 $\argmax_a \Qval(s,a)$?}

\textbf{如果最优策略是随机的(如石头剪刀布),如何用确定性策略表示?}
\end{quote}

Policy-Based 方法提供了一种更直接的思路:\textbf{直接参数化策略 $\policy_\theta(a|s)$,通过梯度上升最大化期望回报}。

\subsection{Value-Based 方法的局限性}

尽管 Q-Learning 和 DQN 取得了很大成功,但存在以下局限:

\begin{enumerate}
    \item \textbf{连续动作空间困难}:$\max_a \Qval(s,a)$ 需要枚举或优化所有动作,在连续空间中无法直接计算

    \item \textbf{函数逼近不稳定}(Deadly Triad):当同时使用函数逼近、Bootstrapping 和 Off-policy 学习时,算法可能发散

    \item \textbf{优化目标间接}:Value-Based 方法最小化 TD 误差,而非直接优化期望回报 $J(\policy)$

    \item \textbf{只能学习确定性策略}:$\argmax$ 输出确定动作,但在某些环境中随机策略更优(如博弈、部分可观测环境)
\end{enumerate}

\begin{figure}[H]
    \centering
    \begin{tikzpicture}[
        box/.style={draw, rounded corners, fill=blue!10, minimum width=3cm, minimum height=1.2cm, align=center},
        arrow/.style={->, thick, >=stealth}
    ]
        % Value-Based
        \begin{scope}[shift={(-4,0)}]
            \node[box, fill=orange!20] (q) at (0,0) {学习 $\Qval^*$};
            \node[box, fill=green!20] (pi) at (0,-2.5) {$\policy^* = \argmax$};
            \draw[arrow] (q) -- (pi);
            \node[font=\bfseries] at (0, 1.2) {Value-Based};
            \node[font=\scriptsize, red] at (0, -3.8) {需要枚举动作};
        \end{scope}

        % Policy-Based
        \begin{scope}[shift={(4,0)}]
            \node[box, fill=purple!20] (policy) at (0,-1.25) {直接学习 $\policy_\theta$};
            \node[font=\bfseries] at (0, 1.2) {Policy-Based};
            \node[font=\scriptsize, green!60!black] at (0, -3.8) {直接输出动作分布};
        \end{scope}

        % VS
        \node[font=\Large] at (0, -1.25) {vs};
    \end{tikzpicture}
    \caption{Value-Based vs. Policy-Based。前者间接导出策略,后者直接参数化策略。}
    \label{fig:value-vs-policy}
\end{figure}

\subsection{参数化策略}

Policy-Based 方法直接参数化策略 $\policy_\theta(a|s)$,通过梯度上升最大化期望回报:
\begin{equation}
    J(\theta) = \E_{\trajectory \sim \policy_\theta} \left[ G_0 \right] = \E_{\trajectory \sim \policy_\theta} \left[ \sum_{t=0}^{T} \discount^t r_t \right]
\end{equation}

策略参数化的常见形式:

\begin{itemize}
    \item \textbf{离散动作空间}:Softmax 输出 Categorical 分布
    \begin{equation}
        \policy_\theta(a|s) = \frac{\exp(f_\theta(s,a))}{\sum_{a'} \exp(f_\theta(s,a'))}
    \end{equation}

    \item \textbf{连续动作空间}:输出高斯分布的参数 $(\mu_\theta(s), \sigma_\theta(s))$
    \begin{equation}
        \policy_\theta(a|s) = \mathcal{N}(a \mid \mu_\theta(s), \sigma_\theta^2(s))
    \end{equation}
\end{itemize}

\begin{figure}[H]
    \centering
    \begin{tikzpicture}[scale=0.9]
        % Discrete
        \begin{scope}[shift={(-4,0)}]
            \node[font=\bfseries] at (0, 2) {离散动作};
            \draw[->] (-1.5, 0) -- (1.5, 0) node[right, font=\scriptsize] {动作};
            \draw[->] (-1.5, 0) -- (-1.5, 1.5) node[above, font=\scriptsize] {概率};

            % Bars
            \fill[blue!60] (-1, 0) rectangle (-0.5, 1.2);
            \fill[blue!60] (-0.2, 0) rectangle (0.3, 0.6);
            \fill[blue!60] (0.6, 0) rectangle (1.1, 0.3);

            \node[font=\tiny] at (-0.75, -0.3) {$a_1$};
            \node[font=\tiny] at (0.05, -0.3) {$a_2$};
            \node[font=\tiny] at (0.85, -0.3) {$a_3$};

            \node[font=\scriptsize] at (0, -1) {Softmax 输出};
        \end{scope}

        % Continuous
        \begin{scope}[shift={(4,0)}]
            \node[font=\bfseries] at (0, 2) {连续动作};
            \draw[->] (-2, 0) -- (2, 0) node[right, font=\scriptsize] {动作 $a$};
            \draw[->] (-2, 0) -- (-2, 1.5) node[above, font=\scriptsize] {概率密度};

            % Gaussian curve
            \draw[thick, blue, domain=-1.8:1.8, samples=50] plot (\x, {1.2*exp(-\x*\x/0.5)});

            \draw[dashed] (0, 0) -- (0, 1.2);
            \node[font=\tiny] at (0, -0.3) {$\mu_\theta(s)$};
            \draw[<->] (-0.7, 0.7) -- (0.7, 0.7);
            \node[font=\tiny] at (0, 0.9) {$\sigma_\theta(s)$};

            \node[font=\scriptsize] at (0, -1) {高斯分布输出};
        \end{scope}
    \end{tikzpicture}
    \caption{策略参数化:离散动作用 Softmax,连续动作用高斯分布}
    \label{fig:policy-parameterization}
\end{figure}

\begin{keypoint}
Policy-Based 方法的优势:
\begin{enumerate}
    \item \textbf{处理连续动作}:直接输出动作分布,无需 $\argmax$
    \item \textbf{学习随机策略}:可以输出动作的概率分布
    \item \textbf{直接优化目标}:梯度上升直接最大化 $J(\theta)$
    \item \textbf{更好的收敛性质}:策略参数的小变化导致策略的小变化(光滑)
\end{enumerate}
\end{keypoint}

% ------------------------------------------
\section{Policy Gradient 定理}
\label{sec:policy-gradient}

Policy Gradient 定理是 Policy-Based RL 的理论基础,它给出了目标函数 $J(\theta)$ 关于参数 $\theta$ 的梯度表达式。

\subsection{目标函数定义}

\begin{definition}[策略性能指标]
\begin{equation}
    J(\theta) = \E_{\trajectory \sim \policy_\theta} \left[ R(\trajectory) \right] = \int p(\trajectory | \theta) R(\trajectory) d\trajectory
\end{equation}
其中 $R(\trajectory) = G_0 = \sum_{t=0}^{T} \discount^t r_t$ 是轨迹的总回报。
\end{definition}

问题:如何计算 $\nabla_\theta J(\theta)$?

直接求导会遇到困难:$R(\trajectory)$ 依赖于环境动力学 $P(s'|s,a)$,而我们通常不知道 $P$。

\subsection{Log-Derivative Trick}

计算 $\nabla_\theta J(\theta)$ 的关键技巧是 \textbf{Log-Derivative Trick}(也称为 REINFORCE trick 或 Score Function):

\begin{lemma}[Log-Derivative Trick]
\label{lem:log-derivative}
对于任意概率分布 $p(x|\theta)$:
\begin{equation}
    \nabla_\theta p(x|\theta) = p(x|\theta) \nabla_\theta \log p(x|\theta)
\end{equation}
\end{lemma}

\begin{proof}
由对数的求导法则(链式法则):
\begin{equation}
    \nabla_\theta \log p(x|\theta) = \frac{\nabla_\theta p(x|\theta)}{p(x|\theta)}
\end{equation}
两边同乘 $p(x|\theta)$:
\begin{equation}
    p(x|\theta) \nabla_\theta \log p(x|\theta) = \nabla_\theta p(x|\theta)
\end{equation}
\end{proof}

\begin{note}
Log-Derivative Trick 的妙处:将对 $p(x|\theta)$ 的求导转化为对 $\log p(x|\theta)$ 的求导,而后者往往更容易计算,特别是当 $p$ 是乘积形式时。
\end{note}

\subsection{Policy Gradient 定理的完整推导}

\begin{theorem}[Policy Gradient Theorem]
\label{thm:policy-gradient}
\begin{equation}
    \nabla_\theta J(\theta) = \E_{\trajectory \sim \policy_\theta} \left[ \sum_{t=0}^{T} \nabla_\theta \log \policy_\theta(a_t|s_t) \cdot G_t \right]
    \label{eq:policy-gradient}
\end{equation}
其中 $G_t = \sum_{k=0}^{T-t} \discount^k r_{t+k}$ 是从时刻 $t$ 开始的 reward-to-go。
\end{theorem}

\begin{proof}
\textbf{Step 1}:应用 Log-Derivative Trick

从目标函数的定义出发:
\begin{align}
    \nabla_\theta J(\theta) &= \nabla_\theta \int p(\trajectory|\theta) R(\trajectory) d\trajectory \\
    &= \int \nabla_\theta p(\trajectory|\theta) R(\trajectory) d\trajectory \quad \text{(交换积分和求导)}\\
    &= \int p(\trajectory|\theta) \nabla_\theta \log p(\trajectory|\theta) R(\trajectory) d\trajectory \quad \text{(Log-Derivative Trick)}\\
    &= \E_{\trajectory \sim \policy_\theta} \left[ \nabla_\theta \log p(\trajectory|\theta) \cdot R(\trajectory) \right]
\end{align}

\textbf{Step 2}:展开 $\nabla_\theta \log p(\trajectory|\theta)$

回顾轨迹概率分解(第一章公式 \eqref{eq:trajectory-prob}):
\begin{equation}
    p(\trajectory|\theta) = p(s_0) \prod_{t=0}^{T-1} \policy_\theta(a_t|s_t) P(s_{t+1}|s_t,a_t)
\end{equation}

取对数:
\begin{equation}
    \log p(\trajectory|\theta) = \log p(s_0) + \sum_{t=0}^{T-1} \log \policy_\theta(a_t|s_t) + \sum_{t=0}^{T-1} \log P(s_{t+1}|s_t,a_t)
\end{equation}

对 $\theta$ 求梯度:
\begin{equation}
    \nabla_\theta \log p(\trajectory|\theta) = \underbrace{\nabla_\theta \log p(s_0)}_{=0} + \sum_{t=0}^{T-1} \nabla_\theta \log \policy_\theta(a_t|s_t) + \underbrace{\sum_{t=0}^{T-1} \nabla_\theta \log P(s_{t+1}|s_t,a_t)}_{=0}
\end{equation}

\begin{important}
\textbf{关键观察}:$p(s_0)$ 是环境的初始状态分布,$P(s_{t+1}|s_t,a_t)$ 是环境的动力学模型,它们都与策略参数 $\theta$ 无关,因此梯度为零!

最终:
\begin{equation}
    \nabla_\theta \log p(\trajectory|\theta) = \sum_{t=0}^{T-1} \nabla_\theta \log \policy_\theta(a_t|s_t)
    \label{eq:grad-log-trajectory}
\end{equation}

这意味着:\textbf{即使不知道环境动力学 $P$,我们也能计算 Policy Gradient}——这是 Policy Gradient 方法能够 Model-Free 的根本原因。
\end{important}

\textbf{Step 3}:代入得到基本形式

将 \eqref{eq:grad-log-trajectory} 代入:
\begin{equation}
    \nabla_\theta J(\theta) = \E_{\trajectory \sim \policy_\theta} \left[ \left( \sum_{t=0}^{T} \nabla_\theta \log \policy_\theta(a_t|s_t) \right) \cdot R(\trajectory) \right]
    \label{eq:pg-basic}
\end{equation}

\textbf{Step 4}:引入 Reward-to-go(因果性)

展开 \eqref{eq:pg-basic}:
\begin{equation}
    \nabla_\theta J(\theta) = \E \left[ \sum_{t=0}^{T} \nabla_\theta \log \policy_\theta(a_t|s_t) \cdot \sum_{t'=0}^{T} r_{t'} \right]
\end{equation}

考虑交叉项 $\nabla_\theta \log \policy_\theta(a_t|s_t) \cdot r_{t'}$,当 $t' < t$ 时:
\begin{itemize}
    \item $r_{t'}$ 是在时刻 $t'$ 获得的奖励,发生在动作 $a_t$ 之前
    \item $a_t$ 的选择不可能影响过去的奖励 $r_{t'}$
    \item 因此 $\E[\nabla_\theta \log \policy_\theta(a_t|s_t) \cdot r_{t'}] = 0$(由引理 \ref{lem:score-zero})
\end{itemize}

只有 $t' \geq t$ 的奖励才与 $a_t$ 相关,因此可以用 reward-to-go $G_t = \sum_{k=0}^{T-t} \discount^k r_{t+k}$ 替代 $R(\trajectory)$:
\begin{equation}
    \nabla_\theta J(\theta) = \E_{\trajectory \sim \policy_\theta} \left[ \sum_{t=0}^{T} \nabla_\theta \log \policy_\theta(a_t|s_t) \cdot G_t \right]
\end{equation}
\end{proof}

\begin{lemma}[Score Function 的期望为零]
\label{lem:score-zero}
对于任意策略 $\policy_\theta$:
\begin{equation}
    \E_{a \sim \policy_\theta(\cdot|s)} \left[ \nabla_\theta \log \policy_\theta(a|s) \right] = 0
\end{equation}
\end{lemma}

\begin{proof}
\begin{align}
    \E_{a \sim \policy_\theta} \left[ \nabla_\theta \log \policy_\theta(a|s) \right]
    &= \sum_a \policy_\theta(a|s) \cdot \frac{\nabla_\theta \policy_\theta(a|s)}{\policy_\theta(a|s)} \\
    &= \sum_a \nabla_\theta \policy_\theta(a|s) \\
    &= \nabla_\theta \sum_a \policy_\theta(a|s) \\
    &= \nabla_\theta 1 = 0
\end{align}
\end{proof}

\begin{figure}[H]
    \centering
    \begin{tikzpicture}[scale=0.85]
        % Timeline
        \draw[->, thick] (0,0) -- (12,0) node[right] {时间};

        % Time points
        \foreach \x/\t in {1/0, 3/1, 5/2, 7/t, 9/{t+1}, 11/T} {
            \fill (\x, 0) circle (2pt);
            \node[below] at (\x, -0.2) {$\t$};
        }

        % Actions
        \foreach \x in {1, 3, 5, 7, 9} {
            \node[above, font=\scriptsize, blue] at (\x, 0.3) {$a$};
        }

        % Rewards
        \foreach \x in {2, 4, 6, 8, 10} {
            \node[above, font=\scriptsize, red] at (\x, 0.6) {$r$};
        }

        % Highlight current action
        \fill[blue!30] (6.7, -0.3) rectangle (7.3, 0.3);
        \node[below, font=\small] at (7, -0.6) {当前动作 $a_t$};

        % Past rewards (not affected)
        \draw[decorate, decoration={brace, amplitude=5pt, raise=5pt, mirror}] (1,-0.8) -- (6,-0.8)
            node[midway, below=10pt, font=\scriptsize] {过去奖励:与 $a_t$ 无关};

        % Future rewards (affected)
        \draw[decorate, decoration={brace, amplitude=5pt, raise=5pt}] (7,0.9) -- (11,0.9)
            node[midway, above=8pt, font=\scriptsize] {$G_t = r_t + \gamma r_{t+1} + \cdots$};
    \end{tikzpicture}
    \caption{因果性:动作 $a_t$ 只影响未来奖励,不影响过去。因此 Policy Gradient 只需要 reward-to-go $G_t$。}
    \label{fig:causality}
\end{figure}

\begin{keypoint}
Policy Gradient 定理的直观理解:
\begin{itemize}
    \item $\nabla_\theta \log \policy_\theta(a_t|s_t)$ 是"增加动作 $a_t$ 概率"的方向
    \item $G_t$ 是该动作之后获得的累积奖励
    \item 如果 $G_t > 0$:沿梯度方向更新,增加 $a_t$ 的概率
    \item 如果 $G_t < 0$:反向更新,减少 $a_t$ 的概率
\end{itemize}
简言之:\textbf{好的动作更可能被选择,坏的动作更少被选择}。
\end{keypoint}

% ------------------------------------------
\section{REINFORCE 算法}
\label{sec:reinforce}

REINFORCE 是最简单的 Policy Gradient 算法,直接使用蒙特卡洛采样来估计梯度。

\begin{algorithm}[H]
\caption{REINFORCE}
\label{alg:reinforce}
\KwInput{学习率 $\alpha$,初始策略参数 $\theta$}
\ForEach{episode}{
    采样轨迹 $\trajectory = (s_0, a_0, r_0, \ldots, s_T)$ 按策略 $\policy_\theta$\;
    \ForEach{$t = 0, 1, \ldots, T$}{
        计算 $G_t = \sum_{k=0}^{T-t} \discount^k r_{t+k}$\;
    }
    更新参数:$\theta \leftarrow \theta + \alpha \sum_{t=0}^{T} \nabla_\theta \log \policy_\theta(a_t|s_t) \cdot G_t$\;
}
\end{algorithm}

\subsection{无偏性}

\begin{theorem}[REINFORCE 是无偏估计]
REINFORCE 的梯度估计:
\begin{equation}
    \hat{g} = \sum_{t=0}^{T} \nabla_\theta \log \policy_\theta(a_t|s_t) \cdot G_t
\end{equation}
是 $\nabla_\theta J(\theta)$ 的无偏估计,即 $\E[\hat{g}] = \nabla_\theta J(\theta)$。
\end{theorem}

\begin{proof}
这是 Policy Gradient 定理(定理 \ref{thm:policy-gradient})的直接推论。轨迹 $\tau$ 按 $\policy_\theta$ 采样,$G_t$ 是真实回报,期望就是 $\nabla_\theta J(\theta)$。
\end{proof}

\subsection{高方差问题}

尽管 REINFORCE 无偏,但方差很大:

\begin{itemize}
    \item $G_t$ 累积了从 $t$ 到终止的所有随机性(环境随机 + 策略随机)
    \item 轨迹越长,方差越大
    \item 奖励稀疏时,大部分轨迹的 $G_t \approx 0$,偶尔出现大的 $G_t$
    \item $G_t$ 包含了很多与具体动作 $a_t$ 无关的噪声
\end{itemize}

\begin{figure}[H]
    \centering
    \begin{tikzpicture}[scale=0.8]
        % Axis
        \draw[->] (0,0) -- (8,0) node[right] {Episode};
        \draw[->] (0,-2) -- (0,3) node[above] {$\hat{g}$(梯度估计)};

        % High variance samples
        \foreach \x in {1,2,3,4,5,6,7} {
            \pgfmathsetmacro{\y}{2*rand}
            \fill[blue] (\x, \y) circle (2pt);
        }

        % True gradient
        \draw[red, thick, dashed] (0, 0.5) -- (8, 0.5);
        \node[red, right] at (8, 0.5) {$\nabla J$};

        % Label
        \node[font=\scriptsize] at (4, -2.5) {高方差:每次估计波动很大};
    \end{tikzpicture}
    \caption{REINFORCE 的高方差问题示意图。虽然期望正确,但单次估计波动大。}
    \label{fig:reinforce-variance}
\end{figure}

% ------------------------------------------
\section{Baseline 与方差降低}
\label{sec:baseline}

\subsection{Baseline 技巧}

一个巧妙的技巧是:从 $G_t$ 中减去一个 baseline $b(s_t)$,可以降低方差而不引入偏差。

\begin{theorem}[Baseline 不改变期望]
\label{thm:baseline}
对于任意只依赖于状态 $s$(不依赖于动作 $a$)的函数 $b(s)$:
\begin{equation}
    \E_{a \sim \policy_\theta(\cdot|s)} \left[ \nabla_\theta \log \policy_\theta(a|s) \cdot b(s) \right] = 0
\end{equation}
\end{theorem}

\begin{proof}
由于 $b(s)$ 不依赖于 $a$,可以提出期望外:
\begin{align}
    \E_{a \sim \policy_\theta} \left[ \nabla_\theta \log \policy_\theta(a|s) \cdot b(s) \right]
    &= b(s) \cdot \E_{a \sim \policy_\theta} \left[ \nabla_\theta \log \policy_\theta(a|s) \right] \\
    &= b(s) \cdot \sum_a \policy_\theta(a|s) \cdot \frac{\nabla_\theta \policy_\theta(a|s)}{\policy_\theta(a|s)} \\
    &= b(s) \cdot \sum_a \nabla_\theta \policy_\theta(a|s) \\
    &= b(s) \cdot \nabla_\theta \underbrace{\sum_a \policy_\theta(a|s)}_{= 1} \\
    &= b(s) \cdot \nabla_\theta 1 = 0
\end{align}
\end{proof}

因此,Policy Gradient 可以写成:
\begin{equation}
    \nabla_\theta J(\theta) = \E \left[ \sum_t \nabla_\theta \log \policy_\theta(a_t|s_t) \cdot (G_t - b(s_t)) \right]
    \label{eq:pg-with-baseline}
\end{equation}

减去 baseline 不改变期望,但可以显著降低方差!

\subsection{为什么 Baseline 能降低方差?}

直觉解释:
\begin{itemize}
    \item $G_t$ 可能总是正的(如奖励都是正数),导致所有动作概率都被增加
    \item 减去 $b(s)$(如平均回报),使得 $G_t - b(s_t)$ 有正有负
    \item 好于平均的动作被增强,差于平均的动作被削弱
\end{itemize}

\begin{figure}[H]
    \centering
    \begin{tikzpicture}[scale=0.9]
        % Without baseline
        \begin{scope}[shift={(-4,0)}]
            \node[font=\bfseries] at (0, 3) {无 Baseline};
            \draw[->] (-1.5, 0) -- (1.5, 0) node[right, font=\scriptsize] {动作};
            \draw[->] (-1.5, 0) -- (-1.5, 2.3);
            \node[font=\scriptsize, anchor=south] at (-1.5, 2.3) {$G_t$};

            % All positive
            \fill[blue!60] (-1, 0) rectangle (-0.5, 1.8);
            \fill[blue!60] (-0.2, 0) rectangle (0.3, 1.2);
            \fill[blue!60] (0.6, 0) rectangle (1.1, 1.5);

            \node[font=\tiny, red] at (0, -0.5) {全正,所有动作都被增强};
        \end{scope}

        % With baseline
        \begin{scope}[shift={(4,0)}]
            \node[font=\bfseries] at (0, 3) {有 Baseline};
            \draw[->] (-1.5, 0) -- (1.5, 0) node[right, font=\scriptsize] {动作};
            \draw[->] (-1.5, -1) -- (-1.5, 2.3);
            \node[font=\scriptsize, anchor=south] at (-1.5, 2.3) {$G_t - b$};

            % Mixed
            \fill[green!60] (-1, 0) rectangle (-0.5, 1.0);
            \fill[red!60] (-0.2, 0) rectangle (0.3, -0.6);
            \fill[green!60] (0.6, 0) rectangle (1.1, 0.5);

            % Baseline
            \draw[dashed, gray] (-1.5, 0) -- (1.5, 0);
            \node[font=\tiny, gray, right] at (1.5, 0) {$b(s)$};

            \node[font=\tiny, green!60!black] at (0, -1.5) {好动作增强,坏动作削弱};
        \end{scope}
    \end{tikzpicture}
    \caption{Baseline 的效果:将"绝对好坏"变为"相对好坏",提供更清晰的学习信号。}
    \label{fig:baseline-effect}
\end{figure}

\subsection{最优 Baseline}

\begin{theorem}[最优 Baseline]
在不改变期望的前提下,使方差最小的 baseline 是状态价值函数:
\begin{equation}
    b^*(s) = \Val^\policy(s)
\end{equation}
\end{theorem}

\begin{proof}[证明思路]
方差 $\text{Var}[\hat{g}]$ 关于 $b(s)$ 是二次函数,对 $b(s)$ 求导并令其为零:
\begin{equation}
    \frac{\partial \text{Var}[\hat{g}]}{\partial b(s)} = 0 \implies b^*(s) = \frac{\E[(\nabla \log \policy)^2 \cdot G]}{\E[(\nabla \log \policy)^2]}
\end{equation}
在一定近似下($(\nabla \log \policy)^2$ 与 $G$ 独立),$b^*(s) \approx \E[G|s] = \Val^\policy(s)$。
\end{proof}

当 $b(s) = \Val^\policy(s)$ 时,$G_t - \Val^\policy(s_t)$ 的期望正是 advantage 函数!

% ------------------------------------------
\section{Advantage Function 与 Actor-Critic}
\label{sec:advantage-ac}

\subsection{Advantage 的定义与直觉}

回顾第一章的 Advantage 函数定义:
\begin{equation}
    \advantage^\policy(s,a) = \Qval^\policy(s,a) - \Val^\policy(s)
\end{equation}

当使用 $\Val^\policy(s)$ 作为 baseline 时:
\begin{align}
    \E_\policy \left[ G_t - \Val^\policy(s_t) \mid s_t, a_t \right] &= \E_\policy[G_t | s_t, a_t] - \Val^\policy(s_t) \\
    &= \Qval^\policy(s_t, a_t) - \Val^\policy(s_t) \\
    &= \advantage^\policy(s_t, a_t)
\end{align}

因此,Policy Gradient with Advantage:
\begin{equation}
    \nabla_\theta J(\theta) = \E \left[ \sum_t \nabla_\theta \log \policy_\theta(a_t|s_t) \cdot \advantage^\policy(s_t, a_t) \right]
    \label{eq:pg-advantage}
\end{equation}

\begin{keypoint}
Advantage $\advantage(s,a)$ 的直觉:
\begin{itemize}
    \item $\Val(s)$:在状态 $s$ 的"平均"表现
    \item $\Qval(s,a)$:在状态 $s$ 选择动作 $a$ 的表现
    \item $\advantage(s,a) = \Qval(s,a) - \Val(s)$:动作 $a$ 比平均好多少
\end{itemize}
$\advantage > 0$:这个动作好于平均,应该增加其概率
$\advantage < 0$:这个动作差于平均,应该减少其概率
\end{keypoint}

\subsection{Advantage 的估计方法}

实践中,需要估计 $\advantage^\policy$。常见估计方法:

\begin{enumerate}
    \item \textbf{Monte Carlo 估计}:
    \begin{equation}
        \hat{\advantage}_t^{\text{MC}} = G_t - \hat{\Val}(s_t)
    \end{equation}
    无偏但高方差($G_t$ 累积了整条轨迹的随机性)。

    \item \textbf{TD 估计}(1-step):
    \begin{equation}
        \hat{\advantage}_t^{\text{TD}} = r_t + \discount \hat{\Val}(s_{t+1}) - \hat{\Val}(s_t) = \delta_t
    \end{equation}
    低方差但有偏(依赖于 $\hat{\Val}$ 的准确性)。

    \item \textbf{n-step 估计}:介于两者之间
    \begin{equation}
        \hat{\advantage}_t^{(n)} = \sum_{k=0}^{n-1} \discount^k r_{t+k} + \discount^n \hat{\Val}(s_{t+n}) - \hat{\Val}(s_t)
    \end{equation}

    \item \textbf{GAE}(下一节介绍):通过 $\lambda$ 参数灵活权衡偏差和方差。
\end{enumerate}

\subsection{Actor-Critic 架构}

为了估计 $\hat{\Val}(s)$,我们引入一个 \textbf{Critic} 网络。Actor-Critic 方法同时学习:
\begin{itemize}
    \item \textbf{Actor}:策略网络 $\policy_\theta(a|s)$,输出动作分布
    \item \textbf{Critic}:价值网络 $\hat{\Val}_\phi(s)$,估计状态价值
\end{itemize}

\begin{figure}[H]
    \centering
    \begin{tikzpicture}[scale=0.95,
        box/.style={draw, rounded corners, minimum width=3cm, minimum height=1.2cm, align=center},
        arrow/.style={->, thick, >=stealth}
    ]
        % Actor
        \node[box, fill=orange!30] (actor) at (0, 1) {Actor $\policy_\theta(a|s)$};

        % Critic
        \node[box, fill=purple!25] (critic) at (0, -1.5) {Critic $\hat{\Val}_\phi(s)$};

        % Environment
        \node[box, fill=green!25] (env) at (6, -0.25) {Environment};

        % State input
        \node[circle, draw, fill=blue!20, minimum size=0.8cm] (state) at (-4, -0.25) {$s$};

        % Arrows
        \draw[arrow] (state) -- (actor);
        \draw[arrow] (state) -- (critic);
        \draw[arrow] (actor.east) -- node[above, font=\scriptsize] {$a \sim \policy_\theta$} (env.north west);
        \draw[arrow] (env.south west) -- node[below, font=\scriptsize] {$s', r$} +(-2, 0) |- (critic.east);

        % Advantage signal
        \draw[arrow, blue, thick] (critic.north) -- node[left, font=\scriptsize] {$\hat{\advantage}_t$} (actor.south);

        % Update arrows
        \draw[arrow, red, dashed] (critic.west) -- +(-1, 0) node[left, font=\scriptsize] {TD Loss};
        \draw[arrow, red, dashed] (actor.west) -- +(-1, 0) node[left, font=\scriptsize] {PG Loss};

        % Legend
        \node[font=\scriptsize, text width=3cm] at (6, -2.5) {
            Critic 提供 $\hat{\Val}(s)$\\
            计算 $\hat{\advantage}_t$\\
            用于更新 Actor
        };
    \end{tikzpicture}
    \caption{Actor-Critic 架构。Critic 估计 $\Val(s)$,提供 advantage 信号来更新 Actor。}
    \label{fig:actor-critic}
\end{figure}

\subsection{A2C 算法}

A2C(Advantage Actor-Critic)的核心更新规则:

\textbf{Actor 更新}(Policy Gradient with Advantage):
\begin{equation}
    \theta \leftarrow \theta + \alpha_\theta \sum_t \nabla_\theta \log \policy_\theta(a_t|s_t) \cdot \hat{\advantage}_t
\end{equation}

\textbf{Critic 更新}(Value Function Regression):

方式一:使用 MC target
\begin{equation}
    \phi \leftarrow \phi - \alpha_\phi \nabla_\phi \sum_t \left( \hat{\Val}_\phi(s_t) - G_t \right)^2
\end{equation}

方式二:使用 TD target(更常用)
\begin{equation}
    \phi \leftarrow \phi - \alpha_\phi \nabla_\phi \sum_t \left( \hat{\Val}_\phi(s_t) - (r_t + \discount \hat{\Val}_\phi(s_{t+1})) \right)^2
\end{equation}

\begin{algorithm}[H]
\caption{Advantage Actor-Critic (A2C)}
\label{alg:a2c}
\KwInput{Actor 参数 $\theta$,Critic 参数 $\phi$,学习率 $\alpha_\theta, \alpha_\phi$}
\ForEach{episode}{
    采样轨迹 $(s_0, a_0, r_0, \ldots, s_T)$ 按 $\policy_\theta$\;
    \ForEach{$t = 0, \ldots, T-1$}{
        计算 TD 残差:$\delta_t = r_t + \discount \hat{\Val}_\phi(s_{t+1}) - \hat{\Val}_\phi(s_t)$\;
        或计算 MC advantage:$\hat{\advantage}_t = G_t - \hat{\Val}_\phi(s_t)$\;
    }
    更新 Actor:$\theta \leftarrow \theta + \alpha_\theta \sum_t \nabla_\theta \log \policy_\theta(a_t|s_t) \cdot \hat{\advantage}_t$\;
    更新 Critic:$\phi \leftarrow \phi - \alpha_\phi \nabla_\phi \sum_t (\hat{\Val}_\phi(s_t) - \text{target})^2$\;
}
\end{algorithm}

\begin{keypoint}
为什么需要 Critic?
\begin{itemize}
    \item 提供 $\hat{\Val}(s)$ 来计算 advantage $\hat{\advantage}_t$
    \item 比纯 MC(使用 $G_t$)方差更小
    \item 可每步更新,不用等 episode 结束
    \item Critic 的估计虽有偏差,但整体降低了梯度估计的方差
\end{itemize}
\end{keypoint}

% ------------------------------------------
\section{Generalized Advantage Estimation (GAE)}
\label{sec:gae}

GAE 提供了一种在偏差和方差之间灵活权衡的 advantage 估计方法,是现代 Policy Gradient 算法(如 PPO)的核心组件。

\subsection{从 n-step Advantage 到 GAE}

回顾 n-step advantage 估计:
\begin{equation}
    \hat{\advantage}_t^{(n)} = \sum_{k=0}^{n-1} \discount^k r_{t+k} + \discount^n \hat{\Val}(s_{t+n}) - \hat{\Val}(s_t)
\end{equation}

\begin{itemize}
    \item $n=1$:TD advantage,$\hat{\advantage}_t^{(1)} = \delta_t = r_t + \discount \hat{\Val}(s_{t+1}) - \hat{\Val}(s_t)$(低方差,高偏差)
    \item $n=\infty$:MC advantage,$\hat{\advantage}_t^{(\infty)} = G_t - \hat{\Val}(s_t)$(高方差,低偏差)
\end{itemize}

自然的问题:能否组合不同 $n$ 的估计,取得更好的权衡?答案是 \textbf{GAE}——通过对所有 n-step advantage 进行指数加权平均,用一个参数 $\lambda$ 灵活控制偏差-方差的平衡点。

\subsection{GAE 的定义与推导}

\begin{definition}[Generalized Advantage Estimation]
\begin{equation}
    \hat{\advantage}_t^{\text{GAE}(\gamma, \lambda)} = \sum_{l=0}^{\infty} (\discount \lambda)^l \delta_{t+l}
    \label{eq:gae}
\end{equation}
其中 $\delta_t = r_t + \discount \hat{\Val}(s_{t+1}) - \hat{\Val}(s_t)$ 是 TD 残差,$\lambda \in [0,1]$ 是衰减参数。
\end{definition}

\begin{theorem}[GAE 等价于 n-step Advantage 的加权和]
\label{thm:gae-equivalence}
\begin{equation}
    \hat{\advantage}_t^{\text{GAE}} = (1-\lambda) \sum_{n=1}^{\infty} \lambda^{n-1} \hat{\advantage}_t^{(n)}
\end{equation}
\end{theorem}

\begin{proof}
首先,注意到 n-step advantage 可以写成 TD 残差的和:
\begin{equation}
    \hat{\advantage}_t^{(n)} = \sum_{k=0}^{n-1} \discount^k \delta_{t+k}
\end{equation}

这可以通过展开验证:
\begin{align}
    \sum_{k=0}^{n-1} \discount^k \delta_{t+k} &= \sum_{k=0}^{n-1} \discount^k \left( r_{t+k} + \discount \hat{\Val}(s_{t+k+1}) - \hat{\Val}(s_{t+k}) \right) \\
    &= \sum_{k=0}^{n-1} \discount^k r_{t+k} + \sum_{k=0}^{n-1} \discount^{k+1} \hat{\Val}(s_{t+k+1}) - \sum_{k=0}^{n-1} \discount^k \hat{\Val}(s_{t+k})
\end{align}

后两项是 telescoping sum:
\begin{align}
    &= \sum_{k=0}^{n-1} \discount^k r_{t+k} + \discount^n \hat{\Val}(s_{t+n}) - \hat{\Val}(s_t) \\
    &= \hat{\advantage}_t^{(n)}
\end{align}

现在计算 GAE 的加权和形式:
\begin{align}
    (1-\lambda) \sum_{n=1}^{\infty} \lambda^{n-1} \hat{\advantage}_t^{(n)}
    &= (1-\lambda) \sum_{n=1}^{\infty} \lambda^{n-1} \sum_{k=0}^{n-1} \discount^k \delta_{t+k}
\end{align}

交换求和顺序。对于固定的 $k$,它出现在 $n > k$ 的所有项中:
\begin{align}
    &= (1-\lambda) \sum_{k=0}^{\infty} \discount^k \delta_{t+k} \sum_{n=k+1}^{\infty} \lambda^{n-1} \\
    &= (1-\lambda) \sum_{k=0}^{\infty} \discount^k \delta_{t+k} \cdot \frac{\lambda^k}{1-\lambda} \\
    &= \sum_{k=0}^{\infty} (\discount\lambda)^k \delta_{t+k} \\
    &= \hat{\advantage}_t^{\text{GAE}}
\end{align}
\end{proof}

\subsection{$\lambda$ 参数的偏差-方差权衡}

\begin{table}[H]
    \centering
    \begin{tabular}{@{}cccl@{}}
        \toprule
        $\lambda$ 值 & 等价形式 & 偏差 & 方差 \\
        \midrule
        $\lambda = 0$ & $\delta_t$(TD) & 高(依赖 $\hat{\Val}$) & 低 \\
        $\lambda = 1$ & $G_t - \hat{\Val}(s_t)$(MC) & 低 & 高 \\
        $\lambda \in (0,1)$ & 加权平均 & 中等 & 中等 \\
        \bottomrule
    \end{tabular}
    \caption{GAE 参数 $\lambda$ 的效果}
    \label{tab:gae-lambda}
\end{table}

\begin{figure}[H]
    \centering
    \begin{tikzpicture}[scale=0.85]
        % Lambda = 0
        \begin{scope}[shift={(0,0)}]
            \node[font=\bfseries] at (2.5, 2) {$\lambda = 0$};
            \draw[->, thick] (0,0) -- (5,0) node[right, font=\scriptsize] {$l$};
            \fill[blue!70] (0.5, 0) rectangle (1, 1.5);
            \node[font=\tiny] at (0.75, -0.3) {$\delta_t$};
            \node[font=\scriptsize] at (2.5, -1) {只用当前 TD 残差};
        \end{scope}

        % Lambda = 0.5
        \begin{scope}[shift={(6,0)}]
            \node[font=\bfseries] at (2.5, 2) {$\lambda = 0.5$};
            \draw[->, thick] (0,0) -- (5,0) node[right, font=\scriptsize] {$l$};
            \fill[blue!70] (0.5, 0) rectangle (1, 1.5);
            \fill[blue!50] (1.5, 0) rectangle (2, 0.75);
            \fill[blue!30] (2.5, 0) rectangle (3, 0.375);
            \fill[blue!15] (3.5, 0) rectangle (4, 0.1875);
            \node[font=\scriptsize] at (2.5, -1) {指数衰减权重};
        \end{scope}

        % Lambda = 1
        \begin{scope}[shift={(12,0)}]
            \node[font=\bfseries] at (2.5, 2) {$\lambda = 1$};
            \draw[->, thick] (0,0) -- (5,0) node[right, font=\scriptsize] {$l$};
            \fill[blue!70] (0.5, 0) rectangle (1, 1.5);
            \fill[blue!70] (1.5, 0) rectangle (2, 1.5);
            \fill[blue!70] (2.5, 0) rectangle (3, 1.5);
            \fill[blue!70] (3.5, 0) rectangle (4, 1.5);
            \node[font=\scriptsize] at (2.5, -1) {所有 TD 残差等权};
        \end{scope}
    \end{tikzpicture}
    \caption{GAE 中 $(\gamma\lambda)^l$ 权重随 $l$ 的变化。$\lambda$ 越小,越依赖近期的 TD 残差。}
    \label{fig:gae-weights}
\end{figure}

实践中,$\lambda = 0.95$ 或 $\lambda = 0.97$ 是常用的选择。

\begin{keypoint}
GAE 的直觉理解:
\begin{itemize}
    \item $\delta_t = r_t + \gamma\hat{\Val}(s_{t+1}) - \hat{\Val}(s_t)$ 是"一步后用 Critic 估计剩余价值"的 advantage
    \item GAE 把多步 $\delta$ 加权求和,$(\gamma\lambda)^l$ 让远处的 $\delta$ 权重指数衰减
    \item $\lambda$ 越小,越依赖 Critic 估计(偏差大但方差小)
    \item $\lambda$ 越大,越依赖实际回报(偏差小但方差大)
\end{itemize}
GAE 与第二章的 TD($\lambda$) 有类似的思想,都是通过 $\lambda$ 在 TD 和 MC 之间权衡。
\end{keypoint}

\subsection{GAE 的实际计算}

GAE 可以通过递推高效计算:

\begin{equation}
    \hat{\advantage}_t^{\text{GAE}} = \delta_t + \gamma\lambda \hat{\advantage}_{t+1}^{\text{GAE}}
\end{equation}

边界条件:$\hat{\advantage}_T^{\text{GAE}} = 0$(episode 结束后)。

从后往前计算,复杂度为 $O(T)$。

% ------------------------------------------
\section{重要性采样与 Off-Policy Policy Gradient}
\label{sec:importance-sampling}

\subsection{On-Policy 的问题}

Policy Gradient 是 on-policy 的:每次更新 $\theta$ 后,旧数据的分布 $p(\tau|\theta_{\text{old}})$ 就与新策略 $p(\tau|\theta)$ 不同了。这导致:
\begin{itemize}
    \item 数据只能用一次,样本效率低
    \item 每次更新都需要重新采样
\end{itemize}

重要性采样(Importance Sampling, IS)允许我们复用旧数据。

\subsection{重要性采样原理}

\begin{definition}[重要性采样]
用分布 $q(x)$ 的样本估计 $p(x)$ 下的期望:
\begin{equation}
    \E_{x \sim p}[f(x)] = \E_{x \sim q}\left[ \frac{p(x)}{q(x)} f(x) \right]
\end{equation}
其中 $\rho(x) = \frac{p(x)}{q(x)}$ 称为\textbf{重要性权重}(Importance Weight)。
\end{definition}

\begin{proof}
\begin{equation}
    \E_{x \sim q}\left[ \frac{p(x)}{q(x)} f(x) \right] = \int q(x) \cdot \frac{p(x)}{q(x)} f(x) dx = \int p(x) f(x) dx = \E_{x \sim p}[f(x)]
\end{equation}
\end{proof}

\subsection{应用到 Policy Gradient}

用旧策略 $\policy_{\text{old}}$ 的样本估计新策略 $\policy_\theta$ 下的期望:

\begin{equation}
    J(\theta) = \E_{\tau \sim \policy_{\text{old}}} \left[ \frac{p(\tau|\theta)}{p(\tau|\theta_{\text{old}})} R(\tau) \right]
\end{equation}

轨迹的重要性权重:
\begin{align}
    \frac{p(\tau|\theta)}{p(\tau|\theta_{\text{old}})} &= \frac{\cancel{p(s_0)} \prod_t \policy_\theta(a_t|s_t) \cancel{P(s_{t+1}|s_t,a_t)}}{\cancel{p(s_0)} \prod_t \policy_{\text{old}}(a_t|s_t) \cancel{P(s_{t+1}|s_t,a_t)}} \\
    &= \prod_{t=0}^{T-1} \frac{\policy_\theta(a_t|s_t)}{\policy_{\text{old}}(a_t|s_t)}
\end{align}

环境动力学 $P$ 和初始分布 $p(s_0)$ 消掉了!这再次体现了 Policy Gradient 的 model-free 特性。

对于单步 advantage 估计,重要性权重简化为:
\begin{equation}
    \rho_t(\theta) = \frac{\policy_\theta(a_t|s_t)}{\policy_{\text{old}}(a_t|s_t)}
\end{equation}

\begin{theorem}[Off-policy Policy Gradient]
\begin{equation}
    \nabla_\theta J(\theta) = \E_{(s,a) \sim \policy_{\text{old}}} \left[ \rho_t(\theta) \nabla_\theta \log \policy_\theta(a_t|s_t) \hat{\advantage}_t \right]
\end{equation}
\end{theorem}

这使得可以用旧数据多次更新策略!

\subsection{严格的 Off-Policy 梯度与状态分布修正}

上面的公式省略了一个重要细节。\textbf{严格的 off-policy policy gradient} 不仅需要修正动作概率,还需要修正\textbf{状态分布}:

\begin{equation}
    \nabla_\theta J(\theta) = \E_{s \sim d_{\policy_{\text{old}}}} \left[ \frac{d_{\policy_\theta}(s)}{d_{\policy_{\text{old}}}(s)} \cdot \E_{a \sim \policy_{\text{old}}(\cdot|s)} \left[ \rho_t(\theta) \nabla_\theta \log \policy_\theta(a|s) \hat{\advantage}(s,a) \right] \right]
\end{equation}

其中 $d_\policy(s)$ 是策略 $\policy$ 诱导的状态分布(也称为 discounted state visitation distribution)。

\begin{note}
\textbf{为什么需要状态分布修正?}

直观理解:用旧策略 $\policy_{\text{old}}$ 采样时,不仅动作的分布变了,连访问到的状态分布也变了。例如:
\begin{itemize}
    \item 新策略可能更倾向于进入某些状态
    \item 旧数据中这些状态的样本可能较少
\end{itemize}
因此严格的 off-policy 梯度需要同时修正这两个偏差。
\end{note}

\subsubsection*{PPO/TRPO 的隐式近似}

然而,计算状态分布比 $\frac{d_{\policy_\theta}(s)}{d_{\policy_{\text{old}}}(s)}$ 非常困难——它依赖于整条轨迹的累积效应,无法像动作概率那样直接计算。

PPO/TRPO 的 \textbf{surrogate objective} 实际上做了一个关键近似:

\begin{equation}
    \frac{d_{\policy_\theta}(s)}{d_{\policy_{\text{old}}}(s)} \approx 1
\end{equation}

即假设新旧策略诱导的状态分布相同。

\begin{important}
\textbf{Trust Region 的作用}:当 $\policy_\theta$ 与 $\policy_{\text{old}}$ 足够接近时(KL 散度小),状态分布的差异也会很小。TRPO 的 KL 约束和 PPO 的 clip 机制正是为了保证这一点。
\end{important}

\textbf{总结}:PPO/TRPO 使用的 surrogate objective
\begin{equation}
    L(\theta) = \E_{(s,a) \sim \policy_{\text{old}}} \left[ \rho_t(\theta) \hat{\advantage}_t \right]
\end{equation}
实际上隐式地做了两件事:
\begin{enumerate}
    \item 用 $\rho_t = \frac{\policy_\theta(a|s)}{\policy_{\text{old}}(a|s)}$ 修正动作概率偏差
    \item 假设 $\frac{d_{\policy_\theta}(s)}{d_{\policy_{\text{old}}}(s)} \approx 1$,忽略状态分布偏差
\end{enumerate}

这解释了为什么 PPO/TRPO 能直接使用 token-level 的 $\rho_t$ 而不需要额外的状态分布修正——\textbf{前提是 trust region 约束成立}。

\subsection{方差问题}

重要性采样的问题:当 $\rho_t$ 偏离 1 太多时,方差会急剧增大。

\begin{itemize}
    \item 如果 $\policy_\theta$ 和 $\policy_{\text{old}}$ 差异大,$\rho_t$ 可能非常大或非常小
    \item 大的 $\rho_t$ 导致梯度估计方差爆炸
    \item 需要限制策略更新幅度,保持 $\rho_t \approx 1$
\end{itemize}

\begin{figure}[H]
    \centering
    \begin{tikzpicture}[scale=0.8]
        \draw[->] (0,0) -- (6,0) node[right] {$\rho_t$};
        \draw[->] (0,0) -- (0,3) node[above] {方差};

        % Variance curve
        \draw[thick, blue, domain=0.2:5, samples=50] plot (\x, {0.5 + 0.3*(\x-2)^2});

        % Safe region
        \fill[green!20] (1.5, 0) rectangle (2.5, 2.5);
        \node[font=\scriptsize, green!60!black] at (2, 2.8) {安全区域};

        % Labels
        \node[below] at (2, 0) {$1$};
        \draw[dashed] (2, 0) -- (2, 0.5);

        \node[font=\scriptsize] at (4.5, 2) {$\rho$ 偏离 1 时};
        \node[font=\scriptsize] at (4.5, 1.5) {方差急剧增大};
    \end{tikzpicture}
    \caption{重要性权重 $\rho_t$ 偏离 1 时,方差增大。需要限制策略更新幅度。}
    \label{fig:is-variance}
\end{figure}

% ------------------------------------------
\section{Trust Region 方法:TRPO 与 PPO}
\label{sec:trust-region}

\subsection{动机:限制策略更新幅度}

重要性采样允许复用旧数据,但如果 $\policy_\theta$ 与 $\policy_{\text{old}}$ 差异太大,估计就不可靠。Trust Region 方法通过限制策略更新幅度来解决这个问题。

\subsection{TRPO:KL 约束优化}

TRPO(Trust Region Policy Optimization)通过 KL 散度约束限制策略更新:

\begin{definition}[TRPO 优化问题]
\begin{align}
    \max_\theta \quad & L(\theta) = \E_{(s,a) \sim \policy_{\text{old}}} \left[ \rho_t(\theta) \hat{\advantage}_t \right] \\
    \text{s.t.} \quad & \bar{D}_{\text{KL}}(\policy_{\text{old}} \| \policy_\theta) \leq \delta
\end{align}
其中 $\bar{D}_{\text{KL}}$ 是状态分布上的平均 KL 散度。
\end{definition}

TRPO 的特点:
\begin{itemize}
    \item 理论上保证单调改进(在约束满足时)
    \item 需要计算 KL 散度的 Hessian(二阶优化,Conjugate Gradient)
    \item 实现复杂,计算开销大
\end{itemize}

\subsection{PPO:简化的 Trust Region}

PPO(Proximal Policy Optimization)通过更简单的方式近似 TRPO 的效果。

\subsubsection{PPO-Clip}

\begin{definition}[PPO-Clip 目标]
\begin{equation}
    L^{\text{CLIP}}(\theta) = \E \left[ \min \left( \rho_t \hat{\advantage}_t, \, \text{clip}(\rho_t, 1-\epsilon, 1+\epsilon) \hat{\advantage}_t \right) \right]
    \label{eq:ppo-clip}
\end{equation}
其中 $\text{clip}(x, a, b) = \max(a, \min(x, b))$,$\epsilon$ 通常取 0.1 或 0.2。
\end{definition}

\begin{figure}[H]
    \centering
    \begin{tikzpicture}[scale=0.75]
        % Advantage > 0
        \begin{scope}[shift={(-5,0)}]
            \node[font=\bfseries] at (2.5, 3) {$\hat{\advantage}_t > 0$(好动作)};
            \draw[->] (0,0) -- (5,0) node[right, font=\scriptsize] {$\rho_t$};
            \draw[->] (0,-0.5) -- (0,2.5) node[above, font=\scriptsize] {$L$};

            % Unclipped
            \draw[red, dashed, thick] (0,0) -- (4.5, 2.25);

            % Clipped
            \draw[blue, very thick] (0,0) -- (3,1.5) -- (4.5,1.5);

            % Labels
            \node[below, font=\tiny] at (1.5, 0) {$1{-}\epsilon$};
            \node[below, font=\tiny] at (3, 0) {$1{+}\epsilon$};
            \draw[dotted] (1.5,0) -- (1.5,0.75);
            \draw[dotted] (3,0) -- (3,1.5);

            \node[font=\tiny, blue] at (4.2, 2) {截断};
            \node[font=\scriptsize] at (2.5, -1) {防止过度增加概率};
        \end{scope}

        % Advantage < 0
        \begin{scope}[shift={(5,0)}]
            \node[font=\bfseries] at (2.5, 3) {$\hat{\advantage}_t < 0$(坏动作)};
            \draw[->] (0,0) -- (5,0) node[right, font=\scriptsize] {$\rho_t$};
            \draw[->] (0,-2) -- (0,1) node[above, font=\scriptsize] {$L$};

            % Unclipped
            \draw[red, dashed, thick] (0,0) -- (4.5, -2.25);

            % Clipped
            \draw[blue, very thick] (0,-1.5) -- (1.5,-1.5) -- (4.5, 0);

            % Labels
            \node[below, font=\tiny] at (1.5, 0) {$1{-}\epsilon$};
            \node[below, font=\tiny] at (3, 0) {$1{+}\epsilon$};
            \draw[dotted] (1.5,0) -- (1.5,-1.5);
            \draw[dotted] (3,0) -- (3,-0.75);

            \node[font=\tiny, blue] at (0.8, -2) {截断};
            \node[font=\scriptsize] at (2.5, -2.8) {防止过度减少概率};
        \end{scope}
    \end{tikzpicture}
    \caption{PPO-Clip 机制。当 $\rho_t$ 偏离 $[1-\epsilon, 1+\epsilon]$ 时,目标被截断,防止策略变化过大。}
    \label{fig:ppo-clip-full}
\end{figure}

\begin{theorem}[PPO-Clip 的直觉]
\begin{itemize}
    \item 当 $\hat{\advantage}_t > 0$(好动作):
    \begin{itemize}
        \item 目标是增加 $\policy_\theta(a_t|s_t)$,即让 $\rho_t > 1$
        \item 但当 $\rho_t > 1+\epsilon$ 时截断,防止过度增加
    \end{itemize}
    \item 当 $\hat{\advantage}_t < 0$(坏动作):
    \begin{itemize}
        \item 目标是减少 $\policy_\theta(a_t|s_t)$,即让 $\rho_t < 1$
        \item 但当 $\rho_t < 1-\epsilon$ 时截断,防止过度减少
    \end{itemize}
\end{itemize}
\end{theorem}

\subsubsection{PPO-KL(可选)}

PPO 的另一个变体使用 KL 惩罚项:

\begin{definition}[PPO-KL 目标]
\begin{equation}
    L^{\text{KL}}(\theta) = \E \left[ \rho_t \hat{\advantage}_t \right] - \beta \cdot \text{KL}(\policy_{\text{old}} \| \policy_\theta)
\end{equation}
其中 $\beta$ 是自适应调整的系数:
\begin{itemize}
    \item 如果 KL 太大,增大 $\beta$
    \item 如果 KL 太小,减小 $\beta$
\end{itemize}
\end{definition}

\subsection{Entropy Bonus}

为了鼓励探索,PPO 通常还会加入 entropy bonus:

\begin{equation}
    L^{\text{total}}(\theta) = L^{\text{CLIP}}(\theta) + c_1 \cdot H(\policy_\theta)
\end{equation}

其中 $H(\policy_\theta) = -\E[\log \policy_\theta(a|s)]$ 是策略的熵,$c_1$ 是权重系数(通常 0.01)。

\begin{note}
Entropy bonus 的作用:
\begin{itemize}
    \item 鼓励策略保持一定的随机性,避免过早收敛到确定性策略
    \item 促进探索,防止陷入局部最优
    \item 熵越大,策略越"均匀",对各动作的概率分布越平坦
\end{itemize}
\end{note}

\subsection{PPO 完整算法}

\begin{algorithm}[H]
\caption{Proximal Policy Optimization (PPO)}
\label{alg:ppo}
\KwInput{初始参数 $\theta_0$, $\phi_0$,clip 参数 $\epsilon$,每轮更新次数 $K$,GAE 参数 $\lambda$}
\For{iteration $= 1, 2, \ldots$}{
    \textbf{数据收集}:用当前策略 $\policy_{\theta}$ 收集 $N$ 条轨迹\;
    计算 GAE advantage $\hat{\advantage}_t$(使用 $\hat{\Val}_\phi$)\;
    计算 return $\hat{R}_t = \hat{\advantage}_t + \hat{\Val}_\phi(s_t)$(作为 Critic target)\;
    记录 $\policy_{\text{old}} = \policy_{\theta}$(固定,用于计算 $\rho_t$)\;

    \textbf{策略更新}:\For{$k = 1, \ldots, K$}{
        对 mini-batch 数据计算:
        \begin{itemize}
            \item $\rho_t = \frac{\policy_\theta(a_t|s_t)}{\policy_{\text{old}}(a_t|s_t)}$
            \item $L^{\text{CLIP}} = \E[\min(\rho_t \hat{\advantage}_t, \text{clip}(\rho_t, 1{-}\epsilon, 1{+}\epsilon) \hat{\advantage}_t)]$
            \item $L^{\text{Value}} = \E[(\hat{\Val}_\phi(s_t) - \hat{R}_t)^2]$
            \item $L^{\text{Entropy}} = -\E[\log \policy_\theta(a_t|s_t)]$
        \end{itemize}
        总目标:$L = L^{\text{CLIP}} - c_1 L^{\text{Value}} + c_2 L^{\text{Entropy}}$\;
        梯度上升更新 $\theta$,梯度下降更新 $\phi$\;
    }
}
\end{algorithm}

\begin{keypoint}
PPO 的成功原因:
\begin{enumerate}
    \item \textbf{简单高效}:只需一阶优化,不需要计算 Hessian
    \item \textbf{样本效率}:可多次复用同一批数据($K$ 次更新)
    \item \textbf{稳定性}:clip 机制防止策略剧烈变化,避免"走太远"
    \item \textbf{鲁棒性}:对超参数不敏感,适用于多种任务
\end{enumerate}
PPO 是目前最常用的 Policy Gradient 算法,也是 RLHF(第五章)中的标准选择。
\end{keypoint}

% ------------------------------------------
\section{本章小结}
\label{sec:policy-summary}

\begin{keypoint}
本章核心内容:

\begin{enumerate}
    \item \textbf{Policy Gradient 定理}
    \begin{itemize}
        \item 给出了目标函数梯度的解析形式:$\nabla_\theta J = \E[\sum_t \nabla \log \policy \cdot G_t]$
        \item Log-Derivative Trick 是推导的关键
        \item 环境动力学与 $\theta$ 无关,实现了 Model-Free
    \end{itemize}

    \item \textbf{方差降低技术}
    \begin{itemize}
        \item Baseline 技巧:减去 $b(s)$ 不改变期望但降低方差
        \item 最优 baseline 是 $\Val^\policy(s)$
        \item 使用 Advantage $\advantage = \Qval - \Val$ 替代 $G_t$
    \end{itemize}

    \item \textbf{Actor-Critic 架构}
    \begin{itemize}
        \item Actor(策略网络)+ Critic(价值网络)
        \item Critic 提供 $\hat{\Val}(s)$ 来估计 advantage
    \end{itemize}

    \item \textbf{GAE}
    \begin{itemize}
        \item $\hat{\advantage}^{\text{GAE}} = \sum_l (\gamma\lambda)^l \delta_{t+l}$
        \item $\lambda$ 控制偏差-方差权衡(类似 TD($\lambda$))
    \end{itemize}

    \item \textbf{Trust Region 方法}
    \begin{itemize}
        \item 重要性采样允许复用旧数据,但需要限制策略变化
        \item TRPO:KL 约束优化,实现复杂
        \item PPO:clip 机制,简单高效,是实践中的首选
    \end{itemize}
\end{enumerate}
\end{keypoint}

\begin{figure}[H]
    \centering
    \begin{tikzpicture}[
        box/.style={draw, rounded corners, fill=blue!10, minimum width=2.5cm, minimum height=0.9cm, align=center, font=\small},
        arrow/.style={->, thick}
    ]
        % Timeline
        \node[box] (reinforce) at (0, 0) {REINFORCE};
        \node[box] (baseline) at (3.5, 0) {+ Baseline};
        \node[box] (ac) at (7, 0) {Actor-Critic};
        \node[box] (gae) at (10.5, 0) {+ GAE};
        \node[box, fill=green!20] (ppo) at (10.5, -2) {PPO};

        \draw[arrow] (reinforce) -- (baseline);
        \draw[arrow] (baseline) -- (ac);
        \draw[arrow] (ac) -- (gae);
        \draw[arrow] (gae) -- (ppo);

        % Annotations
        \node[font=\tiny, below=0.1cm of reinforce] {无偏高方差};
        \node[font=\tiny, below=0.1cm of baseline] {降低方差};
        \node[font=\tiny, below=0.1cm of ac] {学习 Critic};
        \node[font=\tiny, below=0.1cm of gae] {$\lambda$ 权衡};
        \node[font=\tiny, below=0.1cm of ppo] {稳定高效};
    \end{tikzpicture}
    \caption{Policy Gradient 方法的演进:从 REINFORCE 到 PPO}
    \label{fig:pg-evolution}
\end{figure}

\begin{note}
Policy-Based vs. Value-Based 的选择:
\begin{itemize}
    \item \textbf{连续动作空间}:优先选择 Policy-Based(如 PPO、SAC)
    \item \textbf{离散动作空间}:两者都可以,DQN 可能更样本高效
    \item \textbf{需要随机策略}:使用 Policy-Based
    \item \textbf{LLM 对齐}:PPO 是 RLHF 的标准选择(第五章详述)
\end{itemize}
\end{note}
