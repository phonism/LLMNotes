% ==========================================
% 附录:详细推导与常见问题
% ==========================================

\chapter{详细推导与常见问题}
\label{chap:appendix}

% ------------------------------------------
\section{DPO 详细推导}
\label{sec:appendix-dpo}

本节给出 DPO Loss 的完整推导过程。

\subsection{Step 1:RLHF 目标函数}

RLHF 的优化目标是:
\begin{equation}
    \max_\policy \E_{x \sim \mathcal{D}, y \sim \policy} \left[ r(x, y) \right] - \beta \cdot \text{KL}(\policy(y|x) \| \policy_{\text{ref}}(y|x))
\end{equation}

\subsection{Step 2:展开 KL 散度}

展开 KL 散度,转为最小化问题:
\begin{align}
    J(\theta) &= \max_\policy \E_{x, y \sim \policy} \left[ r(x, y) - \beta \log \frac{\policy(y|x)}{\policy_{\text{ref}}(y|x)} \right] \\
    &= \min_\policy \E_{x, y \sim \policy} \left[ \beta \log \frac{\policy(y|x)}{\policy_{\text{ref}}(y|x)} - r(x, y) \right] \\
    &= \min_\policy \E_{x, y \sim \policy} \left[ \log \frac{\policy(y|x)}{\policy_{\text{ref}}(y|x)} - \frac{1}{\beta} r(x, y) \right]
\end{align}

\subsection{Step 3:引入配分函数 $Z(x)$}

定义配分函数:
\begin{equation}
    Z(x) = \sum_y \policy_{\text{ref}}(y|x) \exp\left( \frac{r(x,y)}{\beta} \right)
\end{equation}

在目标函数中加减 $\log Z(x)$(不影响优化):
\begin{align}
    J(\theta) &= \min_\policy \E_{x, y \sim \policy} \left[ \log \frac{\policy(y|x)}{\policy_{\text{ref}}(y|x)} - \frac{1}{\beta} r(x, y) + \log Z(x) - \log Z(x) \right] \\
    &= \min_\policy \E_{x, y \sim \policy} \left[ \log \frac{\policy(y|x)}{\frac{1}{Z(x)} \policy_{\text{ref}}(y|x) \exp\left( \frac{r(x,y)}{\beta} \right)} - \log Z(x) \right]
\end{align}

\subsection{Step 4:最优策略的闭式解}

定义:
\begin{equation}
    \policy^*(y|x) = \frac{1}{Z(x)} \policy_{\text{ref}}(y|x) \exp\left( \frac{r(x,y)}{\beta} \right)
\end{equation}

可以验证 $\policy^*$ 是合法的概率分布(因为 $Z(x)$ 是归一化常数)。

目标函数变为:
\begin{equation}
    J(\theta) = \min_\policy \E_x \left[ \text{KL}(\policy(\cdot|x) \| \policy^*(\cdot|x)) - \log Z(x) \right]
\end{equation}

由于 KL 散度非负且在 $\policy = \policy^*$ 时为零,最优解为 $\policy = \policy^*$。

\begin{keypoint}
这是 DPO 的核心洞察:KL 正则 RL 问题有闭式解!最优策略是参考策略按 $\exp(r/\beta)$ 重新加权的结果。
\end{keypoint}

\subsection{Step 5:反解 reward}

从 $\policy^*$ 的定义反解 reward:
\begin{align}
    \policy^*(y|x) &= \frac{\policy_{\text{ref}}(y|x)}{Z(x)} \exp\left( \frac{r(x,y)}{\beta} \right) \\
    \log \policy^*(y|x) &= \log \policy_{\text{ref}}(y|x) - \log Z(x) + \frac{r(x,y)}{\beta}
\end{align}

解出 $r(x,y)$:
\begin{equation}
    r(x,y) = \beta \log \frac{\policy^*(y|x)}{\policy_{\text{ref}}(y|x)} + \beta \log Z(x)
    \label{eq:reward-implicit-app}
\end{equation}

\subsection{Step 6:Bradley-Terry 偏好模型}

人类偏好遵循 Bradley-Terry 模型:
\begin{equation}
    P(y_w \succ y_l | x) = \sigma(r(x, y_w) - r(x, y_l))
\end{equation}

\subsection{Step 7:代入 reward,$Z(x)$ 消除}

将 \eqref{eq:reward-implicit-app} 代入:
\begin{align}
    r(x, y_w) - r(x, y_l) &= \beta \log \frac{\policy^*(y_w|x)}{\policy_{\text{ref}}(y_w|x)} + \cancel{\beta \log Z(x)} - \beta \log \frac{\policy^*(y_l|x)}{\policy_{\text{ref}}(y_l|x)} - \cancel{\beta \log Z(x)} \\
    &= \beta \left[ \log \frac{\policy^*(y_w|x)}{\policy_{\text{ref}}(y_w|x)} - \log \frac{\policy^*(y_l|x)}{\policy_{\text{ref}}(y_l|x)} \right]
\end{align}

关键:$\beta \log Z(x)$ 项相消!

\subsection{Step 8:DPO Loss}

最大化 log-likelihood:
\begin{equation}
    \boxed{
    L_{\text{DPO}}(\theta) = -\E_{(x, y_w, y_l)} \left[ \log \sigma \left( \beta \left[ \log \frac{\policy_\theta(y_w|x)}{\policy_{\text{ref}}(y_w|x)} - \log \frac{\policy_\theta(y_l|x)}{\policy_{\text{ref}}(y_l|x)} \right] \right) \right]
    }
\end{equation}

% ------------------------------------------
\section{KL 估计器推导}
\label{sec:appendix-kl}

本节推导三种 KL 散度估计器:k1、k2、k3。

\subsection{KL 散度的定义}

\begin{equation}
    \text{KL}(\policy_\theta \| \policy_{\text{ref}}) = \E_{y \sim \policy_\theta} \left[ \log \frac{\policy_\theta(y|x)}{\policy_{\text{ref}}(y|x)} \right] = \E_{y \sim \policy_\theta} \left[ -\log r \right]
\end{equation}
其中 $r = \frac{\policy_{\text{ref}}(y|x)}{\policy_\theta(y|x)}$。

\subsection{k1:直接估计}

\begin{definition}[k1 估计器]
\begin{equation}
    k_1 = -\log r = \log \frac{\policy_\theta}{\policy_{\text{ref}}}
\end{equation}
\end{definition}

性质:
\begin{itemize}
    \item $\E_{\policy_\theta}[k_1] = \text{KL}(\policy_\theta \| \policy_{\text{ref}})$(无偏)
    \item 方差较大,因为 $\log r$ 可能取很大或很小的值
\end{itemize}

\subsection{k2:平方形式}

\begin{definition}[k2 估计器]
\begin{equation}
    k_2 = \frac{1}{2}(\log r)^2 = \frac{1}{2} \left( \log \frac{\policy_{\text{ref}}}{\policy_\theta} \right)^2
\end{equation}
\end{definition}

性质:
\begin{itemize}
    \item $\E_{\policy_\theta}[k_2] \neq \text{KL}$(有偏)
    \item 但 $\nabla_\theta \E[k_2] = \nabla_\theta \text{KL}$(梯度正确)
    \item 更平滑,适合做 loss
\end{itemize}

\begin{proof}[梯度等价性证明]
\begin{align}
    \nabla_\theta k_2 &= \nabla_\theta \frac{1}{2}(\log r)^2 = \log r \cdot \nabla_\theta \log r \\
    &= -\log r \cdot \nabla_\theta \log \policy_\theta = k_1 \cdot \nabla_\theta \log \policy_\theta
\end{align}
由于 $\E[\nabla_\theta \log \policy_\theta] = 0$(score function 的期望为零),可以证明 $\E[\nabla_\theta k_2] = \E[\nabla_\theta k_1]$。
\end{proof}

\subsection{k3:Control Variate}

\begin{definition}[k3 估计器]
\begin{equation}
    k_3 = (r - 1) - \log r = \frac{\policy_{\text{ref}}}{\policy_\theta} - 1 - \log \frac{\policy_{\text{ref}}}{\policy_\theta}
\end{equation}
\end{definition}

性质:
\begin{itemize}
    \item $\E_{\policy_\theta}[r - 1] = 0$(因为 $\E_{\policy_\theta}[r] = \sum_y \policy_\theta \cdot \frac{\policy_{\text{ref}}}{\policy_\theta} = 1$)
    \item $(r-1)$ 作为 control variate,降低方差
    \item $\E_{\policy_\theta}[k_3] = \text{KL}$(无偏)
    \item 方差比 k1 小
\end{itemize}

\begin{keypoint}
k3 的巧妙之处:利用 $\E[r-1]=0$ 这个性质,加入 $(r-1)$ 项不改变期望,但可以降低方差。这是 control variate 技术的典型应用。
\end{keypoint}

\subsection{三种估计器的对比}

\begin{table}[H]
    \centering
    \begin{tabular}{@{}lcccc@{}}
        \toprule
        \textbf{估计器} & \textbf{公式} & \textbf{偏差} & \textbf{方差} & \textbf{推荐用法} \\
        \midrule
        k1 & $-\log r$ & 无偏 & 高 & in reward \\
        k2 & $\frac{1}{2}(\log r)^2$ & 有偏 & 低 & as loss \\
        k3 & $(r-1) - \log r$ & 无偏 & 低 & as loss \\
        \bottomrule
    \end{tabular}
    \caption{KL 估计器对比}
\end{table}

% ------------------------------------------
\section{Off-Policy 状态分布修正的理论基础}
\label{sec:appendix-state-distribution}

本节给出"严格的 off-policy policy gradient 需要状态分布修正"的完整数学推导。

\subsection{性能差引理(Performance Difference Lemma)}

\begin{theorem}[Kakade \& Langford, 2002]
\label{thm:performance-difference}
对于任意两个策略 $\policy_\theta$ 和 $\policy_{\text{old}}$:
\begin{equation}
    J(\policy_\theta) - J(\policy_{\text{old}}) = \frac{1}{1-\discount} \E_{s \sim d_{\policy_\theta}, a \sim \policy_\theta(\cdot|s)} \left[ \advantage^{\policy_{\text{old}}}(s,a) \right]
\end{equation}
其中 $d_\policy(s) = (1-\discount) \sum_{t=0}^{\infty} \discount^t P(s_t = s \mid \policy)$ 是 discounted state visitation distribution。
\end{theorem}

\begin{proof}
定义状态价值函数的差:
\begin{equation}
    J(\policy_\theta) - J(\policy_{\text{old}}) = \E_{s_0} \left[ \Val^{\policy_\theta}(s_0) - \Val^{\policy_{\text{old}}}(s_0) \right]
\end{equation}

利用 advantage 的定义 $\advantage^\policy(s,a) = \Qval^\policy(s,a) - \Val^\policy(s)$,通过 telescope sum 展开:
\begin{align}
    \Val^{\policy_\theta}(s) - \Val^{\policy_{\text{old}}}(s)
    &= \E_{\policy_\theta} \left[ \sum_{t=0}^{\infty} \discount^t \left( r_t + \discount \Val^{\policy_{\text{old}}}(s_{t+1}) - \Val^{\policy_{\text{old}}}(s_t) \right) \,\Big|\, s_0 = s \right] \\
    &= \E_{\policy_\theta} \left[ \sum_{t=0}^{\infty} \discount^t \advantage^{\policy_{\text{old}}}(s_t, a_t) \,\Big|\, s_0 = s \right]
\end{align}

对初始状态求期望并整理:
\begin{equation}
    J(\policy_\theta) - J(\policy_{\text{old}}) = \sum_{t=0}^{\infty} \discount^t \E_{s_t \sim P_{\policy_\theta}^t, a_t \sim \policy_\theta} \left[ \advantage^{\policy_{\text{old}}}(s_t, a_t) \right]
\end{equation}

代入 discounted state visitation distribution 的定义得:
\begin{equation}
    \boxed{J(\policy_\theta) - J(\policy_{\text{old}}) = \frac{1}{1-\discount} \E_{s \sim d_{\policy_\theta}, a \sim \policy_\theta} \left[ \advantage^{\policy_{\text{old}}}(s,a) \right]}
\end{equation}
\end{proof}

\begin{important}
注意右边的期望是关于 \textbf{新策略的状态分布} $d_{\policy_\theta}$,而非旧策略的 $d_{\policy_{\text{old}}}$。这是状态分布修正的理论来源。
\end{important}

\subsection{严格的 Off-Policy 估计}

我们的数据来自旧策略 $\policy_{\text{old}}$:
\begin{itemize}
    \item 状态:$s \sim d_{\policy_{\text{old}}}$
    \item 动作:$a \sim \policy_{\text{old}}(\cdot|s)$
\end{itemize}

要用这些数据无偏地估计性能差引理中的期望,需要两个重要性权重:

\begin{equation}
    \E_{s \sim d_{\policy_\theta}, a \sim \policy_\theta} [f(s,a)] = \E_{s \sim d_{\policy_{\text{old}}}, a \sim \policy_{\text{old}}} \left[ \underbrace{\frac{d_{\policy_\theta}(s)}{d_{\policy_{\text{old}}}(s)}}_{\text{状态分布修正}} \cdot \underbrace{\frac{\policy_\theta(a|s)}{\policy_{\text{old}}(a|s)}}_{\text{动作概率修正}} \cdot f(s,a) \right]
\end{equation}

因此,严格的 off-policy policy gradient 为:
\begin{equation}
    \nabla_\theta J(\theta) = \E_{s \sim d_{\policy_{\text{old}}}} \left[ \frac{d_{\policy_\theta}(s)}{d_{\policy_{\text{old}}}(s)} \cdot \E_{a \sim \policy_{\text{old}}} \left[ \frac{\policy_\theta(a|s)}{\policy_{\text{old}}(a|s)} \nabla_\theta \log \policy_\theta(a|s) \hat{\advantage}(s,a) \right] \right]
\end{equation}

\subsection{PPO/TRPO Surrogate Objective 的近似}

PPO/TRPO 使用的 surrogate objective 为:
\begin{equation}
    L(\theta) = \E_{s \sim d_{\policy_{\text{old}}}, a \sim \policy_{\text{old}}} \left[ \frac{\policy_\theta(a|s)}{\policy_{\text{old}}(a|s)} \hat{\advantage}(s,a) \right]
\end{equation}

对比严格的 off-policy 梯度,\textbf{缺少了状态分布比} $\frac{d_{\policy_\theta}(s)}{d_{\policy_{\text{old}}}(s)}$,即隐式假设:
\begin{equation}
    \frac{d_{\policy_\theta}(s)}{d_{\policy_{\text{old}}}(s)} \approx 1
\end{equation}

\subsection{近似误差界}

\begin{theorem}[TRPO 误差界,Schulman et al., 2015]
Surrogate objective 与真实性能差的偏差有界:
\begin{equation}
    \left| J(\policy_\theta) - J(\policy_{\text{old}}) - \frac{1}{1-\discount} L(\theta) \right| \leq \frac{2\discount \epsilon}{(1-\discount)^2} \cdot \max_s D_{\text{KL}}(\policy_{\text{old}}(\cdot|s) \| \policy_\theta(\cdot|s))
\end{equation}
其中 $\epsilon = \max_{s,a} |\advantage^{\policy_{\text{old}}}(s,a)|$。
\end{theorem}

\begin{keypoint}
这个误差界说明:
\begin{enumerate}
    \item 当 $\policy_\theta$ 与 $\policy_{\text{old}}$ 的 KL 散度足够小时,surrogate objective 是真实性能差的良好近似
    \item TRPO 的 KL 约束和 PPO 的 clip 机制正是为了控制这个误差界
    \item 这解释了为什么 PPO/TRPO 能直接使用 token-level 的 $\rho_t = \frac{\policy_\theta(a|s)}{\policy_{\text{old}}(a|s)}$ 而不需要显式计算状态分布修正——\textbf{前提是 trust region 约束成立}
\end{enumerate}
\end{keypoint}

% ------------------------------------------
\section{常见问题解答}
\label{sec:faq}

\subsection{Q1:为什么 RLHF 要加 KL 正则?}

\textbf{答}:KL 正则有多重作用:

\begin{enumerate}
    \item \textbf{防止 Reward Hacking}:Reward Model 不是完美的,模型可能学到"欺骗" RM 的方式(如生成特定模式获得高分但质量差)。KL 项限制策略偏离 SFT 太远。

    \item \textbf{保持生成质量}:SFT 模型已经学会了基本的语言能力,KL 项确保这些能力不丢失。

    \item \textbf{稳定训练}:约束优化空间,避免策略崩溃或发散。

    \item \textbf{理论保证}:KL 正则 RL 问题有闭式解(DPO 的理论基础)。
\end{enumerate}

\subsection{Q2:长序列 token-level PPO/GRPO 为什么容易崩?}

\textbf{答}:主要原因是 importance sampling 权重的累积:

\begin{enumerate}
    \item \textbf{权重累积}:序列级 IS 权重是 token 级权重的乘积:
    \begin{equation}
        \rho_{\text{seq}} = \prod_{t=1}^T \rho_t
    \end{equation}
    即使每个 $\rho_t$ 接近 1,累积后 $\rho_{\text{seq}}$ 可能非常大或非常小。

    \item \textbf{方差爆炸}:大的 IS 权重导致梯度估计方差急剧增大。

    \item \textbf{Policy Staleness}:长序列生成耗时长,策略 $\policy_\theta$ 在生成过程中已经更新多次,导致 $\policy_\theta$ 和 $\policy_{\text{old}}$ 偏离更大。
\end{enumerate}

\textbf{解决方案}:
\begin{itemize}
    \item GSPO:序列级 IS + 长度归一化
    \item CISPO:clip IS 权重而非 loss
    \item 减少更新步数,保持 policy 接近
\end{itemize}

\subsection{Q3:如何同时优化 Helpfulness / Harmlessness / Honesty?}

\textbf{答}:多目标对齐的常见方法:

\begin{enumerate}
    \item \textbf{多 Reward Model}:为每个目标训练独立的 RM,加权求和:
    \begin{equation}
        r_{\text{total}} = w_1 r_{\text{helpful}} + w_2 r_{\text{harmless}} + w_3 r_{\text{honest}}
    \end{equation}

    \item \textbf{Constitutional AI}:用规则和原则指导生成,自我批评和修正。

    \item \textbf{多阶段训练}:先优化 harmlessness,再优化 helpfulness。

    \item \textbf{Pareto 优化}:寻找多目标的 Pareto 前沿。
\end{enumerate}

挑战:不同目标可能冲突(如 helpful 但 harmful 的回答)。

\subsection{Q4:如何避免 Reward Hacking?}

\textbf{答}:Reward Hacking 是指模型学到获得高奖励但实际质量差的行为。防范方法:

\begin{enumerate}
    \item \textbf{KL 正则}:限制策略偏离参考模型。

    \item \textbf{多样化 RM 训练数据}:覆盖更多场景,减少 RM 的漏洞。

    \item \textbf{RM Ensemble}:使用多个 RM,取平均或最小值。

    \item \textbf{人工评估}:定期人工检查模型输出。

    \item \textbf{对抗训练}:生成可能 hack RM 的样本,加入训练。

    \item \textbf{过程监督}:使用 PRM 提供更细粒度的反馈。
\end{enumerate}

\begin{keypoint}
Reward Hacking 的根本原因是 Reward Model 不完美,无法完全捕捉人类偏好。长期解决方案是提升 RM 质量和多维度评估。
\end{keypoint}
